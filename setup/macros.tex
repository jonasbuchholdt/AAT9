%  A simple AAU report template.
%  2014-09-13 v. 1.1.0
%  Copyright 2010-2014 by Jesper Kjær Nielsen <jkn@es.aau.dk>
%
%  This is free software: you can redistribute it and/or modify
%  it under the terms of the GNU General Public License as published by
%  the Free Software Foundation, either version 3 of the License, or
%  (at your option) any later version.
%
%  This is distributed in the hope that it will be useful,
%  but WITHOUT ANY WARRANTY; without even the implied warranty of
%  MERCHANTABILITY or FITNESS FOR A PARTICULAR PURPOSE.  See the
%  GNU General Public License for more details.
%
%  You can find the GNU General Public License at <http://www.gnu.org/licenses/>.
%
%
%
% see, e.g., http://en.wikibooks.org/wiki/LaTeX/Customizing_LaTeX#New_commands
% for more information on how to create macros

%%%%%%%%%%%%%%%%%%%%%%%%%%%%%%%%%%%%%%%%%%%%%%%%
% Loads user defined variables
%%%%%%%%%%%%%%%%%%%%%%%%%%%%%%%%%%%%%%%%%%%%%%%%
\input{setup/variables.tex}

%%%%%%%%%%%%%%%%%%%%%%%%%%%%%%%%%%%%%%%%%%%%%%%%
% Macros for the titlepage
%%%%%%%%%%%%%%%%%%%%%%%%%%%%%%%%%%%%%%%%%%%%%%%%
%Creates the aau titlepage
\newcommand{\aautitlepage}[3]{%
  {
    %set up various length
    \ifx\titlepageleftcolumnwidth\undefined
      \newlength{\titlepageleftcolumnwidth}
      \newlength{\titlepagerightcolumnwidth}
    \fi
    \setlength{\titlepageleftcolumnwidth}{0.5\textwidth-\tabcolsep}
    \setlength{\titlepagerightcolumnwidth}{\textwidth-2\tabcolsep-\titlepageleftcolumnwidth}
    %create title page
    \thispagestyle{empty}
    \noindent%
    \begin{tabular}{@{}ll@{}}
      \parbox{\titlepageleftcolumnwidth}{
        \iflanguage{danish}{%
          \includegraphics[page=1,width=\titlepageleftcolumnwidth]{figures/aau_logo}
        }{%
          \includegraphics[page=2,width=\titlepageleftcolumnwidth]{figures/aau_logo}
        }
      } &
      \parbox{\titlepagerightcolumnwidth}{\raggedleft\sf\small
        #2
      }\bigskip\\
       #1 &
      \parbox[t]{\titlepagerightcolumnwidth}{%
        \iflanguage{danish}{%
          \textbf{Synopsis:}\smallskip\par
        }{%
          \textbf{Abstract:}\smallskip\par
        }
        \fbox{\parbox{\titlepagerightcolumnwidth-2\fboxsep-2\fboxrule}{%
          #3
        }}
      }\\
    \end{tabular}
    \vfill
    \iflanguage{danish}{%
      \noindent{\footnotesize\emph{Rapportens indhold er frit tilgængeligt, men offentliggørelse (med kildeangivelse) må kun ske efter aftale med forfatterne.}}
    }{%
      \noindent{\footnotesize\emph{The content of this report is freely available, but publication may only be pursued with reference.}}
    }
    \cleardoublepage
  }
}

%Create english project info
\newcommand{\englishprojectinfo}[8]{%
  \parbox[t]{\titlepageleftcolumnwidth}{
    \textbf{Title:}\\ #1\bigskip\par
    \textbf{Theme:}\\ #2\bigskip\par
    \textbf{Project Period:}\\ #3\bigskip\par
    \textbf{Project Group:}\\ #4\bigskip\par
    \textbf{Participants:}\\ #5\bigskip\par
    \textbf{Supervisor:}\\ #6\bigskip\par
    \textbf{Number of Pages:} \pageref{LastPage}\bigskip\par
    \textbf{Date of Completion:}\\ #8
  }
}

%Create danish project info
\newcommand{\danishprojectinfo}[8]{%
  \parbox[t]{\titlepageleftcolumnwidth}{
    \textbf{Titel:}\\ #1\bigskip\par
    \textbf{Tema:}\\ #2\bigskip\par
    \textbf{Projektperiode:}\\ #3\bigskip\par
    \textbf{Projektgruppe:}\\ #4\bigskip\par
    \textbf{Deltagere:}\\ #5\bigskip\par
    \textbf{Vejleder:}\\ #6\bigskip\par
    \textbf{Oplagstal:} #7\bigskip\par
    \textbf{Sidetal:} \pageref{LastPage}\bigskip\par
    \textbf{Afleveringsdato:}\\ #8
  }
}

%%%%%%%%%%%%%%%%%%%%%%%%%%%%%%%%%%%%%%%%%%%%%%%%
% An example environment
%%%%%%%%%%%%%%%%%%%%%%%%%%%%%%%%%%%%%%%%%%%%%%%%
\theoremheaderfont{\normalfont\bfseries}
\theorembodyfont{\normalfont}
\theoremstyle{break}
\def\theoremframecommand{{\color{aaublue!50}\vrule width 5pt \hspace{5pt}}}
\newshadedtheorem{exa}{Example}[chapter]
\newenvironment{example}[1]{%
		\begin{exa}[#1]
}{%
		\end{exa}
}


%%%%%%%%%%%%%%%%%%%%%%%%%%%%%%%%%%%%%%%%%%%%%%%%%
% Exponential function defined as upright e, \exp
%%%%%%%%%%%%%%%%%%%%%%%%%%%%%%%%%%%%%%%%%%%%%%%%%
\renewcommand{\exp}{\text{e}}


%%%%%%%%%%%%%%%%%%%%%%%%%%%%%%%%%%%%%%%%%%%%%%%%%%%
% Command \clearevenpage start chapter on even page
%%%%%%%%%%%%%%%%%%%%%%%%%%%%%%%%%%%%%%%%%%%%%%%%%%%
\makeatletter
\newcommand*{\clearevenpage}{%
  \clearpage
  \if@twoside
    \ifodd\c@page
      \hbox{}%
      \newpage
      \if@twocolumn
        \hbox{}%
        \newpage
      \fi
    \fi
  \fi
}
\makeatother

%%%%%%%%%%%%%%%%%%%%%%%%%%%%%%%%%%%%%%%%%%%%%%%%%%%%%%%
% Command \addunit. Use it to add si units to equations
%%%%%%%%%%%%%%%%%%%%%%%%%%%%%%%%%%%%%%%%%%%%%%%%%%%%%%%
\makeatletter
\providecommand\add@text{}
\newcommand\addunit[1]{%
  \gdef\add@text{\text{[}\ifthenelse{\equal{#1}{}}{\noSIunit}{\si{#1}}\text{]}\gdef\add@text{}}}% 
\renewcommand\tagform@[1]{%
  \maketag@@@{\llap{\add@text\quad}(\ignorespaces#1\unskip\@@italiccorr)}%
}
\makeatother

%%%%%%%%%%%%%%%%%%%%%%%%%%%%%%%%%%%%%%%%%%%%%
% Oversættelser af ord ved brug af \autoref{}
%%%%%%%%%%%%%%%%%%%%%%%%%%%%%%%%%%%%%%%%%%%%%
\addto\extrasdanish{%
  \def\figureautorefname{Figur}%
  \def\subfigureautorefname{Figur}%
  \def\tableautorefname{Tabel}%
  \def\partautorefname{Del}%
  \def\appendixautorefname{Bilag}%
  \def\equationautorefname{Ligning}%
  \def\Itemautorefname{Punkt}%
  \def\chapterautorefname{Kapitel}%
  \def\sectionautorefname{Afsnit}%
  \def\subsectionautorefname{Afsnit}%
  \def\subsubsectionautorefname{Underafsnit}%
  \def\paragraphautorefname{Delafsnit}%
  \def\Hfootnoteautorefname{Fodnote}%
  \def\AMSautorefname{Ligning}%
  \def\theoremautorefname{Sætning}%
  \def\pageautorefname{Side}%
  \def\requirementautorefname{Krav}%
}

\addto\extrasenglish{%
  \def\sectionautorefname{section}%
  \def\subsectionautorefname{section}%
  \def\subsubsectionautorefname{section}%
  \def\requirementautorefname{Requirement}%
  \def\algorithmautorefname{Algorithm}%
}

%%%%%%%%%%%%%%%%%%%%%%%%%%%%%%%%%%%%%%%%%%%%%%%%%%%%%%%%%%%%%%%%%%%%%%
% Tilføjer kommando \fullref{} for både at referere til nummer og navn
%%%%%%%%%%%%%%%%%%%%%%%%%%%%%%%%%%%%%%%%%%%%%%%%%%%%%%%%%%%%%%%%%%%%%%
\renewcommand*{\fullref}[1]{\hyperref[{#1}]{\autoref*{#1} \nameref*{#1}}} % One single link


%%%%%%%%%%%%%%%%%%%%%%%%%%%%%%%%%%%%%%%%%%%%%%%%%%%%%%%%%%%%%%%%%%%%%%
% Tilføjer forkortelser af ord.
%%%%%%%%%%%%%%%%%%%%%%%%%%%%%%%%%%%%%%%%%%%%%%%%%%%%%%%%%%%%%%%%%%%%%%
\newcommand{\AAU}{%
\iflanguage{english}{%
Aalborg University%
}{%
Aalborg Universitet%
}}

\newcommand{\opamp}{%
\iflanguage{english}{%
operational amplifier%
}{%
operationsforstærker%
}}

\newcommand{\hifi}{%
\iflanguage{english}{%
hi-fi amplifier%
}{%
hi-fi-forstærker%
}}

%%%%%%%%%%%%%%%%%%%%%%%%%%%%%%%%%%%%%%%%%%%%%%%%%%%%%%%%%%%%%%%%%%%%%%
% Command \DefVar{VARIBLE_NAME}
% Is used to create a variable.
% Set varible: \VARIBLE_NAME{VALUE}
% Get varible: \getVARIBLE_NAME
%%%%%%%%%%%%%%%%%%%%%%%%%%%%%%%%%%%%%%%%%%%%%%%%%%%%%%%%%%%%%%%%%%%%%%

\makeatletter%
\newcommand{\DefVar}[1]{\@namedef{#1}##1{\global\@namedef{get#1}{##1}}\@nameuse{#1}{}}%
\makeatother%

%%%%%%%%%%%%%%%%%%%%%%%%%%%%%%%%%%%%%%%%%%%%%%%%%%%%%%%%%%%%%%%%%%%%%%
% Requirements environment:
%
% \begin{requirement}
%   \requirement{}
%   \argument{}
%   \fullfilment{}
% \end{requirement}
%%%%%%%%%%%%%%%%%%%%%%%%%%%%%%%%%%%%%%%%%%%%%%%%%%%%%%%%%%%%%%%%%%%%%%

\def\reqPrefixName{??}% Individual prefix for the requirements
\newcounter{reqIDCounter}% Requirement counter for the subsections
\newcounter{requirement}% Absolute requirement counter. Used to trigger label/reference target.

\newlength{\reqBoxWidth}% Create length varible to define width of requirement box
\setlength{\reqBoxWidth}{5cm}% Actual width of requirement box

\newcommand{\reqPrefix}[1]{% Command to define requirement prefix
\ifthenelse{\equal{#1}{}}{\reqPrefixName}{\setcounter{reqIDCounter}{0}\def\reqPrefixName{#1}}%
}

\makeatletter% Define requirementID for references
\newcommand{\reqLabel}{%
\protected@edef\@currentlabel{\reqPrefixName\thereqIDCounter}%
\protected@edef\@currentlabelname{Requirement}%
}\makeatother%

\newenvironment{requirement}%
{\par\vspace{\baselineskip}\noindent\ignorespaces%
\refstepcounter{requirement}%
\stepcounter{reqIDCounter}%
\reqLabel%
\DefVar{requirement}\DefVar{argument}%\DefVar{fullfilment}%
%
\tabularx{1\textwidth}{p{\reqBoxWidth} !{\color{white}{\vrule width 2pt}} X}%
\greyrow \parbox[t]{\reqBoxWidth}{\textsb{Requirement~\reqPrefixName\thereqIDCounter}\\\getrequirement\vskip1mm}%
&%
\parbox[t]{\textwidth-\reqBoxWidth-4\tabcolsep-2pt}{\textsb{Argumentation}\\\getargument\vskip1mm}
%\addlinespace[2pt]%
%\greyrow \multicolumn{2}{l}{\parbox{\textwidth-2\tabcolsep}{\vskip1mm\textsb{Conditions of fulfilment}\\\getfullfilment\vskip1mm}}%
}%
{\endtabularx\par\ignorespacesafterend}


%%%%%%%%%%%%%%%%%%%%%%%%%%%%%%%%%%%%%%%%%%%%%%%%%%%%%%%%%%%%%%%%%%%%%%
% Tilføjer kommando \numnameref{labelnavn}
% Kommandoen tilføjer en reference til et afsnit med både nummer og navn.
%%%%%%%%%%%%%%%%%%%%%%%%%%%%%%%%%%%%%%%%%%%%%%%%%%%%%%%%%%%%%%%%%%%%%%
\newcommand{\numnameref}[1]{%
\hyperref[#1]{\autoref{#1}: \nameref{#1}}%
}

%%%%%%%%%%%%%%%%%%%%%%%%%%%%%%%%%%%%%%%%%%%%%%%%%%%%%%%%%%%%%%%%%%%%%%
% Defines subscript as upright text if it contains more than one character.
% 
%%%%%%%%%%%%%%%%%%%%%%%%%%%%%%%%%%%%%%%%%%%%%%%%%%%%%%%%%%%%%%%%%%%%%%

\catcode`\_=12% Makes underscore an inactive character
\begingroup\lccode`~=`\_% Loads underscore character for redefinition
\lowercase{\endgroup\def~}#1{% Start definition of underscore
\StrLen{#1}[\subscriptstringlen]% Determine length of subscript string
\ifthenelse{\subscriptstringlen=1}% Test if subscript string contain one character
{\sb{#1}}% Prints subscript as italic if only one character is present
{\sb{\mathrm{#1}}}}% Prints subscript as upright text if there are more than one character
\AtBeginDocument{\mathcode`\_=\string"8000 }% Makes underscore active only in math mode


%%%%%%%%%%%%%%%%%%%%%%%%%%%%%%%%%%%%%%%%%%%%%%%%%%%%%%%%%%%%%%%%%%%%%%
% Defind counter
% Bruges til forklaringer af ligninger og equpment
%%%%%%%%%%%%%%%%%%%%%%%%%%%%%%%%%%%%%%%%%%%%%%%%%%%%%%%%%%%%%%%%%%%%%%
\newcounter{firstexplain}% Holder styr på om det er den første symbolforklaring

%%%%%%%%%%%%%%%%%%%%%%%%%%%%%%%%%%%%%%%%%%%%%%%%%%%%%%%%%%%%%%%%%%%%%%
% Defind counter
% Bruges til forklaringer af ligninger og equpment
%%%%%%%%%%%%%%%%%%%%%%%%%%%%%%%%%%%%%%%%%%%%%%%%%%%%%%%%%%%%%%%%%%%%%%

%%%%%%%%%%%%%%%%%%%%%%%%%%%%%%%%%%%%%%%%%%%%%%%%%%%%%%%%%%%%%%%%%%%%%%
% Tilføjer kommando \startexplain, \stopexplain og \explain{variable}{test}{is unit}
% Bruges til forklaringer af ligninger.
%%%%%%%%%%%%%%%%%%%%%%%%%%%%%%%%%%%%%%%%%%%%%%%%%%%%%%%%%%%%%%%%%%%%%%
%\newcounter{firstexplain}% Holder styr på om det er den første symbolforklaring
%
%\def\startexplain{%
%	\setcounter{firstexplain}{1}%
%	{\noindent}%
%	{\par\noindent}
%	\begin{tabular}{@{}p{.06\columnwidth}p{.76\columnwidth}@{\hskip.04\columnwidth}p{.02\columnwidth}@{}}}%
%\def\stopexplain{\end{tabular}\\[10pt]}%
%
%\newcommand{\explain}[2]{%
%	\ifthenelse{\thefirstexplain=1}{%
%	Where:\\
%	&#1&[\ifthenelse{\equal{#2}{}}{\noSIunit}{#2}]\\\setcounter{firstexplain}{0}
%	}{%
%	&#1&[\ifthenelse{\equal{#2}{}}{\noSIunit}{#2}]\\%
%	}%
%}%




\def\startexplain{\setcounter{firstexplain}{1}{\noindent}{\par\noindent}\begin{tabular}{@{}p{.06\columnwidth}p{.04\columnwidth}@{}p{.72\columnwidth}@{\hskip.04\columnwidth}p{.02\columnwidth}@{}}}%

\def\stopexplain{\end{tabular}\\[10pt]}%

\newcommand{\explain}[3]{\ifthenelse{\thefirstexplain=1}{Where:\\
	&#1&#2&[\ifthenelse{\equal{#3}{}}{\noSIunit}{#3}]\\\setcounter{firstexplain}{0}
	}{%	
	&#1&#2&[\ifthenelse{\equal{#3}{}}{\noSIunit}{#3}]\\%
	}%
}%

%%%%%%%%%%%%%%%%%%%%%%%%%%%%%%%%%%%%%%%%%%%%%%%%%%%%%%%%%%%%%%%%%%%%%%
% Tilføjer kommando \startexplain, \stopexplain og \explain{}{}{}
% Bruges til forklaringer af ligninger.
%%%%%%%%%%%%%%%%%%%%%%%%%%%%%%%%%%%%%%%%%%%%%%%%%%%%%%%%%%%%%%%%%%%%%%

%%%%%%%%%%%%%%%%%%%%%%%%%%%%%%%%%%%%%%%%%%%%%%%%%%%%%%%%%%%%%%%%%%%%%%
% Tilføjer kommando \starteq, \stopequp og \equp{variable}{test}{is unit}{}
% Bruges til forklaringer af ligninger.
%%%%%%%%%%%%%%%%%%%%%%%%%%%%%%%%%%%%%%%%%%%%%%%%%%%%%%%%%%%%%%%%%%%%%%


\def\startequipment{\setcounter{firstexplain}{1}{\noindent}{\par\noindent}
\begin{table}[!ht]
\caption{Equipment list}
\begin{tabular}{@{}p{.23\columnwidth}|p{.40\columnwidth}@{}|p{.20\columnwidth}@{}|p{.10\columnwidth}@{}}}%

\def\stopequipment{\end{tabular}
\end{table}\\[10pt]}%

\newcommand{\equipment}[4]{\ifthenelse{\thefirstexplain=1}{
Description & Model & Serial-no & AAU-no \\ \hline
	#1 & #2 & #3 & #4 \\ \setcounter{firstexplain}{0}
	}{%	
	#1 & #2 & #3 & #4\\ %
	}%
}%


%%%%%%%%%%%%%%%%%%%%%%%%%%%%%%%%%%%%%%%%%%%%%%%%%%%%%%%%%%%%%%%%%%%%%%
% Tilføjer kommando \starteq, \stopequp og \equp{variable}{test}{is unit}{}
% Bruges til forklaringer af ligninger.
%%%%%%%%%%%%%%%%%%%%%%%%%%%%%%%%%%%%%%%%%%%%%%%%%%%%%%%%%%%%%%%%%%%%%%


%%%%%%%%%%%%%%%%%%%%%%%%%%%%%%%%%%%%%%%%%%%%%%%%%%%%%%%%%%%%%%%%%%%%%%
% Tilføjer figur
%%%%%%%%%%%%%%%%%%%%%%%%%%%%%%%%%%%%%%%%%%%%%%%%%%%%%%%%%%%%%%%%%%%%%%

\newcommand{\fig}[3]{
\begin{figure}[H]
\centering
\includegraphics[width=1\textwidth]{#1}
\caption{#2}
\label{#3}
\end{figure}}%

\newcommand{\plot}[3]{
 \begin{figure}[H]
	\centering
	\include{#1}
		\caption{#2}
		\label{#3}
\end{figure}}

\newcommand{\plotsettings}[0]{
width=120mm, 
height=74mm, 
at={(5mm,5mm)}, 
scale only axis,
grid=both,
grid style={line width=.1pt, draw=gray!10},
major grid style={line width=.2pt,draw=gray!50}, 
minor tick num=4,
axis x line*=bottom,
axis y line*=left,
axis background/.style={fill=white},
}

%%%%%%%%%%%%%%%%%%%%%%%%%%%%%%%%%%%%%%%%%%%%%%%%%%%%%%%%%%%%%%%%%%%%%%
% Tilføjer figur
%%%%%%%%%%%%%%%%%%%%%%%%%%%%%%%%%%%%%%%%%%%%%%%%%%%%%%%%%%%%%%%%%%%%%%


%%%%%%%%%%%%%%%%%%%%%%%%%%%%%%%%%%%%%%%%%%%%%%%%%%%%%%%%%%%%%%%%%%%%%%
% Tilføjer kommando \hyph
% Bruges til bindestreger i ord så de stadig kan deles korrekt af Latex.
%%%%%%%%%%%%%%%%%%%%%%%%%%%%%%%%%%%%%%%%%%%%%%%%%%%%%%%%%%%%%%%%%%%%%%
\def\hyph{-\penalty0\hskip0pt\relax}


%%%%%%%%%%%%%%%%%%%%%%%%%%%%%%%%%%%%%%%%%%%%%%%%%%%%%%%%%%%%%%%%%%%%%%
% Tilføjer kommando \hex og \bin \decibel
% Bruges til at skrive hexidecimal og binære tal.
%%%%%%%%%%%%%%%%%%%%%%%%%%%%%%%%%%%%%%%%%%%%%%%%%%%%%%%%%%%%%%%%%%%%%%
\newcommand{\hex}[1]{%
	\texttt{#1$_{16}$}
}%

\newcommand{\bin}[1]{%
	\texttt{#1$_{2}$}
}%

\newcommand{\dB}[1]{%
	\SI{#1}{\decibel}
}%


\newcommand{\Hz}[1]{%
	\SI{#1}{\hertz}
}%

\newcommand{\kHz}[1]{%
	\SI{#1}{\kilo\hertz}
}%

\newcommand{\dif}[0]{%
	$\Delta$
}%

\newcommand{\greyrow}{%
	\rowcolor{lightGrey}
}%

\newcommand{\calnumber}[1]{%
\def\variable{5}
\ifnum#1>1
    TRUE
  \else
    FALSE
\fi
}

%%%%%%%%%%%%%%%%%%%%%%%%%%%%%%%%%%%%%%%%%%%%%%%%%%%%%%%%%%%%%%%%%%%%%%
% Tilføjer kommando \citeref
% Bruges til at referere til egen rapport/bilag.
%%%%%%%%%%%%%%%%%%%%%%%%%%%%%%%%%%%%%%%%%%%%%%%%%%%%%%%%%%%%%%%%%%%%%%
\newcommand{\citeref}[1]{%
	[\autoref{#1}]%
}%


%%%%%%%%%%%%%%%%%%%%%%%%%%%%%%%%%%%%%%%%%%%%%%%%%%%%%%%%%%%%%%%%%%%%%%
% Tilføjer kommando \rot{}
% Bruges fx til at rotere kollonneoverskrifter i en tabel.
%%%%%%%%%%%%%%%%%%%%%%%%%%%%%%%%%%%%%%%%%%%%%%%%%%%%%%%%%%%%%%%%%%%%%%
\newcommand*\rot{\rotatebox{60}}


%%%%%%%%%%%%%%%%%%%%%%%%%%%%%%%%%%%%%%%%%%%%%%%%%%%%%%%%%%%%%%%%%%%%%%
% Tilføjer kommando \file{}
% Bruges til at indsætte et filnavn i teksten.
%%%%%%%%%%%%%%%%%%%%%%%%%%%%%%%%%%%%%%%%%%%%%%%%%%%%%%%%%%%%%%%%%%%%%%
\newcommand{\file}[1]{\texttt{#1}}

\input{setup/kodeudsnit_bibliotek}


\makeatletter
%\newcommand*{\getlength}[1]{\strip@pt#1}
% Or rounded back to `mm` (there will be some rounding errors!)
\newcommand*{\getlength}[1]{\strip@pt\dimexpr0.35146\dimexpr#1\relax\relax}
%
\makeatother

%%%%%%%%%%%%%%%%%%%%%%%%%%%%%%%%%%%%%%%%%%%%%%%%%%%%%%%%%%%%%%%%%%%%%%
% Tilføjer MATALB
% Bruges i stedet for gls
%%%%%%%%%%%%%%%%%%%%%%%%%%%%%%%%%%%%%%%%%%%%%%%%%%%%%%%%%%%%%%%%%%%%%%
\newcommand{\matlab}{MATLAB\textsuperscript{\textregistered} }

