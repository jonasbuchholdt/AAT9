\section{Single source}\label{ch:single_speaker_source}
To gain understanding about the interaction of several sources emitting sound simultaneously, it is essential to first investigate the behaviour of a single sound source. In the context of this project loudspeakers are investigated. The goal of this section is to find a way to represent the sound emission of a loudspeaker within the frequencies boundaries of this project (\SIrange{60}{300}{\hertz}).
The representation must be sufficiently accurate but still relatively simple to calculate in order to be useful in an optimization process.

%This section aims to introduce and analyse the fundamental for a single source, by analyse the behaviour of a baffled circular plane piston source. The pressure around the piston source will be analysed analytically, to determine the radiation of a single speaker from \SI{60}{\hertz} and upwards and compare it with the measurement from \autoref{ch:polar_response}. The \SI{60}{\hertz} lower limit enable the simulation to be validated by measurement in the AAU anechoic chamber and is a used lower limit for the low/mid driver in some line source array \citep{V-DOSC}.  The analyse shall end out with a approximated simulation model of the \gls{dut}.
\subsection{Pressure emission from an omidirectional source}\label{ssec:omni}
As explained in \autoref{ch:polar_response}, loudspeakers are commonly treated as omnidirectional sources at low frequencies. Taking advantage of this approximation, the sound pressure emission in free field conditions can be described conveniently. \citep[p. 171]{Kinsler2000} states the following equation \autoref{eq:omni_source} for a pulsating sphere. It has to be noted, that the position of the sphere in the given case is the acoustic center of the loudspeaker (see \autoref{sec:ac_center}) and therefore the radius $a$ is not the size of the diaphragm, but the radius of the pulsating sphere.
\begin{equation}\label{eq:omni_source}
p(r,t)\,=\,\rho_0 c V_0 \left(\frac{a}{r}\right)\cos \theta_a \textbf{\textit{e}}^{j\left(\omega t - k(r-a)+\theta_a\right)}
\end{equation}
\startexplain
\explain{$p$ is the sound pressure.}{\si{\pascal}}
\explain{$r$ is the distance, for which the pressure is being calculated, \(r>a\).}{\si{\meter}}
\explain{$t$ is the time, for which the pressure is being calculated.}{\si{\meter}}
\explain{$\rho_0$ is the specific density of air.}{\si{\kilo\gram\per\cubic\meter}}
\explain{$c$ is the speed of sound.}{\si{\meter\per\second}}
\explain{$V_0$ is the peak velocity at the surface of the spherical source.}{\si{\meter\per\second}}
\explain{$a$ is the radius of the spherical source.}{\si{\meter}}
\explain{$\theta_a$ is equal to $\tan(ka)$.}{\si{\radian}}
\explain{$\omega$ is the angular frequency.}{\si{\second^{-1}}}
\explain{$k$ is the wavenumber.}{\si{\meter^{-1}}}
\stopexplain

\subsection{Pressure emission from a baffled piston}\label{ssec:piston}
Another  approach to analytically approximate the behaviour of a cone loudspeaker is given by \citep[p. 179 ff.]{Kinsler2000}, where the behaviour of the sound radiated from a plane circular piston surrounded by an infinite rigid baffle is described. The piston model is derived from the model of a continuous line source as described in \citep[p. 176 f.]{Kinsler2000}, which states the following equation \autoref{eq:piston_source}.


%To characterize the directional properties of an analytical model of the \gls{dut}, the source will be modulated in two dimension as one baffled circular plane piston source. The analysis of a piston source in two dimension built on a thin piston source in three dimension where the x axis is fixed in the plot. This is possible because of vertical beam patten symmetry of \autoref{fig:continues_line_source} \citep{Kinsler2000}. The piston lays flat down so i look like a line source and have radius $a$. The baffled circular plane piston source can be considered as many continuous line sources, which detonate dx and is pointing towards the reader on \autoref{fig:continues_line_source}. The calculated pressure point will be in far field where $r>>a$ holds and the surface integral can be rewritten to a Bessel function \citep{Kinsler2000}. The pressure formula is therefore:

\begin{equation}\label{eq:piston_source}
p(r,\theta ,t)=\frac{j}{2} \rho_{0}c  V_{0}\frac{a}{r}ka \left ( \frac{2J_1(ka\, sin(\theta ))}{ka\, sin(\theta )} \right )e^{j(\omega t-kr)}
\end{equation}
\startexplain
	\explain{$r$ is the distance, for which the pressure is being calculated, \(r>a\).}{\si{\meter}}
	\explain{$\theta$ is the angle at which the pressure is calculated, relative to the direction, in which the piston is moving.}{\si{\radian}}
	\explain{$t$ is the time, for which the pressure is being calculated.}{\si{\meter}}
	\explain{$j$ is the imaginary unit}{\si{1}}
	\explain{$\rho_0$ is the specific density of air.}{\si{\kilo\gram\per\cubic\meter}}
	\explain{$c$ is the speed of sound.}{\si{\meter\per\second}}
	\explain{$V_0$ ist the peak velocity of the piston.}{\si{\meter\per\second}}
	\explain{$a$ is the radius of the piston.}{\si{\meter}}
	\explain{$k$ is the wavenumber.}{\si{\meter^{-1}}}
	\explain{$J_1$ is the Bessel function of the first kind of order 1.}{\si{1}}
	\explain{$\omega$ is the angular frequency.}{\si{\second^{-1}}}
\stopexplain

A major disadvantage of utilizing the piston model in the context of the project is, that no information about the cabinet of the \gls{dut} is incorporated. The concept of an infinite rigid baffle leads to results that are axis symmetric along the \SIrange{-90}{90}{\degree}-line in a polar plot. This is illustrated in \autoref{fig:piston_analytical}.
\begin{figure}[H]
	\centering
	\includegraphics[width=0.7\textwidth]{piston.eps}
	\caption{Normed SPL according to \autoref{eq:piston_source}, $a=165.3$\si{\milli\meter}$\,\approx 13$'', corresponding to the size of \citep{seas33}.}
		\label{fig:piston_analytical}
\end{figure}
\autoref{fig:piston_analytical} displays almost perfect omnidirectional behaviour for lower frequencies and lacks pressure towards $\pm$\SI{90}{\degree} for higher frequencies. The deviation from omnidirectionality of the upper part of the polar plot has the same qualitative appearance as the measured pressure curves in \autoref{fig:03_02_pressure_main}, however they significantly differ in attenuation when the angle approaching $\pm$\SI{90}{\degree}. When comparing the lower part of the piston model and the measured loudspeaker, the differences reach an unacceptable extent. The phase derived from the measured polar response of the speaker is shown in  \autoref{fig:03_02_m1_phase_main}  
\begin{figure}[H]
	\centering
	\includegraphics[width=0.7\textwidth]{03_02_meas1_pressure.eps}
	\caption{Normed \gls{spl}, measured at a distance \(d=\)\SI{2.74}{\meter}, see Appendix \ref{ax:directional_2}}
		\label{fig:03_02_pressure_main}
\end{figure}

\begin{figure}[H]
	\centering
	\includegraphics[width=0.7\textwidth]{03_02_meas1_phase.eps}
	\caption{Phase, measured at a distance \(d=\)\SI{2.74}{\meter}, see Appendix \ref{ax:directional_2}}
		\label{fig:03_02_m1_phase_main}
\end{figure}

\autoref{fig:03_02_pressure_main} illustrates, that the solutions obtained with \autoref{eq:piston_source}, where the resulting polar plot is symmetric along the \SIrange{-90}{90}{\degree}-axis, do not realistically represent how the pressure is emitted on the back side of the loudspeaker. Specifically, the indents in the curves for frequencies $\ge$\SI{200}{\hertz} around $\pm$\SI{125}{\degree} in \autoref{fig:03_02_pressure_main} deviate from the shape that is drawn by the piston model. 
%The movement of the surface is described with

%\begin{equation}
%v = V_{0} \cdot exp(j \omega t)
%\end{equation}

%\startexplain
%    	\explain{$v$ is the complex speed of the line source }{\si{1}}
%        \explain{$V_{0}$ is the Amplitude}{\si{1}}
%        \explain{$j$ is the imaginary unit }{\si{1}}
%        \explain{$\omega$ is the angular velocity }{\si{1}}
%        \explain{$t$ is the time }{\si{1}}
%\stopexplain
    
%Each small sources is treated as an baffled simple source with a width of $dx$ and the source strange can be modulated as following      
%
%\begin{equation}
%dQ = V_{0} 2 a\, sin(\phi) \, dx
%\end{equation}
%
%    \startexplain
%    		\explain{$dQ$ is the simple source strange }{\si{1}}
%        \explain{$V_{0}$ is the Amplitude}{\si{1}}
%        \explain{$a$ is the radius for cylinder }{\si{1}}
%        \explain{$\phi$ is the angle between the radius $a$ and the x axis}{\si{1}}
%        \explain{$dx$ is the length for the simple source }{\si{1}}
%    \stopexplain    
% 
%The following \autoref{fig:continues_line_source} shows an example of the continues line source where one of the small source is showed with width $dx$ and length $2a \, sin(\phi)$. 
%
%\begin{figure}[H]
%	\centering
%\begin{picture}(0,0)%
%\includegraphics{line_source.pdf}%
%\end{picture}%
%\setlength{\unitlength}{746sp}%
%%
%\begingroup\makeatletter\ifx\SetFigFont\undefined%
%\gdef\SetFigFont#1#2#3#4#5{%
%  \reset@font\fontsize{#1}{#2pt}%
%  \fontfamily{#3}\fontseries{#4}\fontshape{#5}%
%  \selectfont}%
%\fi\endgroup%
%\begin{picture}(31223,16833)(-5411,-436)
%\put(12106,7139){r'}%
%\put(24121,15599){P(r,$\theta$,t)}%
%\put(946,2684){$\theta$}%
%\put(23851,659){x}%
%\put(-89,16094){z}%
%\put(11341,8354){r}%
%\put(3466,-571){$a$}%
%\put(1441,-376){$dx$}%
%\put(-134,-421){0}%
%\put(-4049,-571){$-a$}%
%\put(226,1649){$\Delta$r}%
%\end{picture}%
%	\caption{The model of a continues line source where y axis is pointing towards the reader. (ref the book)}
%		\label{fig:continues_line_source}
%\end{figure}




%\subsection{Comparison: calculations based on the baffled piston and measured polar respone}
%In this section, the \gls{dut} will be simulated as a baffled circular plane piston source as described in \autoref{ch:single_speaker_source} and compared to the actually measurement of the \gls{dut} \autoref{ax:directional_2}. An piston simulated model of \citep{seas33} in MATLAB shows in \autoref{fig:piston_model_of_seas33}
%
%\begin{figure}[H]
%	\centering
%	\includegraphics[width=0.8\textwidth]{piston_model.pdf}
%	\caption{The figure shows  The \gls{dut} which correspond to this figure is a \citep{seas33}}
%		\label{fig:piston_model_of_seas33}
%\end{figure}
%
%The actually measurement of the \citep{seas33}  \autoref{ax:directional_2}.
%
%\begin{figure}[H]
%	\centering
%	\includegraphics[width=0.8\textwidth]{meas1_seas.pdf}
%	\caption{The figure shows  The \gls{dut} which correspond to this figure is a \citep{seas33}}
%		\label{fig:speaker_model}
%\end{figure}

\subsection{Fitting the omnidirectional source model to the \gls{dut}}\label{sec:correction}
The analytical descriptions of both the pulsating sphere (see \autoref{ssec:omni}) and the piston (see \autoref{ssec:piston}) do not describe the measured behaviour of the \gls{dut} in an adequate manner over the whole frequency range of interest. In order to be able to perform an optimization of the signal processing parameters for the speaker array (see \autoref{ch:optimization}), an enhanced model has to be obtained. For the sake of this project, it has been chosen to rely on measured data.
Because the pressure and phase characteristic of the \gls{dut} have been ascertained in measurements (see Appendices \ref{ax:directional_1},\ref{ax:directional_2}), the measurements results can be processed to obtain a correction table relating to the model of the omnidirectional source. This is achieved by calculating a difference of the measured data from the ideal values, that would suggest omnidirectionality at any given frequency. The resolution of the correction grid is determined by angular resolution and the frequency resolution of the polar response measurement, that it is based on.  \autoref{fig:correction_3d} shows a pressure correction contour plot, based on the polar response from Appendix \ref{ax:directional_2}.
\begin{figure}[H]
	\centering
	\includegraphics[width=1\textwidth]{correction_3d.eps}
	\caption{Contour plot of the pressure correction data, based on measurement data from Appendix \ref{ax:directional_2}}
		\label{fig:correction_3d}
\end{figure}
With the correction values applied, an augmented form of \autoref{eq:omni_source} can be established. Note, that while the deviation from omnidirectionality is visualized in \autoref{fig:correction_3d} in a \si{\decibel}-scale, for the usage in the \textsf{Dev}-function in \autoref{eq:aug_omni}, data has to be presented linearly.
\begin{equation}\label{eq:aug_omni}
p(r,t)\,=\,\rho_0 c V_0 \left(\frac{a}{r}\right)\cos \theta_a \textbf{\textit{e}}^{j\left(\omega t - k(r-a)+\theta_a\right)}\cdot\textsf{Dev}(f,\phi)
\end{equation}

\startexplain
\explain{$\textsf{Dev}$ is the pressure deviation from the correction table.}{\si{1}}
\explain{$f$ is the frequency.}{\si{\hertz}}
\explain{$\phi$ is the angle, for which the pressure is being calculated relative to the main axis of the speaker.}{\si{\radian}}
\stopexplain



\section{First order gradient source}\label{sec:first_order_speaker}
The aim of this section is to do an analysis of a well known and commonly used low frequency beamforming technique with two loudspeakers \citep{McCarthy2016}. The technique is often referred to as the first order gradient speaker and it is based on using a phase difference to make beamforming \citep{olson1973gradient}.
It can be seen in \autoref{ssec:piston} that the single speaker is nearly omnidirectional both in phase and pressure. Because the phase is omnidirectional, it is possible to place the two speaker in a line with a shifted phase, such that the signals in front of both loudspeakers are in phase, whereas the back the phase of the signal from the two speakers is \SI{180}{\degree} apart. This means that the pressure in the front is added and the pressure in back cancels out. Sometimes this beamforming technique also named as the Two-Element In-Line Technique. A cardioid low frequency array can be set up with only two speakers, where both speakers are pointing towards the audience while standing in line width the horizontal main lobe. The benefit of placing the speakers this way is, that the phase between the two speaker can be controlled such that the pressure at the back is lowered compared to the pressure at the front. One arising problem with this technique is that the speaker position is constant no matter which frequency the speaker is playing. Therefore this array can only set up optimally for one frequency, where the distance between the speaker and the phase is designed to work. Outside this exact frequency the array front pressure will drop unless the pressure emitted by both of the speakers is increased. This problem is arising, because in the front, the phase between the signals from both speakers is not \SI{0}{\degree} anymore. The following \autoref{fig:gradient_problem} shows the frequency response characteristic of such an array where the back speaker is operating in with a phase shift of \SI{180}{\degree} related to the front speaker.

\begin{figure}[H]
	\centering
	\includegraphics[width=1\textwidth]{first_order_speaker.pdf}
	\caption{Schematic diagram, frequency response characteristic, and polar directional characteristics of a unidirectional first-order gradient sound radiator. \citep{olson1973gradient}}
		\label{fig:gradient_problem}
\end{figure}

It can be seen in the frequency response in \autoref{fig:gradient_problem} that the technique works best when the ratio between the distance between the two speakers $D$ and the wavelength $\lambda$ is $0.5$. When the ratio between $D$ and $\lambda$ grows larger than $0.5$ the beamforming shape starts to radiate to the sides instead of to the front. When the ratio is less than $0.5$ the beamforming keeps the cardioid shape, but the pressure in the front drops. This indicates, that it is the highest frequency of interest, that shall be used to determine the distance between the speakers with the $D$ and $\lambda$ relation, in order to keep cardioid shape over the whole frequency range of interest.\\ 

The formerly described technique does enable the speakers array to radiate in a cardioid shape, but the direction of the cardioid shape can only be controlled forward and backwards and not in any other direction.

\section{Conclusion on Sound Sources}
Investigating analytical ways to express the behaviour of a speaker by incorporating no other information than its size leads to results that are not particulary accurate and are therefore of limited use. While it appears to be possible to get better estimates by incorporating more information into a numerical model (e.g. \citep{vanderkooy10}), the given objective of using the model for an optimization makes this approach unfeasible. Because measurements have been conducted investigating the directional characteristics of the loudspeaker, that will later be used, a more empirical approach has been found. Exploiting information from the measurements, the relatively simple model of a pulsating sphere can be augmented, so that it fits the behaviour of the \gls{dut}.\\
The first-order gradient source technique has been analysed. It has been documented, that the first-order loudspeaker array is able to do beamforming resulting a cardioid directional characteristic. Unfortunately the first-order loudspeaker array is only able to do cardioid shape in the frontal and backwards direction and cannot angle the beam by changing the signal parameters. Another beamforming array has to be developed for this project.