\section{Speech in communication}

This section will analyse the human speech intelligibility with respect to the frequency range for the human through communication channel e.g. air, bone or electronic systems. The intelligibility of a speech depends on the sound quality, noise and the frequency range of the voice. This section will focus on the frequency of the the voice, to determined the necessary frequency range in electronic systems for communication. 

The intelligibility of the voice is an important factor in communication and are therefore a factor in the frequency range of the human voice. The fact that the fundamental frequency for the voice is in the low frequency area does not necessary mean that all high frequency above that fundamental frequency area can be filtered away. The overtone of the voice might affect the intelligibility of speech. Therefore the intelligibility of speech is the focus area of this reachurch of the voice frequency range. 

One on the most important acoustics factor in the human speech is the glottal oscillation, which is fundamental frequency of the voice. The fundamental frequency is person dependent but the mean differs from male and female and the type of communication e.g. song speech or jelling. For male the mean fundamental frequency for conversational speech is \SI{120}{\hertz} and for female is is \SI{200}{\hertz}. The fact that the fundamental frequency is that low does not mean that those frequency are important in the frequency range for the human conversational speech output of the mouth. The throat speaking system contain a transfer function from the the glottal oscillation to the output of a mouth \citep{pulkki2015}. The following \autoref{fig:speech_system} shows the path way for the air from the glottal to the lips.


 \begin{figure}[H]
	\centering
		\includegraphics[width=1\textwidth]{glottal}
		\caption{The figure shows the path way from the glottal to the lips \citep{pulkki2015}}
		\label{fig:speech_system}
\end{figure}

The transfer function is human depending and more specific throat depending. The mean length for the throat from generation of the glottal oscillation to the lips is \SI{17}{\centi\meter} for male and \SI{14}{\centi\meter} for female. glottal oscillation is mixed with unvoiced airflow in the throat which make turbulence in the transfer for voice. This generate e.g. noise and explosive transient-like sound in the voice output \citep{pulkki2015}. The following \autoref{fig:speech_transfer_system} shows a signal block diagram for the speech transfer function.

 \begin{figure}[H]
	\centering
		\includegraphics[width=1\textwidth]{speech_transfer}
		\caption{The figure shows a block diagram for the speech transfer function \citep{pulkki2015}}
		\label{fig:speech_transfer_system}
\end{figure}


In fact with the speech transfer function the frequency range can start below \SI{100}{\hertz} and go upon several \si{\kilo\hertz}. In language depending speech there is some low frequency fundamental component generating by the tongue there goes down to \SI{20}{\hertz}. The tongue is generating it by trilling e.g. "r'' in Spanish. The highest frequency component is generated by singing and goes up to \SI{7}{\kilo\hertz} \citep{pulkki2015}.

At this point all frequency above  \SI{7}{\kilo\hertz} do not affect the intelligibility of the voice, since the voice do not generate frequency above. The range from \SI{20}{\hertz} to \SI{7}{\kilo\hertz} is then the outer limit but the frequency area might not be that wide when it is only speech. Filtering away some of the frequency might not affect the speech intelligibility. 

