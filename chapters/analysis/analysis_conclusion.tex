\section{Conclusion}

It can be concluded that sound from a \gls{bct} is mostly said to be transferred through the bone, but research shows that there are multiple pathways contributing to sound transmission and that elements such as skin and the auditory canal have a big influence on the sound perception through \gls{bc}. %It was discovered in \autoref{sec:bonepaths} that the skin can attenuate the sound approximate \SI{20}{\decibel} in the \gls{bc} frequency range. It was also discovered that the sound are not only going through the bone, but also through the rest of the auditory canal. The bone and skin in the ear canal is radiating sound intro air which make the eardrum to move. The ossicular chain transfer the air born sound to the cochlear but the chain is also affected it self by vibration in the skull, which makes the bone to move. The fluid on the inside of the skull is also affected by the skull vibration and transfer sound through the cochlear fluid canal.  
It can also be concluded that the excitation of the \gls{bm} from \gls{bc} sound is similar as per \gls{ac}. %Study have shown that \gls{bc} sound can be successfully cancelled with present of \gls{ac} sound where only the phase and amplitude is manipulated in the \gls{ac}. 

%It can be concluded that the fundamental frequency of the voice lays down in the the octave band of \SI{125}{\hertz} and \SI{250}{\hertz} but only \SI{5}{\percent} of the speech intelligibility are within those two octave band. The throat, mouth and the nose is a transfer system which change the sound from the glottal in such way that the frequency range for speech goes from \SI{20}{\hertz} to \SI{7}{\kilo\hertz}, where the highest speech intelligibility lays in the octave band of \SI{500}{\hertz}, \SI{1}{\kilo\hertz}, \SI{2}{\kilo\hertz} and \SI{4}{\kilo\hertz}. The necessary frequency range for good speech intelligibility is therefore from \SI{355}{\hertz} to \SI{5.68}{\kilo\hertz} when including the outer limit of the octave band.

%It can be concluded that the Radioear B71 does not cover the full frequency area that is needed for good speech intelligibility but cover the most. The problem with the B71 is that the distortion from \SI{1}{\kilo\hertz} and below is higher than \SI{10}{\percent} and therefore the B71 is concluded to be infeasible for communication use. The newer model, Radioear B81 are therefore analysed and it can be concluded that the B81 is feasible for communication, since the \gls{best} principle makes the frequency range wider because of lower distortion. In fact the frequency range from \SI{355}{\hertz} to \SI{5.68}{\kilo\hertz} is covered with the B81 and the distortion is generally less than \SI{4}{\percent} and in most of the interesting frequency range the distortion is lower than \SI{1}{\percent}.

After researching available \gls{bct}, it is decided that this project will use the Radioear B81, focusing in the frequency range 
%and the research frequency shall at least cover the frequency range 
from \SI{355}{\hertz} to \SI{5.68}{\kilo\hertz} to cover more than \SI{90}{\percent} of the speech intelligibility frequency range. It could be possible to use a \gls{bct} with a wider frequency response, but not with a narrower one.
%The frequency range is allowed to be wider but not smaller to ensure high probability for high speech intelligibility.