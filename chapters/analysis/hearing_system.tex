\section{Auditory System}
The term \textit{auditory system} refers to the sensory system for the perception of sound. In context with the given project, an overview of the paths through which sound waves reach the nerve cells on the basilar membrane. There are two general principles: the \gls{ac} and \gls{bc}. Airborne waves lead to cochlea stimulation by moving the tympanic membrane whereas boneborne sound is stimulating the cochlea through vibrations of the skull. The pathway for airborne sound, which starts from the pinna, through the ear canal to the tympanic membrane and through the middle ear to the basilar membrane in the cochlea has been studied extensively and is regarded to be well known. Opposed to this, the details of \gls{bc} have not been investigated entirely \citep{stenfelt_2005} and are the subject of recent studies. The following section outlines both hearing through \gls{ac} and \gls{bc}.


\section{Air Conduction of Sound}
\label{sec:ear_functions}
The typical way to stimulate the nerve cells on the basilar membrane, and therefore perceive sound, is via \gls{ac}. The pathway from airborne wave outside the ear to the cochlea can be divided into three sections(\autoref{fig:hearing_system}). 
 \begin{figure}[H]
	\centering
		\includegraphics[width=1\textwidth]{anatomy-of-ear.jpg}
		\caption{Breakdown of the different ear parts.}
		\label{fig:hearing_system}
\end{figure}

In the first section, the sound wave is travelling through the outer ear, which consists of the pinna and the ear canal.The shape of the pinna differs amongst the population. A unique transfer function is attributable to the pinna. Combining the influence of the pinna, with that of the ear canal and the torso, the resulting transfer function is referred to as \gls{hrtf}. It has to be noted, that \gls{hrtf} are dependent on the incidence direction of sound.% An individual's brain is conditioned to the exact \gls{hrtf} of that person. Presenting sounds with another \gls{hrtf} to a person will initially lead to confusion, but over time the brain can adapt a new \gls{hrtf} if necessary. 
After sound has entered the outer ear and has been modified by the \gls{hrtf}, it reaches the tympanic membrane, also called the eardrum. It is set into motiong according to the air pressure variation in the ear canal and transmits sound into the middle ear.

The second section is comprised by the middle ear, where the so called Tympanic membrane, that seperates the middle ear and the outer ear, converts airborne waves into mechanical movement. The middle ear consist of three auditory bones (\textit{ossicles}): the hammer (Malleus), the anvil (Incus) and the stirrup (Stapes). Through them, the sound wave is transmitted mechanically. The Stapes is attached to the oval window and the Malleus is attached to the tympanic membrane. The bones act as an impedance adaptation from airborne wave to a liquid-borne wave. The displacement of the tympanic membrane, which is the input of the impedance adaption, is adapted, such that the displacement of the oval window is approx. 20 times that of the tympanic membrane. The middle ear is connected to the Pharynx by the Eustachian tube. Its purpose is to equalise changes in air pressure on both sides on the eardrum. A pressure build up in the middle ear will affect the hearing negatively. %The middle ear consists of three auditory bones (\textit{ossicles}) and is joined by the Eustachian tube. One of the ossicles is connected to the Tympanic membrane and another ossicle is connected to the oval window. The remaining ossicle connects the formerly mentioned ones. 

The third section is the inner ear, which consists of the cochlea and the vestibular system. The latter is a system for equilibrioception and does not have any influence on hearing. For the hearing only the cochlea is of interest in the inner ear . The oval window, that is set into motion by the ossicles in the middle ear, transfers the motion into a fluid-borne wave \citep{ho_2017}.
Vibrations of the oval window cause a wave travelling in the cochlea fluid. Those fluid-borne waves cause a displacement of the basilar membrane \citep{ho_2017}. The vibration of the basilar membrane stimulates the inner hair cells at the organ of corti. These then generate an electric response that is transmitted through the auditory nerve to the brain.



\section{Bone Conduction of Sound}\label{sec:bonepaths}
At the moment there is not a general delimited definiton of what \gls{bc} means, but it is mostly accepted as the sound transmission through the bones of the skull. However, \gls{bc} is not only limited to bones, since it usually also involves transmission through soft tissue and cartilage. Therefore, it can be identified as the transmission of sound energy through the body (bones, soft tissue and cartilage). 
\gls{bc}, although being performed in a localised way, involves the outer, middle and inner ear, as well as the bones surrounding the hearing apparatus and both its fluids and \gls{csf} \citep{puria_2013}.
 \begin{figure}[H]
	\centering
		\includegraphics[width=1\textwidth]{more-on-the-vestibular-system-of-anatomy-of-cochlear-aqueduct.jpg}
		\caption{Detail of middle and inner ear}
		\label{fig:hearing_system_detail}
\end{figure}
Stenfelt's \citep[section 6.1.1]{puria_2013} starts with the following quote: \enquote{One of the quintessential questions about BC hearing is the end organ for
transforming BC vibration in the skull to neural code}. From this quote, we can infer, that the determination of the endpoint for \gls{bc} is not an easy task and could be the pinnacle in fully understanding \gls{bc} sound transmission.

As explained throughout the referenced section, several studies and experiments have been conducted over the years, leading to the result of identifying the cochlea as such organ. Once the endpoint was identified, it was primordial to understand the ways in which the sound is propagating to reach it.
\subsection{Propagation Factors}

When analysing \gls{bc} propagation, we have to take into account three main factors:

\begin{itemize}
\item \textbf{Frequency}: there have been several studies and experiments over time both focused on skull and cochlea vibration patterns. These studies have served as ground for identifying four main modes within the human skull vibration pattern for frequencies below \SI{10}{\kilo\hertz}.
\begin{enumerate}
\item Below \SI{400}{\hertz}: within this range the skull moves as a rigid body (Fig \ref{fig:skull_vibration_pattern}a).\citep{stenfelt_2005b}
\item Between \SI{400}{\hertz} and \SI{1}{\kilo\hertz}: within this range the skull's motion can be described as a mass-spring system(Fig \ref{fig:skull_vibration_pattern}b).
\item Between \SI{1}{\kilo\hertz} and \SI{2}{\kilo\hertz}: at \SI{1}{\kilo\hertz}, the first free resonance of the skull appears \citep{hakansson_1994} and the motion transitions from mass-spring system to a system dominated by wave transmission.
\item Above \SI{2}{\kilo\hertz}: when reaching this range, wave transmission dominates the skull vibration pattern completely, with differences between the cranial vault and the skull base (Fig \ref{fig:skull_vibration_pattern}c).
\end{enumerate}
\begin{figure}[H]
	\centering
		\includegraphics[width=1\textwidth]{skull_vibration_pattern.pdf}
		\caption{Two-dimensional illustration of the vibration modes of the human skull at frequencies between 0.1 and 10 kHz. The thick arrows indicate the stimulation position and the thin arrows indicate the response directions. The rigid body response at the lowest frequencies is illustrated in (a), while the response at frequencies between approximately 0.3–1.0 kHz that is similar to a massspring system is shown in (b), where three sections of the skull move sequentially in opposite directions. In (c) the vibration responses for frequencies above 2 kHz is illustrated differently for the skull base and the cranial vault: at the skull base longitudinal wave propagation dominates the response while a mixture of vibration modes including bending waves is present at the cranial vault. Image source: \citep{puria_2013}}
		\label{fig:skull_vibration_pattern}
\end{figure}
\item \textbf{Skin and soft tissue}: generally, \gls{bc} transducers are placed pressed on skin-covered bone, meaning that the sound transmission will be affected by the presence of skin and soft tissue between the transducer and the bone. This often means that the sound transmission is attenuated when the skin is present, with an increase of up to \SI{20}{\decibel} in the acceleration threshold level. The threshold level shift caused by the skin is generally not frequency depended and stays mostly the same from \SI{250}{\hertz} to \SI{8}{\kilo\hertz} \citep{hakansson_1985}.
\item \textbf{Position of the transducer}: the stimulation area is one of the key points in terms of increasing the sensitivity to \gls{bc} sound. A good sensitivity can be achieved by positioning the stimuli close to the cochlea, which usually leads to the mastoid being selected as stimulation area. The reason for the improved sensitivity in this area could be related to the in-line position of the bone with regards to the petrous bone encapsulating the cochlea, or it could involve any of the skull bone structures \citep{puria_2013}. Therefore, further testing will be performed in order to identify the best position for our study case.
\end{itemize}
\subsection{Influence of Inner/Middle/Outer ear}

Early theories suggested that there were two main pathways dominating the \gls{bc} perception. However, more recent studies suggest, that there are five reported pathways for sound vibration to propagate or transmit to the cochlea \citep{zhang_2016}:
\begin{enumerate}
\item External auditory canal sound radiation (outer ear).
\item Inertia of the ossicular chain (middle ear).
\item Inertia of the cochlear fluid (inner ear).
\item Cochlear walls' compression (inner ear).
\item \gls{csf} pressure (inner ear).
\end{enumerate}

These pathways can work independently, but all of them contribute to the quality of the signal received by the cochlea. A failure in one of the pathways leads to a decrease in the quality of the received stimuli. In  \autoref{fig:hearing_system_pathway}, the relation between \gls{bc} and \gls{ac} stimuli is depicted, clearly showing how \gls{bc} stimulation can generate \gls{ac}-like responses in the outer, middle and inner ear.

 \begin{figure}[H]
	\centering
		\includegraphics[width=1\textwidth]{ovidweb}
		\caption{Transmission paths for \gls{ac} and \gls{bc}, image source:  \citep{stenfelt_2005}}
		\label{fig:hearing_system_pathway}
\end{figure}

\subsection*{Outer ear}

%The outer ear consists of the pinna and the ear canal, meaning that half its structure is surrounded by cartilage and half by bone. 
When using \gls{bc} stimulation, sound pressure is produced in the ear canal due to vibration transmission in the ear canal walls. This sound pressure is transmitted to the cochlea in a similar way as \gls{ac} sound does, meaning that the \gls{bc} component of the outer ear needs to go through the middle ear before reaching the cochlea. The frequency response of the ear canal, as well as the sound pressure generated inside it can be manipulated e.g. by occlusion of the ear canal. A method for estimating the contribution of the ear canal sound pressure to \gls{bc} perception is described in \citep{puria_2013}. Through \gls{ac} and \gls{bc} separately, tones that are perceived to have the same loudness are generated. The sound pressure is then measured in the open ear canal. With this test, it has been observed, that above \SI{500}{\hertz}, the pressure that is present during \gls{bc} stimulation is significantly smaller than during \gls{ac} stimulation. Therefore, it can be concluded that the ear canal sound pressure is not the dominant contributor to \gls{bc} sound.

\subsection*{Middle ear}

%The middle ear is the portion of the ear between the eardrum and the oval window of the inner ear. As explained in \autoref{sec:ear_functions}, it consists of three ossicles (maellus, incus and stapes), the tympanic cavity and the Eustachian tube. 
The middle ear contributes to \gls{bc} sound perception by transmission through ossicular inertia and sound pressure in the tympanic cavity. The influence of the middle ear ossicles on \gls{bc} has been thoroughly studied and references can be found in \citep[Sec. 6.5]{puria_2013}. The contribution to \gls{bc} perception widely varies depending on the frequency range and on the status of the middle ear ossicles themselves.

\subsection*{Inner ear}

%The inner ear is the innermost part of the ear, composed by the cochlea and the vestibular system. As stated previously, the cochlea is the end-organ for \gls{bc} sound transmission, and therefore will be the endpoint for all \gls{bc} transmission components or pathways. 
The inner ear is the dominating contibutor for \gls{bc} sound perception and although the middle and outer ear are also involved in the transmission, their effect on \gls{bc} sensitivity is very little compared to the inner ear. This is the main reason why \gls{bc} tresholds are compared with \gls{ac} tresholds in audiology in order to diagnose conductive impairment. However, there is still an ongoing debate over the exact processes that result in cochlear's \gls{bc} sound perception. Some of the major recent theories can be found in \citep[Sec. 6.3.3]{puria_2013}.
