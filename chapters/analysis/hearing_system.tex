\section{Hearing system}
The hearing system refers to the pathway of the sound wave to the nerve cells on the basilar membrane, where only the natural hearing ways are of interest in this project. The natural hearing ways are the \gls{ac} and \gls{bc} sound way, where the airborne wave is stimulating the cochlea by moving the tympanic membrane, and the boneborne wave is stimulating the cochlea by vibration in the skull. The pathway from the tympanic membrane to the basilar membrane in the cochlea is well known and well understood, while the \gls{bc} is not completely understood \citep{stenfelt_2005} The following section will give an introduction to both the \gls{ac} and \gls{bc} hearing.


\section{Air conduction of sound}
The natural way to stimulate the nerve cells in the basilar membrane is the \gls{ac} with airborne sound. The pathway from airborne wave outside the ear to the cochlea can be divided intro three paths. In the first path, the sound wave is travelling through is the outer ear, which consists of the pinna and the ear canal. The second pathway is the middle ear, where the Tympanic membrane is the membrane which convert airborne wave to mechanical movement. The middle ear consists of three auditory bones and the Eustachian tube. One bone is connected to the Tympanic membrane and one bone is connected to the oval and the last bone connect those bones together. The third pathway is the inner ear where the oval window is a membrane that convert mechanical borne wave intro liquid borne wave \citep{ho_2017}. The inner ear consists of the cochlea and the vestibular system, where the vestibular is a system for balance and do not have any hearing influence. For the hearing only the cochlea is of interest in the inner ear. The following \autoref{fig:hearing_system} shows the full hearing system.


 \begin{figure}[H]
	\centering
		\includegraphics[width=1\textwidth]{anatomy-of-ear.jpg}
		\caption{The figure shows the outer ear, the middle ear and the inner ear.}
		\label{fig:hearing_system}
\end{figure}

\subsection{Functions for ear parts}
The pinna of the outer ear is shaped differently for every person and is very important for localisation of sound event. The shape of the pinna makes amplification, peaks and dips in the sound wave and including the ear canal the changing of the sound in the outer ear is called the \gls{hrtf}. Since the pinna is different for every person, every person have there own \gls{hrtf} and the brain have learned from birth the exact \gls{hrtf}. By changing the \gls{hrtf} on a person, the brain will be confused, but with time the brain can learn a new \gls{hrtf} if necessary. After the sound waves have entered the outer ear and changed by the pinna and the ear canal the tympanic membrane, also called the eardrum, is moving accordingly to the air pressure variation in the ear canal and transmitted to the middle ear.  

The middle ear consist of the three auditory bones, the hammer (malleus), the anvil (incus) and the stirrup (stapes), where the sound wave is travelling mechanically. The bone act as a impedance adaptation from air to liquid and have an amplification of approximation 20 times. The Eustachian tube function is to equalise the air pressure on both side on the eardrum. A build up pressure in the middle ear will affect the hearing negatively. The bone Mallus is attached to the oval window where stapes is attached to the tympanic membrane.

Vibration of the oval window cause wave travelling in the cochlea fluids, those fluid borne wave makes the basilar membrane to vibrate because pressure difference between the cochlear perilymphatic \citep{ho_2017}. The vibration of the basilar membrane gets the inner hair cell to move and generate electric response signal that is transmitted through the auditory nerve to the brain.



\section{Bone conduction of sound}
At the moment there is not a general delimited definiton of what \gls{bc} means, but it is mostly accepted as the sound transmission through the bones of the skull. However, \gls{bc} is not only limited to bones, since it usually also involves transmission through soft tissue and cartilage. Therefore, it can be identified as the transmission of sound energy through the body (bones, soft tissue and cartilage). 
\gls{bc}, although being performed in a localised way, involves the outer, middle and inner ear, as well as the bones surrounding the hearing apparatus and both its fluids and cerebrospinal fluid (CSF) \citep{puria_2013}.
 \begin{figure}[H]
	\centering
		\includegraphics[width=1\textwidth]{more-on-the-vestibular-system-of-anatomy-of-cochlear-aqueduct.jpg}
		\caption{The figure shows the outer ear, the middle ear and the inner ear in detail}
		\label{fig:hearing_system_detail}
\end{figure}
Stenfelt's section 6.1.1 of \citep{puria_2013},starts with the following quote: \enquote{One of the quintessential questions about BC hearing is the end organ for
transforming BC vibration in the skull to neural code}. From this quote, we can infere that the determination of the endpoint for \gls{bc} is not an easy task, and could be the pinnacle in fully understanding \gls{bc} sound transmission.

As explained throughout the referenced section, several studies and experiments have been conducted over the years, leading to the result of identifying the cochlea as such organ. Once the endpoint was identified, it was primordial to understand the ways in which the sound is propagating to reach it.
\subsection{Propagation factors}

When analysing \gls{bc} propagation, we have to take into account three main factors:

\begin{itemize}
\item \textbf{Frequency}: there have been several studies and experiments over time both focused and skull and cochlea vibration patterns. These studies have served as ground for identifying four main modes within the human skull vibration pattern for frequencies below \SI{10}{\kilo\hertz}.
\begin{enumerate}
\item Below \SI{400}{\hertz}: within this range the skull moves as a rigid body (Fig \ref{fig:skull_vibration_pattern}a).\citep{stenfelt_2005b}
\item Between \SI{400}{\hertz} and \SI{1}{\kilo\hertz}: within this range the skull's motion can be described as a mass-spring system(Fig \ref{fig:skull_vibration_pattern}b).
\item Between \SI{1}{\kilo\hertz} and \SI{2}{\kilo\hertz}: at \SI{1}{\kilo\hertz}, the first free resonance of the skull appears \citep{hakansson_1994} and the motion transitions from mass-spring system to a system dominated by wave transmission.
\item Above \SI{2}{\kilo\hertz}: when reaching this range, wave transmission dominates the skull vibration pattern completely, with differences between the cranial vault and the skull base (Fig \ref{fig:skull_vibration_pattern}c).
\end{enumerate}
\begin{figure}[H]
	\centering
		\includegraphics[width=1\textwidth]{skull_vibration_pattern.pdf}
		\caption{Two-dimensional illustration of the vibration modes of the human skull at frequencies between 0.1 and 10 kHz. The thick arrows indicate the stimulation position and the thin arrows indicate the response directions. The rigid body response at the lowest frequencies is illustrated in (a) while the response at frequencies between approximately 0.3–1.0 kHz that is similar to a massspring system is shown in (b) where three sections of the skull move sequentially in opposite directions. In (c) the vibration responses for frequencies above 2 kHz is illustrated differently for the skull base and the cranial vault: at the skull base longitudinal wave propagation dominates the response while a mixture of vibration modes including bending waves is present at the cranial vault \citep{puria_2013}}
		\label{fig:skull_vibration_pattern}
\end{figure}
\item \textbf{Skin} and soft tissue: test
\item \textbf{Position of the transducer}: test
\end{itemize}
\subsection{Influence of Inner/Middle/Outer ear}

Early theories suggested that there were just a couple of pathways dominating the \gls{bc} perception. However, more recent studies suggest that there are five reported pathways for sound vibration to propagate or transmit to the cochlea \citep{zhang_2016}:
\begin{enumerate}
\item External auditory canal sound radiation (outer ear).
\item Inertia of the ossicular chain (middle ear).
\item Inertia of the cochlear fluid (inner ear).
\item Cochlear walls' compression (inner ear).
\item CSF pressure (inner ear).
\end{enumerate}

These pathways can work independently, but all of them add up to the quality of the signal received by the cochlea, and a failure in one of the pathways leads to a decrease in the quality of the received stimuli. In figure \ref{fig:hearing_system_pathway}, this correlation between \gls{bc} and \gls{ac} stimuli is depicted, clearly showing how \gls{bc} stimulation can generate \gls{ac}-like responses in the outer, middle and inner ear.

 \begin{figure}[H]
	\centering
		\includegraphics[width=1\textwidth]{ovidweb}
		\caption{The figure shows the transmission path for \gls{bc} \citep{stenfelt_2005}}
		\label{fig:hearing_system_pathway}
\end{figure}


