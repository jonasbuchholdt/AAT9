\section{Relation of Airborne and Bone Induced Sound}
For a considerable amount of time there has been an interest in characterizing the way, that \gls{bc}-induced sound is perceived in comparison to the perception of  airborne sound. In order to do so, knowledge about the anatomical and physiological aspects, that contribute to the perception of sound is necessary.
The pathways, that contribute to \gls{bc} causing the excitation in the \gls{bm} have been described in refXXXXX. A relatively recent summary can also be found in \citep{dauman_2013}.\\
In \citep{bekesy_1932}, the author is successfully cancelling the perception of a bone induced tone on a human subject with airborne sound. In order to do so, phase and amplitude of the airborne sound are manipulated. This led to the assumption, that inducing sound via the bones result in similar patterns on the \gls{bm} (see XXXXX) as airborne sound.
In \citep{lowy_1942}, this cancellation of airborne and bone induced sound is investigated in greater detail. The electrical cochlear response is measured on anesthesized cats and guniea pigs, while an airborne tone is cancelled with a bone induced tone in a frequency range from \SIrange{250}{3000}{\hertz}. It is shown, that, when the signal parameters are adjusted for cancellation at one position of the cochlea, the cancellation is also achieved, when measuring on another position of the cochlea. This again suggests, that the excitation patterns on the\gls{bm} caused by \gls{bc} are fairly similar to those caused by airborne sound. More recently, \citep{stenfelt_2007} conducted cancellation experiments with multiple tones. This is motivated by the observation of phenomena in different studies, which point towards differences between \gls{ac}- and \gls{bc}-induced sound in the context of otoacoustic emissions and auditory brainstem response. 
Stenfelt is presenting three test subjects with cancellation tasks, some of which involving more than one tone. Some minor deviations from the expected behaviour regarding the phase- and amplitude-parameters for cancellation are observed. However, no statistically significant tendencies can be obtained from the test data, that would indicate, that the auditory system does not behave like a linear system for \gls{ac} and \gls{bc} at the investigated frequencies (\SI{700}{\hertz},\SI{1100}{\hertz}, \SIrange{40}{60}{\decibel}HL).
Also, the author alludes, that the cancellation parameters are fairly sensitive to movements of the head and jaw of the subjects. 