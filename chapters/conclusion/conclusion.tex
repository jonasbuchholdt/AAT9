\section{Conclusion}\label{sec:conclusion}
As stated in \autoref{sec:problem_statement}, the main objective of this study is to assess the performance of a \gls{bc}-based communications channel, as compared with an \gls{ac}-based one, in terms of intelligibility.
In order to do so, it has been decided to design and perform a subject-based perceptual intelligibility test to obtain relevant data. However, extensive literature research about the matter of \gls{bc} sound and the state of the art in transducers was needed beforehand.
During the course of the project, the questions presented in \autoref{sec:problem_statement} have mostly been answered, and all relevant data for the final intelligibility evaluation has been extracted.
Based on relevant literature, the RadioEar B81 \gls{bct} was chosen for all subsequent applications placed on the condyle.
%With this in mind, two of the first aspects that needed to be covered were the choice of an appropriate \gls{bct} and the position to place it. This is resolved by researching the relevant literature, and it is decided to use a RadioEar B81 \gls{bct} placed on the condyle.
A key point of the project has been finding a way of linking the perceived levels from both the \gls{bct} and the \gls{ace} for the intelligibility test test. 
In order to do so, a level matching routine has been developed, and although it does not provide exact data about cochlear excitation caused by the \gls{bct}, it provides a framework in which the inter-subject assessment of intelligibility can be performed by looking at the relative performance difference for each individual subject. 
The final decision has been made on the type of test to perform. Based on the potential application and the background of the project proposal, the \gls{hint} was deemed to be most suitable both in terms of obtainable results as well as practical feasibility. This test is based on short sentences, that are unknown to the test subjects.

After analyzing the results from the \gls{hint}, it can be concluded that there is a tendency, that the performance of \gls{bct} regarding intelligibility is might be worse than the performance of classic \gls{ac}-based systems. 
With the obtained data, this tendency could not be shown to be statistically significant. A higher variance of the performance of subjects could be observed for the \gls{bct}.
However, the difference in performance appears not to be so drastic, so that the difference does not render \gls{bct} an invalid option for communication applications. The performance could likely be further increased by flattening the frequency response of the \gls{bct} with signal processing. Also a possibility of combining \gls{ac} and \gls{bc} in hybrid systems in order to increase the overall performance seems feasible as a subject of future research.





