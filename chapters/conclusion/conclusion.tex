\section{Conclusion}\label{sec:conclusion}
As stated in \autoref{sec:intro}, the main objective of this study is to assess the performance of a \gls{bc}-based communications system, as compared with an \gls{ac}-based one, in terms of intelligibility.

In order to do so, it was decided that the correct approach would be to design and perform a subject-based test to obtain the relevant data. However, extensive research about the matter of \gls{bc} sound and the state of the art used transducers was needed beforehand.
While doing so, most of the presented questions in \autoref{sec:intro} have been answered, and all relevant data for the final intelligibility evaluation has been extracted.
With this in mind, two of the first aspects that needed to be covered were the choice of an appropriate \gls{bct} and the position to place it. This is resolved by researching the relevant literature, and it is decided to use a RadioEar B81 \gls{bct} placed on the condyle.
Once the transducer is chosen, a key point is to ensure that the perceived levels from both the \gls{bct} and the \gls{ace} are the same during the test. In order to do so, a level matching routine is implemented, and although it does not provide exact data about cochlear excitation for the \gls{bct}, it provides a framework in which the inter-subject assessment of intelligibility can be performed by looking at the relative performance difference for each individual subject. Taking this into account, the last remaining decision is the type of test to perform, and for this case, based on the potential application and the background for the proposal, \gls{hint} is what fits the best. This type of test is based on short sentences, random in the sense that their content cannot be predicted.

After analyzing the results from the \gls{hint}, it can be concluded that the performance of \gls{bct} regarding intelligibility is lower than the performance of classic \gls{ac}-based systems. However, it is likely that this difference does not prevent \gls{bct} to be a valid option, and its performance could be further increased by flattening the transducer's frequency response. This result also leads to the possibility of combining \gls{ac} and \gls{bc} in hybrid systems in order to increase the overall performance.





