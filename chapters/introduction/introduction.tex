Sound transmission has been bound to be typically airborne for most of its application. However, \gls{bc} has been extensively used for medical purposes \citep{iso_389-3}. The main application of \gls{bc} is to perform pure-tone audiometries in subjects with hearing loss in order to assess damage in the outer or middle ear, since it has been researched that \gls{bc} mostly excites the inner ear.

This medical use leads to the question of whether \gls{bc} could further extend its use for day-to-day applications. This expand in use could range from leisure activities, such as music playback, to specialised usage, such as communications systems that do not block the user's ears. With this objective in mind, this project was proposed by RTX?? as breaking ground for \gls{bc} usage in communication systems.

Due to RTX?? background, the main focus of the research is testing the feasability of incorporating \gls{bct} in communications equipment and comparing its performance in comparison with classic \gls{ac} systems. However, there are different applications within this research area, such as incorporating \gls{bct} in order to not block the ears and maintain environmental awareness in high-risk situations (e.g. firefighters and tactical teams). Another possible application would be to substitute \gls{ac} transducers in equipment used in high noise level environments, adding the possibility to isolate the user's ears with both earplugs and protective headphones and thus reducing the possibility of suffering hearing damage (e.g. helicopter pilots and F1 mechanics). For any of these situations, the main concern is the assessment of intelligibility of the transmitted signal.

It has been decided to delimit the scope of the project and focus on \gls{bct} use in noisy environments, comparing its performance to the one of an \gls{ac}-based system in terms of a subject-based intelligibility test.