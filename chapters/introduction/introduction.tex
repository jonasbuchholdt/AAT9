\label{sec:intro}
Human sound perception has typically been based on airborne sound for a majority of situations. However, bone conducted sound has been extensively used for diagnostic purposes \citep{iso_389-3}. The main application of \gls{bc} is to perform pure-tone audiometries in subjects with hearing loss. In conjunction, with \gls{ac} audiometry, it can be assessed, where the damage is situated. It has been established that \gls{bc} circumvents the in the outer and middle ear, which allows for conclusions based on a difference between \gls{ac} and \gls{bc} audiometry.

This medical use leads to the question of whether the use of \gls{bc} could be extended further to day-to-day applications. This expansion of usage could range from leisure activities, such as music playback, to specialised usage, such as communications systems, that do not block the user's ears or conversly, that are used in noisy enviroments, where hearing protection is an imperative. With this objective in mind, the project at hand was proposed by RTX A/S, as foundational research for the usage of \gls{bc} in communication systems.

The focus of the project is testing the feasability of incorporating \gls{bct} in communications equipment by comparing its performance in comparison with classic \gls{ac}-based systems. However, there are different applications within this research area, such as utilising a \gls{bct} in order to not block the ear canals and maintain environmental awareness in high-risk situations (e.g. firefighters and law enforcement). Another possible application is to substitute \gls{ac} transducers in equipment used in high noise level environments, adding the possibility to isolate the user's ears with both earplugs and protective headphones and thus reducing the risk of hearing damage (e.g. helicopter pilots and F1 mechanics). For any of these situations, the main concern is the assessment of intelligibility of the transmitted signal.

\section{Problem Statement}
It has been decided to constrain the scope of the project and focus on \gls{bct} use in noisy environments, comparing its performance to the one of an \gls{ac}-based system in terms of a subject-based speech intelligibility test. In the course of doing so, the following aspects will be investigated:
\begin{itemize}
\item How does \gls{bc} differ from \gls{ac} in terms of speech intelligibility?
\begin{itemize}
\item Which intelligibility test method is suitable for the task at hand?
\item Can it be ensured that the perceived level is the same for \gls{ac} and \gls{bc}?
\end{itemize}
\item What types of \gls{bct} are available at the moment?
\item Where should the \gls{bct} be placed?
\end{itemize}