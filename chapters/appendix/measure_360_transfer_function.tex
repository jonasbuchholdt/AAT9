\chapter{Speaker measuring manual} \label{appendix:measuring_manual}
In this appendix the polar response measurement software will be briefly explained. The goal of this software is to measure the polar response and calculate transfer functions of any loudspeaker in a free field environment with calibrated measuring equipment. During the measurement of the polar response, impulse responses are measured with a specified degree step size all around the speaker. E.g. if the step is one degree, the loudspeaker will be turned 1 degree for every  impulse response measurement, until \SI{360}{\degree} is achieved.

\section*{Materials and setup}
To measure the polar response of a loud speaker, the following materials are used:
\begin{itemize}
\item Outline ET 250-3D (Turntable)
\begin{itemize}[noitemsep]
\item AAU-number: -
\item Serial number: REIBO012
\end{itemize}
\item RME FIREFACE ucx (Soundcard)
\begin{itemize}[noitemsep]
\item AAU-number: 108230
\item Serial number: 23811948
\end{itemize}
\item Pioneer A-616 with a gain of \SI{44}{\decibel} (amplifier)
\begin{itemize}[noitemsep]
\item AAU-number: B4-109-C-8
\item Serial number: HJ9404841S
\end{itemize}
\item Microphone with preamp
\item MATLAB 2017b (PC - Software)
\item Ethernet cable 
\item \gls{bandk} connector to XLR
\item Jack to Phone cable
\item XLR speaker cable
\item speaker stand
\end{itemize}

%\begin{figure}[htbp!]
%\centering
%\def\svgwidth{\columnwidth}
%\input{figures/appendix/guitar_frequency_test.pdf_tex}
%\caption{Setup for measuring frequency area on a guitar.}
%		\label{fig:appendix:test}
%\end{figure}

\section*{Calibration procedure}


\begin{enumerate}
\item The materials are set up as in \autoref{appendix:calibration}.
\item NOTE! The calibration shall be done in the same order as the following description.
\item The sound card is calibrated according to the first step in \autoref{appendix:calibration}   
\item The microphone is calibrated according to the second step in \autoref{appendix:calibration}
\end{enumerate}

\section*{Test procedure}


\begin{enumerate}
\item The materials are set up as in \autoref{fig:measurement_setup}.
\item The turn degree step is chosen and the turntable control software \autoref{appendix:turntable} is used.  
\item  All of the transfer function are stored in complex values together with the impulse responses.
\end{enumerate}


%\begin{figure}[htbp!]
%	\centering
	%	\includegraphics[width=1\textwidth]{guitar_low_E_neck.pdf}
		%\caption{Measurement of the low E note on the neck pickup.}
%		\label{fig:appendix:low_E_neck}
%\end{figure}

\section*{The MATLAB function}

\includeCode{transfer_measure.m}{matlab}{1}{55}{The polar response measurement software}{code:transfer_measure}{./code/acoustics_center_002_02_2018/}

