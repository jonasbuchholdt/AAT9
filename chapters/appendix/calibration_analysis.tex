\chapter*{Calibration analysis of Affinity}
A test was made to get a view of the Affinity input CD1 and CD2. To decide on the matching between the \gls{ac} and the \gls{bc} is have to be found out if the input CD1 and CD2 is calibrated. With calibrating it means that the input voltage relate to a force output of the \gls{bc}. To test if the Affinity 2.0 is calibrated, the input output voltage relation shall change along the frequency to rich the correct force for every frequency since the transfer function of the \gls{bc} does not match the transfer from the standard ...   

\section*{Materials and setup}
To measure the output voltage for difference frequency input, the following materials are used:

\begin{table}[H]
\centering
\caption{Equipment list}
\begin{tabular}{l|l|l|l l}
Description         	& Model                                        & Serial-no  						& AAU-no \\ \hline
PC        			 		& Macbook                                   & W89242W966H  			& -  \\
RME  					& Fireface UFX                             &  23811948 			 	& 108230 \\
Radioear   				&  B81                            & -   									& - \\
Affinity     				& 2.0                            				& -   									& -  \\
Fluke    				& -                           				& -   									& -  \\
Analysis software   & MATLAB \textsuperscript{\textregistered} R2018b & -          & -     
\end{tabular}
\end{table}

%\begin{figure}[htbp!]
%\centering
%\def\svgwidth{\columnwidth}
%\input{figures/appendix/guitar_frequency_test.pdf_tex}
%\caption{Setup for measuring frequency area on a guitar.}
%		\label{fig:appendix:test}
%\end{figure}

\section*{Test procedure}


\begin{enumerate}
\item The materials are set up as in \autoref{fig:appendix:test}.
\item 
\item  
\item  
\item 
\item 
\end{enumerate}

\section*{Results}

%\begin{figure}[htbp!]
%	\centering
%		\includegraphics[width=1\textwidth]{guitar_low_E_neck.pdf}
%		\caption{Measurement of the low E note on the neck pickup.}
%		\label{fig:appendix:low_E_neck}
%\end{figure}

On  \autoref{fig:appendix:low_E_neck} it is seen that the lowest significant frequency is around \SI{80}{\hertz} and the highest significant frequency is around \SI{400}{\hertz}, when playing the low E note on the guitar, using the neck pickup.

