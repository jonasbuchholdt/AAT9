\chapter{Test a guitars frequency area}
In this appendix the calibration of the \gls{ace} and \gls{acs} is done.

\section*{Materials and setup}
The calibration materials and setup are as following:


\startequipment
\equipment{PC}{Macbook}{W89242W966H}{-}
\equipment{PC}{Dell Latitude E5430}{-}{78316}
\equipment{\gls{ace} }{ETYM$\bar{O}$TIC RESEARCH ER4S}{-}{02049}
\equipment{Audio interface}{Interacoustics Affinity 2.0}{-}{108207}
\equipment{Audio interface}{RME Fireface UCX }{23811948}{108230}
\equipment{Analysis software}{\matlab R2018b}{-}{-}
\equipment{Head and Torso simulator}{B\&K Type 4128}{1407972}{08453-00}
\equipment{Microphone power supply}{B\&k Type 2804}{533884}{07304-00}
\equipment{\gls{acs} transducer}{Radioear SP90A}{Gd13080406}{-}
\equipment{Calibrator}{B\&k Type 4220}{965532}{07792-00}
\stopequipment


\begin{figure}[htbp!]
\centering
\def\svgwidth{\columnwidth}
\input{figures/appendix/guitar_frequency_test.pdf_tex}
\caption{Setup for measuring frequency area on a guitar.}
		\label{fig:appendix:test}
\end{figure}

\section*{Test procedure}


\begin{enumerate}
\item The materials are set up as in \autoref{fig:appendix:test}.
\item 
\item  
\item  
\item 
\item 
\end{enumerate}

\section*{Results}

\begin{figure}[htbp!]
	\centering
		\includegraphics[width=1\textwidth]{guitar_low_E_neck.pdf}
		\caption{Measurement of the low E note on the neck pickup.}
		\label{fig:appendix:low_E_neck}
\end{figure}

On  \autoref{fig:appendix:low_E_neck} it is seen that the lowest significant frequency is around \SI{80}{\hertz} and the highest significant frequency is around \SI{400}{\hertz}, when playing the low E note on the guitar, using the neck pickup.

