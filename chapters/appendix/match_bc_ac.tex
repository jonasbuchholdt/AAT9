\chapter{\gls{bc} vs \gls{ac} matching signals}
\label{apend_matching_signals}
Due to limitations in time and number of test subjects, the matching test was designed to be performed using the noise from the \gls{hint} as a starting point. However, this decision arose a potential issue, since the transfer function for \gls{bc} and \gls{ac} is not the same. The transfer function differs mostly in the low frequency, below \SI{500}{\hertz}, and in the high, above \SI{2}{\kilo\hertz}. Therefore, the low frequency from the \gls{ac} might work as a masker and the \gls{ac} therefore is perceived as louder in level in the frequency range of interest. To test this hypothesis, the original noise's performance from the \gls{hint} is compared to different filtered versions and to pure tones. All selected filtered versions and pure tones used through this test are contained in \autoref{apen:test_signals}. 


\begin{table}[H]
\caption{Test signal description, All filter is 8th order butterworth filters}
\begin{tabularx}{\textwidth}{l | X l}
Matching signal       & Description \\ \hline
Original signal         & Original unfiltered signal from the \gls{hint} test \citep{hint_2011}.      \\
Highpass        & Original signal filtered with a highpass filter at \SI{400}{\hertz}.           \\
Bandpass 1        & Original signal filtered with a bandpass filter from \SI{1}{\kilo\hertz} to \SI{2}{\kilo\hertz}.           \\
Bandpass 2        & Original signal filtered with a bandpass filter from \SI{1}{\kilo\hertz} to \SI{4}{\kilo\hertz}.          \\
1/3 octave & Original signal filtered with a 1/3 octave filter centred at \SI{2}{\kilo\hertz}.          \\
\SI{1}{\kilo\hertz}                  & Pure \SI{1}{\kilo\hertz} tone.      \\
\SI{2}{\kilo\hertz}                 & Pure \SI{2}{\kilo\hertz} tone.          \\
\SI{4}{\kilo\hertz}                 & Pure \SI{4}{\kilo\hertz} tone.        
\end{tabularx}
\label{apen:test_signals}
\end{table}

\section*{Materials and setup}
To match the perceived level from the \gls{bc} vs \gls{ac}, the following materials were used:

\begin{table}[H]
\centering
\caption{Equipment list}
\begin{tabular}{l|l|l|l l}
Description         	& Model                                        & Serial-no  						& AAU-no \\ \hline
PC        			 		& Macbook                                   & W89242W966H  			& -  \\
Audio Interface  					& RME Fireface UCX                             &  23811948 			 	& 108230 \\
Headphones     	&   ETYM$\bar{O}$TIC RESEARCH ER4S            & -   									& 02049 \\
\gls{bc} transducer   				&  Radioear B81                            & -   									& - \\
Audio interface     				& Affinity 2.0                            				& -   									& -  \\
Analysis software   & MATLAB \textsuperscript{\textregistered} R2018b & -          & -     
\end{tabular}
\end{table}




\begin{figure}[H]
\centering
\def\svgwidth{\columnwidth}
\input{figures/appendix/match_test.pdf_t}
\caption{Setup for match test of \gls{bc} vs \gls{ac}.}
		\label{fig:appendix:match_meas_system}
\end{figure}

\section*{Test procedure}


\begin{enumerate}
\item Set up material as indicated in \autoref{fig:appendix:match_meas_system}.
\item Place the \gls{bc} transducer with a rubber hairband according to \autoref{sec:bc_pos}.
\item Insert headphones into the ears.
\item Set Affinity to Match_bone_air mode (see \autoref{apend:aff_bc_ac_match}).
\item Play matching noise alternatively through \gls{ac} and \gls{bc} with a time step of \SI{2}{\second}.
\item The  subject indicates the conductor of the test if the level of the signal delivered through \gls{bc} should be lower, higher or if it matches the level for the \gls{ac}, since it will be fixed.
\item After the subject indicates the level is the same, the conductor will modify the level for \gls{bc} until two other matches are obtained in order to get the convergence level.
\item The matching test is done for every signal in \autoref{apen:test_signals}.
\end{enumerate}

\section*{Results}

The results for three subjects are as depicted in \autoref{apen:match_result}. 

\begin{table}[H]
\centering
\caption{Test result}
\begin{tabular}{l|lll|ll}
Matching test in \si{\decibel}       & 1\textsuperscript{st}  Subject & 1\textsuperscript{nd}  Subject & 3\textsuperscript{rd}  Subject & $\mu$ & $\sigma$ \\ \hline
Original signal         & 46        & 45        & 45        & 45.3        & 0.58               \\
Highpass         & 47        & 44        & 46        & 45.7        & 1.53               \\
Bandpass 1         & 38        & 40        & 45        & 41          & 3.61               \\
Bandpass 2        & 41        & 41        & 47        & 43          & 3.46               \\
1/3 octave & 57        & 50        & 60+       & 55.7        & 5.13               \\
\SI{1}{\kilo\hertz}                   & 34        & 32        & 36        & 34          & 2                  \\
\SI{2}{\kilo\hertz}                  & 45        & 51        & 60+       & 52          & 7.55               \\
\SI{4}{\kilo\hertz}                  & 47        & 49        & 59        & 51.7        &  6.43                 
\end{tabular}
\label{apen:match_result}
\end{table}

In \autoref{apen:match_result} it can be seen that the tone matching relays highly on the frequency and the standard deviation is high for \SI{2}{\kilo\hertz}  and \SI{4}{\kilo\hertz}. All tone matching frequency deviate with a standard deviation of \SI{10.31}{\decibel} and is therefore not feasible for the level matching. The 1/3 octave filtered signal with center frequency at \SI{2}{\kilo\hertz}, the Bandpass 1 and Bandpass 2 do also have a high standard deviation but the averaged of the two bandpass filters is not far from each other. The original signal compared to the Highpassed version of the signal does not differ more than \SI{0.4}{\decibel} and the original signal have the lowest standard deviation. From the present data, the hypothesis that the \gls{hint} noise signal might mask in the \gls{ac} does not hold, and the original signal seems to be the most suitable signal for matching in the \gls{bc} vs \gls{ac} matching test.

It can be concluded that the original signal of the \gls{hint} test is a valid choose over pure tone and filtered \gls{hint} noise test signal.

