

\chapter{Matching analysis in \gls{bier}}
\label{apend:matching_in_bier}
This annex contains the results of the matching test recorded from the \gls{bier} obtained data.

\section*{Materials and setup}


\begin{table}[H]
\centering
\caption{Equipment list for matching test}
\begin{tabular}{l|l|l|l l}
Description         	& Model                                        & Serial-no  						& AAU-no \\ \hline
PC        			 		& Macbook                                   & W89242W966H  			& -  \\
Audio interface  					& RME Fireface UCX                             &  23811948 			 	& 108230 \\
Headphones     	&   ETYM$\bar{O}$TIC RESEARCH ER4S            & -   									& 02049 \\
\gls{bc} transducer   				&  Radioear B81                            & -   									& - \\
Audio interface     				& Affinity 2.0                            				& -   									& -  \\
Analysis software   & \matlab R2018b & -          & -     
\end{tabular}
\end{table}



\begin{figure}[H]
\centering
\def\svgwidth{\columnwidth}
\input{figures/appendix/match_test.pdf_t}
\caption{Setup for match test of \gls{bc} vs \gls{ac}.}
		\label{fig:appendix:match_meas_system_bier}
\end{figure}

\section*{Test procedure}


\begin{enumerate}
\item Set up material as indicated in \autoref{fig:appendix:match_meas_system_bier}.
\item Place the \gls{bc} transducer with a rubber hairband according to \autoref{sec:bc_pos}.
\item Insert headphones into the ears.
\item Set Affinity to Match_bone_air mode (see \autoref{apend:aff_bc_ac_match}).
\item Play matching noise alternatively through \gls{ac} and \gls{bc} with a time step of \SI{2}{\second}.
\item The  subject indicates the conductor of the test if the level of the signal delivered through \gls{bc} should be lower, higher or if it matches the level for the \gls{ac}, since it will be fixed.
\item After the subject indicates the level is the same, the conductor will modify the level for \gls{bc} until two other matches are obtained in order to get the convergence level.
\item This procedure will be repeated between 2 and 3 times depending on subject's consistency.
\end{enumerate}

\section*{Results}

The results for the familiarisation are displayed in \autoref{apen:match_bier_fam} for all ten subjects. 

\begin{table}[H]
\centering
\caption{Familiarisation of match}
\begin{tabular}{l|lll|ll}
Matching in bier \si{\decibel}  &1\textsuperscript{st} trial & 2\textsuperscript{nd} trial & 3\textsuperscript{rd} trial & $\mu$   & $\sigma$ \\ \hline
Subject 1           & 52    & 53    & 53    & 52.7 & 0.58  \\
Subject 2          & 45    & 50    & 47    & 47.3 & 2.52   \\
Subject 3           & 45    & 48    & 51    & 48   & 3     \\
Subject 4           & 51    & 51    & 50    & 50.7 & 0.58  \\
Subject 5           & FP    & 55    & 56    & 55.5   & 0.71  \\
Subject 6           & 50    & 49    & 48    & 49   & 1     \\
Subject 7           & 50    & 51    & 51    & 50.7 & 0.58  \\
Subject 8           & 49    & 48    & 49    & 49.7 & 0.58  \\
Subject 9           & 48    & 48    & 51    & 49   & 1.73  \\
Subject 10          & 45    & 47    & 45    & 45.7 & 1.15  \\
\end{tabular}
\label{apen:match_bier_fam} 
\end{table}

in \autoref{apen:match_bier_fam}  it can be observed that the average and the standard deviation does not differ much from the initial test in \autoref{apend:match_field_init} where the average was \SI{50.2}{\decibel} and the standard deviation was \SI{2.5}{\decibel}. The dynamic range chosen in  \autoref{apend:match_field_init} then is proven to hold in this conducted \gls{bier}.


The results for the main part of this section are as displayed in \autoref{apen:match_bier} for ten subjects. 
\begin{table}[H]
\centering
\caption{Test result}
\begin{tabular}{lllllllll}
\multicolumn{1}{l|}{Matching }   & 1\textsuperscript{st}  & 2\textsuperscript{nd}  & 3\textsuperscript{rd}  & 4\textsuperscript{th} & 5\textsuperscript{th}  & \multicolumn{1}{l|}{6\textsuperscript{th} }                &    &  \\
\multicolumn{1}{l|}{in \gls{bier} [\si{\decibel}] }   & 1trial & trial &  trial &  trial &  trial & \multicolumn{1}{l|}{trial}                & $\mu$   & $\sigma$ \\ \hline
\multicolumn{1}{l|}{Subject 1}  & 52    & 53    & 51    & 52    & ND    & \multicolumn{1}{l|}{ND} & 52   & 0.82  \\
\multicolumn{1}{l|}{Subject 2} & 50    & 47    & 47    & 47    & 51    & \multicolumn{1}{l|}{48} & 48.3 & 1.75  \\
\multicolumn{1}{l|}{Subject 3}  & 45    & 50    & 47    & 47    & 49    & \multicolumn{1}{l|}{50} & 48   & 2     \\
\multicolumn{1}{l|}{Subject 4}  & 45    & 50    & 47    & 50    & ND    & \multicolumn{1}{l|}{ND} & 48   & 2.45  \\
\multicolumn{1}{l|}{Subject 5}  & 45    & FP    & 47    & 47    & ND    & \multicolumn{1}{l|}{ND} & 46.3 & 1.15  \\
\multicolumn{1}{l|}{Subject 6}  & 50    & 50    & 48    & 47    & ND    & \multicolumn{1}{l|}{ND} & 48.8 & 1.5   \\
\multicolumn{1}{l|}{Subject 7}  & 51    & 51    & ND    & ND    & ND    & \multicolumn{1}{l|}{ND} & 51   & 0     \\
\multicolumn{1}{l|}{Subject 8}  & 46    & 48    & 45    & 45    & ND    & \multicolumn{1}{l|}{ND} & 46   & 1.41  \\
\multicolumn{1}{l|}{Subject 9}  & 50    & 52    & 50    & 52    & ND    & \multicolumn{1}{l|}{ND} & 51   & 1.15  \\
\multicolumn{1}{l|}{Subject 10}  & 43    & 42    & 40    & 43    & ND    & \multicolumn{1}{l|}{ND} & 42   & 1.41
\end{tabular}
\label{apen:match_bier} 
\end{table}

In \autoref{apen:match_bier} the subject was asked to repeat the same process as in the familiarisation twice, unless not deemed as necessary by the test conductors (as in the case of Subject 2). The tendency of the two last matchings compared to the familiarisation is that two of the subjects with high standard deviation in the familiarisation have lower standard deviation in the real test. The standard deviation is raised by 0.13 in the real test compared to the familiarisation, but the standard deviation between subjects seems to be more consistent. 





