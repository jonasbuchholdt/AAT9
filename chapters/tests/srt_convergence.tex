\section{\gls{srt} convergence}
In the previous \autoref{sec:result_srt}, it is discovered that the range of values, across which the \gls{srt} of the subjects span, is much larger for the \gls{bct} than for the \gls{ace}. 
It stands out, that the first six subject in \autoref{tab:srtn} are relative consistent in their \gls{srtn} for both the \gls{bct} and the \gls{ace}, whereas the last four seem to struggle with either the \gls{bct} or both the \gls{bct} and the \gls{ace}. Subject 9 has higher \gls{srtn} in both the \gls{bct} and the \gls{ace}, which might be due to a lower insertion loss of the earphones (see also \autoref{sec:disc_res}). Subject 7, 8 and 10 have at least \dB{10} difference between the \gls{srt} of \gls{bct} and \gls{ace}. 


During the \gls{hint} part of the \gls{bier}, it has been observed, that for some subjects, the presentation \gls{snr} did not converge within the 20 sentences per list. 
Therefore, the \gls{snr} during the test is plotted in \autoref{fig:4842964_BC} for an exemplary subject with a high \gls{srtn} difference between \gls{bct} and the \gls{ace}. 

\plot{chapters/tests/4842965_BC}{Subject 7, \gls{bct}}{fig:4842965_BC}


In \autoref{fig:4842965_BC} it can be observed that the \gls{snr} did not converge during the \gls{hint}. The subject starts converging within the first 10 sentences of the first run, but afterwards the performance varies increasingly. Similar behaviour is displayed during the second run.
This trend cannot be observed for all subjects 7, 8 and 10, (see \autoref{append:srt_measurement}) and the some of the other subjects did not show converging behaviour either. 
To have a comparison, \autoref{fig:9823032_BC} displays the same curve for another subject, that had a closer to average performance, but also does not converge in the first run.
\plot{chapters/tests/9823032_BC}{Subject 2, \gls{bct}}{fig:9823032_BC}

It appears, the converge tendencies among the test subject cannot be linked with their respective \gls{srtn} score. Eliminating subjects from the datapool for subsequent analysis is not feasible.