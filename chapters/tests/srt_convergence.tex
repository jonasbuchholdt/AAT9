\section{\gls{srt} convergence}
In the previous \autoref{sec:result_srt}, it is discovered that the standard deviation of the \gls{bct} is \dB{3.06} higher than the \gls{ace}. This is due to that the first six subject in \autoref{tab:srtn} is relative consistent in their \gls{srtn} for both the \gls{bct} and the \gls{ace}, but the last four seems to struggle with the \gls{bct} or both the \gls{bct} and the \gls{ace}. Subject 9 have higher \gls{srtn} in both the \gls{bct} and the \gls{ace} which might be due to a lower insertion loss, but subject 7, 8 and 10 have at least \dB{10} difference between the \gls{bct} and \gls{ace}. 


Doing the \gls{hint} of the subject, it is observed that some subject did not convergence within the 20 sentences. Therefore, the high difference between the \gls{bct} and the \gls{ace} is compared with their convergence in the \gls{srtn} list score. The following \autoref{sec:4842965_BC} is the \gls{bct} \gls{srtn} list for subject 7.

\plot{chapters/tests/4842965_BC}{Subject 7 \gls{bct}}{sec:4842965_BC}


In \autoref{sec:4842965_BC} it can be observed that the subject did not converged doing the \gls{hint}. The subject start converging within the first 10 sentences but afterwards the performance dropped and the subject did not converge nicely. The trend is not equally between the subject 7, 8 and 10, see \autoref{append:srt_measurement} and the some of the six first subject did ether not converge nicely. The following \autoref{sec:9823032_BC} is the \gls{bct} \gls{srtn} list for subject 2 which show the same tendency as subject 7 just with lower \gls{snr}.



\plot{chapters/tests/9823032_BC}{Subject 2 \gls{bct}}{sec:9823032_BC}


Solid choises can not be done 

The converge tendency among all test subject can not be linked with their respective \gls{srtn} score for the further analysis. All \gls{srtn} can be founded in \autoref{append:srt_measurement}.