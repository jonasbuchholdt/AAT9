\section{Discussion: Methods}\label{sec:disc_meth}

\subsection*{Equalization of \gls{bct}}
During the conduction of the level matching tests (see \autoref{sec:match_sig_cho}) the difference in the transfer function for both transducers, \gls{ace} and \gls{bct}, stood up to be more prominent than anticipated after analysing the different options for the choice of \gls{bct}. Due to this difference, subjects expressed certain difficulty when trying to match the perceived level of the \gls{bct} with the perceived level from the headphones. This has led to the realisation that the difference on the transmission of sound, and therefore its perception, could have affected the results of the test.
Although the chosen \gls{bct} behaves correctly within the needed frequency range (see \autoref{sec:intel}), the difference with the selected \gls{ac} transducer could be further diminished by equalising it. With a proper equalisation, the \gls{bct} could generate a flatter frequency response in the range of interest. This would make the level matching process easier, as the output signals would be more similar, and could improve the performance of the \gls{bct} during the \gls{hint}.

\subsection*{Changes in setup}
As stated in \autoref{sec:aircon}, there are several types of \gls{ac} transducers that could have been used during this test. In addition, configurations in which both the speech and the noise go through the \gls{ace} have to also be taken into account. Therefore, and depending on the aim of the study, the test could be also conducted in the following ways:
\begin{itemize}
\item Using two \gls{bct}: by adding an extra \gls{bct} to the current setup, any possible localisation issue would be eliminated, such as identification of the impinging sound. However, and due to crosstalk between the two ears while using a \gls{bct}, the conditions of the experiment could be altered in a negative way (e.g. phase issues).
\item Closed circumaural headphones: by choosing this \gls{ac} transducer, the system would resemble more accurately a communication system for noisy environments. In this case, the test procedure and the calibration wouldn't be affected. However, the fitting of the \gls{bct} on the selected area has been proven to be difficult with this type of headphones.
\item Open ears, speech and noise through \gls{acs}: with this configuration, the calibration process turns simpler than with the one used for this project, since it is easier to calculate the \gls{spl} produced by the speaker at the listening point. However, this method would have the drawback of not comparing both transducers in the same conditions, as for \gls{bc} the speech and noise would be transmitted through different transducers.
\end{itemize}

\subsection*{\gls{bc}/\gls{ac} hybrid system}
This project's aim is to compare intelligibility ratings between \gls{ac}
and \gls{bc} systems. As a starting point, this comparison has been performed between simple systems consisting of a single type of transducer. However, and based on the results of this experiment, it would be interesting to test different combinations of \gls{ac}+\gls{bc} systems. By using these hybrid systems, it could be possible to map different frequency ranges for each transducer depending on its frequency response. Another option could be to weigh the output level for both transducers, testing different combinations in which the level for one of the transducers is lower than the other. 
