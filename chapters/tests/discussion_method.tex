\section{Method discussion}\label{sec:disc_meth}

\subsection*{Equalization of \gls{bct}}
During the conduction of the level matching tests (see \autoref{sec:match_sig_cho}) the difference in the transfer function for both transducers, \gls{ace} and \gls{bct}, stood up to be more prominent than anticipated after analysing the different options for the choice of \gls{bct}. Due to this difference, subjects expressed certain difficulty when trying to match the perceived level of the \gls{bct} with the perceived level from the headphones. This has led to the realisation that the difference on the transmission of sound, and therefore its perception, could have affected the results of the test.
Although the chosen \gls{bct} behaves correctly within the needed frequency range (see \autoref{sec:intel}), the difference with the selected \gls{ac} transducer could be further diminished by equalising it. With a proper equalisation, the \gls{bct} could generate a flatter frequency response in the range of interest. This would make the level matching process easier, as the output signals would be more similar, and could improve the performance of the \gls{bct} during the \gls{hint}.

\subsection*{Changes in setup}
As stated in \autoref{sec:aircon}, there are several types of \gls{ac} transducers that could have been used during this test. In addition, configurations in which both the speech and the noise go through the \gls{ace} have to also be taken into account.

\subsection*{Localization issue}

\subsection*{\gls{bc}/\gls{ac} hybrid system}


