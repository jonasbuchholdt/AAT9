\section{Introduction}
Assessing intelligibility is an essential component of the project at hand. 
In \citep[p. 11]{cote_2011}, the author defines intelligibility as \enquote{the ability to extract the content of the speech signal (\dots) from the recognized phonemes}. Phonemes are commonly defined as the smallest distinctive units of speech.
The approach to acquiring data in order to evaluate intelligibility, may have a substantial influence on the outcome of the assessment.
Recalling from autorefsec:problem_statement, the goal of the investigations is assessing whether there is a difference in intelligibility between \gls{ac} and \gls{bc} speech signals during the presence of ambient noise. 

According to \citep{arl_us_army}, there are three basic priniciples, that are being employed for intelligibility assessment:
\begin{itemize}
\item Speech intelligibility rating scales
\item Perceptual intelligibility tests
\item Technical speech intelligibility predictors
\end{itemize}