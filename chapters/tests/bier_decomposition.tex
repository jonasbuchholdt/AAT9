\section{\gls{bier}: \gls{srtn} Analysis}


During the \gls{bier} test, a for each of the of the ten subjects four \gls{srtn}s have been measured, evenly split between \gls{ac} and \gls{bc}.
The values two values for each conductor for each subject are averaged, resulting in a total of 10 pairs of observations to perform statistics on (see \autoref{tab:srtn}).
It is assumed, that the result of the loudness matching part of \gls{bier} (reference to the LM section here ??) leads to a reliable estimate of the \gls{snr} for the \gls{bct} for the \gls{hint} part of \gls{bier}.
In order to determine, if under this assumption there is a statistically significant difference in intelligibility between \gls{ac} and \gls{bc}, a paired-sample t-Test can be performed.\\

\begin{table}[H]
\centering
\caption{Average \gls{srtn} of the subjects for \gls{ac} and \gls{bc} obtained with \gls{bier}, values are rounded to one digit after the decimal point for representation in the table, not for the underlying calculations.}
\label{tab:srtn}
\begin{tabular}{l|rrrrrrrrrr}
Subject     & 1   & 2    & 3    & 4    & 5    & 6    & 7     & 8     & 9    & 10   \\ \hline
Avg. \gls{srtn}\textsubscript{,\gls{ac}} & 2.8 & 1.0  & -2.3 & -3.8 & -7.1 & -2.2 & 11.5  & 0.4   & 11.0 & 4.8  \\
Avg. \gls{srtn}\textsubscript{,\gls{bc}} & 0.7 & 1.6  & 1.1  & -7.2 & -1.3 & -0.6 & 21.5  & 11.4  & 9.7  & 17.5 \\
\gls{srtn}\textsubscript{,AC}-\gls{srtn}\textsubscript{,BC}  & 2.0 & -0.6 & -3.4 & -3.3 & -5.8 & -1.6 & -10.0 & -11.0 & 1.3  & 12.8
\end{tabular}
\end{table}

The null hypothesis of the test is, that the difference between the observation pairs is normally distributed and zero mean. If the former hypothesis can be rejected, it can be concluded, that there is a statistically significant difference between the mean \gls{srtn} of the two conductors.
The MATLAB\textsuperscript{\textregistered} function \texttt{ttest} has been applied to the available data from \autoref{tab:srtn}. The null hypothesis can be rejected with a $p$-value of 0.0644. This does reach the commonly apllied significance level of $\alpha=$\SI{5}{\percent}.\\
A normal probability plot of the \gls{srtn} is depicted in \autoref{fig:srtn_normal}. The lines in plot correspond to the normal distribution, that has been derived from estimating mean and variance from the available observations. 
The points depict those observations. Points being close to their corresponding lines indicate, that a normal distribution is likely to underlay the observations. The datapoints in \autoref{fig:srtn_normal} can be considered reasonably close to the corresponding normal distributions, however the number of points is small.\\
\begin{figure}[H]
\centering
% This file was created by matlab2tikz.
%
%The latest updates can be retrieved from
%  http://www.mathworks.com/matlabcentral/fileexchange/22022-matlab2tikz-matlab2tikz
%where you can also make suggestions and rate matlab2tikz.
%
\begin{tikzpicture}

\begin{axis}[%
width=120mm, 
height=50mm, 
at={(5mm,5mm)}, 
scale only axis,
xmin=-7.9175,
xmax=22.2175,
xlabel style={font=\color{white!15!black}},
xlabel={Data},
ymin=-1.95996398454005,
ymax=1.95996398454005,
ytick={-3.09023230616781,-2.74778138544499,-2.32634787404084,-2.05374891063182,-1.64485362695147,-1.2815515655446,-0.674489750196082,0,0.674489750196082,1.2815515655446,1.64485362695147,2.05374891063182,2.32634787404084,2.74778138544499,3.09023230616781},
yticklabels={{0.001},{0.003},{0.01},{0.02},{0.05},{0.10},{0.25},{0.50},{0.75},{0.90},{0.95},{0.98},{0.99},{0.997},{0.999}},
ylabel style={font=\color{white!15!black}},
ylabel={Probability},
axis background/.style={fill=white},
title style={font=\bfseries},
title={Normal Probability Plot},
axis x line*=bottom,
axis y line*=left,
xmajorgrids,
ymajorgrids,
legend style={legend cell align=left, align=left, draw=white!15!black}
]


\addplot [color=color1]
  table[row sep=crcr]{%
-2.25	-0.674489750196082\\
4.75	0.674489750196082\\
};
\addlegendentry{\gls{ac}, Gaussian Fit}

\addplot [color=color2, draw=none, mark=+, mark options={solid, blue}]
  table[row sep=crcr]{%
-7.1	-1.64485362695147\\
-3.85	-1.03643338949379\\
-2.25	-0.674489750196082\\
-2.2	-0.385320466407568\\
0.365	-0.125661346855074\\
1.025	0.125661346855074\\
2.75	0.385320466407568\\
4.75	0.674489750196082\\
11	1.03643338949379\\
11.5	1.64485362695147\\
};
\addlegendentry{\gls{ac}, Subject Data}



\addplot [color=color3]
  table[row sep=crcr]{%
-0.603	-0.674489750196082\\
11.35	0.674489750196082\\
};
\addlegendentry{\gls{bc}, Gaussian Fit}

\addplot [color=color4, draw=none, mark=+, mark options={solid, blue}]
  table[row sep=crcr]{%
-7.2	-1.64485362695147\\
-1.285	-1.03643338949379\\
-0.603	-0.674489750196082\\
0.747	-0.385320466407568\\
1.145	-0.125661346855074\\
1.625	0.125661346855074\\
9.7	0.385320466407568\\
11.35	0.674489750196082\\
17.5	1.03643338949379\\
21.5	1.64485362695147\\
};
\addlegendentry{\gls{bc}, Subject Data}
\addplot [color=red, dashdotted]
  table[row sep=crcr]{%
-7.2	-1.41900725744005\\
21.5	1.81998811286491\\
};
%\addlegendentry{data4}
\addplot [color=red, dashdotted]
  table[row sep=crcr]{%
-7.1	-1.60913983261065\\
11.5	1.97529141128853\\
};
%\addlegendentry{data1}

\end{axis}
\end{tikzpicture}%
\caption{Bone Air Thing}
\label{fig:srtn_normal}
\end{figure}
For the paired-sample t-test to be viable, it has to be assumed, that the pairs of observations are normally distributed. Hence the distribution of the difference between the two also has to be normally distributed. 
This can be checked by means of a Lilliefors test.
The Lilliefors test checks the null hypothesis, that the input data is coming from a normally distributed population. 
In the given context, the intrasubject difference between the the \gls{ac} \gls{srtn} and the \gls{bc} \gls{srtn} (see  is tested.
The population mean and variance are estimated based on the input data. The null hypothesis is rejected or confirmed, based on the discrepancy between the Gaussian \gls{cdf}, that is derived from the estimated mean and variance and the empirical distribution.
This procedure is conveniently implemented in the MATLAB\textsuperscript{\textregistered} function \texttt{lillietest}.
The null hypothesis, that the intrasubject \gls{srtn} difference is normally distributed, is not rejected at a \SI{5}{\percent} significance level based on the data at hand.\\

\section{Psychometric functions derived from \gls{bier}}



\begin{figure}[H]
\centering
% This file was created by matlab2tikz.
%
%The latest updates can be retrieved from
%  http://www.mathworks.com/matlabcentral/fileexchange/22022-matlab2tikz-matlab2tikz
%where you can also make suggestions and rate matlab2tikz.
%

%
\begin{tikzpicture}

\begin{axis}[%
width=120mm, 
height=74mm, 
at={(5mm,5mm)}, 
scale only axis,
xmin=-15,
xmax=15,
tick align=outside,
xlabel style={font=\color{white!15!black}},
xlabel={Normed Level},
ymin=0,
ymax=1,
ylabel style={font=\color{white!15!black}},
ylabel={Proportion Correct},
axis background/.style={fill=white},
axis x line*=bottom,
axis y line*=left,
grid=both,
grid style={line width=.1pt, draw=gray!10},
major grid style={line width=.2pt,draw=gray!50}, 
minor tick num=4,
legend style={legend cell align=left, align=left, draw=white!15!black},
legend entries={\gls{ac} Fit Curve,
                \gls{ac} Data,
                \gls{bc} Fit Curve,
                \gls{bc} Data},
legend pos=north west
]
\addlegendimage{mycolor1}
\addlegendimage{only marks,mycolor2}
\addlegendimage{mycolor3}
\addlegendimage{only marks, mycolor4}
\addplot [color=mycolor2, draw=none, mark size=3.0pt, mark=*, mark options={solid, mycolor2}]
  table[row sep=crcr]{%
-4	0.0769230769230769\\
};


\addplot [color=mycolor2, draw=none, mark size=4.9pt, mark=*, mark options={solid, mycolor2}]
  table[row sep=crcr]{%
-3	0.117647058823529\\
};


\addplot [color=mycolor2, draw=none, mark size=4.6pt, mark=*, mark options={solid, mycolor2}]
  table[row sep=crcr]{%
-2	0.2\\
};


\addplot [color=mycolor2, draw=none, mark size=7.6pt, mark=*, mark options={solid, mycolor2}]
  table[row sep=crcr]{%
-1	0.457831325301205\\
};


\addplot [color=mycolor2, draw=none, mark size=6.0pt, mark=*, mark options={solid, mycolor2}]
  table[row sep=crcr]{%
0	0.519230769230769\\
};


\addplot [color=mycolor2, draw=none, mark size=7.0pt, mark=*, mark options={solid, mycolor2}]
  table[row sep=crcr]{%
1	0.704225352112676\\
};


\addplot [color=mycolor2, draw=none, mark size=4.9pt, mark=*, mark options={solid, mycolor2}]
  table[row sep=crcr]{%
2	0.771428571428571\\
};


\addplot [color=mycolor2, draw=none, mark size=4.7pt, mark=*, mark options={solid, mycolor2}]
  table[row sep=crcr]{%
3	0.75\\
};


\addplot [color=mycolor2, draw=none, mark size=3.1pt, mark=*, mark options={solid, mycolor2}]
  table[row sep=crcr]{%
5	0.928571428571429\\
};


\addplot [color=mycolor2, draw=none, mark size=2.5pt, mark=*, mark options={solid, mycolor2}]
  table[row sep=crcr]{%
-5	0.111111111111111\\
};


\addplot [color=mycolor2, draw=none, mark size=1.7pt, mark=*, mark options={solid, mycolor2}]
  table[row sep=crcr]{%
7	1\\
};


\addplot [color=mycolor2, draw=none, mark size=0.8pt, mark=*, mark options={solid, mycolor2}]
  table[row sep=crcr]{%
-6	0\\
};


\addplot [color=mycolor2, draw=none, mark size=3.0pt, mark=*, mark options={solid, mycolor2}]
  table[row sep=crcr]{%
4	0.846153846153846\\
};


\addplot [color=mycolor2, draw=none, mark size=0.8pt, mark=*, mark options={solid, mycolor2}]
  table[row sep=crcr]{%
8	1\\
};


\addplot [color=mycolor2, draw=none, mark size=1.9pt, mark=*, mark options={solid, mycolor2}]
  table[row sep=crcr]{%
6	1\\
};


\addplot [color=mycolor2, draw=none, mark size=0.8pt, mark=*, mark options={solid, mycolor2}]
  table[row sep=crcr]{%
10	1\\
};


\addplot [color=mycolor2, draw=none, mark size=0.8pt, mark=*, mark options={solid, mycolor2}]
  table[row sep=crcr]{%
-8	0\\
};


\addplot [color=mycolor2, draw=none, mark size=0.8pt, mark=*, mark options={solid, mycolor2}]
  table[row sep=crcr]{%
9	1\\
};

\addplot [color=mycolor1, line width=2.0pt]
  table[row sep=crcr]{%
-8	0.00697553558180215\\
-7.98198198198198	0.00708637601787237\\
-7.96396396396396	0.00719876762879163\\
-7.94594594594595	0.00731272846423847\\
-7.92792792792793	0.00742827672444899\\
-7.90990990990991	0.00754543076055167\\
-7.89189189189189	0.00766420907488707\\
-7.87387387387387	0.00778463032131255\\
-7.85585585585586	0.00790671330549159\\
-7.83783783783784	0.00803047698516753\\
-7.81981981981982	0.00815594047042133\\
-7.8018018018018	0.00828312302391333\\
-7.78378378378378	0.00841204406110839\\
-7.76576576576577	0.00854272315048465\\
-7.74774774774775	0.00867518001372498\\
-7.72972972972973	0.00880943452589169\\
-7.71171171171171	0.00894550671558335\\
-7.69369369369369	0.00908341676507422\\
-7.67567567567568	0.0092231850104355\\
-7.65765765765766	0.00936483194163843\\
-7.63963963963964	0.00950837820263886\\
-7.62162162162162	0.00965384459144304\\
-7.6036036036036	0.00980125206015442\\
-7.58558558558559	0.00995062171500108\\
-7.56756756756757	0.0101019748163437\\
-7.54954954954955	0.0102553327786636\\
-7.53153153153153	0.0104107171705309\\
-7.51351351351351	0.010568149714552\\
-7.4954954954955	0.0107276522872967\\
-7.47747747747748	0.0108892469192048\\
-7.45945945945946	0.0110529557944706\\
-7.44144144144144	0.011218801250907\\
-7.42342342342342	0.0113868057797876\\
-7.40540540540541	0.0115569920256666\\
-7.38738738738739	0.0117293827861771\\
-7.36936936936937	0.0119040010118067\\
-7.35135135135135	0.0120808698056505\\
-7.33333333333333	0.0122600124231408\\
-7.31531531531532	0.0124414522717547\\
-7.2972972972973	0.0126252129106967\\
-7.27927927927928	0.0128113180505593\\
-7.26126126126126	0.0129997915529583\\
-7.24324324324324	0.013190657430145\\
-7.22522522522523	0.0133839398445932\\
-7.20720720720721	0.0135796631085618\\
-7.18918918918919	0.0137778516836332\\
-7.17117117117117	0.013978530180225\\
-7.15315315315315	0.0141817233570782\\
-7.13513513513514	0.0143874561207182\\
-7.11711711711712	0.0145957535248911\\
-7.0990990990991	0.0148066407699731\\
-7.08108108108108	0.0150201432023542\\
-7.06306306306306	0.015236286313795\\
-7.04504504504505	0.0154550957407568\\
-7.02702702702703	0.0156765972637053\\
-7.00900900900901	0.0159008168063856\\
-6.99099099099099	0.0161277804350719\\
-6.97297297297297	0.0163575143577875\\
-6.95495495495495	0.0165900449234985\\
-6.93693693693694	0.0168253986212784\\
-6.91891891891892	0.0170636020794454\\
-6.9009009009009	0.0173046820646701\\
-6.88288288288288	0.0175486654810555\\
-6.86486486486486	0.0177955793691875\\
-6.84684684684685	0.018045450905157\\
-6.82882882882883	0.0182983073995519\\
-6.81081081081081	0.0185541762964205\\
-6.79279279279279	0.0188130851722045\\
-6.77477477477477	0.0190750617346427\\
-6.75675675675676	0.0193401338216448\\
-6.73873873873874	0.0196083294001345\\
-6.72072072072072	0.0198796765648627\\
-6.7027027027027	0.0201542035371902\\
-6.68468468468468	0.0204319386638395\\
-6.66666666666667	0.0207129104156159\\
-6.64864864864865	0.0209971473860978\\
-6.63063063063063	0.0212846782902957\\
-6.61261261261261	0.0215755319632796\\
-6.59459459459459	0.0218697373587759\\
-6.57657657657658	0.022167323547732\\
-6.55855855855856	0.022468319716849\\
-6.54054054054054	0.0227727551670838\\
-6.52252252252252	0.0230806593121178\\
-6.5045045045045	0.0233920616767947\\
-6.48648648648649	0.0237069918955254\\
-6.46846846846847	0.0240254797106612\\
-6.45045045045045	0.0243475549708341\\
-6.43243243243243	0.0246732476292649\\
-6.41441441441441	0.0250025877420389\\
-6.3963963963964	0.0253356054663486\\
-6.37837837837838	0.025672331058704\\
-6.36036036036036	0.0260127948731095\\
-6.34234234234234	0.0263570273592091\\
-6.32432432432432	0.0267050590603971\\
-6.30630630630631	0.0270569206118969\\
-6.28828828828829	0.0274126427388066\\
-6.27027027027027	0.0277722562541111\\
-6.25225225225225	0.0281357920566613\\
-6.23423423423423	0.0285032811291198\\
-6.21621621621622	0.0288747545358744\\
-6.1981981981982	0.0292502434209165\\
-6.18018018018018	0.0296297790056882\\
-6.16216216216216	0.0300133925868949\\
-6.14414414414414	0.0304011155342849\\
-6.12612612612613	0.0307929792883958\\
-6.10810810810811	0.0311890153582676\\
-6.09009009009009	0.0315892553191227\\
-6.07207207207207	0.0319937308100122\\
-6.05405405405405	0.0324024735314293\\
-6.03603603603604	0.0328155152428895\\
-6.01801801801802	0.0332328877604778\\
-6	0.0336546229543624\\
-5.98198198198198	0.0340807527462754\\
-5.96396396396396	0.034511309106961\\
-5.94594594594595	0.0349463240535906\\
-5.92792792792793	0.0353858296471443\\
-5.90990990990991	0.0358298579897607\\
-5.89189189189189	0.0362784412220536\\
-5.87387387387387	0.036731611520396\\
-5.85585585585586	0.0371894010941715\\
-5.83783783783784	0.0376518421829939\\
-5.81981981981982	0.0381189670538941\\
-5.8018018018018	0.0385908079984749\\
-5.78378378378378	0.039067397330034\\
-5.76576576576577	0.0395487673806549\\
-5.74774774774775	0.0400349504982666\\
-5.72972972972973	0.040525979043671\\
-5.71171171171171	0.0410218853875398\\
-5.69369369369369	0.0415227019073791\\
-5.67567567567568	0.0420284609844643\\
-5.65765765765766	0.042539195000743\\
-5.63963963963964	0.0430549363357078\\
-5.62162162162162	0.0435757173632388\\
-5.6036036036036	0.0441015704484151\\
-5.58558558558559	0.0446325279442975\\
-5.56756756756757	0.0451686221886803\\
-5.54954954954955	0.0457098855008139\\
-5.53153153153153	0.0462563501780986\\
-5.51351351351351	0.0468080484927483\\
-5.4954954954955	0.0473650126884267\\
-5.47747747747748	0.0479272749768535\\
-5.45945945945946	0.0484948675343836\\
-5.44144144144144	0.0490678224985578\\
-5.42342342342342	0.0496461719646252\\
-5.40540540540541	0.05022994798204\\
-5.38738738738739	0.0508191825509285\\
-5.36936936936937	0.0514139076185324\\
-5.35135135135135	0.052014155075623\\
-5.33333333333333	0.0526199567528909\\
-5.31531531531532	0.0532313444173094\\
-5.2972972972973	0.0538483497684721\\
-5.27927927927928	0.0544710044349067\\
-5.26126126126126	0.055099339970362\\
-5.24324324324324	0.0557333878500728\\
-5.22522522522523	0.0563731794669994\\
-5.20720720720721	0.0570187461280436\\
-5.18918918918919	0.0576701190502426\\
-5.17117117117117	0.0583273293569388\\
-5.15315315315315	0.058990408073928\\
-5.13513513513514	0.0596593861255857\\
-5.11711711711712	0.0603342943309714\\
-5.0990990990991	0.061015163399912\\
-5.08108108108108	0.0617020239290654\\
-5.06306306306306	0.0623949063979622\\
-5.04504504504505	0.0630938411650295\\
-5.02702702702703	0.0637988584635943\\
-5.00900900900901	0.0645099883978685\\
-4.99099099099099	0.0652272609389155\\
-4.97297297297297	0.0659507059205993\\
-4.95495495495495	0.0666803530355158\\
-4.93693693693694	0.0674162318309078\\
-4.91891891891892	0.0681583717045635\\
-4.9009009009009	0.0689068019006992\\
-4.88288288288288	0.0696615515058268\\
-4.86486486486486	0.0704226494446065\\
-4.84684684684685	0.0711901244756861\\
-4.82882882882883	0.0719640051875252\\
-4.81081081081081	0.0727443199942079\\
-4.79279279279279	0.0735310971312423\\
-4.77477477477477	0.0743243646513479\\
-4.75675675675676	0.0751241504202325\\
-4.73873873873874	0.0759304821123571\\
-4.72072072072072	0.0767433872066924\\
-4.7027027027027	0.0775628929824636\\
-4.68468468468468	0.0783890265148887\\
-4.66666666666667	0.0792218146709063\\
-4.64864864864865	0.0800612841048969\\
-4.63063063063063	0.0809074612543974\\
-4.61261261261261	0.0817603723358077\\
-4.59459459459459	0.0826200433400933\\
-4.57657657657658	0.0834865000284818\\
-4.55855855855856	0.0843597679281545\\
-4.54054054054054	0.0852398723279353\\
-4.52252252252252	0.0861268382739755\\
-4.5045045045045	0.0870206905654368\\
-4.48648648648649	0.0879214537501715\\
-4.46846846846847	0.0888291521204029\\
-4.45045045045045	0.0897438097084041\\
-4.43243243243243	0.0906654502821775\\
-4.41441441441441	0.0915940973411353\\
-4.3963963963964	0.0925297741117816\\
-4.37837837837838	0.0934725035433967\\
-4.36036036036036	0.0944223083037248\\
-4.34234234234234	0.0953792107746649\\
-4.32432432432432	0.0963432330479666\\
-4.30630630630631	0.0973143969209317\\
-4.28828828828829	0.09829272389212\\
-4.27027027027027	0.0992782351570644\\
-4.25225225225225	0.100270951603991\\
-4.23423423423423	0.101270893809547\\
-4.21621621621622	0.102278082034544\\
-4.1981981981982	0.1032925362197\\
-4.18018018018018	0.104314275981399\\
-4.16216216216216	0.105343320607464\\
-4.14414414414414	0.106379689052936\\
-4.12612612612613	0.107423399935867\\
-4.10810810810811	0.10847447153313\\
-4.09009009009009	0.109532921776239\\
-4.07207207207207	0.110598768247188\\
-4.05405405405405	0.111672028174304\\
-4.03603603603604	0.112752718428112\\
-4.01801801801802	0.113840855517227\\
-4	0.114936455584258\\
-3.98198198198198	0.11603953440173\\
-3.96396396396396	0.117150107368027\\
-3.94594594594595	0.118268189503359\\
-3.92792792792793	0.119393795445746\\
-3.90990990990991	0.120526939447027\\
-3.89189189189189	0.121667635368885\\
-3.87387387387387	0.122815896678908\\
-3.85585585585586	0.123971736446666\\
-3.83783783783784	0.12513516733981\\
-3.81981981981982	0.126306201620208\\
-3.8018018018018	0.127484851140098\\
-3.78378378378378	0.128671127338274\\
-3.76576576576577	0.129865041236299\\
-3.74774774774775	0.131066603434746\\
-3.72972972972973	0.132275824109471\\
-3.71171171171171	0.133492713007915\\
-3.69369369369369	0.134717279445442\\
-3.67567567567568	0.1359495323017\\
-3.65765765765766	0.137189480017027\\
-3.63963963963964	0.138437130588882\\
-3.62162162162162	0.139692491568315\\
-3.6036036036036	0.140955570056473\\
-3.58558558558559	0.142226372701136\\
-3.56756756756757	0.143504905693301\\
-3.54954954954955	0.144791174763795\\
-3.53153153153153	0.14608518517993\\
-3.51351351351351	0.147386941742196\\
-3.4954954954955	0.148696448780999\\
-3.47747747747748	0.15001371015343\\
-3.45945945945946	0.151338729240089\\
-3.44144144144144	0.15267150894194\\
-3.42342342342342	0.154012051677216\\
-3.40540540540541	0.155360359378365\\
-3.38738738738739	0.15671643348904\\
-3.36936936936937	0.158080274961142\\
-3.35135135135135	0.159451884251895\\
-3.33333333333333	0.160831261320984\\
-3.31531531531532	0.162218405627729\\
-3.2972972972973	0.163613316128313\\
-3.27927927927928	0.165015991273058\\
-3.26126126126126	0.166426429003749\\
-3.24324324324324	0.167844626751012\\
-3.22522522522523	0.169270581431742\\
-3.20720720720721	0.170704289446581\\
-3.18918918918919	0.172145746677451\\
-3.17117117117117	0.173594948485138\\
-3.15315315315315	0.175051889706936\\
-3.13513513513514	0.176516564654337\\
-3.11711711711712	0.177988967110782\\
-3.0990990990991	0.179469090329469\\
-3.08108108108108	0.180956927031213\\
-3.06306306306306	0.182452469402364\\
-3.04504504504505	0.183955709092792\\
-3.02702702702703	0.185466637213915\\
-3.00900900900901	0.1869852443368\\
-2.99099099099099	0.188511520490318\\
-2.97297297297297	0.190045455159356\\
-2.95495495495495	0.191587037283098\\
-2.93693693693694	0.193136255253364\\
-2.91891891891892	0.194693096913005\\
-2.9009009009009	0.196257549554373\\
-2.88288288288288	0.197829599917844\\
-2.86486486486486	0.19940923419041\\
-2.84684684684685	0.200996438004332\\
-2.82882882882883	0.202591196435862\\
-2.81081081081081	0.204193494004028\\
-2.79279279279279	0.205803314669485\\
-2.77477477477477	0.20742064183343\\
-2.75675675675676	0.209045458336592\\
-2.73873873873874	0.210677746458279\\
-2.72072072072072	0.212317487915501\\
-2.7027027027027	0.213964663862157\\
-2.68468468468468	0.215619254888291\\
-2.66666666666667	0.217281241019421\\
-2.64864864864865	0.218950601715929\\
-2.63063063063063	0.220627315872536\\
-2.61261261261261	0.222311361817828\\
-2.59459459459459	0.224002717313872\\
-2.57657657657658	0.225701359555888\\
-2.55855855855856	0.227407265172\\
-2.54054054054054	0.22912041022306\\
-2.52252252252252	0.230840770202542\\
-2.5045045045045	0.232568320036503\\
-2.48648648648649	0.234303034083629\\
-2.46846846846847	0.236044886135346\\
-2.45045045045045	0.237793849416\\
-2.43243243243243	0.239549896583127\\
-2.41441441441441	0.241312999727777\\
-2.3963963963964	0.243083130374928\\
-2.37837837837838	0.244860259483965\\
-2.36036036036036	0.246644357449241\\
-2.34234234234234	0.248435394100703\\
-2.32432432432432	0.250233338704606\\
-2.30630630630631	0.252038159964289\\
-2.28828828828829	0.253849826021039\\
-2.27027027027027	0.255668304455017\\
-2.25225225225225	0.257493562286275\\
-2.23423423423423	0.259325565975835\\
-2.21621621621622	0.261164281426852\\
-2.1981981981982	0.263009673985853\\
-2.18018018018018	0.264861708444046\\
-2.16216216216216	0.266720349038714\\
-2.14414414414414	0.268585559454675\\
-2.12612612612613	0.27045730282583\\
-2.10810810810811	0.272335541736775\\
-2.09009009009009	0.274220238224503\\
-2.07207207207207	0.276111353780167\\
-2.05405405405405	0.278008849350931\\
-2.03603603603604	0.279912685341895\\
-2.01801801801802	0.281822821618092\\
-2	0.283739217506563\\
-1.98198198198198	0.285661831798511\\
-1.96396396396396	0.28759062275153\\
-1.94594594594595	0.289525548091904\\
-1.92792792792793	0.291466565016991\\
-1.90990990990991	0.293413630197675\\
-1.89189189189189	0.295366699780897\\
-1.87387387387387	0.297325729392263\\
-1.85585585585586	0.299290674138721\\
-1.83783783783784	0.301261488611321\\
-1.81981981981982	0.30323812688804\\
-1.8018018018018	0.305220542536692\\
-1.78378378378378	0.307208688617904\\
-1.76576576576577	0.309202517688167\\
-1.74774774774775	0.31120198180297\\
-1.72972972972973	0.31320703251999\\
-1.71171171171171	0.315217620902372\\
-1.69369369369369	0.317233697522072\\
-1.67567567567568	0.319255212463277\\
-1.65765765765766	0.321282115325891\\
-1.63963963963964	0.323314355229102\\
-1.62162162162162	0.32535188081501\\
-1.6036036036036	0.327394640252335\\
-1.58558558558559	0.329442581240191\\
-1.56756756756757	0.331495651011928\\
-1.54954954954955	0.333553796339049\\
-1.53153153153153	0.33561696353519\\
-1.51351351351351	0.337685098460179\\
-1.4954954954955	0.339758146524146\\
-1.47747747747748	0.341836052691721\\
-1.45945945945946	0.343918761486285\\
-1.44144144144144	0.346006216994289\\
-1.42342342342342	0.348098362869648\\
-1.40540540540541	0.350195142338191\\
-1.38738738738739	0.352296498202181\\
-1.36936936936937	0.354402372844898\\
-1.35135135135135	0.356512708235288\\
-1.33333333333333	0.358627445932668\\
-1.31531531531532	0.360746527091506\\
-1.2972972972973	0.36286989246625\\
-1.27927927927928	0.364997482416229\\
-1.26126126126126	0.367129236910607\\
-1.24324324324324	0.369265095533402\\
-1.22522522522523	0.371404997488559\\
-1.20720720720721	0.373548881605091\\
-1.18918918918919	0.375696686342264\\
-1.17117117117117	0.377848349794857\\
-1.15315315315315	0.380003809698458\\
-1.13513513513514	0.382163003434833\\
-1.11711711711712	0.384325868037342\\
-1.0990990990991	0.386492340196409\\
-1.08108108108108	0.388662356265048\\
-1.06306306306306	0.390835852264439\\
-1.04504504504505	0.393012763889559\\
-1.02702702702703	0.39519302651486\\
-1.00900900900901	0.397376575199998\\
-0.990990990990991	0.399563344695619\\
-0.972972972972973	0.401753269449177\\
-0.954954954954955	0.403946283610818\\
-0.936936936936937	0.406142321039298\\
-0.918918918918919	0.408341315307949\\
-0.900900900900901	0.410543199710697\\
-0.882882882882883	0.412747907268114\\
-0.864864864864865	0.414955370733521\\
-0.846846846846847	0.417165522599128\\
-0.828828828828829	0.419378295102218\\
-0.810810810810811	0.42159362023137\\
-0.792792792792793	0.423811429732724\\
-0.774774774774775	0.426031655116278\\
-0.756756756756757	0.42825422766223\\
-0.738738738738738	0.430479078427354\\
-0.72072072072072	0.432706138251406\\
-0.702702702702703	0.434935337763572\\
-0.684684684684685	0.437166607388948\\
-0.666666666666667	0.439399877355046\\
-0.648648648648648	0.441635077698341\\
-0.63063063063063	0.44387213827084\\
-0.612612612612613	0.446110988746689\\
-0.594594594594595	0.448351558628797\\
-0.576576576576577	0.450593777255497\\
-0.558558558558558	0.452837573807231\\
-0.54054054054054	0.455082877313259\\
-0.522522522522523	0.457329616658388\\
-0.504504504504505	0.459577720589735\\
-0.486486486486487	0.461827117723499\\
-0.468468468468468	0.464077736551766\\
-0.45045045045045	0.466329505449323\\
-0.432432432432432	0.4685823526805\\
-0.414414414414415	0.470836206406024\\
-0.396396396396397	0.47309099468989\\
-0.378378378378378	0.475346645506251\\
-0.36036036036036	0.477603086746319\\
-0.342342342342342	0.479860246225278\\
-0.324324324324325	0.482118051689218\\
-0.306306306306307	0.484376430822065\\
-0.288288288288288	0.486635311252539\\
-0.27027027027027	0.488894620561101\\
-0.252252252252252	0.49115428628693\\
-0.234234234234235	0.493414235934884\\
-0.216216216216216	0.495674396982484\\
-0.198198198198198	0.497934696886893\\
-0.18018018018018	0.500195063091902\\
-0.162162162162162	0.502455423034918\\
-0.144144144144144	0.50471570415395\\
-0.126126126126126	0.506975833894597\\
-0.108108108108108	0.509235739717039\\
-0.0900900900900901	0.511495349103016\\
-0.0720720720720722	0.513754589562812\\
-0.0540540540540544	0.516013388642231\\
-0.0360360360360357	0.51827167392957\\
-0.0180180180180178	0.520529373062577\\
0	0.522786413735413\\
0.0180180180180187	0.525042723705598\\
0.0360360360360357	0.527298230800944\\
0.0540540540540544	0.529552862926483\\
0.0720720720720713	0.531806548071384\\
0.0900900900900901	0.534059214315844\\
0.108108108108109	0.536310789837982\\
0.126126126126126	0.538561202920702\\
0.144144144144144	0.54081038195855\\
0.162162162162161	0.543058255464545\\
0.18018018018018	0.545304752076999\\
0.198198198198199	0.547549800566307\\
0.216216216216216	0.549793329841729\\
0.234234234234235	0.552035268958135\\
0.252252252252251	0.554275547122741\\
0.27027027027027	0.556514093701809\\
0.288288288288289	0.558750838227327\\
0.306306306306306	0.560985710403666\\
0.324324324324325	0.563218640114199\\
0.342342342342342	0.565449557427907\\
0.36036036036036	0.567678392605936\\
0.378378378378379	0.569905076108142\\
0.396396396396396	0.572129538599593\\
0.414414414414415	0.574351710957043\\
0.432432432432432	0.576571524275368\\
0.45045045045045	0.578788909873976\\
0.468468468468469	0.58100379930317\\
0.486486486486486	0.583216124350483\\
0.504504504504505	0.585425817046974\\
0.522522522522523	0.587632809673481\\
0.54054054054054	0.589837034766839\\
0.558558558558559	0.592038425126059\\
0.576576576576576	0.594236913818456\\
0.594594594594595	0.59643243418575\\
0.612612612612613	0.598624919850107\\
0.63063063063063	0.600814304720149\\
0.648648648648649	0.603000522996915\\
0.666666666666666	0.605183509179769\\
0.684684684684685	0.607363198072276\\
0.702702702702704	0.609539524788013\\
0.72072072072072	0.611712424756343\\
0.738738738738739	0.613881833728141\\
0.756756756756756	0.616047687781458\\
0.774774774774775	0.618209923327147\\
0.792792792792794	0.620368477114426\\
0.810810810810811	0.622523286236399\\
0.828828828828829	0.624674288135515\\
0.846846846846846	0.626821420608975\\
0.864864864864865	0.628964621814087\\
0.882882882882884	0.631103830273561\\
0.900900900900901	0.63323898488075\\
0.918918918918919	0.635370024904834\\
0.936936936936936	0.637496889995944\\
0.954954954954955	0.639619520190226\\
0.972972972972974	0.641737855914853\\
0.990990990990991	0.643851837992965\\
1.00900900900901	0.64596140764856\\
1.02702702702703	0.648066506511318\\
1.04504504504505	0.65016707662136\\
1.06306306306306	0.65226306043395\\
1.08108108108108	0.654354400824129\\
1.0990990990991	0.656441041091292\\
1.11711711711712	0.658522924963689\\
1.13513513513514	0.660599996602874\\
1.15315315315315	0.662672200608074\\
1.17117117117117	0.664739482020509\\
1.18918918918919	0.666801786327629\\
1.20720720720721	0.66885905946729\\
1.22522522522523	0.670911247831868\\
1.24324324324324	0.672958298272295\\
1.26126126126126	0.675000158102032\\
1.27927927927928	0.67703677510097\\
1.2972972972973	0.679068097519263\\
1.31531531531532	0.681094074081094\\
1.33333333333333	0.683114653988359\\
1.35135135135135	0.685129786924296\\
1.36936936936937	0.687139423057029\\
1.38738738738739	0.689143513043048\\
1.40540540540541	0.691142008030615\\
1.42342342342342	0.693134859663095\\
1.44144144144144	0.695122020082218\\
1.45945945945946	0.697103441931267\\
1.47747747747748	0.699079078358189\\
1.4954954954955	0.701048883018638\\
1.51351351351351	0.70301281007894\\
1.53153153153153	0.704970814218984\\
1.54954954954955	0.706922850635041\\
1.56756756756757	0.708868875042507\\
1.58558558558559	0.710808843678568\\
1.6036036036036	0.712742713304796\\
1.62162162162162	0.714670441209664\\
1.63963963963964	0.716591985210991\\
1.65765765765766	0.718507303658303\\
1.67567567567568	0.720416355435129\\
1.69369369369369	0.722319099961215\\
1.71171171171171	0.724215497194657\\
1.72972972972973	0.726105507633973\\
1.74774774774775	0.727989092320079\\
1.76576576576577	0.729866212838209\\
1.78378378378378	0.731736831319742\\
1.8018018018018	0.733600910443964\\
1.81981981981982	0.735458413439745\\
1.83783783783784	0.737309304087149\\
1.85585585585586	0.739153546718958\\
1.87387387387387	0.74099110622213\\
1.89189189189189	0.742821948039169\\
1.90990990990991	0.744646038169429\\
1.92792792792793	0.746463343170336\\
1.94594594594595	0.748273830158536\\
1.96396396396396	0.75007746681097\\
1.98198198198198	0.751874221365865\\
2	0.753664062623657\\
2.01801801801802	0.755446959947837\\
2.03603603603604	0.757222883265716\\
2.05405405405405	0.758991803069125\\
2.07207207207207	0.760753690415029\\
2.09009009009009	0.762508516926074\\
2.10810810810811	0.764256254791056\\
2.12612612612613	0.765996876765314\\
2.14414414414414	0.767730356171055\\
2.16216216216216	0.769456666897598\\
2.18018018018018	0.771175783401546\\
2.1981981981982	0.772887680706891\\
2.21621621621622	0.774592334405034\\
2.23423423423423	0.776289720654746\\
2.25225225225225	0.777979816182042\\
2.27027027027027	0.779662598279997\\
2.28828828828829	0.78133804480848\\
2.30630630630631	0.783006134193817\\
2.32432432432432	0.784666845428395\\
2.34234234234234	0.786320158070175\\
2.36036036036036	0.787966052242153\\
2.37837837837838	0.789604508631742\\
2.3963963963964	0.791235508490083\\
2.41441441441441	0.792859033631297\\
2.43243243243243	0.794475066431653\\
2.45045045045045	0.796083589828682\\
2.46846846846847	0.797684587320217\\
2.48648648648649	0.799278042963362\\
2.5045045045045	0.800863941373403\\
2.52252252252252	0.802442267722642\\
2.54054054054054	0.804013007739179\\
2.55855855855856	0.805576147705613\\
2.57657657657658	0.807131674457689\\
2.59459459459459	0.80867957538288\\
2.61261261261261	0.810219838418897\\
2.63063063063063	0.811752452052146\\
2.64864864864865	0.813277405316117\\
2.66666666666667	0.81479468778971\\
2.68468468468468	0.816304289595505\\
2.7027027027027	0.817806201397963\\
2.72072072072072	0.819300414401573\\
2.73873873873874	0.820786920348939\\
2.75675675675676	0.822265711518803\\
2.77477477477477	0.823736780724018\\
2.79279279279279	0.825200121309449\\
2.81081081081081	0.826655727149836\\
2.82882882882883	0.828103592647583\\
2.84684684684685	0.829543712730499\\
2.86486486486486	0.830976082849488\\
2.88288288288288	0.832400698976173\\
2.9009009009009	0.833817557600474\\
2.91891891891892	0.835226655728136\\
2.93693693693694	0.83662799087819\\
2.95495495495495	0.838021561080384\\
2.97297297297297	0.839407364872538\\
2.99099099099099	0.840785401297869\\
3.00900900900901	0.842155669902257\\
3.02702702702703	0.843518170731461\\
3.04504504504505	0.844872904328291\\
3.06306306306306	0.846219871729729\\
3.08108108108108	0.847559074464004\\
3.0990990990991	0.848890514547625\\
3.11711711711712	0.850214194482359\\
3.13513513513514	0.851530117252174\\
3.15315315315315	0.852838286320136\\
3.17117117117117	0.854138705625255\\
3.18918918918919	0.855431379579301\\
3.20720720720721	0.85671631306357\\
3.22522522522523	0.857993511425611\\
3.24324324324324	0.859262980475913\\
3.26126126126126	0.860524726484552\\
3.27927927927928	0.861778756177804\\
3.2972972972973	0.863025076734713\\
3.31531531531532	0.864263695783627\\
3.33333333333333	0.865494621398696\\
3.35135135135135	0.866717862096336\\
3.36936936936937	0.867933426831655\\
3.38738738738739	0.86914132499485\\
3.40540540540541	0.870341566407568\\
3.42342342342342	0.871534161319231\\
3.44144144144144	0.872719120403341\\
3.45945945945946	0.873896454753741\\
3.47747747747748	0.875066175880854\\
3.4954954954955	0.87622829570789\\
3.51351351351351	0.877382826567029\\
3.53153153153153	0.878529781195567\\
3.54954954954955	0.879669172732048\\
3.56756756756757	0.880801014712358\\
3.58558558558559	0.881925321065804\\
3.6036036036036	0.883042106111159\\
3.62162162162162	0.884151384552692\\
3.63963963963964	0.885253171476171\\
3.65765765765766	0.886347482344844\\
3.67567567567568	0.8874343329954\\
3.69369369369369	0.888513739633909\\
3.71171171171171	0.889585718831742\\
3.72972972972973	0.890650287521475\\
3.74774774774775	0.891707462992769\\
3.76576576576577	0.892757262888239\\
3.78378378378378	0.893799705199304\\
3.8018018018018	0.894834808262022\\
3.81981981981982	0.895862590752906\\
3.83783783783784	0.896883071684734\\
3.85585585585586	0.897896270402337\\
3.87387387387387	0.89890220657838\\
3.89189189189189	0.899900900209132\\
3.90990990990991	0.900892371610217\\
3.92792792792793	0.901876641412365\\
3.94594594594595	0.902853730557145\\
3.96396396396396	0.903823660292697\\
3.98198198198198	0.904786452169449\\
4	0.905742128035827\\
4.01801801801802	0.906690710033968\\
4.03603603603604	0.907632220595411\\
4.05405405405405	0.908566682436802\\
4.07207207207207	0.909494118555574\\
4.09009009009009	0.910414552225643\\
4.10810810810811	0.911328006993086\\
4.12612612612613	0.912234506671825\\
4.14414414414414	0.913134075339308\\
4.16216216216216	0.914026737332186\\
4.18018018018018	0.914912517241995\\
4.1981981981982	0.915791439910833\\
4.21621621621622	0.916663530427044\\
4.23423423423423	0.917528814120897\\
4.25225225225225	0.918387316560279\\
4.27027027027027	0.919239063546374\\
4.28828828828829	0.920084081109364\\
4.30630630630631	0.920922395504123\\
4.32432432432432	0.921754033205918\\
4.34234234234234	0.922579020906124\\
4.36036036036036	0.923397385507931\\
4.37837837837838	0.924209154122073\\
4.3963963963964	0.925014354062554\\
4.41441441441441	0.925813012842387\\
4.43243243243243	0.926605158169346\\
4.45045045045045	0.927390817941717\\
4.46846846846847	0.928170020244068\\
4.48648648648649	0.92894279334303\\
4.5045045045045	0.929709165683081\\
4.52252252252252	0.930469165882355\\
4.54054054054054	0.931222822728449\\
4.55855855855856	0.931970165174254\\
4.57657657657658	0.932711222333797\\
4.59459459459459	0.933446023478094\\
4.61261261261261	0.934174598031023\\
4.63063063063063	0.934896975565204\\
4.64864864864865	0.935613185797908\\
4.66666666666667	0.936323258586967\\
4.68468468468468	0.937027223926716\\
4.7027027027027	0.937725111943937\\
4.72072072072072	0.938416952893836\\
4.73873873873874	0.939102777156025\\
4.75675675675676	0.939782615230534\\
4.77477477477477	0.940456497733836\\
4.79279279279279	0.941124455394891\\
4.81081081081081	0.941786519051213\\
4.82882882882883	0.942442719644962\\
4.84684684684685	0.943093088219043\\
4.86486486486486	0.943737655913247\\
4.88288288288288	0.944376453960391\\
4.9009009009009	0.945009513682503\\
4.91891891891892	0.945636866487012\\
4.93693693693694	0.946258543862972\\
4.95495495495495	0.946874577377305\\
4.97297297297297	0.947484998671067\\
4.99099099099099	0.948089839455744\\
5.00900900900901	0.948689131509568\\
5.02702702702703	0.949282906673858\\
5.04504504504505	0.949871196849388\\
5.06306306306306	0.950454033992782\\
5.08108108108108	0.951031450112936\\
5.0990990990991	0.951603477267456\\
5.11711711711712	0.952170147559143\\
5.13513513513514	0.952731493132482\\
5.15315315315315	0.953287546170178\\
5.17117117117117	0.953838338889708\\
5.18918918918919	0.954383903539904\\
5.20720720720721	0.954924272397567\\
5.22522522522523	0.955459477764105\\
5.24324324324324	0.955989551962205\\
5.26126126126126	0.95651452733253\\
5.27927927927928	0.957034436230444\\
5.2972972972973	0.957549311022776\\
5.31531531531532	0.958059184084601\\
5.33333333333333	0.95856408779606\\
5.35135135135135	0.959064054539206\\
5.36936936936937	0.959559116694886\\
5.38738738738739	0.960049306639643\\
5.40540540540541	0.960534656742662\\
5.42342342342342	0.961015199362736\\
5.44144144144144	0.961490966845271\\
5.45945945945946	0.961961991519317\\
5.47747747747748	0.962428305694634\\
5.4954954954955	0.962889941658789\\
5.51351351351351	0.963346931674282\\
5.53153153153153	0.963799307975711\\
5.54954954954955	0.964247102766958\\
5.56756756756757	0.964690348218421\\
5.58558558558559	0.965129076464265\\
5.6036036036036	0.965563319599716\\
5.62162162162162	0.965993109678382\\
5.63963963963964	0.966418478709607\\
5.65765765765766	0.966839458655859\\
5.67567567567568	0.967256081430153\\
5.69369369369369	0.967668378893501\\
5.71171171171171	0.968076382852401\\
5.72972972972973	0.968480125056356\\
5.74774774774775	0.968879637195427\\
5.76576576576577	0.969274950897818\\
5.78378378378378	0.969666097727495\\
5.8018018018018	0.97005310918184\\
5.81981981981982	0.970436016689336\\
5.83783783783784	0.970814851607284\\
5.85585585585586	0.971189645219557\\
5.87387387387387	0.971560428734385\\
5.89189189189189	0.971927233282173\\
5.90990990990991	0.972290089913354\\
5.92792792792793	0.972649029596274\\
5.94594594594595	0.973004083215109\\
5.96396396396396	0.973355281567818\\
5.98198198198198	0.97370265536413\\
6	0.974046235223557\\
6.01801801801802	0.974386051673449\\
6.03603603603604	0.974722135147077\\
6.05405405405405	0.97505451598175\\
6.07207207207207	0.975383224416965\\
6.09009009009009	0.975708290592593\\
6.10810810810811	0.976029744547088\\
6.12612612612613	0.976347616215743\\
6.14414414414414	0.976661935428966\\
6.16216216216216	0.976972731910596\\
6.18018018018018	0.977280035276249\\
6.1981981981982	0.977583875031695\\
6.21621621621622	0.977884280571271\\
6.23423423423423	0.978181281176323\\
6.25225225225225	0.978474906013679\\
6.27027027027027	0.978765184134159\\
6.28828828828829	0.979052144471112\\
6.30630630630631	0.979335815838986\\
6.32432432432432	0.979616226931929\\
6.34234234234234	0.979893406322426\\
6.36036036036036	0.980167382459958\\
6.37837837837838	0.980438183669701\\
6.3963963963964	0.980705838151253\\
6.41441441441441	0.980970373977389\\
6.43243243243243	0.981231819092849\\
6.45045045045045	0.981490201313158\\
6.46846846846847	0.981745548323472\\
6.48648648648649	0.981997887677462\\
6.5045045045045	0.982247246796217\\
6.52252252252252	0.982493652967183\\
6.54054054054054	0.982737133343137\\
6.55855855855856	0.982977714941176\\
6.57657657657658	0.983215424641751\\
6.59459459459459	0.983450289187719\\
6.61261261261261	0.983682335183427\\
6.63063063063063	0.983911589093829\\
6.64864864864865	0.984138077243624\\
6.66666666666667	0.984361825816427\\
6.68468468468468	0.984582860853967\\
6.7027027027027	0.984801208255314\\
6.72072072072072	0.985016893776129\\
6.73873873873874	0.985229943027949\\
6.75675675675676	0.98544038147749\\
6.77477477477477	0.985648234445989\\
6.79279279279279	0.985853527108555\\
6.81081081081081	0.986056284493569\\
6.82882882882883	0.986256531482087\\
6.84684684684685	0.986454292807289\\
6.86486486486486	0.98664959305394\\
6.88288288288288	0.986842456657886\\
6.9009009009009	0.987032907905564\\
6.91891891891892	0.987220970933552\\
6.93693693693694	0.987406669728131\\
6.95495495495495	0.987590028124879\\
6.97297297297297	0.987771069808284\\
6.99099099099099	0.987949818311388\\
7.00900900900901	0.988126297015448\\
7.02702702702703	0.988300529149624\\
7.04504504504505	0.98847253779069\\
7.06306306306306	0.988642345862769\\
7.08108108108108	0.988809976137091\\
7.0990990990991	0.988975451231766\\
7.11711711711712	0.989138793611595\\
7.13513513513514	0.989300025587888\\
7.15315315315315	0.989459169318311\\
7.17117117117117	0.989616246806755\\
7.18918918918919	0.989771279903222\\
7.20720720720721	0.989924290303738\\
7.22522522522523	0.990075299550281\\
7.24324324324324	0.990224329030734\\
7.26126126126126	0.990371399978857\\
7.27927927927928	0.990516533474274\\
7.2972972972973	0.99065975044249\\
7.31531531531532	0.990801071654921\\
7.33333333333333	0.99094051772894\\
7.35135135135135	0.991078109127948\\
7.36936936936937	0.991213866161466\\
7.38738738738739	0.991347808985235\\
7.40540540540541	0.991479957601344\\
7.42342342342342	0.991610331858371\\
7.44144144144144	0.991738951451543\\
7.45945945945946	0.991865835922917\\
7.47747747747748	0.991991004661567\\
7.4954954954955	0.992114476903804\\
7.51351351351351	0.992236271733398\\
7.53153153153153	0.992356408081827\\
7.54954954954955	0.992474904728532\\
7.56756756756757	0.992591780301199\\
7.58558558558559	0.992707053276048\\
7.6036036036036	0.992820741978142\\
7.62162162162162	0.992932864581707\\
7.63963963963964	0.993043439110471\\
7.65765765765766	0.993152483438015\\
7.67567567567568	0.993260015288141\\
7.69369369369369	0.993366052235249\\
7.71171171171171	0.993470611704732\\
7.72972972972973	0.993573710973384\\
7.74774774774775	0.993675367169821\\
7.76576576576577	0.993775597274913\\
7.78378378378378	0.993874418122232\\
7.8018018018018	0.993971846398509\\
7.81981981981982	0.994067898644107\\
7.83783783783784	0.994162591253506\\
7.85585585585586	0.994255940475792\\
7.87387387387387	0.99434796241517\\
7.89189189189189	0.994438673031477\\
7.90990990990991	0.994528088140716\\
7.92792792792793	0.99461622341559\\
7.94594594594595	0.994703094386055\\
7.96396396396396	0.994788716439881\\
7.98198198198198	0.994873104823223\\
8	0.994956274641199\\
8.01801801801802	0.995038240858483\\
8.03603603603604	0.995119018299902\\
8.05405405405405	0.995198621651046\\
8.07207207207207	0.995277065458886\\
8.09009009009009	0.995354364132398\\
8.10810810810811	0.995430531943202\\
8.12612612612613	0.9955055830262\\
8.14414414414414	0.99557953138023\\
8.16216216216216	0.995652390868724\\
8.18018018018018	0.995724175220375\\
8.1981981981982	0.995794898029809\\
8.21621621621622	0.995864572758267\\
8.23423423423423	0.995933212734295\\
8.25225225225225	0.996000831154434\\
8.27027027027027	0.996067441083924\\
8.28828828828829	0.996133055457411\\
8.30630630630631	0.996197687079659\\
8.32432432432432	0.996261348626273\\
8.34234234234234	0.99632405264442\\
8.36036036036036	0.996385811553559\\
8.37837837837838	0.996446637646182\\
8.3963963963964	0.99650654308855\\
8.41441441441441	0.99656553992144\\
8.43243243243243	0.996623640060896\\
8.45045045045045	0.996680855298986\\
8.46846846846847	0.996737197304556\\
8.48648648648649	0.996792677623999\\
8.5045045045045	0.996847307682017\\
8.52252252252252	0.996901098782399\\
8.54054054054054	0.996954062108789\\
8.55855855855856	0.997006208725467\\
8.57657657657658	0.997057549578133\\
8.59459459459459	0.997108095494687\\
8.61261261261261	0.997157857186018\\
8.63063063063063	0.997206845246795\\
8.64864864864865	0.997255070156259\\
8.66666666666667	0.997302542279018\\
8.68468468468468	0.997349271865843\\
8.7027027027027	0.99739526905447\\
8.72072072072072	0.997440543870399\\
8.73873873873874	0.997485106227696\\
8.75675675675676	0.997528965929802\\
8.77477477477477	0.997572132670333\\
8.79279279279279	0.997614616033892\\
8.81081081081081	0.997656425496876\\
8.82882882882883	0.997697570428282\\
8.84684684684685	0.997738060090523\\
8.86486486486486	0.997777903640236\\
8.88288288288288	0.997817110129089\\
8.9009009009009	0.997855688504604\\
8.91891891891892	0.997893647610956\\
8.93693693693694	0.997930996189797\\
8.95495495495495	0.997967742881058\\
8.97297297297297	0.99800389622377\\
8.99099099099099	0.998039464656872\\
9.00900900900901	0.99807445652002\\
9.02702702702703	0.998108880054404\\
9.04504504504505	0.998142743403553\\
9.06306306306306	0.99817605461415\\
9.08108108108108	0.998208821636835\\
9.0990990990991	0.99824105232702\\
9.11711711711712	0.998272754445689\\
9.13513513513514	0.998303935660207\\
9.15315315315315	0.998334603545127\\
9.17117117117117	0.998364765582987\\
9.18918918918919	0.998394429165118\\
9.20720720720721	0.998423601592439\\
9.22522522522523	0.998452290076258\\
9.24324324324324	0.998480501739067\\
9.26126126126126	0.998508243615336\\
9.27927927927928	0.998535522652309\\
9.2972972972973	0.998562345710791\\
9.31531531531532	0.998588719565937\\
9.33333333333333	0.998614650908038\\
9.35135135135135	0.998640146343305\\
9.36936936936937	0.998665212394652\\
9.38738738738739	0.99868985550247\\
9.40540540540541	0.998714082025408\\
9.42342342342342	0.998737898241143\\
9.44144144144144	0.998761310347152\\
9.45945945945946	0.99878432446148\\
9.47747747747748	0.998806946623504\\
9.4954954954955	0.998829182794695\\
9.51351351351351	0.99885103885938\\
9.53153153153153	0.99887252062549\\
9.54954954954955	0.99889363382532\\
9.56756756756757	0.998914384116277\\
9.58558558558559	0.998934777081621\\
9.6036036036036	0.998954818231213\\
9.62162162162162	0.998974513002252\\
9.63963963963964	0.998993866760011\\
9.65765765765766	0.999012884798567\\
9.67567567567568	0.999031572341532\\
9.69369369369369	0.999049934542779\\
9.71171171171171	0.999067976487156\\
9.72972972972973	0.999085703191213\\
9.74774774774775	0.999103119603909\\
9.76576576576577	0.999120230607321\\
9.78378378378378	0.999137041017354\\
9.8018018018018	0.999153555584442\\
9.81981981981982	0.99916977899424\\
9.83783783783784	0.999185715868326\\
9.85585585585586	0.999201370764886\\
9.87387387387387	0.999216748179397\\
9.89189189189189	0.999231852545316\\
9.90990990990991	0.99924668823475\\
9.92792792792793	0.999261259559131\\
9.94594594594595	0.999275570769888\\
9.96396396396396	0.999289626059107\\
9.98198198198198	0.999303429560193\\
10	0.999316985348527\\
};

\addplot [color=mycolor1, dashed, line width=2.0pt]
  table[row sep=crcr]{%
-11.6	0.000164976679416167\\
-11.5636363636364	0.000172365421221753\\
-11.5272727272727	0.000180062831486768\\
-11.4909090909091	0.000188080756544743\\
-11.4545454545455	0.00019643145044526\\
-11.4181818181818	0.000205127586996969\\
-11.3818181818182	0.000214182272086343\\
-11.3454545454545	0.000223609056275406\\
-11.3090909090909	0.000233421947681536\\
-11.2727272727273	0.000243635425142366\\
-11.2363636363636	0.000254264451668695\\
-11.2	0.000265324488188221\\
-11.1636363636364	0.000276831507582768\\
-11.1272727272727	0.000288802009021585\\
-11.0909090909091	0.000301253032593111\\
-11.0545454545455	0.000314202174237543\\
-11.0181818181818	0.000327667600982293\\
-10.9818181818182	0.00034166806648234\\
-10.9454545454545	0.000356222926867295\\
-10.9090909090909	0.000371352156896814\\
-10.8727272727273	0.000387076366425822\\
-10.8363636363636	0.000403416817180815\\
-10.8	0.000420395439848332\\
-10.7636363636364	0.000438034851476436\\
-10.7272727272727	0.000456358373189874\\
-10.6909090909091	0.000475390048219334\\
-10.6545454545455	0.000495154660244977\\
-10.6181818181818	0.000515677752054208\\
-10.5818181818182	0.000536985644513363\\
-10.5454545454545	0.000559105455852756\\
-10.5090909090909	0.000582065121264218\\
-10.4727272727273	0.000605893412810026\\
-10.4363636363636	0.000630619959641803\\
-10.4	0.000656275268527653\\
-10.3636363636364	0.000682890744685516\\
-10.3272727272727	0.000710498712920389\\
-10.2909090909091	0.000739132439062731\\
-10.2545454545455	0.000768826151705011\\
-10.2181818181818	0.000799615064233072\\
-10.1818181818182	0.000831535397148515\\
-10.1454545454545	0.000864624400678033\\
-10.1090909090909	0.000898920377665211\\
-10.0727272727273	0.000934462706739899\\
-10.0363636363636	0.000971291865759889\\
-10	0.00100944945551919\\
-9.96363636363636	0.00104897822371686\\
-9.92727272727273	0.00108992208917973\\
-9.89090909090909	0.00113232616633219\\
-9.85454545454545	0.00117623678990558\\
-9.81818181818182	0.00122170153987911\\
-9.78181818181818	0.00126876926664437\\
-9.74545454545455	0.00131749011638418\\
-9.70909090909091	0.00136791555665674\\
-9.67272727272727	0.00142009840217521\\
-9.63636363636364	0.00147409284077244\\
-9.6	0.00152995445953999\\
-9.56363636363636	0.00158774027113018\\
-9.52727272727273	0.00164750874020935\\
-9.49090909090909	0.00170931981004987\\
-9.45454545454546	0.00177323492924806\\
-9.41818181818182	0.00183931707855458\\
-9.38181818181818	0.00190763079780319\\
-9.34545454545455	0.00197824221292345\\
-9.30909090909091	0.0020512190630222\\
-9.27272727272727	0.00212663072751801\\
-9.23636363636364	0.00220454825331264\\
-9.2	0.00228504438198231\\
-9.16363636363636	0.00236819357697173\\
-9.12727272727273	0.00245407205077266\\
-9.09090909090909	0.00254275779206857\\
-9.05454545454545	0.00263433059282611\\
-9.01818181818182	0.00272887207531391\\
-8.98181818181818	0.00282646571902805\\
-8.94545454545455	0.00292719688750353\\
-8.90909090909091	0.0030311528549902\\
-8.87272727272727	0.003138422832971\\
-8.83636363636364	0.00324909799649984\\
-8.8	0.00336327151033595\\
-8.76363636363636	0.00348103855485087\\
-8.72727272727273	0.00360249635168349\\
-8.69090909090909	0.00372774418911845\\
-8.65454545454545	0.00385688344716211\\
-8.61818181818182	0.00399001762229018\\
-8.58181818181818	0.00412725235184015\\
-8.54545454545454	0.00426869543802158\\
-8.50909090909091	0.00441445687151626\\
-8.47272727272727	0.00456464885464013\\
-8.43636363636364	0.00471938582403815\\
-8.4	0.0048787844728828\\
-8.36363636363636	0.00504296377254647\\
-8.32727272727273	0.00521204499371744\\
-8.29090909090909	0.00538615172692871\\
-8.25454545454545	0.00556540990246854\\
-8.21818181818182	0.00574994780964103\\
-8.18181818181818	0.00593989611534476\\
-8.14545454545454	0.00613538788193684\\
-8.10909090909091	0.00633655858434982\\
-8.07272727272727	0.00654354612642806\\
-8.03636363636364	0.0067564908564499\\
-8	0.00697553558180215\\
};


\addplot [color=mycolor1, dashed, line width=2.0pt]
  table[row sep=crcr]{%
10	0.999316985348527\\
10.0363636363636	0.99934360528689\\
10.0727272727273	0.999369264912341\\
10.1090909090909	0.999393995634251\\
10.1454545454545	0.999417827963449\\
10.1818181818182	0.999440791533049\\
10.2181818181818	0.999462915118986\\
10.2545454545455	0.999484226660231\\
10.2909090909091	0.999504753278704\\
10.3272727272727	0.999524521298871\\
10.3636363636364	0.999543556267034\\
10.4	0.999561882970298\\
10.4363636363636	0.999579525455237\\
10.4727272727273	0.999596507046236\\
10.5090909090909	0.99961285036352\\
10.5454545454545	0.999628577340873\\
10.5818181818182	0.999643709243039\\
10.6181818181818	0.999658266682816\\
10.6545454545455	0.999672269637828\\
10.6909090909091	0.999685737466994\\
10.7272727272727	0.999698688926688\\
10.7636363636364	0.999711142186583\\
10.8	0.999723114845195\\
10.8363636363636	0.999734623945117\\
10.8727272727273	0.999745685987959\\
10.9090909090909	0.999756316948975\\
10.9454545454545	0.9997665322914\\
10.9818181818182	0.99977634698049\\
11.0181818181818	0.999785775497265\\
11.0545454545455	0.999794831851963\\
11.0909090909091	0.999803529597208\\
11.1272727272727	0.99981188184089\\
11.1636363636364	0.999819901258765\\
11.2	0.999827600106773\\
11.2363636363636	0.999834990233085\\
11.2727272727273	0.999842083089871\\
11.3090909090909	0.999848889744807\\
11.3454545454545	0.99985542089231\\
11.3818181818182	0.999861686864511\\
11.4181818181818	0.999867697641976\\
11.4545454545455	0.99987346286416\\
11.4909090909091	0.999878991839626\\
11.5272727272727	0.999884293555999\\
11.5636363636364	0.99988937668969\\
11.6	0.999894249615372\\
11.6363636363636	0.999898920415222\\
11.6727272727273	0.999903396887933\\
11.7090909090909	0.999907686557492\\
11.7454545454545	0.999911796681741\\
11.7818181818182	0.999915734260709\\
11.8181818181818	0.99991950604473\\
11.8545454545455	0.999923118542354\\
11.8909090909091	0.99992657802804\\
11.9272727272727	0.999929890549647\\
11.9636363636364	0.999933061935726\\
12	0.999936097802615\\
12.0363636363636	0.999939003561331\\
12.0727272727273	0.999941784424289\\
12.1090909090909	0.999944445411817\\
12.1454545454545	0.999946991358502\\
12.1818181818182	0.999949426919354\\
12.2181818181818	0.999951756575791\\
12.2545454545455	0.999953984641459\\
12.2909090909091	0.999956115267881\\
12.3272727272727	0.999958152449948\\
12.3636363636364	0.999960100031241\\
12.4	0.999961961709206\\
12.4363636363636	0.99996374104017\\
12.4727272727273	0.999965441444211\\
12.5090909090909	0.999967066209882\\
12.5454545454545	0.999968618498791\\
12.5818181818182	0.999970101350044\\
12.6181818181818	0.999971517684552\\
12.6545454545455	0.999972870309204\\
12.6909090909091	0.999974161920917\\
12.7272727272727	0.999975395110549\\
12.7636363636364	0.999976572366701\\
12.8	0.999977696079395\\
12.8363636363636	0.999978768543633\\
12.8727272727273	0.999979791962844\\
12.9090909090909	0.999980768452226\\
12.9454545454545	0.999981700041971\\
12.9818181818182	0.999982588680395\\
13.0181818181818	0.999983436236955\\
13.0545454545455	0.99998424450518\\
13.0909090909091	0.999985015205495\\
13.1272727272727	0.999985749987954\\
13.1636363636364	0.999986450434888\\
13.2	0.999987118063454\\
13.2363636363636	0.999987754328107\\
13.2727272727273	0.999988360622982\\
13.3090909090909	0.9999889382842\\
13.3454545454545	0.999989488592091\\
13.3818181818182	0.999990012773341\\
13.4181818181818	0.999990512003069\\
13.4545454545455	0.999990987406825\\
13.4909090909091	0.999991440062524\\
13.5272727272727	0.99999187100231\\
13.5636363636364	0.999992281214351\\
13.6	0.999992671644575\\
};


\addplot [color=black]
  table[row sep=crcr]{%
-0.181735076939918	0\\
-0.181735076939918	0.5\\
};


\addplot [color=black, dotted]
  table[row sep=crcr]{%
-11.6	1\\
13.6	1\\
};


\addplot [color=black, dotted]
  table[row sep=crcr]{%
-11.6	0\\
13.6	0\\
};



\addplot [color=mycolor4, draw=none, mark size=3.4pt, mark=*, mark options={solid, mycolor4}]
  table[row sep=crcr]{%
-4	0.0588235294117647\\
};


\addplot [color=mycolor4, draw=none, mark size=4.4pt, mark=*, mark options={solid, mycolor4}]
  table[row sep=crcr]{%
-3	0.357142857142857\\
};


\addplot [color=mycolor4, draw=none, mark size=6.5pt, mark=*, mark options={solid, mycolor4}]
  table[row sep=crcr]{%
-2	0.426229508196721\\
};


\addplot [color=mycolor4, draw=none, mark size=6.0pt, mark=*, mark options={solid, mycolor4}]
  table[row sep=crcr]{%
-1	0.431372549019608\\
};


\addplot [color=mycolor4, draw=none, mark size=6.6pt, mark=*, mark options={solid, mycolor4}]
  table[row sep=crcr]{%
0	0.483870967741935\\
};


\addplot [color=mycolor4, draw=none, mark size=5.9pt, mark=*, mark options={solid, mycolor4}]
  table[row sep=crcr]{%
1	0.64\\
};


\addplot [color=mycolor4, draw=none, mark size=5.9pt, mark=*, mark options={solid, mycolor4}]
  table[row sep=crcr]{%
2	0.66\\
};


\addplot [color=mycolor4, draw=none, mark size=3.8pt, mark=*, mark options={solid, mycolor4}]
  table[row sep=crcr]{%
3	0.904761904761905\\
};


\addplot [color=mycolor4, draw=none, mark size=1.7pt, mark=*, mark options={solid, mycolor4}]
  table[row sep=crcr]{%
7	0.75\\
};


\addplot [color=mycolor4, draw=none, mark size=2.6pt, mark=*, mark options={solid, mycolor4}]
  table[row sep=crcr]{%
-6	0.1\\
};


\addplot [color=mycolor4, draw=none, mark size=3.2pt, mark=*, mark options={solid, mycolor4}]
  table[row sep=crcr]{%
4	0.666666666666667\\
};


\addplot [color=mycolor4, draw=none, mark size=1.9pt, mark=*, mark options={solid, mycolor4}]
  table[row sep=crcr]{%
6	0.8\\
};


\addplot [color=mycolor4, draw=none, mark size=1.2pt, mark=*, mark options={solid, mycolor4}]
  table[row sep=crcr]{%
8	0.5\\
};


\addplot [color=mycolor4, draw=none, mark size=0.8pt, mark=*, mark options={solid, mycolor4}]
  table[row sep=crcr]{%
10	1\\
};


\addplot [color=mycolor4, draw=none, mark size=2.9pt, mark=*, mark options={solid, mycolor4}]
  table[row sep=crcr]{%
-5	0.0833333333333333\\
};


\addplot [color=mycolor4, draw=none, mark size=2.2pt, mark=*, mark options={solid, mycolor4}]
  table[row sep=crcr]{%
5	0.714285714285714\\
};


\addplot [color=mycolor4, draw=none, mark size=0.8pt, mark=*, mark options={solid, mycolor4}]
  table[row sep=crcr]{%
9	0\\
};


\addplot [color=mycolor4, draw=none, mark size=1.2pt, mark=*, mark options={solid, mycolor4}]
  table[row sep=crcr]{%
-7	0\\
};


\addplot [color=mycolor4, draw=none, mark size=0.8pt, mark=*, mark options={solid, mycolor4}]
  table[row sep=crcr]{%
-10	0\\
};


\addplot [color=mycolor3, line width=2.0pt]
  table[row sep=crcr]{%
-10	0.0439620479437708\\
-9.97997997997998	0.044284225640334\\
-9.95995995995996	0.0446083036290211\\
-9.93993993993994	0.0449342892307989\\
-9.91991991991992	0.0452621897640966\\
-9.8998998998999	0.045592012544528\\
-9.87987987987988	0.0459237648846123\\
-9.85985985985986	0.0462574540934941\\
-9.83983983983984	0.0465930874766613\\
-9.81981981981982	0.0469306723356612\\
-9.7997997997998	0.0472702159678162\\
-9.77977977977978	0.047611725665937\\
-9.75975975975976	0.047955208718035\\
-9.73973973973974	0.0483006724070331\\
-9.71971971971972	0.0486481240104749\\
-9.6996996996997	0.0489975708002326\\
-9.67967967967968	0.0493490200422135\\
-9.65965965965966	0.0497024789960653\\
-9.63963963963964	0.0500579549148795\\
-9.61961961961962	0.0504154550448939\\
-9.5995995995996	0.0507749866251936\\
-9.57957957957958	0.0511365568874104\\
-9.55955955955956	0.0515001730554214\\
-9.53953953953954	0.0518658423450453\\
-9.51951951951952	0.0522335719637388\\
-9.4994994994995	0.05260336911029\\
-9.47947947947948	0.052975240974512\\
-9.45945945945946	0.0533491947369342\\
-9.43943943943944	0.0537252375684929\\
-9.41941941941942	0.0541033766302201\\
-9.3993993993994	0.0544836190729315\\
-9.37937937937938	0.0548659720369133\\
-9.35935935935936	0.0552504426516069\\
-9.33933933933934	0.0556370380352938\\
-9.31931931931932	0.0560257652947779\\
-9.2992992992993	0.0564166315250673\\
-9.27927927927928	0.0568096438090552\\
-9.25925925925926	0.0572048092171986\\
-9.23923923923924	0.0576021348071969\\
-9.21921921921922	0.0580016276236687\\
-9.1991991991992	0.0584032946978279\\
-9.17917917917918	0.058807143047158\\
-9.15915915915916	0.0592131796750862\\
-9.13913913913914	0.0596214115706556\\
-9.11911911911912	0.0600318457081967\\
-9.0990990990991	0.0604444890469979\\
-9.07907907907908	0.0608593485309745\\
-9.05905905905906	0.0612764310883374\\
-9.03903903903904	0.0616957436312598\\
-9.01901901901902	0.0621172930555441\\
-8.998998998999	0.0625410862402862\\
-8.97897897897898	0.0629671300475407\\
-8.95895895895896	0.0633954313219833\\
-8.93893893893894	0.0638259968905737\\
-8.91891891891892	0.0642588335622165\\
-8.8988988988989	0.064693948127422\\
-8.87887887887888	0.0651313473579656\\
-8.85885885885886	0.0655710380065463\\
-8.83883883883884	0.0660130268064443\\
-8.81881881881882	0.0664573204711785\\
-8.7987987987988	0.0669039256941617\\
-8.77877877877878	0.0673528491483564\\
-8.75875875875876	0.0678040974859289\\
-8.73873873873874	0.0682576773379031\\
-8.71871871871872	0.0687135953138129\\
-8.6986986986987	0.0691718580013546\\
-8.67867867867868	0.069632471966038\\
-8.65865865865866	0.0700954437508371\\
-8.63863863863864	0.0705607798758395\\
-8.61861861861862	0.0710284868378963\\
-8.5985985985986	0.0714985711102698\\
-8.57857857857858	0.0719710391422816\\
-8.55855855855856	0.0724458973589599\\
-8.53853853853854	0.0729231521606856\\
-8.51851851851852	0.0734028099228387\\
-8.4984984984985	0.0738848769954433\\
-8.47847847847848	0.0743693597028122\\
-8.45845845845846	0.0748562643431917\\
-8.43843843843844	0.0753455971884044\\
-8.41841841841842	0.0758373644834932\\
-8.3983983983984	0.0763315724463631\\
-8.37837837837838	0.076828227267424\\
-8.35835835835836	0.0773273351092322\\
-8.33833833833834	0.0778289021061315\\
-8.31831831831832	0.0783329343638941\\
-8.2982982982983	0.078839437959361\\
-8.27827827827828	0.0793484189400824\\
-8.25825825825826	0.0798598833239567\\
-8.23823823823824	0.0803738370988704\\
-8.21821821821822	0.0808902862223369\\
-8.1981981981982	0.081409236621135\\
-8.17817817817818	0.0819306941909478\\
-8.15815815815816	0.0824546647960003\\
-8.13813813813814	0.0829811542686979\\
-8.11811811811812	0.0835101684092634\\
-8.0980980980981	0.0840417129853751\\
-8.07807807807808	0.0845757937318037\\
-8.05805805805806	0.0851124163500497\\
-8.03803803803804	0.0856515865079802\\
-8.01801801801802	0.0861933098394659\\
-7.997997997998	0.0867375919440178\\
-7.97797797797798	0.0872844383864238\\
-7.95795795795796	0.0878338546963858\\
-7.93793793793794	0.0883858463681557\\
-7.91791791791792	0.0889404188601725\\
-7.8978978978979	0.0894975775946985\\
-7.87787787787788	0.0900573279574563\\
-7.85785785785786	0.0906196752972654\\
-7.83783783783784	0.0911846249256786\\
-7.81781781781782	0.0917521821166193\\
-7.7977977977978	0.0923223521060187\\
-7.77777777777778	0.0928951400914519\\
-7.75775775775776	0.0934705512317763\\
-7.73773773773774	0.0940485906467685\\
-7.71771771771772	0.0946292634167622\\
-7.6976976976977	0.0952125745822858\\
-7.67767767767768	0.0957985291437009\\
-7.65765765765766	0.0963871320608406\\
-7.63763763763764	0.0969783882526483\\
-7.61761761761762	0.0975723025968165\\
-7.5975975975976	0.0981688799294267\\
-7.57757757757758	0.0987681250445887\\
-7.55755755755756	0.099370042694081\\
-7.53753753753754	0.0999746375869914\\
-7.51751751751752	0.100581914389358\\
-7.4974974974975	0.10119187772381\\
-7.47747747747748	0.101804532169211\\
-7.45745745745746	0.102419882260301\\
-7.43743743743744	0.103037932487339\\
-7.41741741741742	0.103658687295746\\
-7.3973973973974	0.10428215108575\\
-7.37737737737738	0.104908328212031\\
-7.35735735735736	0.105537222983366\\
-7.33733733733734	0.106168839662276\\
-7.31731731731732	0.10680318246467\\
-7.2972972972973	0.107440255559495\\
-7.27727727727728	0.108080063068382\\
-7.25725725725726	0.108722609065298\\
-7.23723723723724	0.109367897576192\\
-7.21721721721722	0.110015932578649\\
-7.1971971971972	0.110666718001538\\
-7.17717717717718	0.111320257724665\\
-7.15715715715716	0.111976555578428\\
-7.13713713713714	0.112635615343469\\
-7.11711711711712	0.113297440750327\\
-7.0970970970971	0.1139620354791\\
-7.07707707707708	0.114629403159093\\
-7.05705705705706	0.115299547368483\\
-7.03703703703704	0.115972471633974\\
-7.01701701701702	0.116648179430458\\
-6.996996996997	0.117326674180674\\
-6.97697697697698	0.118007959254874\\
-6.95695695695696	0.11869203797048\\
-6.93693693693694	0.119378913591754\\
-6.91691691691692	0.120068589329459\\
-6.8968968968969	0.120761068340529\\
-6.87687687687688	0.121456353727733\\
-6.85685685685686	0.122154448539347\\
-6.83683683683684	0.122855355768824\\
-6.81681681681682	0.123559078354462\\
-6.7967967967968	0.124265619179081\\
-6.77677677677678	0.124974981069697\\
-6.75675675675676	0.125687166797192\\
-6.73673673673674	0.126402179075999\\
-6.71671671671672	0.127120020563773\\
-6.6966966966967	0.127840693861078\\
-6.67667667667668	0.128564201511059\\
-6.65665665665666	0.129290545999135\\
-6.63663663663664	0.130019729752673\\
-6.61661661661662	0.130751755140684\\
-6.5965965965966	0.131486624473502\\
-6.57657657657658	0.132224340002475\\
-6.55655655655656	0.132964903919661\\
-6.53653653653654	0.13370831835751\\
-6.51651651651652	0.134454585388567\\
-6.4964964964965	0.135203707025164\\
-6.47647647647648	0.135955685219115\\
-6.45645645645646	0.136710521861419\\
-6.43643643643644	0.137468218781957\\
-6.41641641641642	0.138228777749197\\
-6.3963963963964	0.138992200469898\\
-6.37637637637638	0.139758488588815\\
-6.35635635635636	0.140527643688406\\
-6.33633633633634	0.141299667288546\\
-6.31631631631632	0.142074560846234\\
-6.2962962962963	0.142852325755309\\
-6.27627627627628	0.143632963346167\\
-6.25625625625626	0.144416474885474\\
-6.23623623623624	0.145202861575891\\
-6.21621621621622	0.145992124555792\\
-6.1961961961962	0.14678426489899\\
-6.17617617617618	0.14757928361446\\
-6.15615615615616	0.14837718164607\\
-6.13613613613614	0.149177959872308\\
-6.11611611611612	0.149981619106016\\
-6.0960960960961	0.150788160094126\\
-6.07607607607608	0.151597583517391\\
-6.05605605605606	0.152409889990129\\
-6.03603603603604	0.153225080059962\\
-6.01601601601602	0.154043154207559\\
-5.995995995996	0.154864112846382\\
-5.97597597597598	0.155687956322432\\
-5.95595595595596	0.156514684914002\\
-5.93593593593594	0.15734429883143\\
-5.91591591591592	0.158176798216848\\
-5.8958958958959	0.159012183143946\\
-5.87587587587588	0.159850453617729\\
-5.85585585585586	0.160691609574279\\
-5.83583583583584	0.16153565088052\\
-5.81581581581582	0.162382577333984\\
-5.7957957957958	0.163232388662582\\
-5.77577577577578	0.164085084524377\\
-5.75575575575576	0.164940664507357\\
-5.73573573573574	0.165799128129211\\
-5.71571571571572	0.166660474837115\\
-5.6956956956957	0.167524704007506\\
-5.67567567567568	0.168391814945875\\
-5.65565565565566	0.169261806886548\\
-5.63563563563564	0.170134678992484\\
-5.61561561561562	0.171010430355062\\
-5.5955955955956	0.17188905999388\\
-5.57557557557558	0.172770566856552\\
-5.55555555555556	0.173654949818513\\
-5.53553553553554	0.174542207682819\\
-5.51551551551552	0.175432339179958\\
-5.4954954954955	0.176325342967658\\
-5.47547547547548	0.177221217630698\\
-5.45545545545546	0.178119961680728\\
-5.43543543543544	0.179021573556087\\
-5.41541541541542	0.179926051621619\\
-5.3953953953954	0.180833394168505\\
-5.37537537537538	0.181743599414087\\
-5.35535535535536	0.182656665501698\\
-5.33533533533534	0.183572590500495\\
-5.31531531531532	0.184491372405299\\
-5.2952952952953	0.185413009136431\\
-5.27527527527528	0.186337498539555\\
-5.25525525525526	0.187264838385525\\
-5.23523523523524	0.188195026370235\\
-5.21521521521522	0.189128060114466\\
-5.1951951951952	0.190063937163746\\
-5.17517517517518	0.191002654988206\\
-5.15515515515516	0.191944210982442\\
-5.13513513513514	0.192888602465378\\
-5.11511511511512	0.193835826680135\\
-5.0950950950951	0.194785880793902\\
-5.07507507507508	0.195738761897812\\
-5.05505505505506	0.196694467006814\\
-5.03503503503504	0.197652993059559\\
-5.01501501501502	0.198614336918282\\
-4.99499499499499	0.19957849536869\\
-4.97497497497497	0.200545465119851\\
-4.95495495495495	0.201515242804089\\
-4.93493493493493	0.202487824976884\\
-4.91491491491491	0.203463208116767\\
-4.89489489489489	0.204441388625229\\
-4.87487487487487	0.205422362826626\\
-4.85485485485485	0.206406126968092\\
-4.83483483483483	0.20739267721945\\
-4.81481481481481	0.208382009673134\\
-4.79479479479479	0.209374120344108\\
-4.77477477477477	0.210369005169789\\
-4.75475475475475	0.21136666000998\\
-4.73473473473473	0.212367080646795\\
-4.71471471471471	0.213370262784603\\
-4.69469469469469	0.214376202049959\\
-4.67467467467467	0.215384893991549\\
-4.65465465465465	0.216396334080138\\
-4.63463463463463	0.217410517708518\\
-4.61461461461461	0.218427440191459\\
-4.59459459459459	0.219447096765669\\
-4.57457457457457	0.220469482589753\\
-4.55455455455455	0.221494592744175\\
-4.53453453453453	0.22252242223123\\
-4.51451451451451	0.22355296597501\\
-4.49449449449449	0.224586218821382\\
-4.47447447447447	0.225622175537967\\
-4.45445445445445	0.226660830814121\\
-4.43443443443443	0.22770217926092\\
-4.41441441441441	0.228746215411151\\
-4.39439439439439	0.229792933719305\\
-4.37437437437437	0.230842328561574\\
-4.35435435435435	0.231894394235852\\
-4.33433433433433	0.232949124961738\\
-4.31431431431431	0.234006514880547\\
-4.29429429429429	0.23506655805532\\
-4.27427427427427	0.23612924847084\\
-4.25425425425425	0.237194580033653\\
-4.23423423423423	0.238262546572087\\
-4.21421421421421	0.239333141836288\\
-4.19419419419419	0.240406359498239\\
-4.17417417417417	0.241482193151806\\
-4.15415415415415	0.242560636312769\\
-4.13413413413413	0.243641682418868\\
-4.11411411411411	0.244725324829847\\
-4.09409409409409	0.245811556827509\\
-4.07407407407407	0.246900371615761\\
-4.05405405405405	0.24799176232068\\
-4.03403403403403	0.249085721990573\\
-4.01401401401401	0.250182243596037\\
-3.99399399399399	0.251281320030036\\
-3.97397397397397	0.252382944107969\\
-3.95395395395395	0.253487108567749\\
-3.93393393393393	0.254593806069883\\
-3.91391391391391	0.255703029197558\\
-3.89389389389389	0.256814770456729\\
-3.87387387387387	0.257929022276213\\
-3.85385385385385	0.259045777007783\\
-3.83383383383383	0.260165026926269\\
-3.81381381381381	0.261286764229666\\
-3.79379379379379	0.262410981039235\\
-3.77377377377377	0.263537669399623\\
-3.75375375375375	0.264666821278974\\
-3.73373373373373	0.265798428569051\\
-3.71371371371371	0.266932483085357\\
-3.69369369369369	0.268068976567267\\
-3.67367367367367	0.269207900678157\\
-3.65365365365365	0.27034924700554\\
-3.63363363363363	0.271493007061207\\
-3.61361361361361	0.272639172281366\\
-3.59359359359359	0.273787734026795\\
-3.57357357357357	0.274938683582989\\
-3.55355355355355	0.276092012160318\\
-3.53353353353353	0.277247710894184\\
-3.51351351351351	0.278405770845184\\
-3.49349349349349	0.279566182999279\\
-3.47347347347347	0.280728938267963\\
-3.45345345345345	0.281894027488436\\
-3.43343343343343	0.283061441423788\\
-3.41341341341341	0.284231170763174\\
-3.39339339339339	0.285403206122008\\
-3.37337337337337	0.286577538042148\\
-3.35335335335335	0.287754156992091\\
-3.33333333333333	0.288933053367174\\
-3.31331331331331	0.29011421748977\\
-3.29329329329329	0.291297639609502\\
-3.27327327327327	0.292483309903444\\
-3.25325325325325	0.293671218476341\\
-3.23323323323323	0.294861355360823\\
-3.21321321321321	0.296053710517628\\
-3.19319319319319	0.297248273835825\\
-3.17317317317317	0.298445035133045\\
-3.15315315315315	0.299643984155713\\
-3.13313313313313	0.300845110579283\\
-3.11311311311311	0.302048404008482\\
-3.09309309309309	0.303253853977551\\
-3.07307307307307	0.304461449950493\\
-3.05305305305305	0.305671181321328\\
-3.03303303303303	0.306883037414345\\
-3.01301301301301	0.308097007484363\\
-2.99299299299299	0.309313080716994\\
-2.97297297297297	0.31053124622891\\
-2.95295295295295	0.311751493068114\\
-2.93293293293293	0.312973810214214\\
-2.91291291291291	0.3141981865787\\
-2.89289289289289	0.315424611005227\\
-2.87287287287287	0.316653072269903\\
-2.85285285285285	0.317883559081573\\
-2.83283283283283	0.319116060082117\\
-2.81281281281281	0.320350563846742\\
-2.79279279279279	0.321587058884287\\
-2.77277277277277	0.322825533637524\\
-2.75275275275275	0.324065976483468\\
-2.73273273273273	0.325308375733683\\
-2.71271271271271	0.326552719634604\\
-2.69269269269269	0.32779899636785\\
-2.67267267267267	0.32904719405055\\
-2.65265265265265	0.330297300735665\\
-2.63263263263263	0.33154930441232\\
-2.61261261261261	0.332803193006136\\
-2.59259259259259	0.334058954379564\\
-2.57257257257257	0.33531657633223\\
-2.55255255255255	0.336576046601273\\
-2.53253253253253	0.337837352861695\\
-2.51251251251251	0.339100482726709\\
-2.49249249249249	0.340365423748097\\
-2.47247247247247	0.341632163416562\\
-2.45245245245245	0.342900689162091\\
-2.43243243243243	0.34417098835432\\
-2.41241241241241	0.3454430483029\\
-2.39239239239239	0.346716856257868\\
-2.37237237237237	0.347992399410021\\
-2.35235235235235	0.349269664891297\\
-2.33233233233233	0.350548639775148\\
-2.31231231231231	0.351829311076933\\
-2.29229229229229	0.353111665754299\\
-2.27227227227227	0.354395690707575\\
-2.25225225225225	0.355681372780165\\
-2.23223223223223	0.356968698758946\\
-2.21221221221221	0.358257655374665\\
-2.19219219219219	0.359548229302349\\
-2.17217217217217	0.360840407161705\\
-2.15215215215215	0.362134175517534\\
-2.13213213213213	0.363429520880142\\
-2.11211211211211	0.364726429705759\\
-2.09209209209209	0.366024888396954\\
-2.07207207207207	0.367324883303061\\
-2.05205205205205	0.368626400720602\\
-2.03203203203203	0.369929426893718\\
-2.01201201201201	0.371233948014597\\
-1.99199199199199	0.372539950223911\\
-1.97197197197197	0.373847419611251\\
-1.95195195195195	0.37515634221557\\
-1.93193193193193	0.376466704025624\\
-1.91191191191191	0.377778490980419\\
-1.89189189189189	0.37909168896966\\
-1.87187187187187	0.380406283834201\\
-1.85185185185185	0.381722261366504\\
-1.83183183183183	0.38303960731109\\
-1.81181181181181	0.384358307365004\\
-1.79179179179179	0.385678347178277\\
-1.77177177177177	0.386999712354389\\
-1.75175175175175	0.388322388450742\\
-1.73173173173173	0.389646360979126\\
-1.71171171171171	0.390971615406196\\
-1.69169169169169	0.392298137153945\\
-1.67167167167167	0.393625911600188\\
-1.65165165165165	0.394954924079038\\
-1.63163163163163	0.396285159881393\\
-1.61161161161161	0.39761660425542\\
-1.59159159159159	0.398949242407048\\
-1.57157157157157	0.400283059500455\\
-1.55155155155155	0.401618040658564\\
-1.53153153153153	0.40295417096354\\
-1.51151151151151	0.404291435457285\\
-1.49149149149149	0.405629819141945\\
-1.47147147147147	0.406969306980406\\
-1.45145145145145	0.408309883896806\\
-1.43143143143143	0.409651534777038\\
-1.41141141141141	0.410994244469264\\
-1.39139139139139	0.412337997784424\\
-1.37137137137137	0.413682779496753\\
-1.35135135135135	0.415028574344296\\
-1.33133133133133	0.416375367029426\\
-1.31131131131131	0.417723142219368\\
-1.29129129129129	0.41907188454672\\
-1.27127127127127	0.420421578609975\\
-1.25125125125125	0.421772208974055\\
-1.23123123123123	0.423123760170831\\
-1.21121121121121	0.424476216699663\\
-1.19119119119119	0.425829563027924\\
-1.17117117117117	0.427183783591543\\
-1.15115115115115	0.428538862795535\\
-1.13113113113113	0.429894785014542\\
-1.11111111111111	0.431251534593372\\
-1.09109109109109	0.432609095847544\\
-1.07107107107107	0.433967453063827\\
-1.05105105105105	0.435326590500789\\
-1.03103103103103	0.436686492389342\\
-1.01101101101101	0.43804714293329\\
-0.990990990990991	0.439408526309883\\
-0.970970970970971	0.440770626670363\\
-0.950950950950951	0.442133428140521\\
-0.930930930930931	0.443496914821254\\
-0.910910910910911	0.444861070789115\\
-0.890890890890891	0.446225880096874\\
-0.870870870870871	0.447591326774081\\
-0.850850850850851	0.448957394827619\\
-0.830830830830831	0.450324068242271\\
-0.810810810810811	0.451691330981281\\
-0.790790790790791	0.453059166986921\\
-0.77077077077077	0.454427560181053\\
-0.75075075075075	0.455796494465699\\
-0.73073073073073	0.457165953723606\\
-0.71071071071071	0.45853592181882\\
-0.69069069069069	0.45990638259725\\
-0.67067067067067	0.461277319887245\\
-0.65065065065065	0.462648717500163\\
-0.63063063063063	0.464020559230946\\
-0.61061061061061	0.465392828858695\\
-0.59059059059059	0.466765510147243\\
-0.57057057057057	0.468138586845733\\
-0.55055055055055	0.469512042689195\\
-0.53053053053053	0.470885861399123\\
-0.51051051051051	0.472260026684058\\
-0.49049049049049	0.473634522240159\\
-0.47047047047047	0.475009331751791\\
-0.45045045045045	0.476384438892106\\
-0.43043043043043	0.477759827323617\\
-0.41041041041041	0.47913548069879\\
-0.39039039039039	0.480511382660619\\
-0.37037037037037	0.481887516843215\\
-0.35035035035035	0.483263866872387\\
-0.33033033033033	0.484640416366227\\
-0.31031031031031	0.486017148935697\\
-0.29029029029029	0.487394048185209\\
-0.27027027027027	0.488771097713218\\
-0.25025025025025	0.490148281112803\\
-0.23023023023023	0.491525581972255\\
-0.21021021021021	0.492902983875664\\
-0.19019019019019	0.494280470403505\\
-0.17017017017017	0.495658025133226\\
-0.15015015015015	0.497035631639836\\
-0.13013013013013	0.49841327349649\\
-0.11011011011011	0.499790934275082\\
-0.0900900900900901	0.501168597546824\\
-0.07007007007007	0.502546246882843\\
-0.05005005005005	0.503923865854762\\
-0.03003003003003	0.505301438035292\\
-0.01001001001001	0.506678946998816\\
0.01001001001001	0.508056376321981\\
0.03003003003003	0.509433709584281\\
0.05005005005005	0.510810930368647\\
0.07007007007007	0.512188022262033\\
0.0900900900900901	0.513564968856002\\
0.11011011011011	0.514941753747315\\
0.13013013013013	0.516318360538515\\
0.15015015015015	0.517694772838514\\
0.17017017017017	0.519070974263178\\
0.19019019019019	0.520446948435913\\
0.21021021021021	0.521822678988246\\
0.23023023023023	0.523198149560416\\
0.25025025025025	0.524573343801951\\
0.27027027027027	0.525948245372255\\
0.29029029029029	0.527322837941187\\
0.31031031031031	0.528697105189647\\
0.33033033033033	0.530071030810156\\
0.35035035035035	0.531444598507434\\
0.37037037037037	0.532817791998982\\
0.39039039039039	0.534190595015661\\
0.41041041041041	0.53556299130227\\
0.43043043043043	0.536934964618122\\
0.45045045045045	0.538306498737624\\
0.47047047047047	0.539677577450848\\
0.49049049049049	0.54104818456411\\
0.51051051051051	0.542418303900541\\
0.53053053053053	0.543787919300661\\
0.55055055055055	0.54515701462295\\
0.57057057057057	0.546525573744419\\
0.59059059059059	0.54789358056118\\
0.61061061061061	0.549261018989014\\
0.63063063063063	0.550627872963937\\
0.65065065065065	0.551994126442769\\
0.67067067067067	0.553359763403696\\
0.69069069069069	0.554724767846836\\
0.71071071071071	0.556089123794797\\
0.73073073073073	0.557452815293244\\
0.75075075075075	0.558815826411453\\
0.77077077077077	0.560178141242872\\
0.790790790790791	0.561539743905679\\
0.810810810810811	0.562900618543333\\
0.830830830830831	0.564260749325132\\
0.850850850850851	0.565620120446761\\
0.870870870870871	0.566978716130849\\
0.890890890890891	0.56833652062751\\
0.910910910910911	0.569693518214897\\
0.930930930930931	0.571049693199746\\
0.950950950950951	0.572405029917917\\
0.970970970970971	0.573759512734941\\
0.990990990990991	0.575113126046558\\
1.01101101101101	0.576465854279257\\
1.03103103103103	0.57781768189081\\
1.05105105105105	0.579168593370814\\
1.07107107107107	0.580518573241215\\
1.09109109109109	0.581867606056849\\
1.11111111111111	0.583215676405964\\
1.13113113113113	0.58456276891075\\
1.15115115115115	0.585908868227869\\
1.17117117117117	0.587253959048969\\
1.19119119119119	0.588598026101216\\
1.21121121121121	0.589941054147804\\
1.23123123123123	0.591283027988479\\
1.25125125125125	0.592623932460053\\
1.27127127127127	0.593963752436911\\
1.29129129129129	0.595302472831531\\
1.31131131131131	0.596640078594985\\
1.33133133133133	0.59797655471745\\
1.35135135135135	0.599311886228709\\
1.37137137137137	0.600646058198653\\
1.39139139139139	0.601979055737783\\
1.41141141141141	0.603310863997704\\
1.43143143143143	0.604641468171619\\
1.45145145145145	0.605970853494826\\
1.47147147147147	0.607299005245202\\
1.49149149149149	0.608625908743692\\
1.51151151151151	0.609951549354795\\
1.53153153153153	0.611275912487047\\
1.55155155155155	0.612598983593497\\
1.57157157157157	0.613920748172185\\
1.59159159159159	0.61524119176662\\
1.61161161161161	0.616560299966248\\
1.63163163163163	0.617878058406921\\
1.65165165165165	0.619194452771366\\
1.67167167167167	0.620509468789647\\
1.69169169169169	0.621823092239625\\
1.71171171171171	0.623135308947421\\
1.73173173173173	0.624446104787864\\
1.75175175175175	0.62575546568495\\
1.77177177177177	0.627063377612287\\
1.79179179179179	0.628369826593547\\
1.81181181181181	0.629674798702905\\
1.83183183183183	0.630978280065483\\
1.85185185185185	0.632280256857788\\
1.87187187187187	0.633580715308146\\
1.89189189189189	0.634879641697137\\
1.91191191191191	0.63617702235802\\
1.93193193193193	0.637472843677163\\
1.95195195195195	0.638767092094463\\
1.97197197197197	0.640059754103772\\
1.99199199199199	0.641350816253306\\
2.01201201201201	0.642640265146065\\
2.03203203203203	0.643928087440242\\
2.05205205205205	0.645214269849631\\
2.07207207207207	0.646498799144029\\
2.09209209209209	0.647781662149643\\
2.11211211211211	0.649062845749482\\
2.13213213213213	0.650342336883756\\
2.15215215215215	0.651620122550265\\
2.17217217217217	0.652896189804791\\
2.19219219219219	0.65417052576148\\
2.21221221221221	0.655443117593224\\
2.23223223223223	0.656713952532041\\
2.25225225225225	0.657983017869449\\
2.27227227227227	0.65925030095684\\
2.29229229229229	0.660515789205845\\
2.31231231231231	0.661779470088701\\
2.33233233233233	0.663041331138613\\
2.35235235235235	0.664301359950113\\
2.37237237237237	0.665559544179413\\
2.39239239239239	0.666815871544754\\
2.41241241241241	0.668070329826762\\
2.43243243243243	0.669322906868784\\
2.45245245245245	0.670573590577232\\
2.47247247247247	0.671822368921922\\
2.49249249249249	0.673069229936405\\
2.51251251251251	0.674314161718301\\
2.53253253253253	0.675557152429621\\
2.55255255255255	0.676798190297093\\
2.57257257257257	0.678037263612482\\
2.59259259259259	0.679274360732905\\
2.61261261261261	0.680509470081143\\
2.63263263263263	0.68174258014595\\
2.65265265265265	0.682973679482359\\
2.67267267267267	0.684202756711983\\
2.69269269269269	0.685429800523312\\
2.71271271271271	0.686654799672006\\
2.73273273273273	0.687877742981191\\
2.75275275275275	0.689098619341737\\
2.77277277277277	0.69031741771255\\
2.79279279279279	0.691534127120844\\
2.81281281281281	0.69274873666242\\
2.83283283283283	0.693961235501939\\
2.85285285285285	0.695171612873188\\
2.87287287287287	0.696379858079342\\
2.89289289289289	0.697585960493231\\
2.91291291291291	0.69878990955759\\
2.93293293293293	0.699991694785318\\
2.95295295295295	0.70119130575972\\
2.97297297297297	0.702388732134759\\
2.99299299299299	0.703583963635297\\
3.01301301301301	0.704776990057326\\
3.03303303303303	0.705967801268209\\
3.05305305305305	0.70715638720691\\
3.07307307307307	0.708342737884213\\
3.09309309309309	0.709526843382954\\
3.11311311311311	0.710708693858231\\
3.13313313313313	0.711888279537627\\
3.15315315315315	0.713065590721411\\
3.17317317317317	0.714240617782755\\
3.19319319319319	0.715413351167929\\
3.21321321321321	0.716583781396507\\
3.23323323323323	0.717751899061554\\
3.25325325325325	0.718917694829828\\
3.27327327327327	0.720081159441957\\
3.29329329329329	0.721242283712631\\
3.31331331331331	0.722401058530775\\
3.33333333333333	0.723557474859732\\
3.35335335335335	0.724711523737427\\
3.37337337337337	0.725863196276543\\
3.39339339339339	0.727012483664677\\
3.41341341341341	0.728159377164508\\
3.43343343343343	0.729303868113948\\
3.45345345345345	0.730445947926297\\
3.47347347347347	0.73158560809039\\
3.49349349349349	0.732722840170742\\
3.51351351351351	0.733857635807691\\
3.53353353353353	0.73498998671753\\
3.55355355355355	0.736119884692645\\
3.57357357357357	0.737247321601639\\
3.59359359359359	0.738372289389461\\
3.61361361361361	0.739494780077525\\
3.63363363363363	0.74061478576383\\
3.65365365365365	0.741732298623068\\
3.67367367367367	0.742847310906739\\
3.69369369369369	0.743959814943251\\
3.71371371371371	0.745069803138029\\
3.73373373373373	0.746177267973604\\
3.75375375375375	0.747282202009714\\
3.77377377377377	0.748384597883389\\
3.79379379379379	0.749484448309042\\
3.81381381381381	0.750581746078544\\
3.83383383383383	0.751676484061311\\
3.85385385385385	0.752768655204369\\
3.87387387387387	0.753858252532433\\
3.89389389389389	0.754945269147969\\
3.91391391391391	0.756029698231257\\
3.93393393393393	0.757111533040452\\
3.95395395395395	0.758190766911635\\
3.97397397397397	0.75926739325887\\
3.99399399399399	0.760341405574246\\
4.01401401401401	0.761412797427926\\
4.03403403403403	0.76248156246818\\
4.05405405405405	0.763547694421426\\
4.07407407407407	0.764611187092261\\
4.09409409409409	0.76567203436349\\
4.11411411411411	0.766730230196146\\
4.13413413413413	0.767785768629518\\
4.15415415415415	0.768838643781162\\
4.17417417417417	0.769888849846916\\
4.19419419419419	0.770936381100911\\
4.21421421421421	0.771981231895574\\
4.23423423423423	0.773023396661633\\
4.25425425425425	0.77406286990811\\
4.27427427427427	0.775099646222323\\
4.29429429429429	0.77613372026987\\
4.31431431431431	0.777165086794618\\
4.33433433433433	0.778193740618689\\
4.35435435435435	0.779219676642437\\
4.37437437437437	0.780242889844423\\
4.39439439439439	0.78126337528139\\
4.41441441441441	0.78228112808823\\
4.43443443443443	0.783296143477951\\
4.45445445445445	0.784308416741634\\
4.47447447447447	0.785317943248397\\
4.49449449449449	0.786324718445342\\
4.51451451451451	0.787328737857513\\
4.53453453453453	0.788329997087838\\
4.55455455455455	0.789328491817073\\
4.57457457457457	0.790324217803744\\
4.59459459459459	0.791317170884081\\
4.61461461461461	0.792307346971952\\
4.63463463463463	0.793294742058795\\
4.65465465465465	0.794279352213536\\
4.67467467467467	0.79526117358252\\
4.69469469469469	0.796240202389425\\
4.71471471471471	0.79721643493518\\
4.73473473473473	0.798189867597873\\
4.75475475475475	0.799160496832664\\
4.77477477477477	0.800128319171688\\
4.79479479479479	0.801093331223958\\
4.81481481481481	0.802055529675259\\
4.83483483483483	0.80301491128805\\
4.85485485485485	0.803971472901351\\
4.87487487487487	0.804925211430632\\
4.89489489489489	0.805876123867697\\
4.91491491491491	0.806824207280571\\
4.93493493493493	0.807769458813371\\
4.95495495495495	0.808711875686186\\
4.97497497497497	0.809651455194947\\
4.99499499499499	0.810588194711297\\
5.01501501501502	0.811522091682457\\
5.03503503503504	0.812453143631085\\
5.05505505505506	0.813381348155139\\
5.07507507507508	0.81430670292773\\
5.0950950950951	0.815229205696976\\
5.11511511511512	0.816148854285854\\
5.13513513513514	0.817065646592041\\
5.15515515515516	0.817979580587764\\
5.17517517517518	0.818890654319636\\
5.1951951951952	0.819798865908494\\
5.21521521521522	0.820704213549235\\
5.23523523523524	0.821606695510648\\
5.25525525525526	0.822506310135239\\
5.27527527527528	0.823403055839058\\
5.2952952952953	0.824296931111524\\
5.31531531531532	0.825187934515241\\
5.33533533533534	0.826076064685815\\
5.35535535535536	0.826961320331671\\
5.37537537537538	0.827843700233861\\
5.3953953953954	0.828723203245873\\
5.41541541541542	0.829599828293433\\
5.43543543543544	0.830473574374315\\
5.45545545545546	0.831344440558132\\
5.47547547547548	0.83221242598614\\
5.4954954954955	0.833077529871027\\
5.51551551551552	0.833939751496706\\
5.53553553553554	0.834799090218106\\
5.55555555555556	0.835655545460954\\
5.57557557557558	0.836509116721561\\
5.5955955955956	0.837359803566604\\
5.61561561561562	0.8382076056329\\
5.63563563563564	0.839052522627184\\
5.65565565565566	0.839894554325883\\
5.67567567567568	0.840733700574885\\
5.6956956956957	0.841569961289306\\
5.71571571571572	0.842403336453256\\
5.73573573573574	0.843233826119606\\
5.75575575575576	0.84406143040974\\
5.77577577577578	0.844886149513321\\
5.7957957957958	0.845707983688042\\
5.81581581581582	0.846526933259382\\
5.83583583583584	0.847342998620354\\
5.85585585585586	0.848156180231257\\
5.87587587587588	0.848966478619418\\
5.8958958958959	0.84977389437894\\
5.91591591591592	0.85057842817044\\
5.93593593593594	0.851380080720791\\
5.95595595595596	0.85217885282286\\
5.97597597597598	0.85297474533524\\
5.995995995996	0.853767759181982\\
6.01601601601602	0.85455789535233\\
6.03603603603604	0.855345154900448\\
6.05605605605606	0.856129538945143\\
6.07607607607608	0.856911048669595\\
6.0960960960961	0.857689685321072\\
6.11611611611612	0.858465450210659\\
6.13613613613614	0.859238344712969\\
6.15615615615616	0.860008370265861\\
6.17617617617618	0.860775528370157\\
6.1961961961962	0.861539820589353\\
6.21621621621622	0.862301248549325\\
6.23623623623624	0.863059813938045\\
6.25625625625626	0.863815518505283\\
6.27627627627628	0.864568364062312\\
6.2962962962963	0.865318352481613\\
6.31631631631632	0.866065485696575\\
6.33633633633634	0.866809765701193\\
6.35635635635636	0.867551194549768\\
6.37637637637638	0.868289774356603\\
6.3963963963964	0.869025507295695\\
6.41641641641642	0.869758395600431\\
6.43643643643644	0.870488441563276\\
6.45645645645646	0.871215647535462\\
6.47647647647648	0.871940015926681\\
6.4964964964965	0.872661549204763\\
6.51651651651652	0.873380249895369\\
6.53653653653654	0.874096120581668\\
6.55655655655656	0.874809163904022\\
6.57657657657658	0.875519382559664\\
6.5965965965966	0.876226779302378\\
6.61661661661662	0.876931356942179\\
6.63663663663664	0.877633118344982\\
6.65665665665666	0.878332066432282\\
6.67667667667668	0.879028204180829\\
6.6966966966967	0.879721534622291\\
6.71671671671672	0.880412060842936\\
6.73673673673674	0.881099785983292\\
6.75675675675676	0.881784713237819\\
6.77677677677678	0.882466845854579\\
6.7967967967968	0.883146187134893\\
6.81681681681682	0.883822740433016\\
6.83683683683684	0.884496509155792\\
6.85685685685686	0.88516749676232\\
6.87687687687688	0.885835706763613\\
6.8968968968969	0.886501142722264\\
6.91691691691692	0.887163808252095\\
6.93693693693694	0.887823707017825\\
6.95695695695696	0.888480842734719\\
6.97697697697698	0.88913521916825\\
6.996996996997	0.88978684013375\\
7.01701701701702	0.890435709496067\\
7.03703703703704	0.891081831169215\\
7.05705705705706	0.891725209116031\\
7.07707707707708	0.892365847347822\\
7.0970970970971	0.893003749924017\\
7.11711711711712	0.893638920951819\\
7.13713713713714	0.894271364585851\\
7.15715715715716	0.894901085027806\\
7.17717717717718	0.895528086526095\\
7.1971971971972	0.896152373375493\\
7.21721721721722	0.896773949916786\\
7.23723723723724	0.897392820536415\\
7.25725725725726	0.898008989666123\\
7.27727727727728	0.898622461782598\\
7.2972972972973	0.899233241407116\\
7.31731731731732	0.899841333105188\\
7.33733733733734	0.900446741486197\\
7.35735735735736	0.901049471203044\\
7.37737737737738	0.90164952695179\\
7.3973973973974	0.902246913471293\\
7.41741741741742	0.902841635542853\\
7.43743743743744	0.903433697989853\\
7.45745745745746	0.904023105677392\\
7.47747747747748	0.904609863511932\\
7.4974974974975	0.905193976440935\\
7.51751751751752	0.905775449452499\\
7.53753753753754	0.906354287575\\
7.55755755755756	0.906930495876729\\
7.57757757757758	0.90750407946553\\
7.5975975975976	0.908075043488437\\
7.61761761761762	0.908643393131312\\
7.63763763763764	0.909209133618483\\
7.65765765765766	0.909772270212382\\
7.67767767767768	0.910332808213178\\
7.6976976976977	0.91089075295842\\
7.71771771771772	0.911446109822666\\
7.73773773773774	0.91199888421713\\
7.75775775775776	0.912549081589306\\
7.77777777777778	0.913096707422618\\
7.7977977977978	0.913641767236044\\
7.81781781781782	0.914184266583764\\
7.83783783783784	0.914724211054787\\
7.85785785785786	0.915261606272594\\
7.87787787787788	0.915796457894773\\
7.8978978978979	0.916328771612655\\
7.91791791791792	0.91685855315095\\
7.93793793793794	0.917385808267388\\
7.95795795795796	0.917910542752352\\
7.97797797797798	0.918432762428519\\
7.997997997998	0.918952473150494\\
8.01801801801802	0.919469680804451\\
8.03803803803804	0.919984391307769\\
8.05805805805806	0.920496610608671\\
8.07807807807808	0.921006344685864\\
8.0980980980981	0.921513599548176\\
8.11811811811812	0.922018381234194\\
8.13813813813814	0.922520695811909\\
8.15815815815816	0.923020549378349\\
8.17817817817818	0.923517948059226\\
8.1981981981982	0.924012898008569\\
8.21821821821822	0.924505405408374\\
8.23823823823824	0.924995476468238\\
8.25825825825826	0.925483117425005\\
8.27827827827828	0.925968334542408\\
8.2982982982983	0.926451134110711\\
8.31831831831832	0.926931522446353\\
8.33833833833834	0.927409505891593\\
8.35835835835836	0.927885090814151\\
8.37837837837838	0.928358283606859\\
8.3983983983984	0.928829090687301\\
8.41841841841842	0.929297518497463\\
8.43843843843844	0.929763573503378\\
8.45845845845846	0.930227262194776\\
8.47847847847848	0.930688591084728\\
8.4984984984985	0.9311475667093\\
8.51851851851852	0.931604195627198\\
8.53853853853854	0.932058484419423\\
8.55855855855856	0.932510439688915\\
8.57857857857858	0.932960068060215\\
8.5985985985986	0.933407376179108\\
8.61861861861862	0.933852370712281\\
8.63863863863864	0.934295058346977\\
8.65865865865866	0.934735445790649\\
8.67867867867868	0.935173539770615\\
8.6986986986987	0.935609347033717\\
8.71871871871872	0.936042874345977\\
8.73873873873874	0.936474128492255\\
8.75875875875876	0.936903116275908\\
8.77877877877878	0.937329844518453\\
8.7987987987988	0.937754320059223\\
8.81881881881882	0.938176549755035\\
8.83883883883884	0.938596540479847\\
8.85885885885886	0.939014299124427\\
8.87887887887888	0.939429832596015\\
8.8988988988989	0.939843147817989\\
8.91891891891892	0.940254251729534\\
8.93893893893894	0.940663151285307\\
8.95895895895896	0.941069853455107\\
8.97897897897898	0.941474365223548\\
8.998998998999	0.941876693589725\\
9.01901901901902	0.942276845566892\\
9.03903903903904	0.942674828182127\\
9.05905905905906	0.943070648476016\\
9.07907907907908	0.943464313502323\\
9.0990990990991	0.943855830327667\\
9.11911911911912	0.944245206031199\\
9.13913913913914	0.944632447704286\\
9.15915915915916	0.945017562450184\\
9.17917917917918	0.945400557383726\\
9.1991991991992	0.945781439630998\\
9.21921921921922	0.946160216329029\\
9.23923923923924	0.946536894625473\\
9.25925925925926	0.946911481678292\\
9.27927927927928	0.947283984655452\\
9.2992992992993	0.947654410734601\\
9.31931931931932	0.948022767102769\\
9.33933933933934	0.948389060956051\\
9.35935935935936	0.948753299499307\\
9.37937937937938	0.949115489945848\\
9.3993993993994	0.949475639517139\\
9.41941941941942	0.94983375544249\\
9.43943943943944	0.950189844958755\\
9.45945945945946	0.950543915310032\\
9.47947947947948	0.950895973747366\\
9.4994994994995	0.951246027528444\\
9.51951951951952	0.951594083917306\\
9.53953953953954	0.951940150184045\\
9.55955955955956	0.952284233604514\\
9.57957957957958	0.952626341460033\\
9.5995995995996	0.952966481037101\\
9.61961961961962	0.953304659627102\\
9.63963963963964	0.953640884526019\\
9.65965965965966	0.953975163034148\\
9.67967967967968	0.954307502455811\\
9.6996996996997	0.954637910099072\\
9.71971971971972	0.954966393275455\\
9.73973973973974	0.955292959299666\\
9.75975975975976	0.955617615489308\\
9.77977977977978	0.955940369164607\\
9.7997997997998	0.956261227648134\\
9.81981981981982	0.956580198264532\\
9.83983983983984	0.956897288340238\\
9.85985985985986	0.957212505203217\\
9.87987987987988	0.957525856182688\\
9.8998998998999	0.957837348608854\\
9.91991991991992	0.95814698981264\\
9.93993993993994	0.958454787125421\\
9.95995995995996	0.958760747878762\\
9.97997997997998	0.959064879404153\\
10	0.959367189032749\\
};



\addplot [color=mycolor3, dashed, line width=2.0pt]
  table[row sep=crcr]{%
-14	0.00827806772458625\\
-13.959595959596	0.00843681026643058\\
-13.9191919191919	0.0085982184667312\\
-13.8787878787879	0.00876232911675184\\
-13.8383838383838	0.0089291793590326\\
-13.7979797979798	0.00909880668780533\\
-13.7575757575758	0.00927124894934709\\
-13.7171717171717	0.00944654434227057\\
-13.6767676767677	0.00962473141775027\\
-13.6363636363636	0.00980584907968336\\
-13.5959595959596	0.00998993658478418\\
-13.5555555555556	0.010177033542611\\
-13.5151515151515	0.0103671799155245\\
-13.4747474747475	0.0105604160185757\\
-13.4343434343434	0.010756782519324\\
-13.3939393939394	0.0109563204375826\\
-13.3535353535354	0.0111590711450905\\
-13.3131313131313	0.0113650763651114\\
-13.2727272727273	0.0115743781719559\\
-13.2323232323232	0.0117870189904285\\
-13.1919191919192	0.0120030415951962\\
-13.1515151515152	0.0122224891100793\\
-13.1111111111111	0.0124454050072617\\
-13.0707070707071	0.0126718331064214\\
-13.030303030303	0.0129018175737783\\
-12.989898989899	0.0131354029210599\\
-12.9494949494949	0.0133726340043825\\
-12.9090909090909	0.0136135560230481\\
-12.8686868686869	0.0138582145182547\\
-12.8282828282828	0.0141066553717197\\
-12.7878787878788	0.0143589248042158\\
-12.7474747474747	0.0146150693740173\\
-12.7070707070707	0.0148751359752569\\
-12.6666666666667	0.0151391718361907\\
-12.6262626262626	0.0154072245173722\\
-12.5858585858586	0.0156793419097326\\
-12.5454545454545	0.015955572232568\\
-12.5050505050505	0.0162359640314308\\
-12.4646464646465	0.0165205661759262\\
-12.4242424242424	0.0168094278574118\\
-12.3838383838384	0.0171025985865997\\
-12.3434343434343	0.0174001281910599\\
-12.3030303030303	0.017702066812625\\
-12.2626262626263	0.0180084649046945\\
-12.2222222222222	0.018319373229438\\
-12.1818181818182	0.0186348428548969\\
-12.1414141414141	0.0189549251519839\\
-12.1010101010101	0.0192796717913782\\
-12.0606060606061	0.0196091347403178\\
-12.020202020202	0.0199433662592863\\
-11.979797979798	0.0202824188985944\\
-11.9393939393939	0.0206263454948552\\
-11.8989898989899	0.0209751991673528\\
-11.8585858585859	0.0213290333143029\\
-11.8181818181818	0.021687901609006\\
-11.7777777777778	0.0220518579958911\\
-11.7373737373737	0.0224209566864506\\
-11.6969696969697	0.022795252155065\\
-11.6565656565657	0.0231747991347173\\
-11.6161616161616	0.0235596526125967\\
-11.5757575757576	0.0239498678255902\\
-11.5353535353535	0.0243455002556632\\
-11.4949494949495	0.024746605625127\\
-11.4545454545455	0.0251532398917944\\
-11.4141414141414	0.0255654592440211\\
-11.3737373737374	0.025983320095635\\
-11.3333333333333	0.0264068790807506\\
-11.2929292929293	0.0268361930484702\\
-11.2525252525253	0.0272713190574699\\
-11.2121212121212	0.0277123143704721\\
-11.1717171717172	0.0281592364486021\\
-11.1313131313131	0.028612142945631\\
-11.0909090909091	0.0290710917021026\\
-11.0505050505051	0.0295361407393455\\
-11.010101010101	0.03000734825337\\
-10.969696969697	0.0304847726086495\\
-10.9292929292929	0.0309684723317869\\
-10.8888888888889	0.0314585061050654\\
-10.8484848484848	0.0319549327598845\\
-10.8080808080808	0.0324578112700805\\
-10.7676767676768	0.0329672007451323\\
-10.7272727272727	0.033483160423252\\
-10.6868686868687	0.0340057496643611\\
-10.6464646464646	0.0345350279429529\\
-10.6060606060606	0.0350710548408393\\
-10.5656565656566	0.0356138900397856\\
-10.5252525252525	0.0361635933140302\\
-10.4848484848485	0.0367202245226922\\
-10.4444444444444	0.0372838436020663\\
-10.4040404040404	0.0378545105578052\\
-10.3636363636364	0.0384322854569901\\
-10.3232323232323	0.0390172284200907\\
-10.2828282828283	0.0396093996128141\\
-10.2424242424242	0.040208859237844\\
-10.2020202020202	0.0408156675264703\\
-10.1616161616162	0.0414298847301103\\
-10.1212121212121	0.0420515711117223\\
-10.0808080808081	0.0426807869371114\\
-10.040404040404	0.0433175924661297\\
-10	0.0439620479437708\\
};


\addplot [color=mycolor3, dashed, line width=2.0pt]
  table[row sep=crcr]{%
10	0.959367189032749\\
10.040404040404	0.95997178592619\\
10.0808080808081	0.960569052231772\\
10.1212121212121	0.961159048173088\\
10.1616161616162	0.9617418339426\\
10.2020202020202	0.962317469693303\\
10.2424242424242	0.962886015530498\\
10.2828282828283	0.963447531503673\\
10.3232323232323	0.964002077598497\\
10.3636363636364	0.964549713728922\\
10.4040404040404	0.965090499729402\\
10.4444444444444	0.96562449534722\\
10.4848484848485	0.966151760234928\\
10.5252525252525	0.966672353942906\\
10.5656565656566	0.967186335912023\\
10.6060606060606	0.967693765466424\\
10.6464646464646	0.968194701806427\\
10.6868686868687	0.968689204001529\\
10.7272727272727	0.969177330983534\\
10.7676767676768	0.969659141539794\\
10.8080808080808	0.970134694306562\\
10.8484848484848	0.970604047762463\\
10.8888888888889	0.971067260222081\\
10.9292929292929	0.971524389829658\\
10.969696969697	0.971975494552913\\
11.010101010101	0.972420632176968\\
11.0505050505051	0.972859860298399\\
11.0909090909091	0.973293236319399\\
11.1313131313131	0.973720817442049\\
11.1717171717172	0.974142660662717\\
11.2121212121212	0.974558822766556\\
11.2525252525253	0.974969360322134\\
11.2929292929293	0.97537432967616\\
11.3333333333333	0.975773786948341\\
11.3737373737374	0.97616778802634\\
11.4141414141414	0.976556388560851\\
11.4545454545455	0.976939643960794\\
11.4949494949495	0.97731760938861\\
11.5353535353535	0.977690339755678\\
11.5757575757576	0.978057889717841\\
11.6161616161616	0.978420313671041\\
11.6565656565657	0.978777665747068\\
11.6969696969697	0.979129999809416\\
11.7373737373737	0.979477369449254\\
11.7777777777778	0.979819827981499\\
11.8181818181818	0.980157428441005\\
11.8585858585859	0.98049022357885\\
11.8989898989899	0.980818265858744\\
11.9393939393939	0.981141607453534\\
11.979797979798	0.981460300241813\\
12.020202020202	0.981774395804644\\
12.0606060606061	0.98208394542238\\
12.1010101010101	0.982389000071591\\
12.1414141414141	0.982689610422097\\
12.1818181818182	0.982985826834095\\
12.2222222222222	0.983277699355399\\
12.2626262626263	0.983565277718768\\
12.3030303030303	0.983848611339345\\
12.3434343434343	0.98412774931219\\
12.3838383838384	0.984402740409907\\
12.4242424242424	0.984673633080379\\
12.4646464646465	0.984940475444589\\
12.5050505050505	0.98520331529454\\
12.5454545454545	0.985462200091272\\
12.5858585858586	0.985717176962971\\
12.6262626262626	0.985968292703169\\
12.6666666666667	0.986215593769036\\
12.7070707070707	0.986459126279766\\
12.7474747474747	0.986698936015048\\
12.7878787878788	0.986935068413631\\
12.8282828282828	0.987167568571969\\
12.8686868686869	0.987396481242964\\
12.9090909090909	0.987621850834782\\
12.9494949494949	0.987843721409764\\
12.989898989899	0.988062136683412\\
13.030303030303	0.988277140023467\\
13.0707070707071	0.98848877444906\\
13.1111111111111	0.988697082629949\\
13.1515151515152	0.988902106885829\\
13.1919191919192	0.98910388918573\\
13.2323232323232	0.989302471147481\\
13.2727272727273	0.989497894037262\\
13.3131313131313	0.989690198769216\\
13.3535353535354	0.989879425905152\\
13.3939393939394	0.990065615654307\\
13.4343434343434	0.990248807873188\\
13.4747474747475	0.990429042065477\\
13.5151515151515	0.990606357382016\\
13.5555555555556	0.99078079262085\\
13.5959595959596	0.990952386227341\\
13.6363636363636	0.991121176294353\\
13.6767676767677	0.991287200562494\\
13.7171717171717	0.991450496420425\\
13.7575757575758	0.991611100905234\\
13.7979797979798	0.991769050702871\\
13.8383838383838	0.991924382148638\\
13.8787878787879	0.992077131227746\\
13.9191919191919	0.992227333575929\\
13.959595959596	0.992375024480105\\
14	0.992520238879112\\
};


\addplot [color=black]
  table[row sep=crcr]{%
-0.107071998280122	0\\
-0.107071998280122	0.5\\
};


\addplot [color=black, dotted]
  table[row sep=crcr]{%
-14	1\\
14	1\\
};


\addplot [color=black, dotted]
  table[row sep=crcr]{%
-14	0\\
14	0\\
};
\end{axis}


\end{tikzpicture}%
\caption{Probability of correct recognition over \gls{snr}, Subject Group A, the x-calibration is questionable}
\label{fig:srtn_psych}
\end{figure}



