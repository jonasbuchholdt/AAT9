\section{\gls{bier}: \gls{srtn} Analysis}\label{sec:result_srt}


During the \gls{bier} test, a for each of the of the ten subjects four \gls{srtn}s have been measured, evenly split between \gls{ac} and \gls{bc}.
The values two values for each conductor for each subject are averaged, resulting in a total of 10 pairs of observations to perform statistics on.
It is assumed, that the result of the loudness matching part of \gls{bier} (reference to the LM section here ??) leads to a reliable estimate of the \gls{snr} for the \gls{bct} for the \gls{hint} part of \gls{bier}.
The average \gls{srtn} are listed in \autoref{tab:srtn}). For \gls{ac}, values are within a range of \SI{-7.1}{\decibel} to \SI{11.5}{\decibel} and result in a total average of \SI{1.6}{\decibel} with a standard deviation of \SI{6.09}{\decibel}. For \gls{bc}, the values are in a larger range from \SI{-7.2}{\decibel} to  \SI{21.5}{\decibel} with a total average of \SI{5.4}{\decibel} and a standard deviation of \SI{9.15}{\decibel}.
In order to determine, if under this assumption there is a statistically significant difference in intelligibility between \gls{ac} and \gls{bc}, a paired-sample t-Test can be performed.\\

\begin{table}[H]
\centering
\caption{Average \gls{srtn} of the subjects for \gls{ac} and \gls{bc} obtained with \gls{bier}, values are rounded to one digit after the decimal point for representation in the table, not for the underlying calculations, all values in \si{\decibel}}
\label{tab:srtn}
\begin{tabular}{l|rrrrrrrrrr}
Subject     & 1   & 2    & 3    & 4    & 5    & 6    & 7     & 8     & 9    & 10   \\ \hline
Avg. \gls{srtn}\textsubscript{,\gls{ac}} & 2.8 & 1.0  & -2.3 & -3.8 & -7.1 & -2.2 & 11.5  & 0.4   & 11.0 & 4.8  \\
Avg. \gls{srtn}\textsubscript{,\gls{bc}} & 0.7 & 1.6  & 1.1  & -7.2 & -1.3 & -0.6 & 21.5  & 11.4  & 9.7  & 17.5 \\
\gls{srtn}\textsubscript{,AC}-\gls{srtn}\textsubscript{,BC}  & 2.0 & -0.6 & -3.4 & -3.3 & -5.8 & -1.6 & -10.0 & -11.0 & 1.3  & -12.8
\end{tabular}
\end{table}

The null hypothesis of the test is, that the difference between the observation pairs is normally distributed and zero mean. If the former hypothesis can be rejected, it can be concluded, that there is a statistically significant difference between the mean \gls{srtn} of the two conductors.
The \matlab function \texttt{ttest} has been applied to the available data from \autoref{tab:srtn}. The null hypothesis can be rejected with a $p$-value of 0.0644. This is not within the commonly applied significance level of $\alpha=$\SI{5}{\percent}.\\
A normal probability plot of the \gls{srtn} is depicted in \autoref{fig:srtn_normal}. The lines in plot correspond to the normal distribution, that has been derived from estimating mean and variance from the available observations. 
The points depict those observations. Points being close to their corresponding lines indicate, that a normal distribution is likely to underlay the observations. The datapoints in \autoref{fig:srtn_normal} can be considered reasonably close to the corresponding normal distributions, however the number of points is small.\\
\plot{chapters/tests/normal_prob_AB}{Normal probability plot, average performance of each subject with each of the tested transducers.}{fig:srtn_normal}
For the paired-sample t-test to be viable, it has to be assumed, that the pairs of observations are normally distributed. Hence the distribution of the difference between the two also has to be normally distributed. 
This can be checked by means of a Lilliefors test.
The Lilliefors test checks the null hypothesis, that the input data is coming from a normally distributed population. 
In the given context, the intrasubject difference between the the \gls{ac} \gls{srtn} and the \gls{bc} \gls{srtn} (see  is tested.
The population mean and variance are estimated based on the input data. The null hypothesis is rejected or confirmed, based on the discrepancy between the Gaussian \gls{cdf}, that is derived from the estimated mean and variance and the empirical distribution.
This procedure is conveniently implemented in the \matlab function \texttt{lillietest}.
The null hypothesis, that the intrasubject \gls{srtn} difference is normally distributed, is not rejected at a \SI{5}{\percent} significance level based on the data at hand.\\

\section{Psychometric functions derived from \gls{bier}}\label{sec:result_psycho}

In the previous \autoref{sec:result_srt}, the \gls{srtn} have been investigated, to assess, whether there is a statistically signifant difference in \gls{srt} between \gls{ace} and \gls{bct}.
The following section augments the previous statistical investigations by estimating psychometric functions for sentence recognition with \gls{ace} and {bct}.
Unlike for the previous investigations, there are responses from ten subjects to two lists with 20 sentences each per transducer. This results in a dataset of 400 responses for each of the two psychometric functions.\\

In general, psychometric functions can used to assess data collected from psychophysical experiments, that involve a detection or discrimination task (answers can be categorised in yes/no or right/wrong).
The psychometric function is a scaled \gls{cdf}, that maps the probability of a correct response over some physical property of the presented stimulus.\citep{wichmann_01}\\
The psychometric function can be expressed as follows:
\begin{equation}\label{eq:psycho}
\psi(x;\alpha,\beta,\gamma,\lambda) = \gamma + (1-\gamma-\lambda) F(x;\alpha,\beta)
\end{equation}
\startexplain
    \explain{$\psi$}{is the psychometric function}{}
    \explain{$x$}{corresponds to the stimulus intensity and is denoted on the x-axis of plots of the psychometric function.}{}
	\explain{$\alpha$}{is parameter of the sigmoid function $F(x;\alpha,\beta)$, that corresponds to a displacement of the graph along the x-axis. It determines the position of the threshold of the psychometric function, which is defined as the $x$ value, at which $F(x;\alpha,\beta)$ has a value $0.5$ .}{}
	\explain{$\beta$}{is a parameter of the sigmoid function $F(x;\alpha,\beta)$, that corresponds to the slope of the graph}{}
	\explain{$\gamma$}{is a parameter, corresponds to the lower bound of the curve and therefore the probability of a correct answer at an infinitely low stimulus level. For multiple choice experiments, this parameter is the probability of a correct answer by guessing and is therefore sometimes referred to as the \textit{guessing rate}}{}
	\explain{$\lambda$}{is a parameter, that corresponds to the upper bound of the curve. It corresponds to a rate, at which subjects respond incorrectly regardless of stimulus intensity. It is often referred to as \textit{lapse}.}{}
	\explain{$F$}{is a sigmoid function. Common choices include the cumulative Gaussian function, the cumulative Gumbel distribution and the Weibull function.}{}
\stopexplain
When deriving the psychometric functions from the responses given by subjects during the \gls{hint} part of the \gls{bier}, the stimulus variable $x$ corresponds to the \gls{snr}, at which the sentences have been presented during the test.
The absolute difference in average \gls{srtn} has been discussed previously (see \autoref{sec:result_srt}). Hence, the parameter of interest when estimating psychometric functions is their slope, not their threshold.
This allows for a normation to be applied on the data before estimating psychometric functions.
For the set of responses collected from a given subject with a given transducer, the average \gls{srtn} for that particular person with that transducer is subtracted from the \gls{snr}s, that have been tested.
The result of this normation is set responses centered around a level of \SI{0}{\decibel} normed \gls{snr}.
Due to the \gls{hint} being an adaptive procedure and the normation, the normed \gls{snr}s, at which data is available, are not consistent.
In order to determine the proportion of correct responses for the available range normed \gls{snr}, bins are introduced.
The bin size is \SI{1}{\decibel} and the bins are centered around integer normed \gls{snr} values. 
The average performance of all subjects with a given transducer is then calculated by dividing the number of correct responses by the total number of stimuli presented in that bin.\\
With this preprocessing applied to the data, input data for the estimation of the psychometric function is available in the form of the performance, the number of trials and the normed \gls{snr} for every bin.
Some of the parameters of the psychometric functions can be fixed by making assumptions. It is assumed, that there exists a \gls{snr}, at which it is impossible for subjects to repeat the presented sentence correctly. Hence the guessing rate $\gamma$ is fixed at a value of 0.
The \gls{srtn} is defined as \gls{snr}, at which subjects have a \SI{50}{\percent} chance of correctly repeating a presented sentence. Because the input data for the estimation of the psychometric function has been normed with the \gls{srtn}, it is assumed, that the threshold if the psychometric function is at a normed \gls{snr} of \SI{0}{\decibel}. This value is also fixed. 
With the formerly described input data and assumptions, psychometric functions for sentence presentation with \gls{ace} and \gls{bct} were derived by means of the \matlab toolbox \texttt{psignifit 4}. The methods, that the toolbox is based on, are described in \citep{schuett_16}. A cumulative Gaussian function has been used as the sigmoid $F$.
The results are presented in \autoref{fig:srtn_psych}.



%The fit has been performed with the \matlab toolbox \texttt{psignifit 4}. The methods, that the toolbox is based on, is described in \citep{schuett_16}.
\plot{chapters/tests/psych_03}{Psychometric functions for sentence intelligibility with presentation via \gls{ace} and \gls{bct}. The size of the points corresponds to the number of observations per bin}{fig:srtn_psych}

%To quantify the slope of a psychometric function, a width can be defined as the difference between the stimulus levels (in this case \gls{snr}s), at which the function reaches values of $0.05$ and $0.95$. The width of the \gls{ac} psychometric function is \SI{10.4}{\decibel}, the width of the \gls{bc} psychometric function is \SI{18.7}{\decibel}.

%The lapse $\lambda$ has been free during the fit of the psychometric function. The resulting values are $\lambda = 0$ for \gls{ac} and $\lambda = 0.009$.

Parameters derived from the psychometric functions are listed in \autoref{tab:psych}. The width is defined  as the difference between stimulus levels, at which the function reaches given values. 
It can be seen, that for \gls{bc} is shallower than for \gls{ac}.
The size of the dots in \autoref{fig:srtn_psych} corresponds to the number of observations that has resulted in the value that they display. The biggest dots appear close to the threshold at the normed \gls{snr} \SI{0}{\decibel}.
The dots, that correspond to the \gls{ac} psychometric function are tendentially closer to their respective graph, than the dots, that correspond to the \gls{bc} psychometric function.
This is reflected in the confidence interval of the values for the $00.5$ to $0.95$ width in \autoref{tab:psych}. The interval corresponding to \gls{bc} is larger than the one corresponding to \gls{ac}. 



\begin{table}[H]
\caption{Properties of the psychometric functions.}
\label{tab:psych}
\centering
\begin{tabular}{l|rrrrlr}
\multirow{2}{*}{} & \multicolumn{2}{l}{Width [\si{\decibel}] (CI\textsubscript{\SI{95}{\percent}})} & \multicolumn{1}{l}{Width [\si{\decibel}]} & \multicolumn{1}{l}{Slope}   & \multicolumn{2}{l}{\multirow{2}{*}{Lapse (CI\textsubscript{\SI{95}{\percent}})}} \\
                           & \multicolumn{2}{c}{0.05 - 0.95}                                                 & \multicolumn{1}{c}{0.1 - 0.9}             & \multicolumn{1}{c}{SNR = 0} & \multicolumn{2}{l}{}                                                             \\ \hline
\gls{ac}         & 10.4                               & (8.2 - 13.6)                                & 8.1                                       & 0.1257                      & \multicolumn{1}{r}{0}                       & (0.0000 - 0.0747)                  \\
\gls{bc}         & 18.7                               & (13.0 - 27.7)                               & 14.8                                      & 0.0697                      & \multicolumn{1}{r}{0.009}                   & (0.0019 - 0.1238)                 
\end{tabular}
\end{table}

