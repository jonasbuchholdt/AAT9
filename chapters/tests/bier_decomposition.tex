\section{\gls{bier}: \gls{srtn} Analysis}\label{sec:result_srt}


During the \gls{bier} test, a for each of the of the ten subjects four \gls{srtn}s have been measured, evenly split between \gls{ac} and \gls{bc}.
The values two values for each conductor for each subject are averaged, resulting in a total of 10 pairs of observations to perform statistics on (see \autoref{tab:srtn}).
It is assumed, that the result of the loudness matching part of \gls{bier} (reference to the LM section here ??) leads to a reliable estimate of the \gls{snr} for the \gls{bct} for the \gls{hint} part of \gls{bier}.
In order to determine, if under this assumption there is a statistically significant difference in intelligibility between \gls{ac} and \gls{bc}, a paired-sample t-Test can be performed.\\

\begin{table}[H]
\centering
\caption{Average \gls{srtn} of the subjects for \gls{ac} and \gls{bc} obtained with \gls{bier}, values are rounded to one digit after the decimal point for representation in the table, not for the underlying calculations.}
\label{tab:srtn}
\begin{tabular}{l|rrrrrrrrrr}
Subject     & 1   & 2    & 3    & 4    & 5    & 6    & 7     & 8     & 9    & 10   \\ \hline
Avg. \gls{srtn}\textsubscript{,\gls{ac}} & 2.8 & 1.0  & -2.3 & -3.8 & -7.1 & -2.2 & 11.5  & 0.4   & 11.0 & 4.8  \\
Avg. \gls{srtn}\textsubscript{,\gls{bc}} & 0.7 & 1.6  & 1.1  & -7.2 & -1.3 & -0.6 & 21.5  & 11.4  & 9.7  & 17.5 \\
\gls{srtn}\textsubscript{,AC}-\gls{srtn}\textsubscript{,BC}  & 2.0 & -0.6 & -3.4 & -3.3 & -5.8 & -1.6 & -10.0 & -11.0 & 1.3  & 12.8
\end{tabular}
\end{table}

The null hypothesis of the test is, that the difference between the observation pairs is normally distributed and zero mean. If the former hypothesis can be rejected, it can be concluded, that there is a statistically significant difference between the mean \gls{srtn} of the two conductors.
The \matlab function \texttt{ttest} has been applied to the available data from \autoref{tab:srtn}. The null hypothesis can be rejected with a $p$-value of 0.0644. This does reach the commonly apllied significance level of $\alpha=$\SI{5}{\percent}.\\
A normal probability plot of the \gls{srtn} is depicted in \autoref{fig:srtn_normal}. The lines in plot correspond to the normal distribution, that has been derived from estimating mean and variance from the available observations. 
The points depict those observations. Points being close to their corresponding lines indicate, that a normal distribution is likely to underlay the observations. The datapoints in \autoref{fig:srtn_normal} can be considered reasonably close to the corresponding normal distributions, however the number of points is small.\\
\begin{figure}[H]
\centering
% This file was created by matlab2tikz.
%
%The latest updates can be retrieved from
%  http://www.mathworks.com/matlabcentral/fileexchange/22022-matlab2tikz-matlab2tikz
%where you can also make suggestions and rate matlab2tikz.
%
\begin{tikzpicture}

\begin{axis}[%
width=120mm, 
height=50mm, 
at={(5mm,5mm)}, 
scale only axis,
xmin=-7.9175,
xmax=22.2175,
xlabel style={font=\color{white!15!black}},
xlabel={Data},
ymin=-1.95996398454005,
ymax=1.95996398454005,
ytick={-3.09023230616781,-2.74778138544499,-2.32634787404084,-2.05374891063182,-1.64485362695147,-1.2815515655446,-0.674489750196082,0,0.674489750196082,1.2815515655446,1.64485362695147,2.05374891063182,2.32634787404084,2.74778138544499,3.09023230616781},
yticklabels={{0.001},{0.003},{0.01},{0.02},{0.05},{0.10},{0.25},{0.50},{0.75},{0.90},{0.95},{0.98},{0.99},{0.997},{0.999}},
ylabel style={font=\color{white!15!black}},
ylabel={Probability},
axis background/.style={fill=white},
title style={font=\bfseries},
title={Normal Probability Plot},
axis x line*=bottom,
axis y line*=left,
xmajorgrids,
ymajorgrids,
legend style={legend cell align=left, align=left, draw=white!15!black}
]


\addplot [color=color1]
  table[row sep=crcr]{%
-2.25	-0.674489750196082\\
4.75	0.674489750196082\\
};
\addlegendentry{\gls{ac}, Gaussian Fit}

\addplot [color=color2, draw=none, mark=+, mark options={solid, blue}]
  table[row sep=crcr]{%
-7.1	-1.64485362695147\\
-3.85	-1.03643338949379\\
-2.25	-0.674489750196082\\
-2.2	-0.385320466407568\\
0.365	-0.125661346855074\\
1.025	0.125661346855074\\
2.75	0.385320466407568\\
4.75	0.674489750196082\\
11	1.03643338949379\\
11.5	1.64485362695147\\
};
\addlegendentry{\gls{ac}, Subject Data}



\addplot [color=color3]
  table[row sep=crcr]{%
-0.603	-0.674489750196082\\
11.35	0.674489750196082\\
};
\addlegendentry{\gls{bc}, Gaussian Fit}

\addplot [color=color4, draw=none, mark=+, mark options={solid, blue}]
  table[row sep=crcr]{%
-7.2	-1.64485362695147\\
-1.285	-1.03643338949379\\
-0.603	-0.674489750196082\\
0.747	-0.385320466407568\\
1.145	-0.125661346855074\\
1.625	0.125661346855074\\
9.7	0.385320466407568\\
11.35	0.674489750196082\\
17.5	1.03643338949379\\
21.5	1.64485362695147\\
};
\addlegendentry{\gls{bc}, Subject Data}
\addplot [color=red, dashdotted]
  table[row sep=crcr]{%
-7.2	-1.41900725744005\\
21.5	1.81998811286491\\
};
%\addlegendentry{data4}
\addplot [color=red, dashdotted]
  table[row sep=crcr]{%
-7.1	-1.60913983261065\\
11.5	1.97529141128853\\
};
%\addlegendentry{data1}

\end{axis}
\end{tikzpicture}%
\caption{Bone Air Thing}
\label{fig:srtn_normal}
\end{figure}
For the paired-sample t-test to be viable, it has to be assumed, that the pairs of observations are normally distributed. Hence the distribution of the difference between the two also has to be normally distributed. 
This can be checked by means of a Lilliefors test.
The Lilliefors test checks the null hypothesis, that the input data is coming from a normally distributed population. 
In the given context, the intrasubject difference between the the \gls{ac} \gls{srtn} and the \gls{bc} \gls{srtn} (see  is tested.
The population mean and variance are estimated based on the input data. The null hypothesis is rejected or confirmed, based on the discrepancy between the Gaussian \gls{cdf}, that is derived from the estimated mean and variance and the empirical distribution.
This procedure is conveniently implemented in the \matlab function \texttt{lillietest}.
The null hypothesis, that the intrasubject \gls{srtn} difference is normally distributed, is not rejected at a \SI{5}{\percent} significance level based on the data at hand.\\

\section{Psychometric functions derived from \gls{bier}}\label{sec:result_psycho}

In the previous \autoref{sec:result_srt}, the \gls{srtn} have been investigated, to assess, whether there is a statistically signifant difference in \gls{srt} between \gls{ace} and \gls{bct}.
The following section augments the previous statistical investigations by estimating psychometric functions for sentence recognition with \gls{ace} and {bct}.
Unlike for the previous investigations, there are responses from ten subjects to two lists with 20 sentences each per transducer. This results in a dataset of 400 responses for each of the two psychometric functions.\\

In general, psychometric functions can used to assess data collected from psychophysical experiments, that involve a detection or discrimination task (answers can be categorised in yes/no or right/wrong).
The psychometric function is a scaled \gls{cdf}, that maps the probability of a correct response over some physical property of the presented stimulus.\citep{wichmann_01}\\
The psychometric function can be expressed as follows:
\begin{equation}\label{eq:psycho}
\psi(x;\alpha,\beta,\gamma,\lambda) = \gamma + (1-\gamma-\lambda) F(x;\alpha,\beta)
\end{equation}
\startexplain
    \explain{$\psi$ is the psychometric function}{}
    \explain{$x$ corresponds to the stimulus intensity and is denoted on the x-axis of plots of the psychometric function.}{}
	\explain{$\alpha$ is parameter of the sigmoid function $F$, that corresponds to a displacement of the graph along the x-axis}{}
	\explain{$\beta$ is a parameter of the sigmoid function $F$, that corresponds to the slope of the graph}{}
	\explain{$\gamma$ is a parameter, corresponds to the lower bound of the curve and therefore the probability of a correct answer at an infinitely low stimulus level. For multiple choice experiments, this parameter is the probability of a correct answer by guessing and is therefore sometimes referred to as the \textit{guessing rate}}{}
	\explain{$\lambda$ is a parameter, that corresponds to the upper bound of the curve. It corresponds to a rate, at which subjects respond incorrectly regardless of stimulus intensity. It is often referred to as \textit{lapse}.}{}
	\explain{Fis a sigmoid function. Common choices include the cumulative Gaussian function, the cumulative Gumbel distribution and the Weibull function.}{}
\stopexplain


An overview on the background of the psychometric function is given in \citep{wichmann_01}.

The fit has been performed with the \matlab toolbox \texttt{psignifit 4}. The methods, that the toolbox is based on, is described in \citep{schuett_16}.
\begin{figure}[H] 
\centering
% This file was created by matlab2tikz.
%
%The latest updates can be retrieved from
%  http://www.mathworks.com/matlabcentral/fileexchange/22022-matlab2tikz-matlab2tikz
%where you can also make suggestions and rate matlab2tikz.
%

%
\begin{tikzpicture}

\begin{axis}[%
width=120mm, 
height=74mm, 
at={(5mm,5mm)}, 
scale only axis,
xmin=-14,
xmax=14,
tick align=outside,
xlabel style={font=\color{white!15!black}},
xlabel={\gls{snr} normed to \gls{srtn}},
ymin=0,
ymax=1,
ylabel style={font=\color{white!15!black}},
ylabel={Proportion of Correct Responses},
axis background/.style={fill=white},
axis x line*=bottom,
axis y line*=left,
grid=both,
grid style={line width=.1pt, draw=gray!10},
major grid style={line width=.2pt,draw=gray!50}, 
minor tick num=4,
legend style={legend cell align=left, align=left, draw=white!15!black},
legend entries={\gls{ac} Psychometric Function,
                \gls{ac} Data,
                \gls{bc} Psychometric Function,
                \gls{bc} Data},
legend pos=north west
]
\addlegendimage{mycolor1, line width=2.0pt}
\addlegendimage{only marks,mycolor2}
\addlegendimage{mycolor3,line width=2.0pt}
\addlegendimage{only marks, mycolor4}
\addplot [color=mycolor2, draw=none, mark size=3.0pt, mark=*, mark options={solid, mycolor2}]
  table[row sep=crcr]{%
-4	0.0769230769230769\\
};


\addplot [color=mycolor2, draw=none, mark size=4.9pt, mark=*, mark options={solid, mycolor2}]
  table[row sep=crcr]{%
-3	0.117647058823529\\
};


\addplot [color=mycolor2, draw=none, mark size=4.6pt, mark=*, mark options={solid, mycolor2}]
  table[row sep=crcr]{%
-2	0.2\\
};


\addplot [color=mycolor2, draw=none, mark size=7.6pt, mark=*, mark options={solid, mycolor2}]
  table[row sep=crcr]{%
-1	0.457831325301205\\
};


\addplot [color=mycolor2, draw=none, mark size=6.0pt, mark=*, mark options={solid, mycolor2}]
  table[row sep=crcr]{%
0	0.519230769230769\\
};


\addplot [color=mycolor2, draw=none, mark size=7.0pt, mark=*, mark options={solid, mycolor2}]
  table[row sep=crcr]{%
1	0.704225352112676\\
};


\addplot [color=mycolor2, draw=none, mark size=4.9pt, mark=*, mark options={solid, mycolor2}]
  table[row sep=crcr]{%
2	0.771428571428571\\
};


\addplot [color=mycolor2, draw=none, mark size=4.7pt, mark=*, mark options={solid, mycolor2}]
  table[row sep=crcr]{%
3	0.75\\
};


\addplot [color=mycolor2, draw=none, mark size=3.1pt, mark=*, mark options={solid, mycolor2}]
  table[row sep=crcr]{%
5	0.928571428571429\\
};


\addplot [color=mycolor2, draw=none, mark size=2.5pt, mark=*, mark options={solid, mycolor2}]
  table[row sep=crcr]{%
-5	0.111111111111111\\
};


\addplot [color=mycolor2, draw=none, mark size=1.7pt, mark=*, mark options={solid, mycolor2}]
  table[row sep=crcr]{%
7	1\\
};


\addplot [color=mycolor2, draw=none, mark size=0.8pt, mark=*, mark options={solid, mycolor2}]
  table[row sep=crcr]{%
-6	0\\
};


\addplot [color=mycolor2, draw=none, mark size=3.0pt, mark=*, mark options={solid, mycolor2}]
  table[row sep=crcr]{%
4	0.846153846153846\\
};


\addplot [color=mycolor2, draw=none, mark size=0.8pt, mark=*, mark options={solid, mycolor2}]
  table[row sep=crcr]{%
8	1\\
};


\addplot [color=mycolor2, draw=none, mark size=1.9pt, mark=*, mark options={solid, mycolor2}]
  table[row sep=crcr]{%
6	1\\
};


\addplot [color=mycolor2, draw=none, mark size=0.8pt, mark=*, mark options={solid, mycolor2}]
  table[row sep=crcr]{%
10	1\\
};


\addplot [color=mycolor2, draw=none, mark size=0.8pt, mark=*, mark options={solid, mycolor2}]
  table[row sep=crcr]{%
-8	0\\
};


\addplot [color=mycolor2, draw=none, mark size=0.8pt, mark=*, mark options={solid, mycolor2}]
  table[row sep=crcr]{%
9	1\\
};


\addplot [color=mycolor1, line width=2.0pt]
  table[row sep=crcr]{%
-8	0.00586161898432176\\
-7.98198198198198	0.00595683828728672\\
-7.96396396396396	0.00605342655645014\\
-7.94594594594595	0.0061514003163939\\
-7.92792792792793	0.00625077623880617\\
-7.90990990990991	0.00635157114296609\\
-7.89189189189189	0.00645380199621612\\
-7.87387387387387	0.00655748591442195\\
-7.85585585585586	0.00666264016241973\\
-7.83783783783784	0.00676928215445021\\
-7.81981981981982	0.00687742945457977\\
-7.8018018018018	0.0069870997771079\\
-7.78378378378378	0.00709831098696089\\
-7.76576576576577	0.00721108110007161\\
-7.74774774774775	0.00732542828374492\\
-7.72972972972973	0.00744137085700867\\
-7.71171171171171	0.00755892729094983\\
-7.69369369369369	0.00767811620903558\\
-7.67567567567568	0.00779895638741917\\
-7.65765765765766	0.00792146675523011\\
-7.63963963963964	0.00804566639484856\\
-7.62162162162162	0.00817157454216372\\
-7.6036036036036	0.00829921058681571\\
-7.58558558558559	0.00842859407242091\\
-7.56756756756757	0.00855974469678039\\
-7.54954954954955	0.00869268231207124\\
-7.53153153153153	0.00882742692502035\\
-7.51351351351351	0.00896399869706069\\
-7.4954954954955	0.00910241794446938\\
-7.47747747747748	0.00924270513848788\\
-7.45945945945946	0.00938488090542338\\
-7.44144144144144	0.0095289660267316\\
-7.42342342342342	0.00967498143908058\\
-7.40540540540541	0.00982294823439511\\
-7.38738738738739	0.00997288765988166\\
-7.36936936936937	0.0101248211180335\\
-7.35135135135135	0.0102787701666158\\
-7.33333333333333	0.01043475651863\\
-7.31531531531532	0.0105928020422585\\
-7.2972972972973	0.0107529287607871\\
-7.27927927927928	0.0109151588525077\\
-7.26126126126126	0.0110795146505987\\
-7.24324324324324	0.0112460186429838\\
-7.22522522522523	0.0114146934721697\\
-7.20720720720721	0.0115855619350602\\
-7.18918918918919	0.0117586469827493\\
-7.17117117117117	0.0119339717202907\\
-7.15315315315315	0.0121115594064446\\
-7.13513513513514	0.0122914334534013\\
-7.11711711711712	0.0124736174264812\\
-7.0990990990991	0.0126581350438114\\
-7.08108108108108	0.0128450101759782\\
-7.06306306306306	0.0130342668456547\\
-7.04504504504505	0.0132259292272052\\
-7.02702702702703	0.0134200216462637\\
-7.00900900900901	0.0136165685792884\\
-6.99099099099099	0.0138155946530897\\
-6.97297297297297	0.0140171246443345\\
-6.95495495495495	0.0142211834790229\\
-6.93693693693694	0.0144277962319408\\
-6.91891891891892	0.0146369881260844\\
-6.9009009009009	0.0148487845320604\\
-6.88288288288288	0.0150632109674576\\
-6.86486486486486	0.0152802930961927\\
-6.84684684684685	0.0155000567278292\\
-6.82882882882883	0.015722527816868\\
-6.81081081081081	0.0159477324620111\\
-6.79279279279279	0.016175696905398\\
-6.77477477477477	0.0164064475318125\\
-6.75675675675676	0.0166400108678634\\
-6.73873873873874	0.016876413581135\\
-6.72072072072072	0.01711568247931\\
-6.7027027027027	0.0173578445092631\\
-6.68468468468468	0.017602926756126\\
-6.66666666666667	0.0178509564423223\\
-6.64864864864865	0.0181019609265744\\
-6.63063063063063	0.0183559677028791\\
-6.61261261261261	0.0186130043994548\\
-6.59459459459459	0.0188730987776576\\
-6.57657657657658	0.0191362787308684\\
-6.55855855855856	0.0194025722833481\\
-6.54054054054054	0.019672007589064\\
-6.52252252252252	0.0199446129304839\\
-6.5045045045045	0.0202204167173404\\
-6.48648648648649	0.0204994474853637\\
-6.46846846846847	0.0207817338949833\\
-6.45045045045045	0.0210673047299982\\
-6.43243243243243	0.021356188896216\\
-6.41441441441441	0.0216484154200599\\
-6.3963963963964	0.021944013447144\\
-6.37837837837838	0.0222430122408167\\
-6.36036036036036	0.0225454411806724\\
-6.34234234234234	0.0228513297610304\\
-6.32432432432432	0.0231607075893817\\
-6.30630630630631	0.0234736043848038\\
-6.28828828828829	0.023790049976342\\
-6.27027027027027	0.0241100743013592\\
-6.25225225225225	0.0244337074038521\\
-6.23423423423423	0.0247609794327344\\
-6.21621621621622	0.0250919206400884\\
-6.1981981981982	0.0254265613793814\\
-6.18018018018018	0.0257649321036512\\
-6.16216216216216	0.0261070633636569\\
-6.14414414414414	0.0264529858059967\\
-6.12612612612613	0.0268027301711932\\
-6.10810810810811	0.0271563272917439\\
-6.09009009009009	0.0275138080901397\\
-6.07207207207207	0.0278752035768488\\
-6.05405405405405	0.0282405448482675\\
-6.03603603603604	0.0286098630846379\\
-6.01801801801802	0.0289831895479306\\
-6	0.0293605555796955\\
-5.98198198198198	0.0297419925988777\\
-5.96396396396396	0.0301275320996003\\
-5.94594594594595	0.0305172056489133\\
-5.92792792792793	0.0309110448845091\\
-5.90990990990991	0.0313090815124041\\
-5.89189189189189	0.0317113473045867\\
-5.87387387387387	0.0321178740966318\\
-5.85585585585586	0.0325286937852818\\
-5.83783783783784	0.0329438383259935\\
-5.81981981981982	0.0333633397304525\\
-5.8018018018018	0.033787230064053\\
-5.78378378378378	0.0342155414433449\\
-5.76576576576577	0.0346483060334477\\
-5.74774774774775	0.0350855560454304\\
-5.72972972972973	0.0355273237336589\\
-5.71171171171171	0.03597364139311\\
-5.69369369369369	0.0364245413566522\\
-5.67567567567568	0.0368800559922944\\
-5.65765765765766	0.0373402177004003\\
-5.63963963963964	0.0378050589108722\\
-5.62162162162162	0.0382746120802998\\
-5.6036036036036	0.0387489096890787\\
-5.58558558558559	0.0392279842384954\\
-5.56756756756757	0.0397118682477805\\
-5.54954954954955	0.0402005942511296\\
-5.53153153153153	0.0406941947946933\\
-5.51351351351351	0.0411927024335337\\
-5.4954954954955	0.0416961497285517\\
-5.47747747747748	0.0422045692433813\\
-5.45945945945946	0.0427179935412531\\
-5.44144144144144	0.0432364551818273\\
-5.42342342342342	0.0437599867179959\\
-5.40540540540541	0.0442886206926541\\
-5.38738738738739	0.0448223896354415\\
-5.36936936936937	0.0453613260594534\\
-5.35135135135135	0.0459054624579224\\
-5.33333333333333	0.0464548313008701\\
-5.31531531531532	0.0470094650317298\\
-5.2972972972973	0.0475693960639399\\
-5.27927927927928	0.0481346567775091\\
-5.26126126126126	0.0487052795155519\\
-5.24324324324324	0.0492812965807976\\
-5.22522522522523	0.0498627402320702\\
-5.20720720720721	0.0504496426807408\\
-5.18918918918919	0.0510420360871532\\
-5.17117117117117	0.0516399525570224\\
-5.15315315315315	0.0522434241378062\\
-5.13513513513514	0.0528524828150507\\
-5.11711711711712	0.0534671605087102\\
-5.0990990990991	0.0540874890694407\\
-5.08108108108108	0.0547135002748691\\
-5.06306306306306	0.0553452258258367\\
-5.04504504504505	0.0559826973426194\\
-5.02702702702703	0.056625946361122\\
-5.00900900900901	0.057275004329051\\
-4.99099099099099	0.0579299026020624\\
-4.97297297297297	0.0585906724398872\\
-4.95495495495495	0.059257345002435\\
-4.93693693693694	0.0599299513458746\\
-4.91891891891892	0.060608522418694\\
-4.9009009009009	0.0612930890577383\\
-4.88288288288288	0.0619836819842277\\
-4.86486486486486	0.0626803317997552\\
-4.84684684684685	0.063383068982264\\
-4.82882882882883	0.0640919238820058\\
-4.81081081081081	0.0648069267174814\\
-4.79279279279279	0.0655281075713611\\
-4.77477477477477	0.0662554963863888\\
-4.75675675675676	0.0669891229612683\\
-4.73873873873874	0.0677290169465327\\
-4.72072072072072	0.0684752078403983\\
-4.7027027027027	0.0692277249846011\\
-4.68468468468468	0.0699865975602204\\
-4.66666666666667	0.0707518545834861\\
-4.64864864864865	0.0715235249015723\\
-4.63063063063063	0.072301637188378\\
-4.61261261261261	0.0730862199402939\\
-4.59459459459459	0.0738773014719577\\
-4.57657657657658	0.0746749099119975\\
-4.55855855855856	0.075479073198763\\
-4.54054054054054	0.0762898190760477\\
-4.52252252252252	0.0771071750887999\\
-4.5045045045045	0.0779311685788249\\
-4.48648648648649	0.0787618266804781\\
-4.46846846846847	0.0795991763163504\\
-4.45045045045045	0.0804432441929459\\
-4.43243243243243	0.0812940567963525\\
-4.41441441441441	0.0821516403879067\\
-4.3963963963964	0.0830160209998523\\
-4.37837837837838	0.083887224430995\\
-4.36036036036036	0.0847652762423523\\
-4.34234234234234	0.0856502017527993\\
-4.32432432432432	0.0865420260347126\\
-4.30630630630631	0.0874407739096116\\
-4.28828828828829	0.0883464699437978\\
-4.27027027027027	0.0892591384439944\\
-4.25225225225225	0.090178803452984\\
-4.23423423423423	0.0911054887452489\\
-4.21621621621622	0.0920392178226114\\
-4.1981981981982	0.0929800139098762\\
-4.18018018018018	0.0939278999504761\\
-4.16216216216216	0.0948828986021203\\
-4.14414414414414	0.0958450322324481\\
-4.12612612612613	0.096814322914686\\
-4.10810810810811	0.0977907924233114\\
-4.09009009009009	0.098774462229723\\
-4.07207207207207	0.0997653534979173\\
-4.05405405405405	0.100763487080174\\
-4.03603603603604	0.101768883512748\\
-4.01801801801802	0.102781563011575\\
-4	0.103801545467984\\
-3.98198198198198	0.104828850444418\\
-3.96396396396396	0.105863497170174\\
-3.94594594594595	0.106905504537151\\
-3.92792792792793	0.107954891095609\\
-3.90990990990991	0.109011675049945\\
-3.89189189189189	0.110075874254485\\
-3.87387387387387	0.11114750620929\\
-3.85585585585586	0.112226588055977\\
-3.83783783783784	0.113313136573558\\
-3.81981981981982	0.1144071681743\\
-3.8018018018018	0.1155086988996\\
-3.78378378378378	0.11661774441588\\
-3.76576576576577	0.117734320010505\\
-3.74774774774775	0.118858440587718\\
-3.72972972972973	0.119990120664602\\
-3.71171171171171	0.121129374367062\\
-3.69369369369369	0.122276215425827\\
-3.67567567567568	0.123430657172484\\
-3.65765765765766	0.124592712535533\\
-3.63963963963964	0.125762394036465\\
-3.62162162162162	0.126939713785875\\
-3.6036036036036	0.128124683479591\\
-3.58558558558559	0.129317314394844\\
-3.56756756756757	0.130517617386461\\
-3.54954954954955	0.131725602883081\\
-3.53153153153153	0.132941280883419\\
-3.51351351351351	0.134164660952544\\
-3.4954954954955	0.135395752218201\\
-3.47747747747748	0.136634563367161\\
-3.45945945945946	0.137881102641606\\
-3.44144144144144	0.139135377835549\\
-3.42342342342342	0.140397396291292\\
-3.40540540540541	0.141667164895915\\
-3.38738738738739	0.142944690077806\\
-3.36936936936937	0.14422997780323\\
-3.35135135135135	0.145523033572933\\
-3.33333333333333	0.146823862418791\\
-3.31531531531532	0.148132468900489\\
-3.2972972972973	0.149448857102257\\
-3.27927927927928	0.150773030629633\\
-3.26126126126126	0.152104992606276\\
-3.24324324324324	0.153444745670821\\
-3.22522522522523	0.154792291973779\\
-3.20720720720721	0.156147633174482\\
-3.18918918918919	0.15751077043807\\
-3.17117117117117	0.158881704432531\\
-3.15315315315315	0.160260435325782\\
-3.13513513513514	0.1616469627828\\
-3.11711711711712	0.163041285962808\\
-3.0990990990991	0.164443403516495\\
-3.08108108108108	0.165853313583307\\
-3.06306306306306	0.167271013788768\\
-3.04504504504505	0.168696501241867\\
-3.02702702702703	0.170129772532495\\
-3.00900900900901	0.171570823728926\\
-2.99099099099099	0.173019650375364\\
-2.97297297297297	0.174476247489537\\
-2.95495495495495	0.175940609560346\\
-2.93693693693694	0.177412730545575\\
-2.91891891891892	0.178892603869649\\
-2.9009009009009	0.18038022242146\\
-2.88288288288288	0.181875578552239\\
-2.86486486486486	0.183378664073497\\
-2.84684684684685	0.184889470255015\\
-2.82882882882883	0.186407987822905\\
-2.81081081081081	0.187934206957719\\
-2.79279279279279	0.189468117292628\\
-2.77477477477477	0.191009707911658\\
-2.75675675675676	0.192558967347992\\
-2.73873873873874	0.194115883582328\\
-2.72072072072072	0.195680444041308\\
-2.7027027027027	0.197252635596006\\
-2.68468468468468	0.19883244456048\\
-2.66666666666667	0.200419856690391\\
-2.64864864864865	0.202014857181689\\
-2.63063063063063	0.20361743066936\\
-2.61261261261261	0.205227561226245\\
-2.59459459459459	0.20684523236192\\
-2.57657657657658	0.20847042702165\\
-2.55855855855856	0.210103127585405\\
-2.54054054054054	0.211743315866951\\
-2.52252252252252	0.213390973113002\\
-2.5045045045045	0.215046080002449\\
-2.48648648648649	0.216708616645652\\
-2.46846846846847	0.218378562583811\\
-2.45045045045045	0.220055896788396\\
-2.43243243243243	0.221740597660658\\
-2.41441441441441	0.223432643031209\\
-2.3963963963964	0.22513201015967\\
-2.37837837837838	0.226838675734396\\
-2.36036036036036	0.228552615872268\\
-2.34234234234234	0.230273806118569\\
-2.32432432432432	0.232002221446917\\
-2.30630630630631	0.233737836259286\\
-2.28828828828829	0.235480624386095\\
-2.27027027027027	0.237230559086368\\
-2.25225225225225	0.238987613047976\\
-2.23423423423423	0.240751758387947\\
-2.21621621621622	0.242522966652854\\
-2.1981981981982	0.24430120881928\\
-2.18018018018018	0.246086455294355\\
-2.16216216216216	0.247878675916368\\
-2.14414414414414	0.24967783995546\\
-2.12612612612613	0.25148391611439\\
-2.10810810810811	0.253296872529374\\
-2.09009009009009	0.255116676771008\\
-2.07207207207207	0.256943295845261\\
-2.05405405405405	0.258776696194544\\
-2.03603603603604	0.260616843698864\\
-2.01801801801802	0.262463703677047\\
-2	0.26431724088804\\
-1.98198198198198	0.266177419532292\\
-1.96396396396396	0.268044203253207\\
-1.94594594594595	0.269917555138683\\
-1.92792792792793	0.271797437722721\\
-1.90990990990991	0.273683812987111\\
-1.89189189189189	0.275576642363199\\
-1.87387387387387	0.27747588673373\\
-1.85585585585586	0.279381506434768\\
-1.83783783783784	0.281293461257689\\
-1.81981981981982	0.28321171045126\\
-1.8018018018018	0.285136212723786\\
-1.78378378378378	0.28706692624534\\
-1.76576576576577	0.289003808650065\\
-1.74774774774775	0.290946817038556\\
-1.72972972972973	0.292895907980316\\
-1.71171171171171	0.294851037516293\\
-1.69369369369369	0.296812161161486\\
-1.67567567567568	0.298779233907636\\
-1.65765765765766	0.30075221022598\\
-1.63963963963964	0.302731044070095\\
-1.62162162162162	0.304715688878808\\
-1.6036036036036	0.306706097579184\\
-1.58558558558559	0.308702222589586\\
-1.56756756756757	0.310704015822817\\
-1.54954954954955	0.312711428689325\\
-1.53153153153153	0.314724412100496\\
-1.51351351351351	0.316742916472006\\
-1.4954954954955	0.318766891727254\\
-1.47747747747748	0.320796287300872\\
-1.45945945945946	0.322831052142299\\
-1.44144144144144	0.324871134719429\\
-1.42342342342342	0.326916483022338\\
-1.40540540540541	0.328967044567071\\
-1.38738738738739	0.331022766399513\\
-1.36936936936937	0.333083595099317\\
-1.35135135135135	0.335149476783912\\
-1.33333333333333	0.337220357112578\\
-1.31531531531532	0.339296181290586\\
-1.2972972972973	0.341376894073413\\
-1.27927927927928	0.343462439771022\\
-1.26126126126126	0.345552762252207\\
-1.24324324324324	0.347647804949011\\
-1.22522522522523	0.349747510861204\\
-1.20720720720721	0.351851822560833\\
-1.18918918918919	0.353960682196831\\
-1.17117117117117	0.356074031499698\\
-1.15315315315315	0.358191811786239\\
-1.13513513513514	0.360313963964368\\
-1.11711711711712	0.362440428537981\\
-1.0990990990991	0.364571145611878\\
-1.08108108108108	0.366706054896759\\
-1.06306306306306	0.368845095714277\\
-1.04504504504505	0.370988207002143\\
-1.02702702702703	0.373135327319307\\
-1.00900900900901	0.375286394851182\\
-0.990990990990991	0.377441347414933\\
-0.972972972972973	0.379600122464824\\
-0.954954954954955	0.381762657097618\\
-0.936936936936937	0.383928888058036\\
-0.918918918918919	0.386098751744272\\
-0.900900900900901	0.388272184213552\\
-0.882882882882883	0.390449121187761\\
-0.864864864864865	0.392629498059115\\
-0.846846846846847	0.394813249895879\\
-0.828828828828829	0.39700031144815\\
-0.810810810810811	0.399190617153676\\
-0.792792792792793	0.401384101143734\\
-0.774774774774775	0.403580697249048\\
-0.756756756756757	0.405780339005763\\
-0.738738738738738	0.407982959661458\\
-0.72072072072072	0.410188492181209\\
-0.702702702702703	0.412396869253697\\
-0.684684684684685	0.414608023297353\\
-0.666666666666667	0.416821886466557\\
-0.648648648648648	0.419038390657868\\
-0.63063063063063	0.421257467516301\\
-0.612612612612613	0.423479048441644\\
-0.594594594594595	0.425703064594809\\
-0.576576576576577	0.42792944690423\\
-0.558558558558558	0.430158126072284\\
-0.54054054054054	0.432389032581765\\
-0.522522522522523	0.43462209670238\\
-0.504504504504505	0.436857248497283\\
-0.486486486486487	0.439094417829644\\
-0.468468468468468	0.441333534369248\\
-0.45045045045045	0.443574527599122\\
-0.432432432432432	0.445817326822201\\
-0.414414414414415	0.448061861168011\\
-0.396396396396397	0.450308059599389\\
-0.378378378378378	0.452555850919225\\
-0.36036036036036	0.45480516377723\\
-0.342342342342342	0.457055926676731\\
-0.324324324324325	0.459308067981487\\
-0.306306306306307	0.461561515922529\\
-0.288288288288288	0.463816198605019\\
-0.27027027027027	0.466072044015132\\
-0.252252252252252	0.468328980026954\\
-0.234234234234235	0.470586934409406\\
-0.216216216216216	0.472845834833168\\
-0.198198198198198	0.475105608877641\\
-0.18018018018018	0.477366184037903\\
-0.162162162162162	0.479627487731696\\
-0.144144144144144	0.481889447306408\\
-0.126126126126126	0.484151990046084\\
-0.108108108108108	0.486415043178435\\
-0.0900900900900901	0.48867853388186\\
-0.0720720720720722	0.490942389292476\\
-0.0540540540540544	0.493206536511153\\
-0.0360360360360357	0.495470902610561\\
-0.0180180180180178	0.49773541464221\\
0	0.499999999643507\\
0.0180180180180187	0.502264584644805\\
0.0360360360360357	0.504529096676454\\
0.0540540540540544	0.506793462775861\\
0.0720720720720713	0.509057609994539\\
0.0900900900900901	0.511321465405154\\
0.108108108108109	0.513584956108579\\
0.126126126126126	0.51584800924093\\
0.144144144144144	0.518110551980607\\
0.162162162162161	0.520372511555319\\
0.18018018018018	0.522633815249111\\
0.198198198198199	0.524894390409374\\
0.216216216216216	0.527154164453846\\
0.234234234234235	0.529413064877609\\
0.252252252252251	0.53167101926006\\
0.27027027027027	0.533927955271883\\
0.288288288288289	0.536183800681995\\
0.306306306306306	0.538438483364485\\
0.324324324324325	0.540691931305527\\
0.342342342342342	0.542944072610283\\
0.36036036036036	0.545194835509784\\
0.378378378378379	0.547444148367789\\
0.396396396396396	0.549691939687625\\
0.414414414414415	0.551938138119003\\
0.432432432432432	0.554182672464813\\
0.45045045045045	0.556425471687892\\
0.468468468468469	0.558666464917767\\
0.486486486486486	0.56090558145737\\
0.504504504504505	0.563142750789731\\
0.522522522522523	0.565377902584634\\
0.54054054054054	0.567610966705249\\
0.558558558558559	0.569841873214731\\
0.576576576576576	0.572070552382785\\
0.594594594594595	0.574296934692205\\
0.612612612612613	0.57652095084537\\
0.63063063063063	0.578742531770713\\
0.648648648648649	0.580961608629146\\
0.666666666666666	0.583178112820457\\
0.684684684684685	0.585391975989661\\
0.702702702702704	0.587603130033318\\
0.72072072072072	0.589811507105805\\
0.738738738738739	0.592017039625557\\
0.756756756756756	0.594219660281252\\
0.774774774774775	0.596419302037966\\
0.792792792792794	0.59861589814328\\
0.810810810810811	0.600809382133338\\
0.828828828828829	0.602999687838864\\
0.846846846846846	0.605186749391135\\
0.864864864864865	0.6073705012279\\
0.882882882882884	0.609550878099253\\
0.900900900900901	0.611727815073463\\
0.918918918918919	0.613901247542743\\
0.936936936936936	0.616071111228978\\
0.954954954954955	0.618237342189397\\
0.972972972972974	0.620399876822191\\
0.990990990990991	0.622558651872081\\
1.00900900900901	0.624713604435832\\
1.02702702702703	0.626864671967707\\
1.04504504504505	0.629011792284871\\
1.06306306306306	0.631154903572738\\
1.08108108108108	0.633293944390255\\
1.0990990990991	0.635428853675137\\
1.11711711711712	0.637559570749033\\
1.13513513513514	0.639686035322646\\
1.15315315315315	0.641808187500776\\
1.17117117117117	0.643925967787316\\
1.18918918918919	0.646039317090183\\
1.20720720720721	0.648148176726181\\
1.22522522522523	0.65025248842581\\
1.24324324324324	0.652352194338003\\
1.26126126126126	0.654447237034807\\
1.27927927927928	0.656537559515992\\
1.2972972972973	0.658623105213601\\
1.31531531531532	0.660703817996428\\
1.33333333333333	0.662779642174436\\
1.35135135135135	0.664850522503102\\
1.36936936936937	0.666916404187697\\
1.38738738738739	0.668977232887501\\
1.40540540540541	0.671032954719943\\
1.42342342342342	0.673083516264677\\
1.44144144144144	0.675128864567585\\
1.45945945945946	0.677168947144715\\
1.47747747747748	0.679203711986142\\
1.4954954954955	0.68123310755976\\
1.51351351351351	0.683257082815009\\
1.53153153153153	0.685275587186518\\
1.54954954954955	0.687288570597689\\
1.56756756756757	0.689295983464198\\
1.58558558558559	0.691297776697428\\
1.6036036036036	0.69329390170783\\
1.62162162162162	0.695284310408206\\
1.63963963963964	0.697268955216919\\
1.65765765765766	0.699247789061034\\
1.67567567567568	0.701220765379378\\
1.69369369369369	0.703187838125528\\
1.71171171171171	0.705148961770722\\
1.72972972972973	0.707104091306699\\
1.74774774774775	0.709053182248459\\
1.76576576576577	0.710996190636949\\
1.78378378378378	0.712933073041674\\
1.8018018018018	0.714863786563228\\
1.81981981981982	0.716788288835755\\
1.83783783783784	0.718706538029326\\
1.85585585585586	0.720618492852247\\
1.87387387387387	0.722524112553284\\
1.89189189189189	0.724423356923815\\
1.90990990990991	0.726316186299903\\
1.92792792792793	0.728202561564293\\
1.94594594594595	0.730082444148331\\
1.96396396396396	0.731955796033808\\
1.98198198198198	0.733822579754723\\
2	0.735682758398974\\
2.01801801801802	0.737536295609967\\
2.03603603603604	0.73938315558815\\
2.05405405405405	0.741223303092471\\
2.07207207207207	0.743056703441754\\
2.09009009009009	0.744883322516006\\
2.10810810810811	0.74670312675764\\
2.12612612612613	0.748516083172625\\
2.14414414414414	0.750322159331554\\
2.16216216216216	0.752121323370647\\
2.18018018018018	0.753913543992659\\
2.1981981981982	0.755698790467734\\
2.21621621621622	0.75747703263416\\
2.23423423423423	0.759248240899068\\
2.25225225225225	0.761012386239038\\
2.27027027027027	0.762769440200646\\
2.28828828828829	0.764519374900919\\
2.30630630630631	0.766262163027728\\
2.32432432432432	0.767997777840097\\
2.34234234234234	0.769726193168445\\
2.36036036036036	0.771447383414746\\
2.37837837837838	0.773161323552619\\
2.3963963963964	0.774867989127344\\
2.41441441441441	0.776567356255805\\
2.43243243243243	0.778259401626356\\
2.45045045045045	0.779944102498619\\
2.46846846846847	0.781621436703203\\
2.48648648648649	0.783291382641362\\
2.5045045045045	0.784953919284566\\
2.52252252252252	0.786609026174012\\
2.54054054054054	0.788256683420063\\
2.55855855855856	0.789896871701609\\
2.57657657657658	0.791529572265364\\
2.59459459459459	0.793154766925094\\
2.61261261261261	0.794772438060769\\
2.63063063063063	0.796382568617654\\
2.64864864864865	0.797985142105325\\
2.66666666666667	0.799580142596623\\
2.68468468468468	0.801167554726535\\
2.7027027027027	0.802747363691008\\
2.72072072072072	0.804319555245706\\
2.73873873873874	0.805884115704686\\
2.75675675675676	0.807441031939023\\
2.77477477477477	0.808990291375356\\
2.79279279279279	0.810531881994387\\
2.81081081081081	0.812065792329295\\
2.82882882882883	0.813592011464109\\
2.84684684684685	0.815110529031999\\
2.86486486486486	0.816621335213518\\
2.88288288288288	0.818124420734776\\
2.9009009009009	0.819619776865555\\
2.91891891891892	0.821107395417365\\
2.93693693693694	0.82258726874144\\
2.95495495495495	0.824059389726668\\
2.97297297297297	0.825523751797477\\
2.99099099099099	0.82698034891165\\
3.00900900900901	0.828429175558088\\
3.02702702702703	0.829870226754519\\
3.04504504504505	0.831303498045147\\
3.06306306306306	0.832728985498247\\
3.08108108108108	0.834146685703707\\
3.0990990990991	0.835556595770519\\
3.11711711711712	0.836958713324207\\
3.13513513513514	0.838353036504214\\
3.15315315315315	0.839739563961233\\
3.17117117117117	0.841118294854483\\
3.18918918918919	0.842489228848944\\
3.20720720720721	0.843852366112532\\
3.22522522522523	0.845207707313235\\
3.24324324324324	0.846555253616194\\
3.26126126126126	0.847895006680739\\
3.27927927927928	0.849226968657381\\
3.2972972972973	0.850551142184757\\
3.31531531531532	0.851867530386525\\
3.33333333333333	0.853176136868224\\
3.35135135135135	0.854476965714081\\
3.36936936936937	0.855770021483785\\
3.38738738738739	0.857055309209208\\
3.40540540540541	0.8583328343911\\
3.42342342342342	0.859602602995722\\
3.44144144144144	0.860864621451465\\
3.45945945945946	0.862118896645409\\
3.47747747747748	0.863365435919853\\
3.4954954954955	0.864604247068813\\
3.51351351351351	0.865835338334471\\
3.53153153153153	0.867058718403595\\
3.54954954954955	0.868274396403933\\
3.56756756756757	0.869482381900554\\
3.58558558558559	0.87068268489217\\
3.6036036036036	0.871875315807424\\
3.62162162162162	0.87306028550114\\
3.63963963963964	0.874237605250549\\
3.65765765765766	0.875407286751481\\
3.67567567567568	0.87656934211453\\
3.69369369369369	0.877723783861187\\
3.71171171171171	0.878870624919952\\
3.72972972972973	0.880009878622412\\
3.74774774774775	0.881141558699296\\
3.76576576576577	0.88226567927651\\
3.78378378378378	0.883382254871134\\
3.8018018018018	0.884491300387414\\
3.81981981981982	0.885592831112714\\
3.83783783783784	0.886686862713456\\
3.85585585585586	0.887773411231037\\
3.87387387387387	0.888852493077724\\
3.89189189189189	0.889924125032529\\
3.90990990990991	0.890988324237069\\
3.92792792792793	0.892045108191405\\
3.94594594594595	0.893094494749863\\
3.96396396396396	0.89413650211684\\
3.98198198198198	0.895171148842597\\
4	0.896198453819031\\
4.01801801801802	0.897218436275439\\
4.03603603603604	0.898231115774267\\
4.05405405405405	0.899236512206841\\
4.07207207207207	0.900234645789097\\
4.09009009009009	0.901225537057291\\
4.10810810810811	0.902209206863703\\
4.12612612612613	0.903185676372328\\
4.14414414414414	0.904154967054566\\
4.16216216216216	0.905117100684894\\
4.18018018018018	0.906072099336538\\
4.1981981981982	0.907019985377138\\
4.21621621621622	0.907960781464403\\
4.23423423423423	0.908894510541765\\
4.25225225225225	0.90982119583403\\
4.27027027027027	0.91074086084302\\
4.28828828828829	0.911653529343216\\
4.30630630630631	0.912559225377403\\
4.32432432432432	0.913457973252302\\
4.34234234234234	0.914349797534215\\
4.36036036036036	0.915234723044662\\
4.37837837837838	0.916112774856019\\
4.3963963963964	0.916983978287162\\
4.41441441441441	0.917848358899108\\
4.43243243243243	0.918705942490662\\
4.45045045045045	0.919556755094068\\
4.46846846846847	0.920400822970664\\
4.48648648648649	0.921238172606536\\
4.5045045045045	0.922068830708189\\
4.52252252252252	0.922892824198215\\
4.54054054054054	0.923710180210967\\
4.55855855855856	0.924520926088251\\
4.57657657657658	0.925325089375017\\
4.59459459459459	0.926122697815057\\
4.61261261261261	0.92691377934672\\
4.63063063063063	0.927698362098636\\
4.64864864864865	0.928476474385442\\
4.66666666666667	0.929248144703528\\
4.68468468468468	0.930013401726794\\
4.7027027027027	0.930772274302413\\
4.72072072072072	0.931524791446616\\
4.73873873873874	0.932270982340482\\
4.75675675675676	0.933010876325746\\
4.77477477477477	0.933744502900626\\
4.79279279279279	0.934471891715653\\
4.81081081081081	0.935193072569533\\
4.82882882882883	0.935908075405009\\
4.84684684684685	0.93661693030475\\
4.86486486486486	0.937319667487259\\
4.88288288288288	0.938016317302787\\
4.9009009009009	0.938706910229276\\
4.91891891891892	0.93939147686832\\
4.93693693693694	0.94007004794114\\
4.95495495495495	0.940742654284579\\
4.97297297297297	0.941409326847127\\
4.99099099099099	0.942070096684952\\
5.00900900900901	0.942724994957963\\
5.02702702702703	0.943374052925892\\
5.04504504504505	0.944017301944395\\
5.06306306306306	0.944654773461178\\
5.08108108108108	0.945286499012145\\
5.0990990990991	0.945912510217574\\
5.11711711711712	0.946532838778304\\
5.13513513513514	0.947147516471964\\
5.15315315315315	0.947756575149208\\
5.17117117117117	0.948360046729992\\
5.18918918918919	0.948957963199861\\
5.20720720720721	0.949550356606274\\
5.22522522522523	0.950137259054944\\
5.24324324324324	0.950718702706217\\
5.26126126126126	0.951294719771463\\
5.27927927927928	0.951865342509505\\
5.2972972972973	0.952430603223074\\
5.31531531531532	0.952990534255285\\
5.33333333333333	0.953545167986144\\
5.35135135135135	0.954094536829092\\
5.36936936936937	0.954638673227561\\
5.38738738738739	0.955177609651573\\
5.40540540540541	0.95571137859436\\
5.42342342342342	0.956240012569018\\
5.44144144144144	0.956763544105187\\
5.45945945945946	0.957282005745761\\
5.47747747747748	0.957795430043633\\
5.4954954954955	0.958303849558463\\
5.51351351351351	0.958807296853481\\
5.53153153153153	0.959305804492321\\
5.54954954954955	0.959799405035885\\
5.56756756756757	0.960288131039234\\
5.58558558558559	0.960772015048519\\
5.6036036036036	0.961251089597936\\
5.62162162162162	0.961725387206715\\
5.63963963963964	0.962194940376142\\
5.65765765765766	0.962659781586614\\
5.67567567567568	0.96311994329472\\
5.69369369369369	0.963575457930362\\
5.71171171171171	0.964026357893904\\
5.72972972972973	0.964472675553355\\
5.74774774774775	0.964914443241584\\
5.76576576576577	0.965351693253567\\
5.78378378378378	0.965784457843669\\
5.8018018018018	0.966212769222961\\
5.81981981981982	0.966636659556562\\
5.83783783783784	0.967056160961021\\
5.85585585585586	0.967471305501733\\
5.87387387387387	0.967882125190383\\
5.89189189189189	0.968288651982428\\
5.90990990990991	0.96869091777461\\
5.92792792792793	0.969088954402505\\
5.94594594594595	0.969482793638101\\
5.96396396396396	0.969872467187414\\
5.98198198198198	0.970258006688137\\
6	0.970639443707319\\
6.01801801801802	0.971016809739084\\
6.03603603603604	0.971390136202376\\
6.05405405405405	0.971759454438747\\
6.07207207207207	0.972124795710166\\
6.09009009009009	0.972486191196875\\
6.10810810810811	0.97284367199527\\
6.12612612612613	0.973197269115821\\
6.14414414414414	0.973547013481018\\
6.16216216216216	0.973892935923357\\
6.18018018018018	0.974235067183363\\
6.1981981981982	0.974573437907633\\
6.21621621621622	0.974908078646926\\
6.23423423423423	0.97523901985428\\
6.25225225225225	0.975566291883162\\
6.27027027027027	0.975889924985655\\
6.28828828828829	0.976209949310672\\
6.30630630630631	0.976526394902211\\
6.32432432432432	0.976839291697633\\
6.34234234234234	0.977148669525984\\
6.36036036036036	0.977454558106342\\
6.37837837837838	0.977756987046198\\
6.3963963963964	0.97805598583987\\
6.41441441441441	0.978351583866954\\
6.43243243243243	0.978643810390798\\
6.45045045045045	0.978932694557016\\
6.46846846846847	0.979218265392031\\
6.48648648648649	0.979500551801651\\
6.5045045045045	0.979779582569674\\
6.52252252252252	0.98005538635653\\
6.54054054054054	0.98032799169795\\
6.55855855855856	0.980597427003666\\
6.57657657657658	0.980863720556146\\
6.59459459459459	0.981126900509357\\
6.61261261261261	0.98138699488756\\
6.63063063063063	0.981644031584135\\
6.64864864864865	0.98189803836044\\
6.66666666666667	0.982149042844692\\
6.68468468468468	0.982397072530888\\
6.7027027027027	0.982642154777751\\
6.72072072072072	0.982884316807704\\
6.73873873873874	0.983123585705879\\
6.75675675675676	0.983359988419151\\
6.77477477477477	0.983593551755202\\
6.79279279279279	0.983824302381616\\
6.81081081081081	0.984052266825003\\
6.82882882882883	0.984277471470146\\
6.84684684684685	0.984499942559185\\
6.86486486486486	0.984719706190822\\
6.88288288288288	0.984936788319557\\
6.9009009009009	0.985151214754954\\
6.91891891891892	0.98536301116093\\
6.93693693693694	0.985572203055074\\
6.95495495495495	0.985778815807991\\
6.97297297297297	0.98598287464268\\
6.99099099099099	0.986184404633925\\
7.00900900900901	0.986383430707726\\
7.02702702702703	0.986579977640751\\
7.04504504504505	0.986774070059809\\
7.06306306306306	0.98696573244136\\
7.08108108108108	0.987154989111036\\
7.0990990990991	0.987341864243203\\
7.11711711711712	0.987526381860533\\
7.13513513513514	0.987708565833613\\
7.15315315315315	0.98788843988057\\
7.17117117117117	0.988066027566724\\
7.18918918918919	0.988241352304265\\
7.20720720720721	0.988414437351954\\
7.22522522522523	0.988585305814845\\
7.24324324324324	0.988753980644031\\
7.26126126126126	0.988920484636416\\
7.27927927927928	0.989084840434507\\
7.2972972972973	0.989247070526227\\
7.31531531531532	0.989407197244756\\
7.33333333333333	0.989565242768384\\
7.35135135135135	0.989721229120399\\
7.36936936936937	0.989875178168981\\
7.38738738738739	0.990027111627133\\
7.40540540540541	0.990177051052619\\
7.42342342342342	0.990325017847934\\
7.44144144144144	0.990471033260283\\
7.45945945945946	0.990615118381591\\
7.47747747747748	0.990757294148526\\
7.4954954954955	0.990897581342545\\
7.51351351351351	0.991036000589954\\
7.53153153153153	0.991172572361994\\
7.54954954954955	0.991307316974943\\
7.56756756756757	0.991440254590234\\
7.58558558558559	0.991571405214593\\
7.6036036036036	0.991700788700199\\
7.62162162162162	0.991828424744851\\
7.63963963963964	0.991954332892166\\
7.65765765765766	0.992078532531784\\
7.67567567567568	0.992201042899595\\
7.69369369369369	0.992321883077979\\
7.71171171171171	0.992441071996064\\
7.72972972972973	0.992558628430006\\
7.74774774774775	0.992674571003269\\
7.76576576576577	0.992788918186943\\
7.78378378378378	0.992901688300053\\
7.8018018018018	0.993012899509906\\
7.81981981981982	0.993122569832435\\
7.83783783783784	0.993230717132564\\
7.85585585585586	0.993337359124595\\
7.87387387387387	0.993442513372592\\
7.89189189189189	0.993546197290798\\
7.90990990990991	0.993648428144048\\
7.92792792792793	0.993749223048208\\
7.94594594594595	0.99384859897062\\
7.96396396396396	0.993946572730564\\
7.98198198198198	0.994043160999728\\
8	0.994138380302693\\
8.01801801801802	0.994232247017431\\
8.03603603603604	0.994324777375814\\
8.05405405405405	0.994415987464131\\
8.07207207207207	0.994505893223619\\
8.09009009009009	0.994594510451009\\
8.10810810810811	0.994681854799074\\
8.12612612612613	0.994767941777196\\
8.14414414414414	0.994852786751937\\
8.16216216216216	0.994936404947624\\
8.18018018018018	0.995018811446939\\
8.1981981981982	0.995100021191529\\
8.21621621621622	0.995180048982611\\
8.23423423423423	0.995258909481594\\
8.25225225225225	0.995336617210716\\
8.27027027027027	0.995413186553673\\
8.28828828828829	0.995488631756273\\
8.30630630630631	0.995562966927088\\
8.32432432432432	0.99563620603812\\
8.34234234234234	0.995708362925468\\
8.36036036036036	0.99577945129001\\
8.37837837837838	0.995849484698085\\
8.3963963963964	0.995918476582191\\
8.41441441441441	0.995986440241678\\
8.43243243243243	0.99605338884346\\
8.45045045045045	0.996119335422725\\
8.46846846846847	0.996184292883654\\
8.48648648648649	0.996248274000144\\
8.5045045045045	0.996311291416543\\
8.52252252252252	0.996373357648383\\
8.54054054054054	0.996434485083121\\
8.55855855855856	0.99649468598089\\
8.57657657657658	0.996553972475244\\
8.59459459459459	0.996612356573922\\
8.61261261261261	0.996669850159603\\
8.63063063063063	0.996726464990676\\
8.64864864864865	0.996782212702006\\
8.66666666666667	0.996837104805713\\
8.68468468468468	0.996891152691943\\
8.7027027027027	0.996944367629654\\
8.72072072072072	0.996996760767402\\
8.73873873873874	0.997048343134124\\
8.75675675675676	0.997099125639932\\
8.77477477477477	0.99714911907691\\
8.79279279279279	0.997198334119905\\
8.81081081081081	0.997246781327332\\
8.82882882882883	0.997294471141974\\
8.84684684684685	0.997341413891784\\
8.86486486486486	0.997387619790696\\
8.88288288288288	0.99743309893943\\
8.9009009009009	0.997477861326303\\
8.91891891891892	0.997521916828041\\
8.93693693693694	0.997565275210594\\
8.95495495495495	0.997607946129946\\
8.97297297297297	0.997649939132934\\
8.99099099099099	0.997691263658067\\
9.00900900900901	0.997731929036337\\
9.02702702702703	0.997771944492046\\
9.04504504504505	0.997811319143616\\
9.06306306306306	0.997850062004415\\
9.08108108108108	0.997888181983573\\
9.0990990990991	0.997925687886807\\
9.11711711711712	0.997962588417233\\
9.13513513513514	0.997998892176192\\
9.15315315315315	0.998034607664069\\
9.17117117117117	0.998069743281109\\
9.18918918918919	0.998104307328239\\
9.20720720720721	0.998138308007884\\
9.22522522522523	0.998171753424788\\
9.24324324324324	0.998204651586827\\
9.26126126126126	0.998237010405824\\
9.27927927927928	0.998268837698369\\
9.2972972972973	0.998300141186627\\
9.31531531531532	0.998330928499153\\
9.33333333333333	0.998361207171701\\
9.35135135135135	0.998390984648036\\
9.36936936936937	0.998420268280737\\
9.38738738738739	0.998449065332008\\
9.40540540540541	0.998477382974481\\
9.42342342342342	0.998505228292013\\
9.44144144144144	0.998532608280495\\
9.45945945945946	0.998559529848643\\
9.47747747747748	0.998585999818795\\
9.4954954954955	0.998612024927709\\
9.51351351351351	0.998637611827348\\
9.53153153153153	0.998662767085674\\
9.54954954954955	0.998687497187433\\
9.56756756756757	0.998711808534933\\
9.58558558558559	0.998735707448836\\
9.6036036036036	0.998759200168925\\
9.62162162162162	0.998782292854885\\
9.63963963963964	0.998804991587076\\
9.65765765765766	0.998827302367298\\
9.67567567567568	0.998849231119561\\
9.69369369369369	0.998870783690847\\
9.71171171171171	0.998891965851869\\
9.72972972972973	0.998912783297829\\
9.74774774774775	0.998933241649169\\
9.76576576576577	0.998953346452322\\
9.78378378378378	0.99897310318046\\
9.8018018018018	0.998992517234233\\
9.81981981981982	0.999011593942512\\
9.83783783783784	0.999030338563122\\
9.85585585585586	0.999048756283573\\
9.87387387387387	0.999066852221793\\
9.89189189189189	0.999084631426846\\
9.90990990990991	0.999102098879656\\
9.92792792792793	0.999119259493723\\
9.94594594594595	0.999136118115834\\
9.96396396396396	0.999152679526775\\
9.98198198198198	0.999168948442032\\
10	0.999184929512491\\
};


\addplot [color=mycolor1, dashed, line width=2.0pt]
  table[row sep=crcr]{%
-11.6	0.00012882535518264\\
-11.5636363636364	0.00013469931287363\\
-11.5272727272727	0.000140823610305967\\
-11.4909090909091	0.00014720807862959\\
-11.4545454545455	0.00015386289690148\\
-11.4181818181818	0.000160798602749174\\
-11.3818181818182	0.000168026103293636\\
-11.3454545454545	0.000175556686335106\\
-11.3090909090909	0.000183402031805543\\
-11.2727272727273	0.000191574223491166\\
-11.2363636363636	0.000200085761028569\\
-11.2	0.00020894957217777\\
-11.1636363636364	0.000218179025375512\\
-11.1272727272727	0.000227787942572013\\
-11.0909090909091	0.00023779061235429\\
-11.0545454545455	0.000248201803359066\\
-11.0181818181818	0.00025903677797817\\
-10.9818181818182	0.000270311306359211\\
-10.9454545454545	0.000282041680704205\\
-10.9090909090909	0.000294244729868684\\
-10.8727272727273	0.00030693783426368\\
-10.8363636363636	0.000320138941062836\\
-10.8	0.000333866579716754\\
-10.7636363636364	0.000348139877776492\\
-10.7272727272727	0.000362978577027993\\
-10.6909090909091	0.000378403049939008\\
-10.6545454545455	0.000394434316419937\\
-10.6181818181818	0.00041109406089976\\
-10.5818181818182	0.000428404649718057\\
-10.5454545454545	0.000446389148833925\\
-10.5090909090909	0.000465071341852291\\
-10.4727272727273	0.000484475748367996\\
-10.4363636363636	0.000504627642627704\\
-10.4	0.000525553072509457\\
-10.3636363636364	0.000547278878819462\\
-10.3272727272727	0.000569832714905384\\
-10.2909090909091	0.000593243066585183\\
-10.2545454545455	0.000617539272390183\\
-10.2181818181818	0.000642751544120821\\
-10.1818181818182	0.000668910987713193\\
-10.1454545454545	0.000696049624414145\\
-10.1090909090909	0.000724200412262423\\
-10.0727272727273	0.000753397267872969\\
-10.0363636363636	0.000783675088521114\\
-10	0.000815069774523122\\
-9.96363636363636	0.000847618251909092\\
-9.92727272727273	0.000881358495383837\\
-9.89090909090909	0.000916329551571101\\
-9.85454545454545	0.000952571562535854\\
-9.81818181818182	0.000990125789579201\\
-9.78181818181818	0.00102903463729993\\
-9.74545454545455	0.00106934167791622\\
-9.70909090909091	0.00111109167584078\\
-9.67272727272727	0.00115433061250204\\
-9.63636363636364	0.00119910571140367\\
-9.6	0.00124546546341427\\
-9.56363636363636	0.00129345965227833\\
-9.52727272727273	0.00134313938033965\\
-9.49090909090909	0.00139455709446708\\
-9.45454545454546	0.00144776661217279\\
-9.41818181818182	0.00150282314791219\\
-9.38181818181818	0.00155978333955425\\
-9.34545454545455	0.00161870527501066\\
-9.30909090909091	0.00167964851901132\\
-9.27272727272727	0.00174267414001357\\
-9.23636363636364	0.00180784473723165\\
-9.2	0.00187522446777249\\
-9.16363636363636	0.00194487907386343\\
-9.12727272727273	0.00201687591015673\\
-9.09090909090909	0.00209128397109528\\
-9.05454545454545	0.00216817391832332\\
-9.01818181818182	0.00224761810812549\\
-8.98181818181818	0.00232969061887661\\
-8.94545454545455	0.00241446727848451\\
-8.90909090909091	0.00250202569180721\\
-8.87272727272727	0.00259244526802542\\
-8.83636363636364	0.00268580724795042\\
-8.8	0.00278219473124741\\
-8.76363636363636	0.00288169270355296\\
-8.72727272727273	0.00298438806346534\\
-8.69090909090909	0.0030903696493855\\
-8.65454545454545	0.00319972826618612\\
-8.61818181818182	0.0033125567116852\\
-8.58181818181818	0.0034289498029006\\
-8.54545454545454	0.00354900440206086\\
-8.50909090909091	0.00367281944234727\\
-8.47272727272727	0.00380049595334168\\
-8.43636363636364	0.00393213708615373\\
-8.4	0.0040678481382008\\
-8.36363636363636	0.00420773657761337\\
-8.32727272727273	0.0043519120672378\\
-8.29090909090909	0.00450048648820842\\
-8.25454545454545	0.0046535739630596\\
-8.21818181818182	0.0048112908783486\\
-8.18181818181818	0.00497375590675931\\
-8.14545454545454	0.00514109002865611\\
-8.10909090909091	0.00531341655305716\\
-8.07272727272727	0.00549086113799573\\
-8.03636363636364	0.00567355181023751\\
-8	0.00586161898432176\\
};


\addplot [color=mycolor1, dashed, line width=2.0pt]
  table[row sep=crcr]{%
10	0.999184929512491\\
10.0363636363636	0.999216324198493\\
10.0727272727273	0.999246602019141\\
10.1090909090909	0.999275798874752\\
10.1454545454545	0.9993039496626\\
10.1818181818182	0.999331088299301\\
10.2181818181818	0.999357247742894\\
10.2545454545455	0.999382460014624\\
10.2909090909091	0.999406756220429\\
10.3272727272727	0.999430166572109\\
10.3636363636364	0.999452720408195\\
10.4	0.999474446214505\\
10.4363636363636	0.999495371644387\\
10.4727272727273	0.999515523538646\\
10.5090909090909	0.999534927945162\\
10.5454545454545	0.99955361013818\\
10.5818181818182	0.999571594637296\\
10.6181818181818	0.999588905226115\\
10.6545454545455	0.999605564970594\\
10.6909090909091	0.999621596237075\\
10.7272727272727	0.999637020709986\\
10.7636363636364	0.999651859409238\\
10.8	0.999666132707298\\
10.8363636363636	0.999679860345951\\
10.8727272727273	0.999693061452751\\
10.9090909090909	0.999705754557146\\
10.9454545454545	0.99971795760631\\
10.9818181818182	0.999729687980655\\
11.0181818181818	0.999740962509036\\
11.0545454545455	0.999751797483655\\
11.0909090909091	0.99976220867466\\
11.1272727272727	0.999772211344442\\
11.1636363636364	0.999781820261639\\
11.2	0.999791049714837\\
11.2363636363636	0.999799913525986\\
11.2727272727273	0.999808425063523\\
11.3090909090909	0.999816597255209\\
11.3454545454545	0.999824442600679\\
11.3818181818182	0.999831973183721\\
11.4181818181818	0.999839200684265\\
11.4545454545455	0.999846136390113\\
11.4909090909091	0.999852791208385\\
11.5272727272727	0.999859175676708\\
11.5636363636364	0.999865299974141\\
11.6	0.999871173931832\\
11.6363636363636	0.999876807043432\\
11.6727272727273	0.999882208475249\\
11.7090909090909	0.999887387076156\\
11.7454545454545	0.999892351387252\\
11.7818181818182	0.999897109651289\\
11.8181818181818	0.999901669821857\\
11.8545454545455	0.999906039572337\\
11.8909090909091	0.999910226304627\\
11.9272727272727	0.999914237157641\\
11.9636363636364	0.999918079015593\\
12	0.999921758516053\\
12.0363636363636	0.999925282057802\\
12.0727272727273	0.999928655808467\\
12.1090909090909	0.999931885711962\\
12.1454545454545	0.999934977495722\\
12.1818181818182	0.999937936677733\\
12.2181818181818	0.999940768573386\\
12.2545454545455	0.999943478302121\\
12.2909090909091	0.999946070793905\\
12.3272727272727	0.999948550795512\\
12.3636363636364	0.999950922876636\\
12.4	0.999953191435822\\
12.4363636363636	0.999955360706236\\
12.4727272727273	0.999957434761257\\
12.5090909090909	0.999959417519917\\
12.5454545454545	0.999961312752172\\
12.5818181818182	0.999963124084025\\
12.6181818181818	0.999964855002493\\
12.6545454545455	0.999966508860424\\
12.6909090909091	0.999968088881175\\
12.7272727272727	0.999969598163142\\
12.7636363636364	0.999971039684153\\
12.8	0.999972416305731\\
12.8363636363636	0.999973730777218\\
12.8727272727273	0.999974985739776\\
12.9090909090909	0.999976183730263\\
12.9454545454545	0.999977327184984\\
12.9818181818182	0.999978418443324\\
13.0181818181818	0.999979459751268\\
13.0545454545455	0.999980453264806\\
13.0909090909091	0.999981401053226\\
13.1272727272727	0.999982305102307\\
13.1636363636364	0.999983167317402\\
13.2	0.99998398952642\\
13.2363636363636	0.999984773482715\\
13.2727272727273	0.999985520867874\\
13.3090909090909	0.999986233294412\\
13.3454545454545	0.999986912308383\\
13.3818181818182	0.999987559391893\\
13.4181818181818	0.999988175965538\\
13.4545454545455	0.999988763390752\\
13.4909090909091	0.999989322972075\\
13.5272727272727	0.99998985595935\\
13.5636363636364	0.999990363549835\\
13.6	0.999990846890243\\
};


\addplot [color=black]
  table[row sep=crcr]{%
0	0\\
0	0.499999999643507\\
};


\addplot [color=black, dotted]
  table[row sep=crcr]{%
-11.6	0.999999999287014\\
13.6	0.999999999287014\\
};


\addplot [color=black, dotted]
  table[row sep=crcr]{%
-11.6	0\\
13.6	0\\
};


\addplot [color=mycolor4, draw=none, mark size=3.4pt, mark=*, mark options={solid, mycolor4}]
  table[row sep=crcr]{%
-4	0.0588235294117647\\
};


\addplot [color=mycolor4, draw=none, mark size=4.4pt, mark=*, mark options={solid, mycolor4}]
  table[row sep=crcr]{%
-3	0.357142857142857\\
};


\addplot [color=mycolor4, draw=none, mark size=6.5pt, mark=*, mark options={solid, mycolor4}]
  table[row sep=crcr]{%
-2	0.426229508196721\\
};


\addplot [color=mycolor4, draw=none, mark size=6.0pt, mark=*, mark options={solid, mycolor4}]
  table[row sep=crcr]{%
-1	0.431372549019608\\
};


\addplot [color=mycolor4, draw=none, mark size=6.6pt, mark=*, mark options={solid, mycolor4}]
  table[row sep=crcr]{%
0	0.483870967741935\\
};


\addplot [color=mycolor4, draw=none, mark size=5.9pt, mark=*, mark options={solid, mycolor4}]
  table[row sep=crcr]{%
1	0.64\\
};


\addplot [color=mycolor4, draw=none, mark size=5.9pt, mark=*, mark options={solid, mycolor4}]
  table[row sep=crcr]{%
2	0.66\\
};


\addplot [color=mycolor4, draw=none, mark size=3.8pt, mark=*, mark options={solid, mycolor4}]
  table[row sep=crcr]{%
3	0.904761904761905\\
};


\addplot [color=mycolor4, draw=none, mark size=1.7pt, mark=*, mark options={solid, mycolor4}]
  table[row sep=crcr]{%
7	0.75\\
};


\addplot [color=mycolor4, draw=none, mark size=2.6pt, mark=*, mark options={solid, mycolor4}]
  table[row sep=crcr]{%
-6	0.1\\
};


\addplot [color=mycolor4, draw=none, mark size=3.2pt, mark=*, mark options={solid, mycolor4}]
  table[row sep=crcr]{%
4	0.666666666666667\\
};


\addplot [color=mycolor4, draw=none, mark size=1.9pt, mark=*, mark options={solid, mycolor4}]
  table[row sep=crcr]{%
6	0.8\\
};


\addplot [color=mycolor4, draw=none, mark size=1.2pt, mark=*, mark options={solid, mycolor4}]
  table[row sep=crcr]{%
8	0.5\\
};


\addplot [color=mycolor4, draw=none, mark size=0.8pt, mark=*, mark options={solid, mycolor4}]
  table[row sep=crcr]{%
10	1\\
};


\addplot [color=mycolor4, draw=none, mark size=2.9pt, mark=*, mark options={solid, mycolor4}]
  table[row sep=crcr]{%
-5	0.0833333333333333\\
};


\addplot [color=mycolor4, draw=none, mark size=2.2pt, mark=*, mark options={solid, mycolor4}]
  table[row sep=crcr]{%
5	0.714285714285714\\
};


\addplot [color=mycolor4, draw=none, mark size=0.8pt, mark=*, mark options={solid, mycolor4}]
  table[row sep=crcr]{%
9	0\\
};


\addplot [color=mycolor4, draw=none, mark size=1.2pt, mark=*, mark options={solid, mycolor4}]
  table[row sep=crcr]{%
-7	0\\
};


\addplot [color=mycolor4, draw=none, mark size=0.8pt, mark=*, mark options={solid, mycolor4
}]
  table[row sep=crcr]{%
-10	0\\
};


\addplot [color=mycolor3, line width=2.0pt]
  table[row sep=crcr]{%
-10	0.0385394464594078\\
-9.97997997997998	0.0388350709153951\\
-9.95995995995996	0.0391325384162751\\
-9.93993993993994	0.0394318567205697\\
-9.91991991991992	0.0397330335887806\\
-9.8998998998999	0.0400360767831097\\
-9.87987987987988	0.0403409940671788\\
-9.85985985985986	0.0406477932057472\\
-9.83983983983984	0.040956481964427\\
-9.81981981981982	0.0412670681093974\\
-9.7997997997998	0.0415795594071166\\
-9.77977977977978	0.0418939636240323\\
-9.75975975975976	0.0422102885262902\\
-9.73973973973974	0.0425285418794406\\
-9.71971971971972	0.0428487314481439\\
-9.6996996996997	0.0431708649958733\\
-9.67967967967968	0.0434949502846169\\
-9.65965965965966	0.043820995074577\\
-9.63963963963964	0.0441490071238684\\
-9.61961961961962	0.044478994188215\\
-9.5995995995996	0.0448109640206438\\
-9.57957957957958	0.0451449243711783\\
-9.55955955955956	0.0454808829865299\\
-9.53953953953954	0.0458188476097867\\
-9.51951951951952	0.0461588259801023\\
-9.4994994994995	0.0465008258323813\\
-9.47947947947948	0.046844854896964\\
-9.45945945945946	0.0471909208993097\\
-9.43943943943944	0.0475390315596775\\
-9.41941941941942	0.047889194592806\\
-9.3993993993994	0.0482414177075915\\
-9.37937937937938	0.0485957086067644\\
-9.35935935935936	0.0489520749865639\\
-9.33933933933934	0.0493105245364113\\
-9.31931931931932	0.0496710649385814\\
-9.2992992992993	0.0500337038678731\\
-9.27927927927928	0.0503984489912778\\
-9.25925925925926	0.0507653079676459\\
-9.23923923923924	0.0511342884473529\\
-9.21921921921922	0.0515053980719633\\
-9.1991991991992	0.0518786444738925\\
-9.17917917917918	0.0522540352760686\\
-9.15915915915916	0.0526315780915912\\
-9.13913913913914	0.0530112805233898\\
-9.11911911911912	0.0533931501638803\\
-9.0990990990991	0.0537771945946202\\
-9.07907907907908	0.0541634213859621\\
-9.05905905905906	0.0545518380967062\\
-9.03903903903904	0.0549424522737513\\
-9.01901901901902	0.0553352714517439\\
-8.998998998999	0.055730303152727\\
-8.97897897897898	0.0561275548857864\\
-8.95895895895896	0.056527034146696\\
-8.93893893893894	0.0569287484175626\\
-8.91891891891892	0.0573327051664682\\
-8.8988988988989	0.0577389118471117\\
-8.87887887887888	0.058147375898449\\
-8.85885885885886	0.0585581047443319\\
-8.83883883883884	0.0589711057931465\\
-8.81881881881882	0.0593863864374487\\
-8.7987987987988	0.0598039540536002\\
-8.77877877877878	0.0602238160014021\\
-8.75875875875876	0.0606459796237281\\
-8.73873873873874	0.061070452246156\\
-8.71871871871872	0.0614972411765984\\
-8.6986986986987	0.061926353704932\\
-8.67867867867868	0.062357797102626\\
-8.65865865865866	0.0627915786223692\\
-8.63863863863864	0.063227705497696\\
-8.61861861861862	0.0636661849426116\\
-8.5985985985986	0.0641070241512162\\
-8.57857857857858	0.0645502302973275\\
-8.55855855855856	0.0649958105341026\\
-8.53853853853854	0.0654437719936596\\
-8.51851851851852	0.065894121786697\\
-8.4984984984985	0.0663468670021126\\
-8.47847847847848	0.0668020147066222\\
-8.45845845845846	0.067259571944376\\
-8.43843843843844	0.0677195457365753\\
-8.41841841841842	0.0681819430810877\\
-8.3983983983984	0.0686467709520616\\
-8.37837837837838	0.0691140362995399\\
-8.35835835835836	0.0695837460490728\\
-8.33833833833834	0.0700559071013296\\
-8.31831831831832	0.0705305263317102\\
-8.2982982982983	0.0710076105899553\\
-8.27827827827828	0.0714871666997567\\
-8.25825825825826	0.0719692014583653\\
-8.23823823823824	0.0724537216362003\\
-8.21821821821822	0.0729407339764563\\
-8.1981981981982	0.0734302451947103\\
-8.17817817817818	0.0739222619785284\\
-8.15815815815816	0.0744167909870707\\
-8.13813813813814	0.0749138388506972\\
-8.11811811811812	0.0754134121705717\\
-8.0980980980981	0.0759155175182663\\
-8.07807807807808	0.076420161435364\\
-8.05805805805806	0.0769273504330625\\
-8.03803803803804	0.0774370909917762\\
-8.01801801801802	0.0779493895607381\\
-7.997997997998	0.0784642525576014\\
-7.97797797797798	0.0789816863680405\\
-7.95795795795796	0.0795016973453518\\
-7.93793793793794	0.0800242918100541\\
-7.91791791791792	0.0805494760494882\\
-7.8978978978979	0.0810772563174171\\
-7.87787787787788	0.0816076388336246\\
-7.85785785785786	0.0821406297835151\\
-7.83783783783784	0.0826762353177117\\
-7.81781781781782	0.0832144615516554\\
-7.7977977977978	0.083755314565203\\
-7.77777777777778	0.0842988004022252\\
-7.75775775775776	0.084844925070205\\
-7.73773773773774	0.0853936945398352\\
-7.71771771771772	0.0859451147446165\\
-7.6976976976977	0.0864991915804546\\
-7.67767767767768	0.0870559309052583\\
-7.65765765765766	0.0876153385385369\\
-7.63763763763764	0.0881774202609976\\
-7.61761761761762	0.0887421818141432\\
-7.5975975975976	0.0893096288998696\\
-7.57757757757758	0.0898797671800635\\
-7.55755755755756	0.0904526022762002\\
-7.53753753753754	0.0910281397689411\\
-7.51751751751752	0.0916063851977321\\
-7.4974974974975	0.0921873440604017\\
-7.47747747747748	0.092771021812759\\
-7.45745745745746	0.0933574238681926\\
-7.43743743743744	0.0939465555972691\\
-7.41741741741742	0.0945384223273324\\
-7.3973973973974	0.095133029342103\\
-7.37737737737738	0.0957303818812775\\
-7.35735735735736	0.0963304851401291\\
-7.33733733733734	0.0969333442691072\\
-7.31731731731732	0.0975389643734389\\
-7.2972972972973	0.09814735051273\\
-7.27727727727728	0.0987585077005667\\
-7.25725725725726	0.0993724409041174\\
-7.23723723723724	0.099989155043736\\
-7.21721721721722	0.100608654992565\\
-7.1971971971972	0.101230945576137\\
-7.17717717717718	0.101856031571985\\
-7.15715715715716	0.102483917709241\\
-7.13713713713714	0.103114608668242\\
-7.11711711711712	0.103748109080144\\
-7.0970970970971	0.104384423526517\\
-7.07707707707708	0.105023556538964\\
-7.05705705705706	0.105665512598721\\
-7.03703703703704	0.106310296136274\\
-7.01701701701702	0.106957911530959\\
-6.996996996997	0.107608363110584\\
-6.97697697697698	0.108261655151033\\
-6.95695695695696	0.108917791875882\\
-6.93693693693694	0.10957677745601\\
-6.91691691691692	0.110238616009218\\
-6.8968968968969	0.110903311599841\\
-6.87687687687688	0.111570868238363\\
-6.85685685685686	0.11224128988104\\
-6.83683683683684	0.112914580429513\\
-6.81681681681682	0.113590743730433\\
-6.7967967967968	0.114269783575077\\
-6.77677677677678	0.114951703698972\\
-6.75675675675676	0.115636507781519\\
-6.73673673673674	0.116324199445616\\
-6.71671671671672	0.117014782257285\\
-6.6966966966967	0.117708259725298\\
-6.67667667667668	0.118404635300804\\
-6.65665665665666	0.11910391237696\\
-6.63663663663664	0.119806094288563\\
-6.61661661661662	0.12051118431168\\
-6.5965965965966	0.121219185663282\\
-6.57657657657658	0.121930101500878\\
-6.55655655655656	0.122643934922156\\
-6.53653653653654	0.123360688964616\\
-6.51651651651652	0.124080366605212\\
-6.4964964964965	0.124802970759992\\
-6.47647647647648	0.125528504283742\\
-6.45645645645646	0.126256969969629\\
-6.43643643643644	0.126988370548847\\
-6.41641641641642	0.127722708690267\\
-6.3963963963964	0.128459987000081\\
-6.37637637637638	0.129200208021459\\
-6.35635635635636	0.129943374234198\\
-6.33633633633634	0.130689488054378\\
-6.31631631631632	0.131438551834015\\
-6.2962962962963	0.132190567860727\\
-6.27627627627628	0.132945538357385\\
-6.25625625625626	0.133703465481781\\
-6.23623623623624	0.134464351326289\\
-6.21621621621622	0.135228197917531\\
-6.1961961961962	0.135995007216048\\
-6.17617617617618	0.136764781115964\\
-6.15615615615616	0.137537521444663\\
-6.13613613613614	0.138313229962461\\
-6.11611611611612	0.139091908362282\\
-6.0960960960961	0.139873558269334\\
-6.07607607607608	0.140658181240795\\
-6.05605605605606	0.141445778765489\\
-6.03603603603604	0.142236352263575\\
-6.01601601601602	0.14302990308623\\
-5.995995995996	0.143826432515345\\
-5.97597597597598	0.144625941763207\\
-5.95595595595596	0.145428431972201\\
-5.93593593593594	0.146233904214502\\
-5.91591591591592	0.147042359491774\\
-5.8958958958959	0.147853798734872\\
-5.87587587587588	0.148668222803543\\
-5.85585585585586	0.149485632486136\\
-5.83583583583584	0.150306028499304\\
-5.81581581581582	0.151129411487723\\
-5.7957957957958	0.151955782023796\\
-5.77577577577578	0.152785140607377\\
-5.75575575575576	0.153617487665484\\
-5.73573573573574	0.154452823552024\\
-5.71571571571572	0.155291148547514\\
-5.6956956956957	0.156132462858807\\
-5.67567567567568	0.156976766618824\\
-5.65565565565566	0.157824059886285\\
-5.63563563563564	0.158674342645442\\
-5.61561561561562	0.159527614805816\\
-5.5955955955956	0.16038387620194\\
-5.57557557557558	0.161243126593101\\
-5.55555555555556	0.162105365663084\\
-5.53553553553554	0.162970593019924\\
-5.51551551551552	0.163838808195653\\
-5.4954954954955	0.164710010646059\\
-5.47547547547548	0.165584199750444\\
-5.45545545545546	0.166461374811379\\
-5.43543543543544	0.167341535054474\\
-5.41541541541542	0.16822467962814\\
-5.3953953953954	0.169110807603361\\
-5.37537537537538	0.169999917973466\\
-5.35535535535536	0.170892009653906\\
-5.33533533533534	0.171787081482031\\
-5.31531531531532	0.172685132216874\\
-5.2952952952953	0.173586160538936\\
-5.27527527527528	0.174490165049972\\
-5.25525525525526	0.175397144272787\\
-5.23523523523524	0.176307096651029\\
-5.21521521521522	0.177220020548986\\
-5.1951951951952	0.17813591425139\\
-5.17517517517518	0.179054775963221\\
-5.15515515515516	0.179976603809515\\
-5.13513513513514	0.180901395835175\\
-5.11511511511512	0.181829150004789\\
-5.0950950950951	0.182759864202444\\
-5.07507507507508	0.183693536231552\\
-5.05505505505506	0.184630163814675\\
-5.03503503503504	0.185569744593349\\
-5.01501501501502	0.186512276127924\\
-4.99499499499499	0.187457755897394\\
-4.97497497497497	0.188406181299237\\
-4.95495495495495	0.189357549649263\\
-4.93493493493493	0.190311858181455\\
-4.91491491491491	0.19126910404782\\
-4.89489489489489	0.192229284318247\\
-4.87487487487487	0.19319239598036\\
-4.85485485485485	0.194158435939381\\
-4.83483483483483	0.195127401017995\\
-4.81481481481481	0.196099287956217\\
-4.79479479479479	0.197074093411269\\
-4.77477477477477	0.198051813957448\\
-4.75475475475475	0.199032446086015\\
-4.73473473473473	0.200015986205071\\
-4.71471471471471	0.201002430639448\\
-4.69469469469469	0.201991775630602\\
-4.67467467467467	0.202984017336504\\
-4.65465465465465	0.203979151831541\\
-4.63463463463463	0.204977175106419\\
-4.61461461461461	0.20597808306807\\
-4.59459459459459	0.206981871539561\\
-4.57457457457457	0.20798853626001\\
-4.55455455455455	0.208998072884505\\
-4.53453453453453	0.210010476984026\\
-4.51451451451451	0.211025744045369\\
-4.49449449449449	0.21204386947108\\
-4.47447447447447	0.213064848579389\\
-4.45445445445445	0.214088676604146\\
-4.43443443443443	0.215115348694765\\
-4.41441441441441	0.216144859916173\\
-4.39439439439439	0.217177205248756\\
-4.37437437437437	0.218212379588318\\
-4.35435435435435	0.219250377746035\\
-4.33433433433433	0.220291194448425\\
-4.31431431431431	0.221334824337307\\
-4.29429429429429	0.22238126196978\\
-4.27427427427427	0.223430501818193\\
-4.25425425425425	0.224482538270125\\
-4.23423423423423	0.225537365628372\\
-4.21421421421421	0.226594978110931\\
-4.19419419419419	0.227655369850996\\
-4.17417417417417	0.228718534896951\\
-4.15415415415415	0.229784467212373\\
-4.13413413413413	0.230853160676033\\
-4.11411411411411	0.231924609081912\\
-4.09409409409409	0.232998806139205\\
-4.07407407407407	0.234075745472349\\
-4.05405405405405	0.235155420621034\\
-4.03403403403403	0.23623782504024\\
-4.01401401401401	0.237322952100257\\
-3.99399399399399	0.238410795086728\\
-3.97397397397397	0.239501347200684\\
-3.95395395395395	0.240594601558588\\
-3.93393393393393	0.241690551192382\\
-3.91391391391391	0.242789189049542\\
-3.89389389389389	0.243890507993131\\
-3.87387387387387	0.244994500801861\\
-3.85385385385385	0.246101160170158\\
-3.83383383383383	0.247210478708232\\
-3.81381381381381	0.248322448942151\\
-3.79379379379379	0.249437063313918\\
-3.77377377377377	0.250554314181553\\
-3.75375375375375	0.251674193819185\\
-3.73373373373373	0.252796694417135\\
-3.71371371371371	0.253921808082021\\
-3.69369369369369	0.25504952683685\\
-3.67367367367367	0.256179842621127\\
-3.65365365365365	0.257312747290966\\
-3.63363363363363	0.258448232619197\\
-3.61361361361361	0.259586290295489\\
-3.59359359359359	0.260726911926472\\
-3.57357357357357	0.26187008903586\\
-3.55355355355355	0.263015813064587\\
-3.53353353353353	0.264164075370938\\
-3.51351351351351	0.265314867230693\\
-3.49349349349349	0.266468179837268\\
-3.47347347347347	0.267624004301865\\
-3.45345345345345	0.268782331653628\\
-3.43343343343343	0.269943152839795\\
-3.41341341341341	0.271106458725866\\
-3.39339339339339	0.272272240095765\\
-3.37337337337337	0.273440487652014\\
-3.35335335335335	0.274611192015907\\
-3.33333333333333	0.27578434372769\\
-3.31331331331331	0.276959933246744\\
-3.29329329329329	0.278137950951778\\
-3.27327327327327	0.279318387141016\\
-3.25325325325325	0.2805012320324\\
-3.23323323323323	0.281686475763789\\
-3.21321321321321	0.282874108393164\\
-3.19319319319319	0.284064119898842\\
-3.17317317317317	0.285256500179691\\
-3.15315315315315	0.286451239055346\\
-3.13313313313313	0.287648326266437\\
-3.11311311311311	0.288847751474815\\
-3.09309309309309	0.290049504263783\\
-3.07307307307307	0.29125357413834\\
-3.05305305305305	0.292459950525413\\
-3.03303303303303	0.29366862277411\\
-3.01301301301301	0.294879580155968\\
-2.99299299299299	0.296092811865206\\
-2.97297297297297	0.297308307018985\\
-2.95295295295295	0.298526054657673\\
-2.93293293293293	0.299746043745109\\
-2.91291291291291	0.300968263168877\\
-2.89289289289289	0.302192701740582\\
-2.87287287287287	0.303419348196129\\
-2.85285285285285	0.304648191196011\\
-2.83283283283283	0.305879219325595\\
-2.81281281281281	0.307112421095414\\
-2.79279279279279	0.308347784941469\\
-2.77277277277277	0.309585299225528\\
-2.75275275275275	0.310824952235432\\
-2.73273273273273	0.312066732185404\\
-2.71271271271271	0.313310627216369\\
-2.69269269269269	0.314556625396265\\
-2.67267267267267	0.31580471472037\\
-2.65265265265265	0.317054883111631\\
-2.63263263263263	0.318307118420988\\
-2.61261261261261	0.319561408427718\\
-2.59259259259259	0.320817740839768\\
-2.57257257257257	0.322076103294103\\
-2.55255255255255	0.323336483357048\\
-2.53253253253253	0.324598868524646\\
-2.51251251251251	0.32586324622301\\
-2.49249249249249	0.327129603808684\\
-2.47247247247247	0.328397928569006\\
-2.45245245245245	0.329668207722478\\
-2.43243243243243	0.330940428419136\\
-2.41241241241241	0.332214577740926\\
-2.39239239239239	0.333490642702084\\
-2.37237237237237	0.334768610249521\\
-2.35235235235235	0.33604846726321\\
-2.33233233233233	0.337330200556577\\
-2.31231231231231	0.338613796876899\\
-2.29229229229229	0.339899242905699\\
-2.27227227227227	0.341186525259155\\
-2.25225225225225	0.342475630488501\\
-2.23223223223223	0.343766545080445\\
-2.21221221221221	0.345059255457577\\
-2.19219219219219	0.346353747978793\\
-2.17217217217217	0.347650008939713\\
-2.15215215215215	0.34894802457311\\
-2.13213213213213	0.350247781049339\\
-2.11211211211211	0.351549264476772\\
-2.09209209209209	0.35285246090223\\
-2.07207207207207	0.354157356311431\\
-2.05205205205205	0.355463936629429\\
-2.03203203203203	0.356772187721065\\
-2.01201201201201	0.358082095391416\\
-1.99199199199199	0.359393645386254\\
-1.97197197197197	0.360706823392501\\
-1.95195195195195	0.362021615038693\\
-1.93193193193193	0.363338005895448\\
-1.91191191191191	0.364655981475929\\
-1.89189189189189	0.365975527236324\\
-1.87187187187187	0.367296628576315\\
-1.85185185185185	0.368619270839565\\
-1.83183183183183	0.369943439314194\\
-1.81181181181181	0.371269119233268\\
-1.79179179179179	0.372596295775289\\
-1.77177177177177	0.373924954064689\\
-1.75175175175175	0.37525507917232\\
-1.73173173173173	0.376586656115961\\
-1.71171171171171	0.377919669860814\\
-1.69169169169169	0.379254105320016\\
-1.67167167167167	0.38058994735514\\
-1.65165165165165	0.381927180776712\\
-1.63163163163163	0.383265790344727\\
-1.61161161161161	0.384605760769162\\
-1.59159159159159	0.385947076710502\\
-1.57157157157157	0.38728972278026\\
-1.55155155155155	0.388633683541507\\
-1.53153153153153	0.3899789435094\\
-1.51151151151151	0.391325487151717\\
-1.49149149149149	0.392673298889389\\
-1.47147147147147	0.394022363097043\\
-1.45145145145145	0.395372664103541\\
-1.43143143143143	0.396724186192522\\
-1.41141141141141	0.398076913602954\\
-1.39139139139139	0.39943083052968\\
-1.37137137137137	0.400785921123971\\
-1.35135135135135	0.402142169494082\\
-1.33133133133133	0.403499559705809\\
-1.31131131131131	0.404858075783047\\
-1.29129129129129	0.406217701708358\\
-1.27127127127127	0.407578421423532\\
-1.25125125125125	0.408940218830154\\
-1.23123123123123	0.41030307779018\\
-1.21121121121121	0.411666982126503\\
-1.19119119119119	0.413031915623532\\
-1.17117117117117	0.41439786202777\\
-1.15115115115115	0.415764805048389\\
-1.13113113113113	0.417132728357818\\
-1.11111111111111	0.418501615592324\\
-1.09109109109109	0.4198714503526\\
-1.07107107107107	0.421242216204352\\
-1.05105105105105	0.42261389667889\\
-1.03103103103103	0.423986475273725\\
-1.01101101101101	0.425359935453157\\
-0.990990990990991	0.426734260648879\\
-0.970970970970971	0.428109434260572\\
-0.950950950950951	0.429485439656509\\
-0.930930930930931	0.430862260174154\\
-0.910910910910911	0.432239879120771\\
-0.890890890890891	0.433618279774028\\
-0.870870870870871	0.43499744538261\\
-0.850850850850851	0.436377359166822\\
-0.830830830830831	0.437758004319209\\
-0.810810810810811	0.439139364005165\\
-0.790790790790791	0.44052142136355\\
-0.77077077077077	0.441904159507307\\
-0.75075075075075	0.443287561524082\\
-0.73073073073073	0.444671610476844\\
-0.71071071071071	0.446056289404504\\
-0.69069069069069	0.447441581322545\\
-0.67067067067067	0.448827469223638\\
-0.65065065065065	0.450213936078277\\
-0.63063063063063	0.451600964835401\\
-0.61061061061061	0.452988538423026\\
-0.59059059059059	0.454376639748872\\
-0.57057057057057	0.455765251700998\\
-0.55055055055055	0.457154357148433\\
-0.53053053053053	0.458543938941813\\
-0.51051051051051	0.459933979914008\\
-0.49049049049049	0.46132446288077\\
-0.47047047047047	0.46271537064136\\
-0.45045045045045	0.464106685979192\\
-0.43043043043043	0.465498391662469\\
-0.41041041041041	0.466890470444827\\
-0.39039039039039	0.468282905065972\\
-0.37037037037037	0.469675678252327\\
-0.35035035035035	0.471068772717669\\
-0.33033033033033	0.472462171163776\\
-0.31031031031031	0.473855856281071\\
-0.29029029029029	0.475249810749269\\
-0.27027027027027	0.476644017238019\\
-0.25025025025025	0.47803845840755\\
-0.23023023023023	0.479433116909324\\
-0.21021021021021	0.480827975386678\\
-0.19019019019019	0.482223016475472\\
-0.17017017017017	0.483618222804741\\
-0.15015015015015	0.485013576997341\\
-0.13013013013013	0.486409061670599\\
-0.11011011011011	0.487804659436965\\
-0.0900900900900901	0.489200352904656\\
-0.07007007007007	0.490596124678314\\
-0.05005005005005	0.49199195735965\\
-0.03003003003003	0.493387833548097\\
-0.01001001001001	0.494783735841462\\
0.01001001001001	0.496179646836574\\
0.03003003003003	0.497575549129939\\
0.05005005005005	0.498971425318386\\
0.07007007007007	0.500367257999722\\
0.0900900900900901	0.50176302977338\\
0.11011011011011	0.503158723241072\\
0.13013013013013	0.504554321007437\\
0.15015015015015	0.505949805680696\\
0.17017017017017	0.507345159873296\\
0.19019019019019	0.508740366202564\\
0.21021021021021	0.510135407291358\\
0.23023023023023	0.511530265768712\\
0.25025025025025	0.512924924270486\\
0.27027027027027	0.514319365440018\\
0.29029029029029	0.515713571928767\\
0.31031031031031	0.517107526396965\\
0.33033033033033	0.518501211514261\\
0.35035035035035	0.519894609960368\\
0.37037037037037	0.521287704425709\\
0.39039039039039	0.522680477612064\\
0.41041041041041	0.52407291223321\\
0.43043043043043	0.525464991015567\\
0.45045045045045	0.526856696698845\\
0.47047047047047	0.528248012036676\\
0.49049049049049	0.529638919797266\\
0.51051051051051	0.531029402764028\\
0.53053053053053	0.532419443736224\\
0.55055055055055	0.533809025529603\\
0.57057057057057	0.535198130977039\\
0.59059059059059	0.536586742929165\\
0.61061061061061	0.537974844255011\\
0.63063063063063	0.539362417842635\\
0.65065065065065	0.540749446599759\\
0.67067067067067	0.542135913454398\\
0.69069069069069	0.543521801355492\\
0.71071071071071	0.544907093273532\\
0.73073073073073	0.546291772201193\\
0.75075075075075	0.547675821153954\\
0.77077077077077	0.549059223170729\\
0.790790790790791	0.550441961314486\\
0.810810810810811	0.551824018672871\\
0.830830830830831	0.553205378358827\\
0.850850850850851	0.554586023511214\\
0.870870870870871	0.555965937295426\\
0.890890890890891	0.557345102904008\\
0.910910910910911	0.558723503557266\\
0.930930930930931	0.560101122503883\\
0.950950950950951	0.561477943021528\\
0.970970970970971	0.562853948417464\\
0.990990990990991	0.564229122029158\\
1.01101101101101	0.56560344722488\\
1.03103103103103	0.566976907404312\\
1.05105105105105	0.568349485999146\\
1.07107107107107	0.569721166473684\\
1.09109109109109	0.571091932325436\\
1.11111111111111	0.572461767085712\\
1.13113113113113	0.573830654320218\\
1.15115115115115	0.575198577629648\\
1.17117117117117	0.576565520650267\\
1.19119119119119	0.577931467054504\\
1.21121121121121	0.579296400551533\\
1.23123123123123	0.580660304887856\\
1.25125125125125	0.582023163847882\\
1.27127127127127	0.583384961254504\\
1.29129129129129	0.584745680969678\\
1.31131131131131	0.586105306894989\\
1.33133133133133	0.587463822972228\\
1.35135135135135	0.588821213183954\\
1.37137137137137	0.590177461554065\\
1.39139139139139	0.591532552148357\\
1.41141141141141	0.592886469075082\\
1.43143143143143	0.594239196485515\\
1.45145145145145	0.595590718574496\\
1.47147147147147	0.596941019580993\\
1.49149149149149	0.598290083788647\\
1.51151151151151	0.599637895526319\\
1.53153153153153	0.600984439168636\\
1.55155155155155	0.602329699136529\\
1.57157157157157	0.603673659897776\\
1.59159159159159	0.605016305967535\\
1.61161161161161	0.606357621908874\\
1.63163163163163	0.607697592333309\\
1.65165165165165	0.609036201901324\\
1.67167167167167	0.610373435322897\\
1.69169169169169	0.61170927735802\\
1.71171171171171	0.613043712817222\\
1.73173173173173	0.614376726562076\\
1.75175175175175	0.615708303505717\\
1.77177177177177	0.617038428613348\\
1.79179179179179	0.618367086902747\\
1.81181181181181	0.619694263444769\\
1.83183183183183	0.621019943363843\\
1.85185185185185	0.622344111838471\\
1.87187187187187	0.623666754101721\\
1.89189189189189	0.624987855441713\\
1.91191191191191	0.626307401202107\\
1.93193193193193	0.627625376782588\\
1.95195195195195	0.628941767639343\\
1.97197197197197	0.630256559285536\\
1.99199199199199	0.631569737291783\\
2.01201201201201	0.63288128728662\\
2.03203203203203	0.634191194956972\\
2.05205205205205	0.635499446048607\\
2.07207207207207	0.636806026366605\\
2.09209209209209	0.638110921775806\\
2.11211211211211	0.639414118201264\\
2.13213213213213	0.640715601628697\\
2.15215215215215	0.642015358104927\\
2.17217217217217	0.643313373738324\\
2.19219219219219	0.644609634699243\\
2.21221221221221	0.645904127220459\\
2.23223223223223	0.647196837597591\\
2.25225225225225	0.648487752189535\\
2.27227227227227	0.649776857418881\\
2.29229229229229	0.651064139772337\\
2.31231231231231	0.652349585801137\\
2.33233233233233	0.653633182121459\\
2.35235235235235	0.654914915414826\\
2.37237237237237	0.656194772428516\\
2.39239239239239	0.657472739975953\\
2.41241241241241	0.658748804937111\\
2.43243243243243	0.6600229542589\\
2.45245245245245	0.661295174955558\\
2.47247247247247	0.66256545410903\\
2.49249249249249	0.663833778869353\\
2.51251251251251	0.665100136455027\\
2.53253253253253	0.666364514153391\\
2.55255255255255	0.667626899320989\\
2.57257257257257	0.668887279383934\\
2.59259259259259	0.670145641838268\\
2.61261261261261	0.671401974250319\\
2.63263263263263	0.672656264257048\\
2.65265265265265	0.673908499566406\\
2.67267267267267	0.675158667957666\\
2.69269269269269	0.676406757281771\\
2.71271271271271	0.677652755461667\\
2.73273273273273	0.678896650492632\\
2.75275275275275	0.680138430442605\\
2.77277277277277	0.681378083452508\\
2.79279279279279	0.682615597736567\\
2.81281281281281	0.683850961582622\\
2.83283283283283	0.685084163352442\\
2.85285285285285	0.686315191482025\\
2.87287287287287	0.687544034481907\\
2.89289289289289	0.688770680937455\\
2.91291291291291	0.689995119509159\\
2.93293293293293	0.691217338932927\\
2.95295295295295	0.692437328020363\\
2.97297297297297	0.693655075659051\\
2.99299299299299	0.69487057081283\\
3.01301301301301	0.696083802522068\\
3.03303303303303	0.697294759903926\\
3.05305305305305	0.698503432152624\\
3.07307307307307	0.699709808539697\\
3.09309309309309	0.700913878414253\\
3.11311311311311	0.702115631203222\\
3.13313313313313	0.703315056411599\\
3.15315315315315	0.70451214362269\\
3.17317317317317	0.705706882498345\\
3.19319319319319	0.706899262779194\\
3.21321321321321	0.708089274284873\\
3.23323323323323	0.709276906914248\\
3.25325325325325	0.710462150645636\\
3.27327327327327	0.71164499553702\\
3.29329329329329	0.712825431726258\\
3.31331331331331	0.714003449431292\\
3.33333333333333	0.715179038950347\\
3.35335335335335	0.716352190662129\\
3.37337337337337	0.717522895026022\\
3.39339339339339	0.718691142582271\\
3.41341341341341	0.71985692395217\\
3.43343343343343	0.721020229838241\\
3.45345345345345	0.722181051024409\\
3.47347347347347	0.723339378376171\\
3.49349349349349	0.724495202840769\\
3.51351351351351	0.725648515447344\\
3.53353353353353	0.726799307307098\\
3.55355355355355	0.72794756961345\\
3.57357357357357	0.729093293642176\\
3.59359359359359	0.730236470751565\\
3.61361361361361	0.731377092382547\\
3.63363363363363	0.73251515005884\\
3.65365365365365	0.733650635387071\\
3.67367367367367	0.734783540056909\\
3.69369369369369	0.735913855841187\\
3.71371371371371	0.737041574596015\\
3.73373373373373	0.738166688260901\\
3.75375375375375	0.739289188858851\\
3.77377377377377	0.740409068496483\\
3.79379379379379	0.741526319364119\\
3.81381381381381	0.742640933735885\\
3.83383383383383	0.743752903969804\\
3.85385385385385	0.744862222507878\\
3.87387387387387	0.745968881876175\\
3.89389389389389	0.747072874684905\\
3.91391391391391	0.748174193628494\\
3.93393393393393	0.749272831485654\\
3.95395395395395	0.750368781119448\\
3.97397397397397	0.751462035477352\\
3.99399399399399	0.752552587591308\\
4.01401401401401	0.75364043057778\\
4.03403403403403	0.754725557637797\\
4.05405405405405	0.755807962057002\\
4.07407407407407	0.756887637205688\\
4.09409409409409	0.757964576538831\\
4.11411411411411	0.759038773596124\\
4.13413413413413	0.760110222002003\\
4.15415415415415	0.761178915465663\\
4.17417417417417	0.762244847781085\\
4.19419419419419	0.76330801282704\\
4.21421421421421	0.764368404567105\\
4.23423423423423	0.765426017049665\\
4.25425425425425	0.766480844407911\\
4.27427427427427	0.767532880859843\\
4.29429429429429	0.768582120708256\\
4.31431431431431	0.769628558340729\\
4.33433433433433	0.770672188229612\\
4.35435435435435	0.771713004932001\\
4.37437437437437	0.772751003089719\\
4.39439439439439	0.77378617742928\\
4.41441441441441	0.774818522761863\\
4.43443443443443	0.775848033983271\\
4.45445445445445	0.776874706073891\\
4.47447447447447	0.777898534098647\\
4.49449449449449	0.778919513206956\\
4.51451451451451	0.779937638632668\\
4.53453453453453	0.780952905694011\\
4.55455455455455	0.781965309793531\\
4.57457457457457	0.782974846418026\\
4.59459459459459	0.783981511138476\\
4.61461461461461	0.784985299609967\\
4.63463463463463	0.785986207571617\\
4.65465465465465	0.786984230846495\\
4.67467467467467	0.787979365341532\\
4.69469469469469	0.788971607047434\\
4.71471471471471	0.789960952038588\\
4.73473473473473	0.790947396472966\\
4.75475475475475	0.791930936592022\\
4.77477477477477	0.792911568720588\\
4.79479479479479	0.793889289266768\\
4.81481481481481	0.794864094721819\\
4.83483483483483	0.795835981660042\\
4.85485485485485	0.796804946738655\\
4.87487487487487	0.797770986697676\\
4.89489489489489	0.798734098359789\\
4.91491491491491	0.799694278630216\\
4.93493493493493	0.800651524496582\\
4.95495495495495	0.801605833028773\\
4.97497497497497	0.802557201378799\\
4.99499499499499	0.803505626780643\\
5.01501501501502	0.804451106550112\\
5.03503503503504	0.805393638084687\\
5.05505505505506	0.806333218863361\\
5.07507507507508	0.807269846446484\\
5.0950950950951	0.808203518475592\\
5.11511511511512	0.809134232673248\\
5.13513513513514	0.810061986842861\\
5.15515515515516	0.810986778868521\\
5.17517517517518	0.811908606714815\\
5.1951951951952	0.812827468426646\\
5.21521521521522	0.81374336212905\\
5.23523523523524	0.814656286027007\\
5.25525525525526	0.815566238405249\\
5.27527527527528	0.816473217628064\\
5.2952952952953	0.8173772221391\\
5.31531531531532	0.818278250461162\\
5.33533533533534	0.819176301196005\\
5.35535535535536	0.82007137302413\\
5.37537537537538	0.82096346470457\\
5.3953953953954	0.821852575074676\\
5.41541541541542	0.822738703049896\\
5.43543543543544	0.823621847623562\\
5.45545545545546	0.824502007866657\\
5.47547547547548	0.825379182927592\\
5.4954954954955	0.826253372031977\\
5.51551551551552	0.827124574482384\\
5.53553553553554	0.827992789658113\\
5.55555555555556	0.828858017014952\\
5.57557557557558	0.829720256084935\\
5.5955955955956	0.830579506476096\\
5.61561561561562	0.831435767872221\\
5.63563563563564	0.832289040032595\\
5.65565565565566	0.833139322791751\\
5.67567567567568	0.833986616059212\\
5.6956956956957	0.83483091981923\\
5.71571571571572	0.835672234130523\\
5.73573573573574	0.836510559126012\\
5.75575575575576	0.837345895012552\\
5.77577577577578	0.838178242070659\\
5.7957957957958	0.83900760065424\\
5.81581581581582	0.839833971190314\\
5.83583583583584	0.840657354178732\\
5.85585585585586	0.841477750191901\\
5.87587587587588	0.842295159874493\\
5.8958958958959	0.843109583943165\\
5.91591591591592	0.843921023186262\\
5.93593593593594	0.844729478463534\\
5.95595595595596	0.845534950705835\\
5.97597597597598	0.846337440914829\\
5.995995995996	0.847136950162692\\
6.01601601601602	0.847933479591806\\
6.03603603603604	0.848727030414462\\
6.05605605605606	0.849517603912547\\
6.07607607607608	0.850305201437241\\
6.0960960960961	0.851089824408702\\
6.11611611611612	0.851871474315755\\
6.13613613613614	0.852650152715575\\
6.15615615615616	0.853425861233373\\
6.17617617617618	0.854198601562073\\
6.1961961961962	0.854968375461989\\
6.21621621621622	0.855735184760505\\
6.23623623623624	0.856499031351748\\
6.25625625625626	0.857259917196255\\
6.27627627627628	0.858017844320651\\
6.2962962962963	0.858772814817309\\
6.31631631631632	0.859524830844021\\
6.33633633633634	0.860273894623659\\
6.35635635635636	0.861020008443838\\
6.37637637637638	0.861763174656577\\
6.3963963963964	0.862503395677955\\
6.41641641641642	0.86324067398777\\
6.43643643643644	0.863975012129189\\
6.45645645645646	0.864706412708408\\
6.47647647647648	0.865434878394295\\
6.4964964964965	0.866160411918044\\
6.51651651651652	0.866883016072825\\
6.53653653653654	0.86760269371342\\
6.55655655655656	0.86831944775588\\
6.57657657657658	0.869033281177158\\
6.5965965965966	0.869744197014755\\
6.61661661661662	0.870452198366356\\
6.63663663663664	0.871157288389473\\
6.65665665665666	0.871859470301076\\
6.67667667667668	0.872558747377233\\
6.6966966966967	0.873255122952739\\
6.71671671671672	0.873948600420751\\
6.73673673673674	0.87463918323242\\
6.75675675675676	0.875326874896518\\
6.77677677677678	0.876011678979065\\
6.7967967967968	0.87669359910296\\
6.81681681681682	0.877372638947603\\
6.83683683683684	0.878048802248523\\
6.85685685685686	0.878722092796997\\
6.87687687687688	0.879392514439674\\
6.8968968968969	0.880060071078196\\
6.91691691691692	0.880724766668818\\
6.93693693693694	0.881386605222026\\
6.95695695695696	0.882045590802155\\
6.97697697697698	0.882701727527003\\
6.996996996997	0.883355019567452\\
7.01701701701702	0.884005471147077\\
7.03703703703704	0.884653086541763\\
7.05705705705706	0.885297870079315\\
7.07707707707708	0.885939826139073\\
7.0970970970971	0.886578959151519\\
7.11711711711712	0.887215273597893\\
7.13713713713714	0.887848774009794\\
7.15715715715716	0.888479464968796\\
7.17717717717718	0.889107351106051\\
7.1971971971972	0.889732437101899\\
7.21721721721722	0.890354727685472\\
7.23723723723724	0.8909742276343\\
7.25725725725726	0.891590941773919\\
7.27727727727728	0.89220487497747\\
7.2972972972973	0.892816032165306\\
7.31731731731732	0.893424418304597\\
7.33733733733734	0.894030038408929\\
7.35735735735736	0.894632897537907\\
7.37737737737738	0.895233000796759\\
7.3973973973974	0.895830353335933\\
7.41741741741742	0.896424960350704\\
7.43743743743744	0.897016827080767\\
7.45745745745746	0.897605958809844\\
7.47747747747748	0.898192360865277\\
7.4974974974975	0.898776038617635\\
7.51751751751752	0.899356997480304\\
7.53753753753754	0.899935242909095\\
7.55755755755756	0.900510780401836\\
7.57757757757758	0.901083615497973\\
7.5975975975976	0.901653753778167\\
7.61761761761762	0.902221200863893\\
7.63763763763764	0.902785962417039\\
7.65765765765766	0.903348044139499\\
7.67767767767768	0.903907451772778\\
7.6976976976977	0.904464191097582\\
7.71771771771772	0.90501826793342\\
7.73773773773774	0.905569688138201\\
7.75775775775776	0.906118457607831\\
7.77777777777778	0.906664582275811\\
7.7977977977978	0.907208068112833\\
7.81781781781782	0.907748921126381\\
7.83783783783784	0.908287147360325\\
7.85785785785786	0.908822752894521\\
7.87787787787788	0.909355743844412\\
7.8978978978979	0.909886126360619\\
7.91791791791792	0.910413906628548\\
7.93793793793794	0.910939090867982\\
7.95795795795796	0.911461685332684\\
7.97797797797798	0.911981696309996\\
7.997997997998	0.912499130120435\\
8.01801801801802	0.913013993117298\\
8.03803803803804	0.91352629168626\\
8.05805805805806	0.914036032244974\\
8.07807807807808	0.914543221242672\\
8.0980980980981	0.91504786515977\\
8.11811811811812	0.915549970507465\\
8.13813813813814	0.916049543827339\\
8.15815815815816	0.916546591690966\\
8.17817817817818	0.917041120699508\\
8.1981981981982	0.917533137483326\\
8.21821821821822	0.91802264870158\\
8.23823823823824	0.918509661041836\\
8.25825825825826	0.918994181219671\\
8.27827827827828	0.91947621597828\\
8.2982982982983	0.919955772088081\\
8.31831831831832	0.920432856346326\\
8.33833833833834	0.920907475576707\\
8.35835835835836	0.921379636628963\\
8.37837837837838	0.921849346378496\\
8.3983983983984	0.922316611725975\\
8.41841841841842	0.922781439596949\\
8.43843843843844	0.923243836941461\\
8.45845845845846	0.92370381073366\\
8.47847847847848	0.924161367971414\\
8.4984984984985	0.924616515675924\\
8.51851851851852	0.925069260891339\\
8.53853853853854	0.925519610684377\\
8.55855855855856	0.925967572143934\\
8.57857857857858	0.926413152380709\\
8.5985985985986	0.92685635852682\\
8.61861861861862	0.927297197735425\\
8.63863863863864	0.92773567718034\\
8.65865865865866	0.928171804055667\\
8.67867867867868	0.92860558557541\\
8.6986986986987	0.929037028973104\\
8.71871871871872	0.929466141501438\\
8.73873873873874	0.92989293043188\\
8.75875875875876	0.930317403054308\\
8.77877877877878	0.930739566676634\\
8.7987987987988	0.931159428624436\\
8.81881881881882	0.931576996240588\\
8.83883883883884	0.93199227688489\\
8.85885885885886	0.932405277933704\\
8.87887887887888	0.932816006779587\\
8.8988988988989	0.933224470830925\\
8.91891891891892	0.933630677511568\\
8.93893893893894	0.934034634260474\\
8.95895895895896	0.93443634853134\\
8.97897897897898	0.93483582779225\\
8.998998998999	0.935233079525309\\
9.01901901901902	0.935628111226292\\
9.03903903903904	0.936020930404285\\
9.05905905905906	0.93641154458133\\
9.07907907907908	0.936799961292074\\
9.0990990990991	0.937186188083416\\
9.11911911911912	0.937570232514156\\
9.13913913913914	0.937952102154647\\
9.15915915915916	0.938331804586445\\
9.17917917917918	0.938709347401968\\
9.1991991991992	0.939084738204144\\
9.21921921921922	0.939457984606073\\
9.23923923923924	0.939829094230683\\
9.25925925925926	0.94019807471039\\
9.27927927927928	0.940564933686759\\
9.2992992992993	0.940929678810163\\
9.31931931931932	0.941292317739455\\
9.33933933933934	0.941652858141625\\
9.35935935935936	0.942011307691472\\
9.37937937937938	0.942367674071272\\
9.3993993993994	0.942721964970445\\
9.41941941941942	0.94307418808523\\
9.43943943943944	0.943424351118359\\
9.45945945945946	0.943772461778727\\
9.47947947947948	0.944118527781072\\
9.4994994994995	0.944462556845655\\
9.51951951951952	0.944804556697934\\
9.53953953953954	0.94514453506825\\
9.55955955955956	0.945482499691506\\
9.57957957957958	0.945818458306858\\
9.5995995995996	0.946152418657393\\
9.61961961961962	0.946484388489821\\
9.63963963963964	0.946814375554168\\
9.65965965965966	0.947142387603459\\
9.67967967967968	0.947468432393419\\
9.6996996996997	0.947792517682163\\
9.71971971971972	0.948114651229892\\
9.73973973973974	0.948434840798596\\
9.75975975975976	0.948753094151746\\
9.77977977977978	0.949069419054004\\
9.7997997997998	0.94938382327092\\
9.81981981981982	0.949696314568639\\
9.83983983983984	0.950006900713609\\
9.85985985985986	0.950315589472289\\
9.87987987987988	0.950622388610857\\
9.8998998998999	0.950927305894927\\
9.91991991991992	0.951230349089256\\
9.93993993993994	0.951531525957467\\
9.95995995995996	0.951830844261761\\
9.97997997997998	0.952128311762641\\
10	0.952423936218629\\
};


\addplot [color=mycolor3, dashed, line width=2.0pt]
  table[row sep=crcr]{%
-14	0.00670985280736033\\
-13.959595959596	0.00684466228032823\\
-13.9191919191919	0.00698185782302241\\
-13.8787878787879	0.00712147457782473\\
-13.8383838383838	0.00726354805924046\\
-13.7979797979798	0.00740811415499802\\
-13.7575757575758	0.00755520912709164\\
-13.7171717171717	0.00770486961276555\\
-13.6767676767677	0.00785713262543864\\
-13.6363636363636	0.00801203555556806\\
-13.5959595959596	0.00816961617145061\\
-13.5555555555556	0.00832991261996065\\
-13.5151515151515	0.00849296342722324\\
-13.4747474747475	0.00865880749922107\\
-13.4343434343434	0.0088274841223341\\
-13.3939393939394	0.0089990329638105\\
-13.3535353535354	0.00917349407216764\\
-13.3131313131313	0.00935090787752167\\
-13.2727272727273	0.00953131519184475\\
-13.2323232323232	0.00971475720914827\\
-13.1919191919192	0.00990127550559092\\
-13.1515151515152	0.0100909120395103\\
-13.1111111111111	0.0102837091513771\\
-13.0707070707071	0.0104797095636693\\
-13.030303030303	0.0106789563806675\\
-12.989898989899	0.0108814930881674\\
-12.9494949494949	0.01108736355311\\
-12.9090909090909	0.0112966120231272\\
-12.8686868686869	0.0115092831260028\\
-12.8282828282828	0.0117254218690456\\
-12.7878787878788	0.011945073638376\\
-12.7474747474747	0.0121682841981215\\
-12.7070707070707	0.0123950996895241\\
-12.6666666666667	0.0126255666299535\\
-12.6262626262626	0.0128597319118294\\
-12.5858585858586	0.0130976428014486\\
-12.5454545454545	0.0133393469377171\\
-12.5050505050505	0.0135848923307849\\
-12.4646464646465	0.0138343273605847\\
-12.4242424242424	0.0140877007752697\\
-12.3838383838384	0.0143450616895526\\
-12.3434343434343	0.0146064595829426\\
-12.3030303030303	0.0148719442978809\\
-12.2626262626263	0.0151415660377715\\
-12.2222222222222	0.0154153753649079\\
-12.1818181818182	0.0156934231982945\\
-12.1414141414141	0.01597576081136\\
-12.1010101010101	0.0162624398295639\\
-12.0606060606061	0.0165535122278938\\
-12.020202020202	0.0168490303282522\\
-11.979797979798	0.0171490467967324\\
-11.9393939393939	0.0174536146407831\\
-11.8989898989899	0.017762787206259\\
-11.8585858585859	0.0180766181743578\\
-11.8181818181818	0.0183951615584425\\
-11.7777777777778	0.0187184717007472\\
-11.7373737373737	0.0190466032689672\\
-11.6969696969697	0.0193796112527299\\
-11.6565656565657	0.0197175509599493\\
-11.6161616161616	0.0200604780130587\\
-11.5757575757576	0.0204084483451257\\
-11.5353535353535	0.0207615181958446\\
-11.4949494949495	0.0211197441074077\\
-11.4545454545455	0.0214831829202544\\
-11.4141414141414	0.021851891768697\\
-11.3737373737374	0.0222259280764224\\
-11.3333333333333	0.0226053495518704\\
-11.2929292929293	0.0229902141834854\\
-11.2525252525253	0.023380580234844\\
-11.2121212121212	0.0237765062396556\\
-11.1717171717172	0.0241780509966359\\
-11.1313131313131	0.0245852735642548\\
-11.0909090909091	0.0249982332553544\\
-11.0505050505051	0.0254169896316405\\
-11.010101010101	0.0258416024980453\\
-10.969696969697	0.0262721318969607\\
-10.9292929292929	0.0267086381023428\\
-10.8888888888889	0.0271511816136866\\
-10.8484848484848	0.0275998231498716\\
-10.8080808080808	0.0280546236428766\\
-10.7676767676768	0.0285156442313644\\
-10.7272727272727	0.028982946254137\\
-10.6868686868687	0.0294565912434594\\
-10.6464646464646	0.0299366409182533\\
-10.6060606060606	0.0304231571771602\\
-10.5656565656566	0.030916202091474\\
-10.5252525252525	0.031415837897943\\
-10.4848484848485	0.0319221269914415\\
-10.4444444444444	0.0324351319175111\\
-10.4040404040404	0.0329549153647721\\
-10.3636363636364	0.0334815401572045\\
-10.3232323232323	0.0340150692462997\\
-10.2828282828283	0.034555565703083\\
-10.2424242424242	0.0351030927100065\\
-10.2020202020202	0.0356577135527137\\
-10.1616161616162	0.0362194916116762\\
-10.1212121212121	0.0367884903537019\\
-10.0808080808081	0.0373647733233165\\
-10.040404040404	0.0379484041340186\\
-10	0.0385394464594078\\
};


\addplot [color=mycolor3, dashed, line width=2.0pt]
  table[row sep=crcr]{%
10	0.952423936218629\\
10.040404040404	0.953014978544018\\
10.0808080808081	0.95359860935472\\
10.1212121212121	0.954174892324335\\
10.1616161616162	0.95474389106636\\
10.2020202020202	0.955305669125323\\
10.2424242424242	0.95586028996803\\
10.2828282828283	0.956407816974953\\
10.3232323232323	0.956948313431737\\
10.3636363636364	0.957481842520832\\
10.4040404040404	0.958008467313264\\
10.4444444444444	0.958528250760525\\
10.4848484848485	0.959041255686595\\
10.5252525252525	0.959547544780093\\
10.5656565656566	0.960047180586562\\
10.6060606060606	0.960540225500876\\
10.6464646464646	0.961026741759783\\
10.6868686868687	0.961506791434577\\
10.7272727272727	0.961980436423899\\
10.7676767676768	0.962447738446672\\
10.8080808080808	0.96290875903516\\
10.8484848484848	0.963363559528165\\
10.8888888888889	0.96381220106435\\
10.9292929292929	0.964254744575693\\
10.969696969697	0.964691250781076\\
11.010101010101	0.965121780179991\\
11.0505050505051	0.965546393046396\\
11.0909090909091	0.965965149422682\\
11.1313131313131	0.966378109113781\\
11.1717171717172	0.9667853316814\\
11.2121212121212	0.967186876438381\\
11.2525252525253	0.967582802443192\\
11.2929292929293	0.967973168494551\\
11.3333333333333	0.968358033126166\\
11.3737373737374	0.968737454601614\\
11.4141414141414	0.969111490909339\\
11.4545454545455	0.969480199757782\\
11.4949494949495	0.969843638570629\\
11.5353535353535	0.970201864482192\\
11.5757575757576	0.970554934332911\\
11.6161616161616	0.970902904664978\\
11.6565656565657	0.971245831718087\\
11.6969696969697	0.971583771425306\\
11.7373737373737	0.971916779409069\\
11.7777777777778	0.972244910977289\\
11.8181818181818	0.972568221119594\\
11.8585858585859	0.972886764503679\\
11.8989898989899	0.973200595471777\\
11.9393939393939	0.973509768037253\\
11.979797979798	0.973814335881304\\
12.020202020202	0.974114352349784\\
12.0606060606061	0.974409870450142\\
12.1010101010101	0.974700942848472\\
12.1414141414141	0.974987621866676\\
12.1818181818182	0.975269959479742\\
12.2222222222222	0.975548007313128\\
12.2626262626263	0.975821816640265\\
12.3030303030303	0.976091438380155\\
12.3434343434343	0.976356923095094\\
12.3838383838384	0.976618320988484\\
12.4242424242424	0.976875681902767\\
12.4646464646465	0.977129055317452\\
12.5050505050505	0.977378490347251\\
12.5454545454545	0.977624035740319\\
12.5858585858586	0.977865739876588\\
12.6262626262626	0.978103650766207\\
12.6666666666667	0.978337816048083\\
12.7070707070707	0.978568282988512\\
12.7474747474747	0.978795098479915\\
12.7878787878788	0.97901830903966\\
12.8282828282828	0.979237960808991\\
12.8686868686869	0.979454099552034\\
12.9090909090909	0.979666770654909\\
12.9494949494949	0.979876019124926\\
12.989898989899	0.980081889589869\\
13.030303030303	0.980284426297369\\
13.0707070707071	0.980483673114367\\
13.1111111111111	0.980679673526659\\
13.1515151515152	0.980872470638526\\
13.1919191919192	0.981062107172445\\
13.2323232323232	0.981248625468888\\
13.2727272727273	0.981432067486191\\
13.3131313131313	0.981612474800515\\
13.3535353535354	0.981789888605869\\
13.3939393939394	0.981964349714226\\
13.4343434343434	0.982135898555702\\
13.4747474747475	0.982304575178815\\
13.5151515151515	0.982470419250813\\
13.5555555555556	0.982633470058076\\
13.5959595959596	0.982793766506586\\
13.6363636363636	0.982951347122468\\
13.6767676767677	0.983106250052598\\
13.7171717171717	0.983258513065271\\
13.7575757575758	0.983408173550945\\
13.7979797979798	0.983555268523038\\
13.8383838383838	0.983699834618796\\
13.8787878787879	0.983841908100212\\
13.9191919191919	0.983981524855014\\
13.959595959596	0.984118720397708\\
14	0.984253529870676\\
};


\addplot [color=black]
  table[row sep=crcr]{%
0	0\\
0	0.495481691339018\\
};


\addplot [color=black, dotted]
  table[row sep=crcr]{%
-14	0.990963382678036\\
14	0.990963382678036\\
};


\addplot [color=black, dotted]
  table[row sep=crcr]{%
-14	0\\
14	0\\
};
\end{axis}


\end{tikzpicture}%
\caption{Probability of correct recognition over \gls{snr}, Subject Group A, the x-calibration is questionable}
\label{fig:srtn_psych}
\end{figure}



