\section{Bone transducer position}
\label{sec:bc_pos}

%The best positioning location for the \gls{bct} can either be found by subject test or literature research. Because of the project's objective, it has been chosen to do literature research on \gls{bct} positioning instead of subject test. This have been chosen such that the test of speech intelligibility can be done more thorough due to more time for test planning and more time of the test subject for the intelligibility test. The project can not offer money compensation to the test subject and therefore the time for testing the position, would than have lowered the time of speech intelligibility test. 
The best positioning location for the \gls{bct} can either be found by subject test or literature research. Due to time constraints and subject availability, it has been decided to base de decision of the placement position based on literature instead of a subject-based test.
%This section will compare earlier position test of \gls{bc} and conclude on a final position.

\subsection{\gls{bct} position comparison}
The position of the \gls{bct} is important to ensure optimal condition for the use of the transducer during  the subject-based test. The optimal position for \gls{bct} has been the subject of several studies, and it has been concluded that the most favorable locations for optimal signal transmission through \gls{bc} are the condyle and mastoid areas \citep{cat_test}. \autoref{fig:condyle_mastoid} shows the location of these two positions.
%Placement test for \gls{bc} have been research in several study and it have been shown that the condyle and mastoid position is the most favorable locations for optimal hearing \citep{cat_test}. The following \autoref{fig:condyle_mastoid} shows the  condyle and mastoid on a human.

\fig{condyle_mastoid}{Condyle and mastoid position \citep{cat_test}}{fig:condyle_mastoid}{1}

%To chose the finite position of ether condyle or mastoid, the speech intelligibility and loudness perception for both position will be included in the choose. 
In order to choose between these positions, speech intelligibility and loudness perception are taken into account. For this purpose, several studies have been taken into account. The reference to this studies and a little summary on the results can be found in  \autoref{tab:test_methoed}.

%\gls{fmcbct} \citep{freefield_method} compares the two positions for the \gls{bc} with loudness perception. The test is done by playing a tone in a speaker where the test subject shall adjust the gain and phase of the \gls{bc} until the tone from the loudspeaker is cancelled.The \autoref{tab:test_methoed} compare the speech intelligibility and loudness for condyle and mastoid.
%The article \citep{cat_test} introduce a \gls{cat} subject test, which compare  speech intelligibility with \gls{bc} position. The test is described in \autoref{ssec:cat}. \gls{fmcbct} \citep{freefield_method} compare the two position for the \gls{bc} with loudness perception. The test is done by playing a tone in a speaker where the test subject shall adjust the gain and phase of the \gls{bc} until the tone from the loudspeaker is cancelled.The \autoref{tab:test_methoed} compare the speech intelligibility and loudness for condyle and mastoid.
 
\begin{table}[H]
\caption{Results between condyle and mastoid positioning for different intelligibility and loudness tests} 
\begin{tabularx}{\textwidth}{l X l}
\hline
\citep{cat_test}: & Consists of a \gls{cat} comparing the intelligibility between several locations. It shows that there is not statistical evidence that one of the two positions has a higher speech intelligibility. However, it shows that the average speech intelligibility score is marginally better for the mastoid. \\
\citep{freefield_method}: & Based on \gls{fmcbct}, during this test a tone is played through a speaker and the subject has to adjust gain and phase in the \gls{bct} until the tone from the loudspeaker is cancelled. It shows that there is a difference in loudness perception between the two positions for a B71 \gls{bct}. This study also shows that the condyle is more sensitive to bone vibration than the mastoid, specially in the speech intelligibility frequency range from \SI{500}{\hertz} to \SI{4000}{\hertz} where the difference is approximately \SI{5}{\decibel} \citep{freefield_method}. \\
\citep{Influence_without}: & Shows that there is a difference in loudness perception between the two positions for a B71 \gls{bct}. The result of the test indicates that placement over the condylte generates a higher cochlear response, meaning that the hearing threshold is lower.\\
%This means that the hearing threshold is lower at the position of the condyle \citep{Influence_without} with respect to vibration of the B71. For the full result, see the article. \\
\citep{sensitivity_mapping}: & Shows that there is a difference in loudness perception between the two positions for a Oticon A20 \gls{bct}. The test shows that the condyle positioning is more efficient  in terms of sound transfer . \\ \hline
\end{tabularx}
\label{tab:test_methoed}
\end{table}



%\subsection{Conclusion}
%It can be concluded that the test result in \autoref{tab:test_methoed} point to that the speech intelligibility difference when using the \gls{bc} on the mastoid is only marginal better that on the condyle, and statistical it can not be concluded that the mastoid is better that the condyle. With respect to the loudness perception all referred test in this section point to that the condyle is more sensitive that the mastoid. Since the there is no statistical speech intelligibility difference between the mastoid and the condyle and the condyle have a higher sensitivity, it is concluded that the position of the \gls{bc} for the intelligibility test in this project will be on the condyle.


