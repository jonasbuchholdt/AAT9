\label{sec:test_description}
%This section design the subject test. The test design include the calibration of the \gls{acs} and the \gls{ace} as part one and the test protocol description as part to. 
In order to assess the performance of \gls{bc} and \gls{ac}, a subject-based objective test needed to be designed. This section describes the test design, as well as the calibration process for the transducers used.

%The \gls{hint} part of the subject test require that the non-weighted \gls{spl} is known for all transducer. Therefore, the first part of this description explains the procedure of measuring the \gls{spl} for \gls{acs} and \gls{ace}. The second part describe subject test protocol.


%Why the audiometry is used, how the \gls{bct} is related to the \gls{ace}, the differences between the standard use of the Danish \gls{hint} and the use of \gls{hint} in the \gls{bier}. 



\section{\gls{spl} calibration of speaker and earphones}\label{sec:cal}
The chosen test requires that the \gls{spl} is known for all transducers. In the article from which the used version of the \gls{hint} program was taken, the required playback level was A-weighted. In the \gls{hint} version used for the test, this requirement deviates from the original, as non-weighted \gls{spl} is used. Therefore, this section describes the calibration of both the \gls{acs} and the \gls{ace}. To calibrate the two transducers to a known non-weighted \gls{spl} and make the procedure reproducible, there are two ways of calibrating, a calibration based on every single test subject or a calibration based on a standardized head and torso simulator. The single subject based calibration requires a microphone at the eardrum \citep{iso_11904-1} and this is assumed to be uncomfortable of the test subject. Therefore, and in order to make the test as reproduceable as possible, the calibration in \autoref{sec:test_protocol_design} is done with a calibrated standardized head and torso simulator \citep{iso_11904-2}.

The calibration procedure for the \gls{acs} and the \gls{ace} is done in two steps. First \gls{ace} calibration and then \gls{acs} calibration. The calibration is done under the same conditions as for the test subject. This means that the head and torso simulator is placed on the same chair as the test subject. The chair is placed in the center of a standard audiometry room according to standard \citep{iso_8253-2}. The \gls{ace} are plugged into the head and torso simulator's ears and are left in during both calibration procedures. It is assumed that the calibration is performed under free field conditions, as the head and torso simulator is close to the speaker and the reflections are assumed to be low due to the nature of the selected room. 

The calibration of both transducers is done with the \gls{ssn} that is provided in the \gls{hint} program \citep{nilsson_95}. First, the \gls{ssn} is played through the \gls{ace} where the non-weighted \gls{spl} is calculated. Once this is done, the \gls{ssn} is played through the \gls{acs} and the non-weighted \gls{spl} is calculated. The calculated \gls{spl} for the \gls{ace} is adjusted in the soundcard, such that the levels for the \gls{acs} and the \gls{ace} match. Afterwards, the calibrated \gls{spl} for the \gls{acs} is used within the \gls{hint} software as the calibration data. At this point, the resulting \gls{spl}  in the \gls{hint} reflects the calibration for the head and torso simulator for both the \gls{ssn} through the \gls{acs} and the sentences through the \gls{ace}. The calibration data for the used transducers can be found in \autoref{append:cal_of_transducer}.
   
   
   
\section{Test protocol design}
\label{sec:test_protocol_design}   
%In order to assess the performance of \gls{bc} and \gls{ac}, a subject based objective test needed to be designed. However, since not all variables could be assessed from an objective perspective, part of this test relies on the subject's perception and the necessary adjustments derived from it.
%In order to assess the performance of \gls{bct} and \gls{ace}, a subject based objective test needed to be designed. 
One key constraint that needs to be taken into account for this test is time, since the procedure itself demands a high level of concentration from the subject. Therefore, considering the results of some pilot tests, it is decided to limit the total duration to a maximum of \SI{1}{\hour}. This includes an audiometry to ensure that the subjects participating do not have hearing issues, since the project addresses to communication systems for normal hearing.
 
As the main focus of this research is the assessment on intelligibility score, the endpoint of the test is to compare the \gls{hint} performance of the \gls{bct} and \gls{ace}. However, several previous steps are needed in order to assess the \gls{hint} results correctly. Therefore, the test is divided into three main parts, and will be referred to as \gls{bier} from this point onwards. 
 
 
 
\subsection{Subject acceptance questionnaire}
Previous to accepting possible test subjects, it is necessary to send them a questionnaire according to ISO 389-9:2009 \citep{iso_389-9} in order to perform a quick background check and identify possible unfit candidates. This questionnaire can be found in \autoref{apend:quest_for_hearing}.
  
\subsection{Partial audiometry}
The first part of the test is an audiometry in order to ensure normal hearing amongst the subjects. The audiometry is done according to \citep{iso_8253-2} with one exception, as the frequency range is limited to fit the frequency range of interest. Therefore, the frequencies included in the audiometry are \SI{500}{\hertz}, \SI{750}{\hertz}, \SI{1}{\kilo\hertz}, \SI{2}{\kilo\hertz}, \SI{3}{\kilo\hertz} and \SI{4}{\kilo\hertz}. This is done to not extend the \gls{bier} total duration.

This audiometry procedure is divided into tree sub-sections. The first consists on a familiarisation round with two frequencies, \SI{500}{\hertz} and \SI{1}{\kilo\hertz} through the right ear, in order for the subject to fully understand the procedure. Once this part has been completed, the audiometry for both ears is performed, starting with the left ear. The pass criteria for normal hearing is a \gls{hl} of \SI{20}{\decibel} \gls{hl} for every frequency and for both ears. The results of the audiometries for all participants can be found in \autoref{append:audiometry_result}.
   
\subsection{Level match of \gls{ace} and \gls{bct}}\label{ssec:match}
As stated previously (\autoref{sec:loudness_match}), perceived loudness matching is one of the key points in the experiment and is performed as follows for every subject. 

The Affinity is set up with the matching settings as shown in \autoref{apend:aff_bc_ac_match}. The program automatic alternates playing \gls{ssn} between the \gls{bct} and \gls{ace} with a \SI{50}{\percent} duty cycle and a period of \SI{4}{\second}. The \gls{bct} is held in place using a rubber headband with of approx. \SI{4.23}{\newton} (see \autoref{ax:force}). The test subject is asked to judge the level for the \gls{bct} compared to the \gls{ace} every time the sound is \gls{bc}, since the alternating cycles keep repeating without stopping. The level for the \gls{bct} starts at \SI{30}{\decibel} and is adjusted in uprising \SI{5}{\decibel} steps until the same \gls{bct} level has been marked as a match twice. E.g. The test subject asks to adjust upwards from \SI{45}{\decibel} to \SI{50}{\decibel} and then asks to adjust downwards again to \SI{45}{\decibel}. The subjects do not know the step size or level during this procedure. 
After the same level has been reached twice, the step size is reduced to \SI{2}{\decibel}. The matching procedure continues in this mode until the subject matches the level of the transducer twice at the same value. The match level is noted and the  \gls{bct} level is adjusted \SI{5}{\decibel} up. The matching test is continued with a level step of \SI{1}{\decibel}.

When the subject matches the same level two times again, the level is noted and this part of the test is concluded. The number of matching tests is done according to the test procedure with one exception. If the two noted matches for one round are the same level, and that level is present in an earlier matching round for the same subject, the matching test part of \gls{bier} is stopped. 

The mean of all non familiarization resulta is calculated to one single value which is used as the \gls{bct} level for the \gls{hint}. However, there is an exception for the mean calculation. Those noted levels that are at least \SI{5}{\decibel} lower than all other noted level for one subject, are treated as false positives and will be discarded before the single final level is calculated. 

%The match procedure is divided into three sub-sections, first one being familiarisation with at least three match. Next one or two round would consist on at least 2 match and possible up to 4 match, depending on the subject's consistency and the conductors assessment. 


\subsection{\gls{hint} in \gls{bier}}\label{ssec:hint_in_bier}
The last part of the \gls{bier} corresponds to the performance of the modified \gls{hint}. As explained in section \autoref{ssec:hint}, during \gls{hint}, the subject is presented 20 sentences contained in a predefined list as well as \gls{ssn}. For the classic \gls{hint}, both the speech and the \gls{ssn} are presented through the same transducer. However, in order to maintain the same ear canal conditions for the different transducers used in \gls{bier}, the \gls{ssn} is presented through a speaker located inside the room with the subject, and the speech is presented through the  \gls{ace} or the \gls{bct}. By doing this, the blocked ear canal condition that is present during the \gls{ace} use is still maintained while using the \gls{bct}.

This \gls{hint} procedure is divided into five sub-sections, a familiarisation round and four test rounds. During the familiarisation round, half of the sentences are presented through the \gls{ace} and the other half through \gls{bct}. However, for the test rounds, each whole set of sentences is presented through the same transducer. The noise level is maintained the same for the duration of the whole experiment, both inter-subjects and inter-rounds.

The Danish \gls{hint} provides three familiarisation lists, as well as ten sentences lists for \gls{srtn} measurement. The assignation of the sets to each round/subject is done in a randomized way, not allowing list repetition within the subject's test. This also applies for familiarisation lists. The selection of \gls{ac} or \gls{bc} transducers is also randomized, not allowing more than two repetitions of the same transducer throughout the whole test. Therefore, each subject would perform two \gls{ac} rounds and two \gls{bc} rounds, all of them using different lists. The full test protocol can be found in \autoref{apend:test_protocol}. A description of the necessary settings on the Affinity is given in \autoref{apend:aff_bc_ac_match}.


\subsection{The \gls{bier} test setup}
The setup is illustrated in \autoref{fig:sec_bier_setup}.
Additionaly, a list of equipment can be found in \autoref{ax:bier_setup}.

\newpage
\begin{figure}[H]
	\begin{sideways}
	\begin{minipage}{\textheight}
		\centering
\def\svgwidth{\columnwidth}
\input{figures/design/bier_setup.pdf_t}
	\end{minipage}
	\end{sideways}
\caption{The figure shows the setup for \gls{bct} versus \gls{ace} matching and the \gls{hint} }
\label{fig:sec_bier_setup}
\end{figure}


\begin{table}[H]
\centering
\caption{The playback for match test and \gls{hint} in \gls{bier}}
\begin{tabular}{l|lll}
Test playback   & \gls{acs}   & \gls{ace} & \gls{bct} \\ \hline
Match mode      & Not playing & \gls{ssn} & \gls{ssn} \\
\gls{hint} mode & \gls{ssn}   & sentences & sentences
\end{tabular}
\label{sec:playback_bier}
\end{table}


















