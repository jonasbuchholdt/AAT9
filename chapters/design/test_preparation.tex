

\section{Test description}
\label{sec:test_description}
This section design the subject test. The test design include the calibration of the \gls{acs} and the \gls{ace} as part one and the test protocol description as part to. 


%The \gls{hint} part of the subject test require that the non-weighted \gls{spl} is known for all transducer. Therefore, the first part of this description explains the procedure of measuring the \gls{spl} for \gls{acs} and \gls{ace}. The second part describe subject test protocol.


%Why the audiometry is used, how the \gls{bct} is related to the \gls{ace}, the differences between the standard use of the Danish \gls{hint} and the use of \gls{hint} in the \gls{bier}. 



\section{\gls{spl} calibration of speaker and earphone}
The chosen \gls{hint} for the subject test require that the non-weighted \gls{spl} is known for all transducer. Therefore, this section described the calibration of both the \gls{acs} and the \gls{ace}. To calibrate the two transducer to a known non-weighted \gls{spl} and make the procedure reproducible, there is two ways of calibration. A calibration based on every single test subject or a calibration based on a standardized head and torso simulator. The calibration for the subject test in \autoref{sec:test_protocol_design}  is done with a calibrated standardized head and torso simulator \citep{iso_11904-2}, because the single subject based calibration require a microphone at the eardrum \citep{iso_11904-1}. The calibration procedure for the \gls{acs} and the \gls{ace} is done in two steps. First \gls{ace} calibration and then \gls{acs} calibration. The calibration is done under the same condition as for the test subject. This means that the head and torso simulator is placed on the same chair as the test subject. The chair is placed in the center for the room, and the used room is a standard audiometry room according to standard \citep{iso_8253-2}. That the \gls{ace} is plugged intro the head and torso simulators ears under both calibration, and the \gls{ace} is not replugged between the calibration. It is furthermore assumed that the calibration is done under free field condition, because the head and torso simulator is close to the speaker and the reflection from the surface is assumed to be low because of standard of the room. 

The calibration of both transducer is done with the \gls{ssn} that is used in the \gls{hint} \citep{nilsson_95}. First the \gls{ssn} is played in the \gls{ace} where the non-weighted \gls{spl} is calculated. Afterwards the \gls{ssn} is played in the \gls{acs} and the non-weighted \gls{spl} is calculated. The calculated \gls{spl} for the \gls{ace} is adjusted in the sound card, such that the level for the \gls{acs} and teh \gls{ace} match in \gls{spl}. Afterwards the calibrated \gls{spl} for the \gls{acs} is pasted intro the \gls{hint} as the calibration data. At this point the resulting \gls{spl}  in the \gls{hint} reflect the calibration for the head and torso simulator for both the \gls{ssn} through the \gls{acs} and the sentences through the \gls{ace}. The calibration data for the used transducer can be founded in ?? %\autoref{}  
   
   
   
\section{Test protocol design}
\label{sec:test_protocol_design}   
%In order to assess the performance of \gls{bc} and \gls{ac}, a subject based objective test needed to be designed. However, since not all variables could be assessed from an objective perspective, part of this test relies on the subject's perception and the necessary adjustments derived from it.
This section design the test protocol for the subject test. In order to assess the performance of \gls{bct} and \gls{ace}, a subject based objective test needed to be designed. One key constraint that needed to be taken into account for this test is time, since the procedure itself demands a high level of concentration from the subject. Therefore, and after performing some pilot tests, it was decided to limit the total duration to maximum \SI{1}{\hour}. The test will include an audiometry to ensure normal hearing, since the project addresses to communication system for normal hearing.
 
As the main focus of this research is the assessment on intelligibility score, the endpoint of the test is to compare the \gls{hint} performance of the \gls{bct} and \gls{ace}. However, several previous steps are needed in order to assess the \gls{hint} results correctly. Therefore, the test is divided into three main parts ??, and will be referred to as \gls{bier}. 
 
 
 
 \subsection{Subject acceptance questionnaire}
Previous to accepting possible test subjects, it is necessary to send them a questionnaire according to ISO 389-9:2009 \citep{iso_389-9} in order to perform a quick background check and identify possible unfit candidates. This questionnaire can be found in \autoref{{apend:quest_for_hearing}}.
  
\subsection{Partial audiometry}
The first part of the test consisted of a reduced audiometry in order to ensure normal hearing amongst the participants. As a first approach, a full audiometry is proposed, but was later discarded since it considerably increased the duration of the test and its difficulty, since it demands a lot of focus from the participants. Furthermore, the added data was not of use for the experiment due to the bandwidth limitation imposed by the speech intelligibility frequency range. Therefore, a reduced version is selected, containing \SI{500}{\hertz}, \SI{750}{\hertz}, \SI{1}{\kilo\hertz}, \SI{2}{\kilo\hertz}, \SI{3}{\kilo\hertz} and \SI{4}{\kilo\hertz}.

This audiometry procedure is divided into five sub-sections, with a little break between each. The duration of the break was subject-dependant and no longer than 30 seconds.
The first sub-section consisted on a familiarisation round with two frequencies, \SI{500}{\hertz} and \SI{1}{\kilo\hertz} through the right ear in order for the participant to fully understand the procedure. Once this part had been completed, the audiometry for both ears is performed, starting with the left ear.  
   
\subsection{Level match of \gls{ace} and \gls{bct}}
The perceived level matching is done according to \autoref{{sec:loudness_match}} and as following for every subject. The Affinity is put intro the match settings as shown in \autoref{apend:aff_bc_ac_match}. The program automatic alternate between playing \gls{ssn} in the \gls{bct} and \gls{ace} with a \SI{50}{\percent} duty cycle and a period of four second. The test subject is asked to judge the level for the \gls{bct} compare to the \gls{ace} at every time cycle. The level for the \gls{bct} starts at \SI{30}{\decibel} and is adjusted by \SI{5}{\decibel} for every judgement until the same \gls{bct} level have been passed twice. E.g. The test subject ask to adjust upwards from \SI{45}{\decibel} to \SI{50}{\decibel} and then ask to adjust downwards again to \SI{45}{\decibel}. The subject do not know the step size or any number doing the test. After the twice same \gls{ssn} level, the step size is lowered to \SI{2}{\decibel}. The matching procedure continues in this mode until the subject match the level of the transducer twice at the same value. The match level is noted and the  \gls{bct} level is adjusted \SI{5}{\decibel} up. The matching test is continued with a level step of \SI{1}{\decibel}. When the subject match the same level two times again, the level is noted and the test is ended. The number of matching test is done according to the test procedure with one exception. If the two noted match for one match test is the same level and the level is present in an earlier matching test for the same subject, the matching test part of \gls{bier} is stopped. The mean of all non familiarization result is calculated to one single match level which is used as the \gls{bct} level for the \gls{hint}. In the mean calculation there is one exception. Those noted level that is at least \SI{5}{\decibel} lower that all other noted level for one subject is threaded as false positive, and will be discarded before the single level is calculated for \gls{bct}. 

The match procedure is divided into three sub-sections, first one being familiarisation with at least three match. Next one or two round would consist on at least 2 match and possible up to 4 match, depending on the subject's consistency and the conductors assessment. 


\subsection{\gls{hint} in \gls{bier} test}
The last part of the \gls{bier} corresponds to the performance of the \gls{hint}. As explained in section \autoref{ssec:hint}, during \gls{hint} the subject is presented 20 sentences contained in a predefined list as well as \gls{ssn}. For the classic \gls{hint}, both the speech and the \gls{ssn} is presented through the same transducer. However, in order to maintain the same ear canal conditions for the different transducers used in \gls{bier}, the \gls{ssn} is presented through a speaker located inside the room with the subject, and the speech is presented through the  \gls{ace} or the \gls{bct} transducer. By doing this, the blocked ear canal condition that is present during the headphones use was still maintained while using the \gls{bct}.


This \gls{hint} procedure is divided into five sub-sections, a familiarisation round and four test rounds. During the familiarisation round, half of the sentences were presented through the \gls{ace} and the other half through \gls{bct}. While for the test rounds, each whole set of sentences is presented through the same transducer. The noise level was maintained the same for the duration of the whole experiment, both inter-subjects and inter-rounds.

The Danish \gls{hint} provides three familiarisation lists, as well as ten sentences lists for \gls{srtn} measurement. Assigning the five sets to each \gls{bier} was done in a randomized way, not allowing list repetition within the subject's test and as well randomized for familiarisation lists. The selection of \gls{ac} or \gls{bc} transducers was also randomized, not allowing more than two repetitions of the same transducer throughout the whole test. Therefore, each subject would perform two \gls{ac} rounds and two \gls{bc} rounds, all of them using different lists. The full test protocol can be founded in \autoref{apend:test_protocol}.




\startexplain

\explain{1}{test hallo hallo test this is at test ?? yes yes this is a test hallo hallo hallo test test does it work??????? i really need to know..... ?????}{1}

\stopexplain  

















