\section{Test preparation}
This section design the \gls{bier} test. The design described all step from calibration of equipment to the final \gls{bier} test. The \gls{hint} part of the \gls{bier} test require that the non-weighted \gls{spl} is known for all transducer. Therefore, the first part of this description will explain the procedure of measuring the \gls{spl} for \gls{acs} and \gls{ace}. The second part will describe how the \gls{ace} will be related to the \gls{bc}. The third part will describe the differences between the standard use of the Danish \gls{hint} and the use of \gls{hint} in this the \gls{bier}. The last part will describe the test procedure for the subject. 





\subsection{\gls{spl} calibration of speaker and earphone}
This section described the calibration of both the \gls{acs} and the \gls{ace}. To calibrate the \gls{acs} and the  \gls{ace} to a known non-weighted \gls{spl} and make the procedure reproducible, there is at least two ways of calibration. A calibration based on every single test subject or a calibration based on a standardized head and torso simulator. The calibration in the \gls{bier} test will be done with the standardized head and torso simulator, because the single subject based calibration require a microphone at the eardrum. The calibration procedure for the \gls{acs} and the \gls{ace} will be done in two steps. First \gls{ace} calibration and then \gls{acs} calibration. The calibration is be done under same condition for both transducers

To calibrate the \gls{acs} and the \gls{ace} 
   

