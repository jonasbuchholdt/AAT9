


\section{Test preparation}
This section design the \gls{bier} test. The design describe all step from calibration of \gls{spl} of \gls{acs} and  \gls{ace} to the final \gls{bier} test. The \gls{hint} part of the \gls{bier} test require that the non-weighted \gls{spl} is known for all transducer. Therefore, the first part of this description is explaining the procedure of measuring the \gls{spl} for \gls{acs} and \gls{ace}. The second part is describing how the \gls{ace} is related to the \gls{bct}. The third part is describing the differences between the standard use of the Danish \gls{hint} and the use of \gls{hint} in the \gls{bier}. The last part is describing the test procedure for the subject. 





\subsection{\gls{spl} calibration of speaker and earphone}
This section described the calibration of both the \gls{acs} and the \gls{ace}. To calibrate the \gls{acs} and the \gls{ace} to a known non-weighted \gls{spl} and make the procedure reproducible, there is at least two ways of calibration. A calibration based on every single test subject or a calibration based on a standardized head and torso simulator. The calibration in the \gls{bier} test is done with a calibrated standardized head and torso simulator, because the single subject based calibration require a microphone at the eardrum. The calibration procedure for the \gls{acs} and the \gls{ace} is done in two steps. First \gls{ace} calibration and then \gls{acs} calibration. The calibration is done under the same condition as for the test subject . This means that the head and torso simulator is placed on the same chair as the test subject. The chair is placed in the center for the room, and the used room is a listening room according to standard REFMISSING??. It also mean that the \gls{ace} is plugged intro the head and torso simulators ears under both calibration, and the earphone is not replugged between the calibration. It is furthermore assumed that the calibration is done under free field condition, because the head and torso simulator is close to the speaker, the room is acoustical damped and the frequency for interest is above \SI{355}{\hertz}. 

The calibrating of both transducer is done with the noise that is used in the \gls{hint}. First the noise is played in the \gls{ace} where the non-weighted \gls{spl} is calculated. Afterwards the noise is played in the \gls{acs} and the non-weighted \gls{spl} is calculated. The calculated \gls{spl} for the \gls{ace} is adjusted in the sound card, such way that the level for \gls{acs} and \gls{ace} match in \gls{spl}. Afterwards the calibrated number for the \gls{acs} is pasted intro the \gls{hint} as the calibration data. At this point the resulting \gls{spl} number in the \gls{hint} reflect the calibration for the head and torso simulator for both the noise through the \gls{acs} and the sentences through \gls{ace}. The calibrating for the used transducer can be founded in ??%\autoref{}  
   
\subsection{Level match of \gls{ace} and bone}
This section described how the \gls{ace} is related to the \gls{bct} in the \gls{bier} test. The main issue with calibrating of the \gls{bct} is that the the sound is not \gls{ac} but \gls{bc}, so the \gls{bct} level cannot be described as \gls{spl}. Therefore the relation between \gls{ace} and \gls{bct} is founded another way than the \gls{spl}. There is at least two ways of relating the \gls{bct} with the \gls{ace} while using the Affinity. First option is using the standardized calibration of the \gls{bct} to the hearing threshold and assume linearity in the used level range. Secondly match the level of \gls{bct} to the \gls{ace} for every subject by a perceived level matching test.

In the \gls{bier} test the subject based method is chosen. The resent for chosen the subject based method is that the standard ISO 389-3 \citep{iso_389-3} only describe calibration with pure tone and the Affinity do not have any calibration for \gls{bct} while using external input on CD1 or CD2 \autoref{append:affinity_bc_calibration}. Therefore an external calibrating unit between the sound card and the Affinity input is needed, but since a full frequency spectrum calibration standard is not defined this option is discarded.

The perceived level matching is done as following for every subject. The Affinity is put intro the match settings as shown in \autoref{apend:aff_bc_ac_match}. The program automatic alternate between playing noise in the \gls{bct} and \gls{ace} with a \SI{50}{\percent} duty cycle and a period of four second. The test subject is asked to judge the level for the \gls{bct} compare the \gls{ace} at every time cycle. The level for the \gls{bct} starts at \SI{30}{\decibel} and is adjusted by \SI{5}{\decibel} for every judgement until the same \gls{bct} level have been passed twice. E.g. The test subject ask to adjust upwards from \SI{45}{\decibel} to \SI{50}{\decibel} and then ask to adjust downwards again to \SI{45}{\decibel}. The subject do not know the step or any number doing the test. After the twice same hit, the step size is lowered to \SI{2}{\decibel}. The matching procedure continues in this mode until the subject match the level of the transducer. The matched level is noted and the level is adjusted \SI{5}{\decibel} up for the \gls{bct}. The matching test is continued with level step of \SI{1}{\decibel}. When the subject match the level again the level is noted again and the test is over. The number of matching test is done according to the test procedure with one exception. If the two match for one matching test is the same level and the level is present in an earlier matching test for the same subject, the matching test part of \gls{bier} is stopped. The mean of all non familiarization result is calculated to one single match level which is used as the \gls{bct} level. For the mean calculation there is one exception. The lowest result of all run that is lower that the lowest familiarization level match is discarded and is not used in the mean calculation. Those result is threaded as false positive.









\section{Test intro}


