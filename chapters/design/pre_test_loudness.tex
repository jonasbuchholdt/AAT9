
\section{Loudness}


This section described how the \gls{ace} is related to the \gls{bct} in the subject test. The main issue with calibrating of the \gls{bct} is that the the sound is not \gls{ac} but \gls{bc}, so the \gls{bct} level cannot be described as \gls{spl}. Therefore the relation between \gls{ace} and \gls{bct} is founded another way than the \gls{spl}. There is at least two ways of relating the \gls{bct} with the \gls{ace} while using the Affinity. First option is using the standardized calibration of the \gls{bct} to the hearing threshold and assume linearity in the used level range. Secondly using the \gls{hint} \gls{ssn} \citep{nilsson_95} to match level of \gls{bct} to the level of \gls{ace} for every subject by a perceived level matching test.

In the subject test the subject based method is chosen. The resent for chosen the subject based method is that the standard ISO 389-3 \citep{iso_389-3} only describe calibration with pure tone and the Affinity do not have any calibration for \gls{bct} while using external input on CD1 or CD2 \autoref{append:affinity_bc_calibration}. Therefore an external calibrating unit between the sound card and the Affinity input is needed, but since a full frequency spectrum calibration standard is not defined, this option is discarded.

In order to decide which type of signal should be used during this matching phase, a test with several signals was performed (see \autoref{apend_matching_signals}).