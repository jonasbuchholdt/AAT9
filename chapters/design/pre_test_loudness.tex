
\section{Loudness matching between \gls{bct} and \gls{ace}}
\label{sec:loudness_match}
This section is divided intro two parts. One parts where the noise is chosen and one part where the method is examined with pilots.


\subsection{The transducer relating method}

This section described how the \gls{ace} is related to the \gls{bct} in the subject test. The main issue with the frequency response of the \gls{bct} is that the the sound is not \gls{ac} but \gls{bc}, so the \gls{bct} level cannot be described as \gls{spl}. The chosen \gls{bct} belongs to a Hearing Aid Analyzer named Affinity 2.0 \citep{affinity_20} which also can be used as preamp for the \gls{bct} and the \gls{ace} in two inputs, CD1 and CD2. Since the given \gls{hint} is a \matlab program, the Affinity will be used as a preamp for the \gls{bct}. The Affinity is not only for bone but do also have  \gls{ace} posibility. Therefore the Affinity is used as a preamp for both the \gls{ace} and the \gls{bct}. 


To relate the \gls{ace} and \gls{bct} the CD1 and CD2 input frequency response of single tone is measured. This is done to to verify the input and output relation such that it is known if the Affinity change the frequency response for the belonging \gls{bct} as it is necessary according to its datasheet \citep{radioear_b81}. In  \autoref{append:affinity_bc_calibration} it is found out that the Affinity have a linear frequency response, and therefore the \gls{bct} is not adapted for the needed frequency equalization as standardized in ISO 389-3 \citep{iso_389-3}. The standard ISO 389-3 \citep{iso_389-3} also only standardize for pure tone at the Mastoid and not wide band frequency content at the which is contained in speech. Those the relating between the input to the \gls{bct} and the basilar membrane exaltation is uncertain because of the position and the frequency correction. The solution is relating the \gls{bct} to the \gls{ace} by perceived level matching test between the transducers for every subject. The test method is described in \autoref{sec:test_description}.


\subsection{Matching signal chose}
\label{sec:match_sig_cho}
This section design the matching signal for \gls{ace} and \gls{bct} relation. There is several option of choosing the matching signal but tonal match can not be used because the transfer function for both transducer is very difference \citep{microPro_er4},  \citep{radioear_b81}. The tonal match only correspond to its frequency and therefore over a broad band frequency spectrum the loudness will not match. To match the level in the speech frequency range a \gls{ssn} is chosen. The chosen \gls{ssn} is the given noise from the \matlab \gls{hint} test. The differences in the low frequency for the transducer might introduce that the \gls{ace} is perceived louder and therefore the perceived match with the \gls{bct} might be to high. To test this hypotheses, a test is made in \autoref{apend_matching_signals}. The is also done with other frequency band for the \gls{ssn} and in the end with pure tone to confirm the problem with difference transfer function. The test is done according to the test method described in \autoref{sec:test_description}. The following \autoref{sec:match_result_diff} shows the result of three test subject where the mean is calculated.

\begin{table}[H]
\centering
\caption{Test result of matching test for different noise spectrum and pure tone. The specification of each noise can be founded in \autoref{apen:match_result}}
\begin{tabular}{l|lll|ll}
Matching test in \si{\decibel}       & 1\textsuperscript{st}  Subject & 1\textsuperscript{nd}  Subject & 3\textsuperscript{rd}  Subject & $\mu$ & $\sigma$ \\ \hline
Original \gls{ssn}         & 46        & 45        & 45        & 45.3        & 0.58               \\
Highpass         & 47        & 44        & 46        & 45.7        & 1.53               \\
Bandpass 1         & 38        & 40        & 45        & 41          & 3.61               \\
Bandpass 2        & 41        & 41        & 47        & 43          & 3.46               \\
1/3 octave & 57        & 50        & 60+       & 55.7        & 5.13               \\
\SI{1}{\kilo\hertz}                   & 34        & 32        & 36        & 34          & 2                  \\
\SI{2}{\kilo\hertz}                  & 45        & 51        & 60+       & 52          & 7.55               \\
\SI{4}{\kilo\hertz}                  & 47        & 49        & 59        & 51.7        &  6.43                 
\end{tabular}
\label{sec:match_result_diff}
\end{table}


It can be seen in \autoref{sec:match_result_diff} that the difference between original \gls{ssn} and and the highpassed \gls{ssn} is very small. The hypotheses that the low frequency makes a higher \gls{bct} level match is rejected for the measured data. All banpass filtering of the noise was ether not as consistent that the original \gls{ssn}. The pure tone also have high standard diviation and tha perceived level is much difference than the Original \gls{ssn}. The chosen matching noise the the subject test is the original \gls{ssn}



\subsection{Pilot test}
\label{sec:pilot_test}
This section evaluate the chosen \gls{ssn} for \gls{ace} and \gls{bct} matching. The evaluation is done by a pilot test with subject. The test is done according to the matching description in \autoref{sec:test_description} but with only one match test after the familiarization round. Only the non familiarization matching is reported. The result for the pilot test is shown in \autoref{sec:match_result_field} and the corresponding appending of the test is founded in \autoref{apend:match_field_init}.

\begin{table}[H]
\centering
\caption{Test result for pilot level matching}
\begin{tabular}{l|lll|ll}
Matching test in \si{\decibel}   & 1\textsuperscript{st} trial & 2\textsuperscript{nd} trial & 3\textsuperscript{rd} trial & $\mu$ & $\sigma$ \\ \hline
Subject 1  & 48          & 48           & 48          & 48          & 0                  \\
Subject 2  & 52          & 50           & 50          & 50.7        & 1.15               \\
Sbuject 3  & 52          & 52           & 52          & 52          & 0                  \\
Subject 4  & 46          & 46           & ND          & 46          & 0                  \\
Subject 5  & 50          & 48           & ND          & 49          & 1.41               \\
Subject 6  & 52          & 50           & ND          & 51          & 1.41               \\
Subject 7  & 52          & 48           & ND          & 50          & 2.83               \\
Subject 8  & 56          & 52           & 52          & 53.3        & 2.31             
\end{tabular}
\label{sec:match_result_field}
\end{table}

From the result in \autoref{sec:match_result_field} the standard deviation between subject is calculated to be \SI{2.3}{\decibel} and the mean is \SI{50}{\decibel}. The mean and the actual \si{\decibel} for the subject data is not represent the \gls{spl} the subject is exposed to. It is only the \si{\decibel} number the Affinity shows in the display. As stated in the first part of this section, The Affinity  \gls{spl} level in not related to the exposure. Therefore, the mean \gls{bc} level for every subject correspond to there perception of non-weighted \SI{65}{\decibel} \gls{spl}  for the \gls{ace}. The standard deviation between subject result can on the other hand be compared to other loudness perception study. The comparison can indicate if the pilot test result is within the standard deviation as other reviewed research have founded as loudness matching standard deviation. 

In \citep{STENFELT201385} the loudness is measured between to transducer, one Sennheiser HDA200 and two custom made \gls{bct}. The subject test is not done as the pilot test but as an \gls{acalos}. An \gls{acalos} is a test where there is a sociology loudness scale the subject are asked to rate the sound after, e.g. soft, medium and high \gls{spl}. The test in \citep{STENFELT201385} is not done as a matching test between two transducer directly but indirectly. Meaning that the resulting loudness scale rate between the to transducer is compared. The noise there is used in \citep{STENFELT201385} is ether not \gls{ssn} but filtered Gaussian noise. The noise was filtered intro to signal where the test subject is asked to rate both signal on both transducer. The noise band is as following. The first noise is filtered from \SI{600}{\hertz} to \SI{900}{\hertz} and the second noise is filtered from \SI{3}{\kilo\hertz} to \SI{4}{\kilo\hertz}. Since the test is done from inaudible to too loud there is displayed a data fit from \SI{0}{\decibel} to \SI{80}{\decibel}. The pilot test is done with \SI{63.87}{\decibel} \gls{spl} non-weighted, then only the data that correspond to the same level is needed to be compared. The following \autoref{sec:steinflet_result} shows the obtained data from \citep{STENFELT201385} at \SI{63.87}{\decibel} SL.


\begin{table}[H]
\centering
\caption{obtained data from \citep{STENFELT201385} at \SI{63.87}{\decibel} SL}
\begin{tabular}{l|l}
Loudness scale test                        & $\sigma$            \\ \hline
\SI{600}{\hertz} to \SI{900}{\hertz}       & \SI{7.5}{\decibel} \\
\SI{3}{\kilo\hertz} to \SI{4}{\kilo\hertz} & \SI{7.5}{\decibel}
\end{tabular}
\label{sec:steinflet_result}
\end{table}

From the pilot test \autoref{sec:match_result_field} the standard deviation was calculated to be \SI{2.3}{\decibel} where the study in \citep{STENFELT201385} have a mean standard deviation of \SI{7.5}{\decibel}. The test is not exactly preformed the same way, but it is chosen as the reference because very rare research in this field. The pilot test standard deviation is much lower that the result in \citep{STENFELT201385} so the test is concluded to be feasible for the subject test in the this project.

%Loudness functions with air and bone conduction stimulation in normal-hearing subjects using a categorical loudness scaling procedure