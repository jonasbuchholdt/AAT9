In order to assess the performance of \gls{bc} and \gls{ac}, an objective test needed to be designed. However, since not all variables could be assessed from an objective perspective, part of this test relies on the subject's perception and the necessary adjustments derived from it.
\section{Test Description}

As the main focus of this research is the assessment on intelligibility score, the endpoint of it is to compare both types of transducers' performance from the \gls{hint}. However, several previous steps are needed in order to assess the \gls{hint} results correctly. Therefore, the test was divided into three main parts (with an extra step for subject acceptance).

\subsection{Subject acceptance questionnaire}
Previous to accepting possible test subjects, it was necessary to send them a questionnaire according to ISO 389-9:2009REFMISSING	in order to perform a quick background check and identify possible unfit candidates.
This questionnaire can be found in ANNEXREFMISSING.

\subsection{Partial audiometry}
The first part of the test consisted of a reduced audiometry in order to ensure normal hearing amongst the participants. As a first approach, a full audiometry was proposed, but was later discarded since it considerably increased the duration of the test and its difficulty, since it demands a lot of focus from the participants. Furthermore, the added data was not of real use for the experiment due to the bandwidth limitation imposed by the bone transducer.
Therefore, a reduced version was selected, containing the following frequencies: [INSERT FREQS].

This part of the test was divided into three subparts, with a little break between them. The duration of the break was subject-dependant and no longer than 30 seconds.
The first subpart consisted on a familiarisation round with two frequencies (\SI{500}{\hertz} and \SI{1}{\kilo\hertz}) through the right ear in order for the participant to fully understand the procedure. Once this part had been completed, the audiometry for both ears was performed, starting with the left one.

\subsection{Level matching}
One of the key factors in the test was to maintain the conditions for both transducers as close as possible so the comparison could be made. However, when designing the test, a challenging point was pinpointed, since it was necessary that the loudness perceived from both transducers matched.


\subsection{\gls{hint}}
