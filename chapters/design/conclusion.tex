\section{Conclusion}
Before being able to design a properly fitted test to fulfill the contents of the project, it was needed to analyze several parameters such as type of test, used transducers, position of the \gls{bct}, etc. This led to several conclusions and considerations in terms of test execution.

Since the project is oriented to an application in communication systems, it is decided that the intelligibility test in which the routine is based is \gls{hint}. As the scope of the application is not yet delimited within the telecommunications field, the choice has been based on versatility and the assesment of being a more complete test. After analyzing the results of the different studies on \gls{bct} intelligibility depending on the position, it is decided to place the \gls{bct} on the condyle.

In terms of the transducers, it is decided to use the B81 \gls{bct} because of its better frequency response and less distortion compared to the B71. For the \gls{ac} transducer choice, it is decided to use in-ear headphones mainly due to the fact that the conditions for the test of both transducers need to be maintained throughout the whole test, and combining the \gls{bct} with circumaural or supra-aural headphones was proven to not be feasible.

