\section{Conclusion of Considerations for Testing Intelligibility}
In order to design a test procedure suitable to the objective of the project at hand, several aspects have been analysed. These included different intelligibility tests to choose from and model and positioning of a \gls{bct}.
Based on the preceding literatur research, choices are derived.
%Before being able to design a properly fitted test to fulfill the contents of the project, it was needed to analyze several parameters such as type of test, used transducers, position of the \gls{bct}, etc. This led to several conclusions and considerations in terms of test execution.
Since the project is aimed at assessing speech intelligibility in context of a general telecommunications application, the \gls{hint} is chosen to obtain data for a  comparison of the intelligibility of \gls{ac} and \gls{bc} sound.
Based on literature, it is also decided to place the \gls{bct} at the condyle.

%Since the project is oriented to an application in communication systems, it is decided that the intelligibility test in which the routine is based is \gls{hint}. As the scope of the application is not yet delimited within the telecommunications field, the choice has been based on versatility and the assesment of being a more complete test. After analyzing the results of the different studies on \gls{bct} intelligibility depending on the position, it is decided to place the \gls{bct} on the condyle.

In terms of the transducers, it is decided to use the Radioear B81 \gls{bct} because of a wider frequency response and less distortion compared to the Radioear B71. For the \gls{ac} transducer choice, it is decided to use in-ear headphones mainly due to the fact that the conditions for the test of both transducers need to be maintained throughout the whole test, and combining the \gls{bct} with circumaural or supra-aural headphones was proven to not be feasible. Furthermore, with this type of transducer it is possible to replicate a situation in which the user is wearing earplugs, which fits with the objective of the test.