\section{Air conductors}
\label{sec:aircon}
There are mainly three types of \gls{ac} transducers which can be used in the intelligibility test.  %The three type and there sub types will be explained and compared in this section. In the end of the section the \gls{ac} for the subject test will be chosen.
\begin{itemize}
\item Circumaural: this type of transducer completely covers the ear, enclosing it within the frame of the headphones. It can be open or closed, meaning that the frame containing the drivers can be completely sealed(closed) or allow airflow(open). This type of transducer is the largest between the three presented.
\item Supra-aural:  this type of transducer lays on top of the ear, not enclosing it and therefore not blocking all noise generated by external sources. As circumaural headphones, supra-aural can also be open or closed.
\item Insertion: this type of transducer is inserted into the ear, either into the outer part (earphones) or into the ear canal (in-ear headphones). Earphones do not isolate as much as in-ear headphones from external noise since the fitting is not as tight and they do not block the ear canal opening completely. It is the smallest transducer between the three types described.
\end{itemize}
%\subsection{Circumaural \gls{ac}} \label{circumaural_ac}
%Circumaural refers to a type of \gls{act} that covers the full ear. There exist manly to type of circumaural \gls{ac}, active and passive. An active circumaural \gls{ac} is an \gls{ac} with built in active noise reduction, where the passive do not have any active noise reduction but only passive noise reduction. Active noise reduction a technique where an adaptive filter is used to to reduce the noise there is ratiated throug the passive noise reduction of the circumaural \gls{ac} by playing the noise in \SI{180}{\degree} phase. This manly works in the low frequency area due to the long wave length. Passive noise reduction is the reduction the mass and construction of the circumaural \gls{ac} does by it self by covering the ear. The passive circumaural \gls{ac} can be divided intro two subgroups, an open passive circumaural \gls{ac} and a closed passive circumaural. An open passive circumaural \gls{ac} is a design, where the case around of the speaker has openings. The holes is often directly in back of the speaker, which mean the the speaker is not is a closed cabinet, but in an open case where the opening is at the back of the case. Because of the opening in the back, waves from the outside will move the loudspeaker membrane and the wave will travel further to the ear. A problem which may occur with the open passive circumaural \gls{ac}, is that the loudspeaker driver is shorted in the low frequency, if the chamber between the loudspeaker driver and the eardrum is highly open. A low frequency short circuit of a loudspeaker driver will result in a reduced transfer of low frequency between the loudspeaker driver and the eardrum. The closed passive circumaural \gls{ac} is where the loudspeaker is in a back closed case, so no holes in the back and often less shout circuit. Because it is only less shout circuit and not no shout circuit is because the mass construction of the closed passive circumaural \gls{ac} often is light and thin which mean that the case will vibrate along.

%An overview of the circumaural \gls{ac} is then as following.

%\begin{itemize}
%\item active circumaural \gls{ac}.
%\item open passive circumaural \gls{ac}.
%\item closed passive circumaural \gls{ac}.
%\end{itemize}

%\subsection{On-ear \gls{ac}} 
%On-ear \gls{ac} is a \gls{ac} which does not cover the full ear but only part of the ear, and therefore they may allow more noise reaching the eardrum than closed circumaural \gls{ac} and insertion \gls{ac}. This kind of \gls{ac} can be divided intro two types, active and passive. The functionality for active and passive noise cancellation follow the same principle that the circumaural \gls{ac} as explained in \autoref{circumaural_ac} but with the exception that the on-ear \gls{ac} do not cover the full ear. The passive version of the on-ear \gls{ac} can be divided intro to subgroup as the circumaural \gls{ac}, open back and closed back. Those two passive on-ear \gls{ac} also follows the same principle as the circumaural \gls{ac}, which mean that they have the same benefit and weaknesses.

%An overview of the on-ear \gls{ac} is then as following.

%\begin{itemize}
%\item active on-ear \gls{ac}.
%\item open passive on-ear \gls{ac}.
%\item closed passive on-ear \gls{ac}.
%\end{itemize}

%\subsection{insertion \gls{ac}} 